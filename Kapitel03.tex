\newchapter{Variationsungleichungen}

Dieses Kapitel basiert auf \cite{KikOden}, \cite{StarkeVar}, \cite{EPS}, \cite{EPSContact}, \cite{WriggersFEM}, \cite{WriggersContact}, \cite{HlaHas}, \cite{Glow}, \cite{Falk}.

\section{Ein Hindernisproblem}

\begin{itemize}
\item Hindernisproblem: Auslenkung $u$ einer Membran $\Omega$ unter Krafteinwirkung $f$, wobei die Membran durch ein Hindernis $\psi$ behindert wird. Mathematische modelliert bedeutet dies:
\begin{align}\label{eq:Hindernis}
\min_{v\in K} J(v) = \frac 1 2 a(v,v)-(f,v)
\end{align}
mit $K := \{v \in H^1_0(\Omega) \with v\ge \psi$ fast überall in $\Omega\}$. 

\item $J$ gibt wieder die in der Membran gespeicherte Energie an.
\item wobei jetzt die Lösung nicht auf ganz $H^1_0(\Omega)$ gesucht ist, sondern in einer Teilmenge $K \subset H^1_0(\Omega)$.

\item wir können auch hier eine Variationsformulierung, die äquivalent zu \eqref{eq:Hindernis} ist, herleiten

\item zu Beginn noch eine Skizze von einem Hindernisproblem

\begin{figure}[h]
\caption{Ein Hindernisproblem mit Hindernis $\psi$ und Last $f$}
\end{figure}

%\item hierfür zunächst wieder etwas allgemeineres Energiefunktional $J: H \ra \R, J(v) = \frac 1 2 a(v,v)-F(v)$, wobei $H$ Hilbertraum
\end{itemize}



\subsection{Variationsformulierung für das Hindernisproblem}

\begin{itemize}
\item zeigen zunächst, dass $K$ konvex und abgeschlossen ist.
\begin{lemma}\label{lem:3.1}
Die Menge $K =  \{v \in H^1_0(\Omega) \with v\ge \psi\text{ fast überall in }\Omega\}$ ist eine konvexe abgeschlossene Teilmenge von $H^1_0(\Omega)$.
\end{lemma}

\begin{proof}
(i) Es seien $u,v \in K$, d.h. $u \ge \psi$ und $v \ge \psi$ fast überall in $\Omega$. Dann gilt für $t \in [0,1]$
\begin{align*}
	(1-t) u + tv \ge (1-t)\psi + t \psi = \psi  \, ,
\end{align*}
somit ist $(1-t)u + tv \in K$, also $K$ konvex.

(ii) Es sei $(v_n)_{n\in \N}\subset K$ eine konvergente Folge mit $v_n \ra v$ für $n\ra \infty$. Da $H^1_0(\Omega)$ ein abgeschlossener Unterraum von $H^1(\Omega)$ (vgl. auch \cite{Walker} Bemerkung 6.7) ist, folgt direkt $v \in H^1_0(\Omega)$. Da weiter $v_n \ge \psi$ für alle $n \in \N$ gilt, muss auch $v \ge \psi$ gelten und damit folgt $v \in K$, d.h. $K$ ist abgeschlossen.
\end{proof}

\item \begin{satz}\label{satz:3.2}
Es sei $K =   \{v \in H^1_0(\Omega) \with v\ge \psi\text{ fast überall in }\Omega\}$. Das Minimierungsproblem
\begin{align}\label{eq:3.2}
	\min_{v\in K} J(v) = \frac 1 2 a(v,v)-(f,v)
\end{align}
ist äquivalent zur Variationsungleichung: Finde $u \in K$, so dass
\begin{align}\label{eq:3.3}
	a(u,v-u) \ge (f,v-u) \quad \forall \, v \in K \, .
\end{align}
\end{satz}

\begin{proof}
Aus Lemma \ref{lem:2.3} folgt, dass $J$ konvex ist und damit gilt mit Satz \ref{satz:A.10}, dass $u \in K$ genau dann eine Lösung von \eqref{eq:3.2} ist, wenn
\begin{align}\label{eq:3.4}
	\mscr D_{v-u} J(u) \ge 0 \quad \forall \, v \in K 
\end{align}
gilt. Analog zu der berechneten Gâteaux-Ableitung von $J$ in Lemma \ref{lem:2.4}, gilt
\[
	\mscr D_{v-u} J(u) = \frac d{dt} J(u+t(v-u))\Big|_{t=0} = a(u,v-u) -(f,v-u)
\]
und damit folgt mit \eqref{eq:3.4} die Behauptung.
\end{proof}

\item \begin{bem}
Wie man mit Satz \ref{satz:A.10} sehen kann, gilt analog zu Satz \ref{satz:3.2} auch allgemeiner: Es sei $K\subset H$ eine konvexe Teilmenge eines Hilbertraumes $H$. Dann ist
\begin{align*}
	\min_{v\in K} J(v) = \frac 1 2 a(v,v)-F(v)
\end{align*}
äquivalent zur Variationsungleichung: Finde $u \in K$, so dass
\begin{align*}
	a(u,v-u) \ge F(v-u) \quad \forall \, v \in K \, ,
\end{align*}
wobei $F :H\ra \R$ eine lineare stetige Abbildung ist.
\end{bem}

\item auch für das Hindernisproblem gibt es analog zum homogenen Dirichlet-Problem \eqref{eq:2.2} eine äquivalente starke Formulierung

\item \begin{satz}[Starke Formulierung des Hindernisproblems]\label{satz:3.4} Jede Lösung $u \in H^2(\Omega) \cap H^1_0(\Omega)$ des Problems
\begin{align}\label{eq:3.5}
\begin{aligned}
	-\Delta u -f&\ge 0 \\
	u-\psi &\ge 0 \\
	(u-\psi) (-\Delta u &- f) = 0
\end{aligned}
\end{align}
mit $\psi \in H^1(\Omega)$ erfüllt die Variationsungleichung \eqref{eq:3.3}. Umgekehrt ist jede Lösung $u\in H^2(\Omega) \cap K$ von \eqref{eq:3.3} auch eine Lösung von \eqref{eq:3.5}.
\end{satz}

\begin{proof}
"`$\Ra$"' Sei $u \in H^2(\Omega) \cap H^1_0(\Omega)$ eine Lösung von \eqref{eq:3.5}, dann gilt für ein beliebiges $v \in K$
\begin{align*}
	\int_\Omega (-\Delta u - f) (v-u) \, dx & = \underbrace{- \int_\Omega \Delta u  (v-u) \, dx}_{\parbox{3.5cm}{\scriptsize$\stackrel{\text{Green}}= \int_\Omega \nabla u \nabla (v-u) \, dx$ \\ \text{ }\text{ } \text{ }\text{ } $ -\int_\Gamma \underbrace{(v-u)}_{=0} \partial_\nu u \, ds$}} - \int_\Omega  f (v-u) \, dx  \\
	& = \int_\Omega \nabla u \nabla(v-u) \, dx - \int_\Omega f(v-u) \\
	& = a(u,v-u) - (f,v-u) \, .
\end{align*}
Mit $\Omega_0 := \{ x \in \Omega \with u = \psi\}$ folgt, dass $-\Delta u = f$ auf $\Omega_1 := \Omega \setminus \bar\Omega_0$ gelten muss.
\begin{align*}
	 \lra\,   \int_{\Omega = \Omega_0 \cup \Omega_1} \underbrace{(-\Delta u - f)}_{=0 \text{ auf } \Omega_1} (v-u) \, dx  = \int_{\Omega_0}\underbrace{ (-\Delta u - f)}_{\ge0} \underbrace{(v-\psi)}_{\ge 0} \, dx \ge 0
\end{align*}
Damit ist $u$ eine Lösung von \eqref{eq:3.3}
\[
	a(u,v-u) \ge (f,v-u) \quad \forall \, v \in K\, .
\]
"`$\La$"' Es sei $u \in H^2(\Omega) \cap K$ Lösung von \eqref{eq:3.3}. Weiter sei $v \in K$  beliebig, dann gilt
\begin{align}\label{eq:3.6}
\begin{aligned}
	0 & \le a(u,v-u) - (f,v-u) \\
	&= \int_\Omega \nabla u \nabla(v-u) \, dx - \int_\Omega f(v-u) \, dx \\
	& \stackrel{\scriptsize\text{Green}}= \int_\Omega -\Delta u (v-u) \, dx - \int_\Omega f(v-u) \, dx \\
	& = \int_\Omega (-\Delta u - f) (v-u) \, dx \, .
\end{aligned}
\end{align}
Wir nehmen an, dass $-\Delta u -f < 0$ in einem Ball $B_{r_0} := B_{r_0} (x_0)\subset \Omega$ mit Radius $r_0$ um $x_0 \in \Omega$ gilt. Sei weiter $\chi \in C^\infty(\Omega)$ mit $\chi = 0$ auf $\Omega \setminus \bar B_{r_0}, \rho(r) := \(1-\frac r{r_0}\)^2 \chi >0$ und $v := u + \rho (r) \in K$, da $u\in K$ und $\rho (r) >0$. Dann gilt
\[
	\int_\Omega (-\Delta u - f) (v-u) \, dx = \int_{B_{r_0}} \underbrace{(-\Delta u - f)}_{< 0} \underbrace{\rho(r)}_{>0} \, dx < 0 \, ,
\]
was im Widerspruch zu \eqref{eq:3.6} steht. Also muss $-\Delta u - f \ge 0$ gelten.

Nun nehmen wir an, dass $-\Delta u -f > 0$ und $u > \psi$ fast überall in einem Ball $B_{r_0}$ gilt. Wir betrachten $v:= u  + \eps \rho(r) (\psi - u) \in K$ mit $0< \eps\le 1$, dann folgt
\[
	\int_\Omega (-\Delta u - f) (v-u) \, dx = \eps \int_{B_{r_0}} \underbrace{(-\Delta u -f)}_{>0} \underbrace{\rho(r)}_{>0} \underbrace{(\psi-u)}_{<0} \, dx < 0 \, ,
\]
was wiederum im Widerspruch zu \eqref{eq:3.6} steht. Damit muss $u = \psi$ gelten, wenn $-\Delta u = f$ ist. Es folgt, dass $u \in H^2(\Omega) \cap K$ eine Lösung von \eqref{eq:3.5} ist.
\end{proof}
\end{itemize}




\subsection{Existenz und Eindeutigkeit der Lösung}

\begin{itemize}
\item für die Existenz und Eindeutigkeit der Lösung des Problems betrachten wir zunächst wieder das allgemeine reelle quadratische Funktional $J: H \ra \R, J(v) = \frac 1 2 a(v,v) - F(v)$.

\item \begin{vor}
Sei $H$ ein reeller Hilbertraum mit Skalarprodukt $(\cdot,\cdot)_H$ und der damit induzierten Norm $\norm\cdot_H$. Mit $H'$ bezeichnen wir den Dualraum zu $H$. Weiter sei vorausgesetzt:
\begin{enumerate}[(a)]
\item $a: H\times H \ra \R$ ist eine stetige koerzive Bilinearform,
\item $F:H\ra\R$ ist ein stetiges lineares Funktional,
\item $K\not = \emptyset$ ist eine abgeschlossene konvexe Teilmenge von $H$.
\end{enumerate}
\end{vor}

\item \begin{theorem}[Existenz und Eindeutigkeit]\label{theorem:3.5}
Unter den obigen Voraussetzungen hat die Variationsungleichung, finde $u\in K$, so dass
\begin{align}\label{eq:3.7}
	a(u,v-u) \ge F(v-u) \quad \forall \, v \in K
\end{align}
ist, genau eine Lösung.
\end{theorem}

\begin{proof}
(i) Eindeutigkeit: Es seien $u_1,u_2 \in K$ zwei Lösungen der Variationsungleichung \eqref{eq:3.7}, d.h.
\begin{align}\label{eq:3.8}
	a(u_1,v-u_1) \ge F(v-u_1) \quad \forall \, v \in K\, , \\
	a(u_2,v-u_2) \ge F(v-u_2) \quad \forall \, v \in K\, . \label{eq:3.9}
\end{align}
Addieren wir \eqref{eq:3.8} und \eqref{eq:3.9} miteinander und setzen zuvor $v = u_2$ in \eqref{eq:3.8} und $v = u_1$ in \eqref{eq:3.9}, so erhalten wir
\begin{align*}
	0 & \le a(u_1,u_2-u_1) - F(u_2-u_1) + a(u_2,u_1-u_2) \underbrace{- F(u_1-u_2)}_{=F(u_2-u_1)}  \\
	& = a(u_1,u_2-u_1)-a(u_2,u_2-u_1) = -a(u_2-u_1,u_2-u_1) \\
	& \le -\alpha \norm{u_2-u_1}_H^2 \, .
\end{align*}
Also gilt $\norm{u_2-u_1}_H^2 \le 0 \Ra \norm{u_2-u_1}_H^2 = 0$ und damit folgt $u_1 = u_2$.

(ii) Existenz: Aus dem Darstellungssatz von Riesz bzw. das Korollar \ref{kor:2.14} folgt, dass ein $A \in \mcal L(H,H), l \in H$ existiert, so dass
\begin{align*}
	a(u,v) &= (Au,v)_H \quad \forall \, u,v \in H\, , \\
	F(v) &= (l,v)_H \qquad \forall \, v \in H \, .
\end{align*}
Damit gilt
\begin{align*}
	  F(v-u) - a(u,v-u) &= (l,v-u)_H - (Au,v-u)_H \\
	 &=   (l-Au,v-u)_H \le 0 \, .
\end{align*}
Durch Multiplikation mit $\varrho > 0$ und Addition der Null erhalten wir das äquivalente Problem: Finde $u \in K$, so dass
\begin{align}
	(u-\varrho(Au-l)-u,v-u)_H \le 0 \quad \forall \, v \in K \, .
\end{align}
Nach Satz \ref{satz:2.3} ist $u$ damit das Bild der Projektion von $u-\varrho (Au-l)$ auf $K$, d.h.
\[
	u = P_K (u-\varrho (Au-l)) \,.
\]
Es bleibt zu zeigen, dass $W_\varrho : H \ra K, W_\varrho (v) \coloneqq P_K(v-\varrho (Av-l))$ einen Fixpunkt besitzt. Mit Anwendung von Satz \ref{satz:2.4} und der Koerzivität von $a$ rechnen wir nach, dass
\begin{align*}
	\norm{W_\varrho (v_1) - W_\varrho (v_2)}_H^2  & = \norm{ P_K (v_1-\varrho (Av_1-l))- P_K (v_2-\varrho (Av_2-l))  }_H^2 \\
	& \le \norm{v_1-\varrho (Av_1-l)- (v_2-\varrho (Av_2-l))  }_H^2  \\
	& = \norm{(v_1-v_2)-\varrho \, A(v_1-v_2)  }_H^2  \\
	& = \norm{v_1-v_2}_H^2 + \varrho^2 \norm{A(v_1-v_2)}^2_H \\
	& \ \ \, - \underbrace{ \varrho\, (A(v_1-v_2),v_1-v_2)_H - \varrho\, (v_1-v_2,A(v_1-v_2))_H}_{=2\varrho\, (A(v_1-v_2),v_1-v_2)_H  =2\varrho\, a(v_1-v_2,v_1-v_2)} \\
	& \le \norm{v_1-v_2}_H^2 + \varrho^2\, \norm A^2 \norm{v_1-v_2}_H^2 - 2\varrho\alpha \, \norm{v_1-v_2}_H^2 \\
	& = (1-2\varrho \alpha+\varrho^2 \, \norm A^2) \,\norm{v_1-v_2}_H^2
\end{align*}
mit $\norm A := \sup_{v \in H} \frac{\norm{Av}_H}{\norm v_H}$. Also ist die Abbildung $W_\varrho$  eine Kontraktion, wenn gilt
\begin{align*}
	1-2\varrho \alpha+\varrho^2 \, \norm A^2 < 1 \, \lra\, 0 < \varrho < \frac {2\alpha}{\norm A^2} \, .
\end{align*}
Nach dem Banach'scher Fixpunktsatz (vgl. \cite{Stoer} Satz 5.2.3) existiert für solch ein $\varrho$ ein $u \in H$ mit $u = W_\varrho (u) = P_K(u-\varrho (Au-l))$.

Insgesamt gibt es also für das Problem \eqref{eq:3.7} genau eine  Lösung.
\end{proof}

\item \begin{kor}
Das Problem \eqref{eq:Hindernis} hat eine eindeutige Lösung.
\end{kor}

\begin{proof}
Da laut Lemma \ref{lem:3.1} die Menge 
\[
	K=\{v \in H^1_0(\Omega) \with v \ge \psi \text{ fast überall in }\Omega\}
\]
abgeschlossen und konvex ist, $F(v) = (f,v)$ ein stetiges lineares Funktional und 
\[
	a(u,v) = \int_\Omega \nabla u \nabla v \, dx
\]
stetig bilinear und koerziv, sind die Voraussetzungen für Theorem \ref{theorem:3.5} erfüllt. Damit hat das Problem, finde $u \in K$, so dass
\begin{align}\label{eq:3.11}
	a(u,v-u) \ge (f,v-u) \quad \forall \, v \in K \, ,
\end{align}
genau eine Lösung. Nach Satz \ref{satz:3.2} ist \eqref{eq:Hindernis} äquivalent zu \eqref{eq:3.11} und damit folgt die Behauptung.
\end{proof}

\item \begin{bem}
Insbesondere hat auch das Problem \eqref{eq:3.5} nach Satz \ref{satz:3.4} und Theorem \ref{theorem:3.5} eine eindeutige Lösung, wenn $u \in H^2(\Omega) \cap H^1_0(\Omega)$ ist.
\end{bem}
\end{itemize}







\subsection{Lösung des Hindernisproblems mittels FEM}

\begin{itemize}
\item Analog zum vorherigen Kapitel kann man auch im $\R^n$ Existenz und Eindeutigkeit der Lösung unter bestimmten Voraussetzungen zeigen. (vgl. Vug Skript Kapitel 2) $\Ra$ Beachte hierfür auch den Fixpunktsatz von Brouwer.

\item zur Lösung mittels FEM betrachten wir die Variationsungleichung \eqref{eq:3.11} bzgl. eines endlich dimensionalen Unterraum
\[
	K_h \coloneqq \{v_h \in \mcal S_h \with v_h (p) \ge \psi(p) \, \forall \, p \in \mcal N \cap \Omega\}\, ,
\]
wobei $\mcal N$ die Knotenmenge bzgl. der Triangulierung $\mcal T_h$ bezeichne.

\item damit ist \eqref{eq:3.11} in diskreter Form: Finde $u_h \in K_h$, so dass
\begin{align}\label{eq:3.12}
	a(u_h,v_h-u_h) \ge (f,v_h-u_h) \quad \forall \, v_h \in K_h \, .
\end{align}

\item vgl. \cite{Werner} Kapitel 4 Satz 7.15.
\begin{satz}[\idx{Fixpunktsatz von Brouwer}]
Es sei $K \not= \emptyset$ eine kompakte konvexe Teilmenge eines endlich dimensionalen normierten Raumes $H$ und $F: K \ra K$ sei stetig. Dann besitzt $F$ einen Fixpunkt $v \in K$.
\end{satz}

\begin{proof}
Der Beweis ist in \cite{Werner} Kapitel 4 Satz 7.15 zu finden.
\end{proof}

\item \begin{theorem}[Existenz und Eindeutigkeit]
Das Problem \eqref{eq:3.12} hat eine eindeutige Lösung $u_h \in K_h$.
\end{theorem}

\begin{proof}
Der Beweis ist analog zu Theorem \ref{theorem:3.5} zu führen. Wir ersetzen lediglich $H$ durch $V_h$ und $K$ durch $K_h$ und verwenden im endlich dimensionalen Raum $V_h$ den Fixpunktsatz von Brouwer.
\end{proof}

\item \begin{bem*}
In Kapitel 2.2 von \cite{StarkePDE} sind die Argumente bzgl. der Existenz und Eindeutigkeit einer Lösung von \eqref{eq:3.11} für den endlich dimensionalen Fall $K_h$ auch noch einmal im Einzelnen aufgeführt.
\end{bem*}

\item Es sei $\mcal B_h = \{\phi_1,\ldots,\phi_N\}$ eine modale Basis von $\mcal S_h$, d.h. analog zu \eqref{eq:2.10} können wir $u_h$ und $v_h$ mit Koordinaten $\mu_i,\nu_i, i = 1,\ldots,N$ bzgl. $\mcal B_h$ ausdrücken. Dann schreiben wir \eqref{eq:3.12} als
\begin{align*}
	\sum_{i = 1}^N\sum_{j=1}^N \mu_i \, a(\phi_i,\phi_j)\, (\nu_j-\mu_j) & \ge \sum_{j=1}^N  (f,\phi_j) \, (\nu_j-\mu_j)  \\
	\Llra\quad \qquad  \qquad \qquad \bs \mu^T A (\bs \nu- \bs\mu) &  \ge \bs f^T (\bs \nu-\bs \mu)
\end{align*}
mit $A = [a(\phi_j,\phi_i)]_{i,j=1}^N, \bs \mu = [\mu_i]_{i=1}^N,\bs \nu = [\nu_i]_{i=1}^N$ und $\bs f = [(f,\phi_i)]_{i=1}^N$.

\item Die Menge $K_h$ ist bzgl. $\mcal S_h$ äquivalent zu
\begin{align}\label{eq:3.13}
	K_{\mcal S} \coloneqq \{\bs \nu \in \R^N \with \nu_i \ge \psi(p_i) , p_i \in \mcal N \cap \Omega, i = 1,\ldots, N \} \, .
\end{align}
Im Folgenden schreiben wir $\bs \psi \coloneqq [\psi(p_i)]_{i=1}^N$ mit $p_i \in \mcal N\cap \Omega$.

\item \begin{bem*}
$K_{\mcal S}$ ist analog zu $K$ konvex und abgeschlossen.
\end{bem*}

\item Damit erhalten wir aus \eqref{eq:3.12} die diskrete Variationsungleichung: Finde $\bs\mu \in K_{\mcal S}$, so dass
\begin{align}\label{eq:3.14}
	(A\bs \mu-\bs f)^T (\bs \nu- \bs\mu) &  \ge 0 \quad \forall \, \bs \nu \in K_{\mcal S} \, .
\end{align}

\item \begin{satz}\label{satz:3.10}
Das Problem \eqref{eq:3.14} ist äquivalent zum linearen Komplementaritätsproblem\index{lineares Komplementaritätsproblem}: Bestimme $\bs\mu \in K_{\mcal S}$, so dass
\begin{align}\label{eq:3.15}
	A\bs\mu - \bs f \ge \bs 0\quad\text{ und }\quad (A\bs\mu-\bs f)^T(\bs \mu - \bs\psi) = 0
\end{align}
gilt.
\end{satz}

\begin{proof}
"`$\Ra$"' Sei $\bs \mu \in K_{\mcal S}$ Lösung von \eqref{eq:3.14}. Wir setzen $\bs\nu = \bs \mu +\bs e_i \ge \bs\psi$ mit einem beliebigen $i \in \{1,\ldots,N\}$, wobei $\bs e_i $ den $i$-te Einheitsvektor bezeichne. Dann gilt
\[
	0 \le (A\bs\mu - \bs f)^T (\bs\nu - \bs\mu) = (A\bs\mu - \bs f)^T \bs e_i = (A\bs \mu-\bs f)_i \, .
\]
Da $i$ beliebig war, folgt $A\bs\mu - \bs f\ge \bs 0$.

Wir nun nehmen an, dass ein $i \in \{1,\ldots,N\}$ existiert, so dass $(A\bs\mu - \bs f)_i (\bs\mu-\bs\psi)_i >0$ ist. Weiter wählen wir
\[
	\bs\nu = \begin{pmatrix}
				\mu_1 \\
				\vdots \\
				\mu_{i-1} \\
				0 \\
				\mu_{i+1}\\
				\vdots \\
				\mu_N
			\end{pmatrix} + 
			\begin{pmatrix}
				0 \\
				\vdots \\
				0\\
				\psi_i \\
				0\\
				\vdots \\
				0
			\end{pmatrix} \ge \bs\psi
\]
und damit folgt
\begin{align*}
	0 > (A\bs \mu - \bs f)_i (\bs\psi - \bs\mu)_i  = (A\bs\mu-\bs f)^T (\bs \nu-\bs \mu) \ge 0 \, ,
\end{align*}
was im Widerspruch zu \eqref{eq:3.14} steht, daraus folgt die Behauptung.

"`$\La$"' Es sei $\bs \mu \in K_{\mcal S}$ Lösung von \eqref{eq:3.15}. Dann rechnen wir nach, dass für ein beliebiges $\bs\nu \in K_{\mcal S}$ gilt
\begin{align*}
	(A\bs \mu-\bs f)^T (\bs \nu- \bs\mu) & = (A\bs \mu-\bs f)^T (\bs \nu -\bs \psi + \bs\psi- \bs\mu) \\
	& = \underbrace{(A\bs \mu-\bs f)^T}_{\ge \bs 0} \underbrace{(\bs \nu -\bs \psi)}_{\ge \bs 0} -\underbrace{(A\bs \mu-\bs f)^T (\bs\mu-\bs\psi)}_{=0} \\
	& \ge 0\, . \qedhere
\end{align*}
\end{proof}

\item das Problem \eqref{eq:3.14} ist äquivalent zu einem quadratischen Optimierungsproblem
\begin{satz}[Äquivalenz zu quadratischem Programm]\label{satz:3.11}
Das Problem \eqref{eq:3.14} ist äquivalent zum quadratischen Programm\index{quadratisches Programm}
\begin{align}\label{eq:3.16}
	\min_{\bs\nu \in \R^N} J(\bs\nu) = \frac 1 2 \bs \nu^T A \bs \nu - \bs f^T \bs \nu\quad \text{s.t.} \quad \bs \nu \ge \bs \psi \, .
\end{align}
\end{satz}

\begin{proof}
Wir zeigen zunächst die Äquivalenz von \eqref{eq:3.15} zu \eqref{eq:3.16} und dann folgt mit Satz \ref{satz:3.10} die Behauptung.

"`$\Ra$"' Es sei $\bs\mu \in \R^N$ Lösung vom Problem \eqref{eq:3.15}. Dann folgt mit einem beliebigen $\bs\nu \in K_{\mcal S}$
\begin{align*}
	J(\bs\nu) - J(\bs\mu) & =  \frac 1 2 \bs \nu^T A \bs \nu - \bs f^T \bs \nu -  \frac 1 2 \bs \mu^T A \bs \mu + \bs f^T \bs \mu \\
	& = \frac 1 2\underbrace{ (\bs\nu-\bs \mu)^T A(\bs\nu-\bs\mu) }_{\ge 0 \text{ wegen Bem. \ref{bem:2.17}}}+ \bs \mu^T A \bs \nu - \bs \mu ^TA \bs\mu - \bs f^T (\bs\nu-\bs\mu) \\
	%& \ge (A\bs \mu - \bs f)^T(\bs\nu-\bs\mu) \\
	& \ge (A\bs \mu - \bs f)^T(\bs\nu-\bs\psi+\bs\psi-\bs\mu) \\
	& = \underbrace{(A\bs \mu - \bs f)^T}_{\ge 0} \underbrace{(\bs\nu-\bs\psi)}_{\ge 0}- \underbrace{(A\bs \mu - \bs f)^T(\bs\mu-\bs\psi) }_{=0}\\
	&  \ge 0\, .
\end{align*}
Somit ist $\bs\mu \in K_{\mcal S}$ auch Lösung des quadratischen Programms \eqref{eq:3.16}.

"`$\La$"' Sei $\bs\mu \in K_{\mcal S}$ Lösung von \eqref{eq:3.16}, dann gelten nach \cite{NocWri} Kapitel 12, Theorem 12.1 für die Lagrange-Funktion
\[
	\mcal L(\bs\nu,\bs\lambda) = J(\bs\nu)-\bs\lambda^T(\bs\nu-\bs\psi)
\]
die Karush-Kuhn-Tucker Bedingungen für den Optimalpunkt $(\bs \mu,\bs\lambda^*)$
\begin{align}\label{eq:3.17}
	\nabla_{\bs\nu} \mcal L(\bs\mu,\bs\lambda^*) = \nabla J(\bs\mu)-\lambda^* & = A\bs\mu-\bs f - \bs \lambda^* \stackrel != \bs 0 \, ,\\
	\label{eq:3.18}
	\bs\mu - \bs \psi & \ge \bs 0 \, , \\
	\label{eq:3.19}
	\bs \lambda^* & \ge \bs 0 \, , \\
	\label{eq:3.20}
	\lambda_i^* (\mu_i-\psi_i) & = 0 \qquad \forall \, i =1,\ldots,N \, .
\end{align}
Mit \eqref{eq:3.17} gilt also $\bs\lambda^* = A\bs\mu-\bs f$ und daher folgt aus \eqref{eq:3.19}
\[
	A\bs\mu - \bs f \ge\bs 0 \, .
\]
Aus \eqref{eq:3.20} folgt wegen $(A\bs\mu-\bs f)_i (\mu_i-\psi_i) = 0$ für alle $i = 1,\ldots,N$ direkt
\[
	(A\bs\mu - \bs f)^T (\bs \mu-\bs\psi) = 0\, .
\]
Also ist $\mu \in K_{\mcal S}$ auch Lösung von \eqref{eq:3.15}.
\end{proof}

\item \begin{bem}
Analog zu Satz \ref{satz:3.10} und \ref{satz:3.11} können wir auch zeigen, dass das quadratische Programm
\begin{align}\label{eq:3.21}
	\min_{\bs\nu \in \R^N} J(\bs\nu) = \frac 1 2 \bs \nu^T A \bs \nu - \bs f^T \bs \nu \quad \text{s.t.} \quad B\bs\nu \ge \bs \psi
\end{align}
mit $B \in \R^{M\times N}$ äquivalent ist zur Variationsungleichung: Finde $\bs\mu \in \R^N$ mit $B\bs\mu \ge \bs \psi$, so dass
\begin{align}\label{eq:3.22}
	(A\bs\mu - \bs f)^T (\bs\nu-\bs\mu) \ge 0 \quad \forall\, \bs\nu \in \R^N \text{ mit } B \bs \nu \ge \bs\psi \, .
\end{align}
\end{bem}


\item \begin{bem}
Die quadratischen Programme \eqref{eq:3.16} und \eqref{eq:3.21} hat mit den Voraussetzungen aus Bemerkung \ref{bem:2.17} eine globale Lösung $\bs\mu$, wenn diese die KKT-Bedingungen \eqref{eq:KKT} erfüllt (vgl. Theorem \ref{theorem:B.1}).
\end{bem}

\item mit dem im Anhang \ref{anhang:B.2} vorgestellten \idx{Active-Set-Algorithmus} kann ein solches quadratisches Programm gelöst werden

\item es bleibt noch zu prüfen, ob sich die Lösung der Variationsungleichung für Netzverfeinerung an die exakte Lösung konvergiert

\item \begin{vor}
Gegeben sei ein Parameter $h \ra 0$. Weiter seien $(V_h)_h$ eine Familie aus abgeschlossenen Teilmengen von einem Hilbertraum $H, \emptyset \not= K \subset H$ eine konvexe abgeschlossene Teilmenge und $(K_h)_h$ eine Familie von abgeschlossenen konvexen nichtleeren Teilmengen von $H$, so dass $K_h \subset V_h$ für alle $h$.

Dabei sei $K_h$ eine Approximation von $K$ im folgenden Sinne:
\begin{enumerate}[(i)]
\item wenn $(v_h)_h$ eine in $V$ beschränkte Folge mit $v_h \in K_h$ ist, dann folgt $v_h \ra v \in K$,
\item es existiert ein $W \subset V$ mit $\overline W = K$ and ein $I_h: W \ra K_h$, so dass
\[
	\lim_{h\ra 0} I_h v = v
\]
stark in $V$ für alle $v \in W$ konvergiert.
\end{enumerate}
\end{vor}

\item Konvergenz von $u_h$ gegen die exakte Lösung $u$.
\begin{theorem}[a priori Konvergenz]
Mit den obigen Voraussetzungen für $K$ und $(K_h)_h$ gilt für die Lösungen $u$ von \eqref{eq:3.7} und $u_h$ vom approximierten Problem: Finde $u_h \in K_h$, so dass
\begin{align}\label{eq:3.23}
	a(u_h,v_h-u_h) \ge F(v_h-u_h) \quad \forall \, v_h \in K_h \, , 
\end{align}
der Zusammenhang
\begin{align*}
	\lim_{h\ra 0} \norm{u_h-u}_H  = 0 \, .
\end{align*}
\end{theorem}

\begin{proof}
(i) \underline{Abschätzung von $u_h$:} Es sei $u_h$ Lösung von \eqref{eq:3.23}, dann gilt nach einer Umformung für alle $v_h \in K_h$
\begin{align*}
	a(u_h,u_h) & \le a(u_h,v_h)-F(v_h-u_h) \\ 
	& = (Au_h,v_h)_H - (f,v_h-u_h)_H \\
	& \stackrel{\scriptsize\textnormal{CS}}\le \underbrace{\norm{Au_h}_H}_{\le\norm{A} \norm{u_h}_H} \, \norm{v_h}_H + \norm{f}_H\underbrace{\norm{v_h-u_h}_H}_{\le \norm{v_h}_H+\norm{u_h}_H} \\
	& \le \norm A \, \norm{u_h}_H  \norm{v_h}_H + \norm f_H \, (\norm{v_h}_H+\norm{u_h}_H)  \, .
\end{align*}
Zusammen mit der Koerzivität folgt dann
\begin{align}\label{eq:3.24}
	\alpha \, \norm{u_h}^2_H \le \norm A \, \norm{u_h}_H  \norm{v_h}_H + \norm f_H \, (\norm{v_h}_H+\norm{u_h}_H) \, .
\end{align}
Wähle ein festes $v_0 \in W$, sodass $I_h v_0 = v_h \in K_h$ gilt. Aus Voraussetzungen (ii) folgt dann
\[
	\lim_{h\ra 0} I_h v_0 = v_0
\]
und daher muss $v_h$ beschränkt sein, d.h. es existiert ein $m \in \R: \norm{v_h}_H \le m$ für alle $h$. Zusammen mit \eqref{eq:3.24} gilt dann
\begin{align*}
	\norm{u_h}^2_H & \le \frac 1 \alpha (m \, \norm A \, \norm{u_h}_H + \norm f_H (m+\norm{u_h}_H)) \\
	& = \underbrace{\(\frac m \alpha\, \norm A +  \norm f_H \)}_{\eqqcolon c_1} \norm{u_h}_H + \underbrace{\frac m\alpha\norm f_H}_{ \eqqcolon c_2} \\
	& = c_1 \, \norm{u_h}_H + c_2
\end{align*}
und damit können wir durch quadratischer Ergänzung folgern, dass es ein $c \in \R$ gibt mit $\norm{u_h} \le c$ für alle $h$, d.h. $(u_h)_h$ ist gleichmäßig beschränkt.

(ii) \underline{schwache Konvergenz:} Da $(u_h)_h$ in $H$ gleichmäßig beschränkt ist, folgt mit Bemerkung \ref{bem:A.13} (b), dass es eine schwach konvergente Teilfolge $(u_{h_j})_{h_j} \in K_{h_j}$ mit einem Grenzwert $u^*$ in $H$ gibt, d.h.
\[
	u_{h_j} \rightharpoonup u^* \in H \, .
\]
Mit den Voraussetzungen (i) für $(K_h)_h$ folgt direkt $u^* \in K$, außerdem ist $u^*$ nach Bemerkung \ref{bem:A.13} (e) eindeutig.

Wir zeigen nun, dass $u^*$ eine Lösung von \eqref{eq:3.7} ist. Für die oben betrachtete Teilfolge gilt
\begin{align}\label{eq:3.25}
	a(u_{h_j},v_{h_j}-u_{h_j})\ge F(v_{h_j}-u_{h_j}) \quad \forall \, v_{h_j} \in K_{h_j} \, .
\end{align}
Sei $v \in W$ mit $v_{h_j} = I_{h_j} v$. Dann gilt $v_{h_j} = I_{h_j} v \ra v \in W$ für $h_j \ra 0$. Mit \eqref{eq:3.25} folgt
\begin{align}
	\notag a(u_{h_j},u_{h_j}) &\le a(u_{h_j},v_{h_j})-F(v_{h_j}-u_{h_j}) \\
	\notag & = a(u_{h_j},I_{h_j} v)-F(v_{h_j}-u_{h_j}) \\
	\label{eq:3.26} \lra \liminf_{h_j\ra 0} a(u_{h_j},u_{h_j}) & \le  a(u^*,v)-F(v-u^*) \, .
\end{align}
Weiter schätzen wir durch Bemerkung \ref{bem:A.13} (f) nach unten ab
\begin{align}\label{eq:3.27}
	\liminf_{h_j\ra 0} a(u_{h_j},u_{h_j}) = \liminf_{h_j\ra 0} \norm{u_{h_j}}_H^2 \ge \norm{u^*}_H^2 = a(u^*,u^*) \, .
\end{align}
Insgesamt folgt also mit \eqref{eq:3.26} und \eqref{eq:3.27}
\[
	a(u^*,u^*) \le \liminf_{h_j\ra 0} a(u_{h_j},u_{h_j}) \le   a(u^*,v)-F(v-u^*)
\]
und damit nach Umformung
\[
	a(u^*,v-u^*)\ge F(v-u^*) \quad \forall \, v \in W \, .
\]
Da $W$ dicht in $K$ liegt, d.h. $\overline W = K$, und $a, F$ stetig sind, erhalten wir
\[
	a(u^*,v-u^*)\ge F(v-u^*) \quad \forall \, v \in K 
\]
mit $u^*\in K$, also ist $u^*\eqqcolon u$ Lösung von \eqref{eq:3.7}. Da $u$ ein Häufungspunkt von $(u_h)_h$ bzgl. der schwachen Topologie von $H$ ist, konvergiert auch die Folge $(u_h)_h$ schwach gegen $u$.

(iii) \underline{starke Konvergenz:} Aus der Koerzivität von $a$ folgt
\begin{align}\label{eq:3.28}
	\begin{aligned}
		0 &\le \alpha \, \norm{u_h-u}_H^2 \le a(u_h-u,u_h-u) \\
		& \le a(u_h,u_h)-a(u_h,u)-a(u,u_h)+a(u,u) \, ,
	\end{aligned}
\end{align}
wobei $u_h$ Lösung vom approximierten Problem \eqref{eq:3.23} und $u$ Lösung vom exakten Problem \eqref{eq:3.7} ist. Es sei $v \in W$ mit $I_h v = v_h \in K_h$, dann folgt aus \eqref{eq:3.23}
\begin{align}\label{eq:3.29}
	a(u_h,u_h) \le a(u_h,I_h v) - F(I_h v-u_h) \quad \forall \, v \in W \, .
\end{align}
Da $u_h \rightharpoonup u$ in $H$ und $I_h v \ra v$ in $H$ für $h \ra 0$, folgt aus \eqref{eq:3.28} und \eqref{eq:3.29} unter Verwendung von Voraussetzungen (ii)
\begin{align}\label{eq:3.30}
	0 \le \alpha \lim_{h\ra 0} \norm{u_h-u}^2_H \le a(u,v-u)-F(v-u) \quad \forall \, v \in W \, .
\end{align}
Da $a$ und $F$ stetig sind und $W$ dicht in $K$ liegt, gilt \eqref{eq:3.30} auch für alle $v \in K$. Setzen wir dann also $v = u$ in \eqref{eq:3.30}, dann folgt die Behauptung $\lim_{h\ra 0} \norm{u_h - u }_H^2 = 0$. 
\end{proof}

\item \underline{Überlegung:} Inwiefern hält unser $K_h$ diese Voraussetzungen ein.

\item Es lässt sich folglich auch eine a priori Abschätzung für den Fehler von $u$ und $u_h$ machen. Die Herleitung ist detailliert in \cite{Falk} wiederzufinden.
\begin{theorem}[a priori Fehlerabschätzung]
Es seien $u$ und $u_h$ die Lösungen von \eqref{eq:3.7} und \eqref{eq:3.23}. Dann existiert eine Konstante $C \coloneqq C(\Omega, f, \psi)$ unabhängig von $u$, so dass
\[
	\norm{u_h-u}_1 \le C h \, .
\]
\end{theorem}

\begin{proof}
Vgl. \cite{Falk}.
\end{proof}

\item damit führt die Netzverfeinerung also zur exakten Lösung der Variationsungleichung

\item inwiefern adaptive Netzverfeinerung hier sinnvoll ist, wollen wir in Kapitel 4 betrachten
\end{itemize}







\section{Kontaktprobleme}

\subsection{Mathematische Modellierung eines Kontaktproblems}

\begin{itemize}
\item 
\begin{vor}
Wir treffen folgende Annahmen für unser Kontaktmodell:
\begin{enumerate}[(a)]
\item Die in Kontakt stehenden Körper sind beschränkt.
\item Es liegen kleine Deformationen vor und ein linearer Zusammenhang zwischen Spannung und Verzerrung (Hooke'sches Gesetz).
\item Wir betrachten ein konstantes Temperaturfeld, d.h. thermodynamische Prozesse werden ausgeschlossen.
\item Zu Beginn, also in der Ausgangskonfiguration, gilt für die Spannung und Verzerrung: 
\[
	\bs \sigma = \bs 0\, , \quad \bs \eps =\bs 0 \, .
\]
%\item Wir gehen von einem ebenen Problem aus, d.h. $\Omega \subset \R^2$. Im dreidimensionalen analog.
\end{enumerate}
\end{vor}

\item zur Herleitung der starken Kontaktformulierung wollen wir zwei Körper $\mcal B^1, \mcal B^2$ betrachten, welche durch zwei beschränkte Gebiete $\Omega^1, \Omega^2$ mathematisch beschrieben werden können

\item diese Voraussetzung lässt sich noch weiter verallgemeinern (s. \cite{CarWri})

\item die Ränder $\Gamma^i$ von $\Omega^i, i = 1,2$, lassen sich in drei disjunkte Teile unterteilen
\begin{enumerate}
\item[$\Gamma^i_u$:] Der \textit{Dirichlet-Rand}\index{Randbedingungen!Dirichlet}, oder auch \textit{\idx{Verschiebungsrand}}, auf dem die Werte von der Verschiebung $\bs u$ vorgegeben sind.
\item[$\Gamma^i_\sigma$:] Der \textit{Neumann-Rand}\index{Randbedingungen!Neumann}, oder auch \textit{\idx{Spannungsrand}}, auf dem die Oberflächenlast bzw. -spannung $\bs t$ vorgegeben ist.
\item[$\Gamma^i_c$:] Der \textit{Kontaktrand}\index{Randbedingungen!Kontakt}, auf dem die Kontaktbedingungen definiert sind.
\end{enumerate}

\item[Skizze:] zwei Körper, deren Randunterteilung zu erkennen ist.
\begin{figure}[h]
\caption{Körper $\mcal B^1$ und $\mcal B^2$ mit Randbezeichnungen}
\end{figure}

\item zur Kontaktkinematik:

\item für die Formulierung der Kontaktbedingungen werden den Körpern die "`Werte"' \textit{master} und \textit{slave} zugeordnet.

\item slave ist dabei die Menge an Punkten, die überprüft werden, ob sie in die master-Fläche eindringen.

\item Zuordnung ist irrelevant $\ra$ kein Unterschied für das Ergebnis

\item o.B.d.A. sei $\mcal B^1$ slave

\item wir wollen zunächst die Kontaktkinematik etwas allgemeiner als in $\cite{WriggersFEM}$ oder $\cite{WriggersContact}$ beschrieben einführen

\item[Skizze:] Skizze mit zwei Körpern (nichtglatter Rand!!) und den Bezeichnungen $\chi(X)$ und $n_c(X)$.

\begin{figure}[h]
\caption{Kontaktformulierung zwischen zwei Körpern}
\end{figure}

\item für gegebenen Punkt $\bs X \in \Omega^1$, bzw. $\bs X \in \Gamma^1_c$, in der Ausgangskonfiguration ist $\bar{\bs X} \coloneqq \chi(\bs X)$ derjenige Punkt aus $\Gamma^2$, der minimalsten Abstand zu $\bs X$ hat, d.h.
\[
	\norm{\bs X-\bar{\bs X}} = \min \{\norm{\bs X-\bs Y} \with \bs Y \in \Gamma^2\} \, ,
\]
also ist $\chi: \Gamma_c^1\cup \Gamma_c^2 \ra \Gamma^1 \cup \Gamma^2$ eine Abbildung der kleinsten Distanz

\item damit definieren wir entsprechend die kritische Richtung mit Länge 1 als
\begin{align}
	\bs n_c(\bs X) \coloneqq \frac{\chi(\bs X) - \bs X}{\norm{\chi (\bs X)-\bs X}} \, ,
\end{align}
wobei im Falle $\bs X = \chi(\bs X)$, d.h. im Falle des Kontaktes, eine beliebige normierte Richtung gesetzt wird.

\item Bem.: $\bar{\bs X}$ ist daher kritischer Punkt, da er wegen des kleinsten Abstandes zu $\bs X$ der wohlmöglich nächste Punkt ist, der in Kontakt tritt

\item[Vorteil:] diese Formulierung bzgl. der kritischen Richtung kann auch verwendet werden, wenn der Rand der Körper nicht hinreichend glatt ist.

\item in den Koordinaten der Momentankonfiguration gilt $\bs x = \bs X + \bs u$ für das Verschiebungsfeld $\bs u$ und damit ergibt sich die \textit{\idx{Nichtdurchdringungsbedingung}}
\begin{align}\label{eq:3.32}
	(\bar{\bs x} - \bs x) \, \bs n_c (\bs X) \ge 0 \, , 
\end{align}
wobei $\bar{\bs x} \coloneqq \bar{\bs X} + u(\bar{\bs X})$ ist.

\item das bedeutet, dass die Verbindung der Punkte in der Momantankonfiguration mit der kritischen Richtung einen Winkel $\alpha \in [-\frac \pi2,\frac \pi2]$ einschließen muss, ansonsten läge $\bar{\bs x}$ "`hinter"' $\bs x$, d.h. $\mcal B^1$ wäre in $\mcal B^2$ eingedrungen.

\item 

\item aus \eqref{eq:3.32} folgt
\begin{align*}
	0 &  \le (\bar{\bs x} - \bs x) \, \bs n_c (\bs X) = (\bar{\bs X} + u(\bar{\bs X}) -{\bs X} - u({\bs X})) \, \bs n_c(\bs X) \\
	& = \underbrace{(\bs{\bar X} - \bs X) \frac{\bar{\bs X}-\bs X}{\norm{\bar{\bs X}-\bs X}}}_{ = \norm{\bar{\bs X}-\bs X}\eqqcolon g} + ( u(\bar{\bs X})  - u({\bs X})) \, \bs n_c(\bs X) \\
	& = g + ( u(\chi(\bs X))  - u({\bs X})) \, \bs n_c(\bs X) \, .
\end{align*}
damit erhalten wir die Nichtdurchdringungsbedingung bzgl. der Ausgangskonfiguration

\item da wir kleine Deformationen vorausgesetzt haben, gilt unter anderem $\bs X \approx \bs x,  \nabla_{\bs X} \approx \nabla_{\bs x}$ (vgl. \cite{AltKonti} S. 122f). Daher schreiben wir im folgenden immer $\bs x$ statt $\bs X$.

\item damit schreibt sich die Nichtdurchdringungsbedingung als
\begin{align}\label{eq:3.33}
	(\bs u \circ \chi -\bs u) \cdot \bs n_c + g \ge 0 \quad \forall \, \bs x \in \Gamma_c \, .
\end{align}

\item im Folgenden wollen wir der Einfachheit halber wir davon ausgehen, dass die Ränder $\Gamma^i$ hinreichend glatt sind $\Ra$ \eqref{eq:3.33} gilt auch für $\bs n_c$ als Einheitsnormale von $\mcal B^1$

\item weiter wollen wir $\bs u(\bar{\bs x})\equiv \bs 0$ annehmen.

\item damit reduziert sich \eqref{eq:3.33} auf
\[
	\bs u \cdot \bs n - g \le 0
\]


\item Starke Formulierung (s. Wriggers Paper) für Kontaktproblem mit Signorini-Kontakt (ohne Reibung).
\begin{align}
\div \bs \sigma + \bs b &= \bs 0 \text{ in } \Omega\\
\bs \sigma  - \mcal C \bs \eps & = \bs 0 \text{ in } \Omega\\
\bs \sigma \cdot \bs n &= \bs t  \text{ auf } \Gamma_N \\
\bs u &= \bs 0 \text{ auf } \Gamma_D \\
(\bs u \circ \chi - \bs u) \cdot \bs n_c + g& \ge 0 \text{ auf } \Gamma_C
\end{align}
sowie auf $\Gamma_C$ muss $\sigma_n \le 0$ (Normalenkraft $\sigma_n = \bs n\cdot ( \bs \sigma \cdot \bs n)$), $\bs \sigma_t = \bs 0$ (keine Tangentialkraft, da keine Reibung – $\bs \sigma_t = \bs \sigma \cdot \bs n - \sigma_n \bs n$) und $((\bs u \circ \chi - \bs u) \cdot \bs n_c + g)\sigma_n = 0$, d.h. wenn kein Kontakt ist, ist die Normalkraft in den Punkten Null, also herrscht Kräftegleichgewicht.
\item Anreißen von Kontaktproblem mit Tresca-Reibung (vgl. Numerik für Kontaktmechanik von Stephan und Vug von Starke) $\Ra$ Herleitung der Variationsungleichung durch Ableitung nicht mehr möglich, da Reibungspotential nicht mehr differenzierbar.
\end{itemize}

\subsection{Variationsformulierung für Kontaktprobleme}

\begin{itemize}
\item Minimierung von Energiefunktional (vgl. \cite{KikOden} Seite 112 unten) mit $\boldsymbol{u}: \Omega\ra \R^3$:
\begin{align*}
	E(u)& = \frac 1 2 a(u,u)-f(u) \text{ mit } \\
	 a(u,u) &= \int_\Omega {\mcal C} \bs\eps (\bs u): \bs\eps(\bs u) \, d\Omega , \, f(u) = \int_\Omega \bs b \cdot \bs u \, d\Omega + \int_{\Gamma_N} \bs t \cdot \bs u \, d\Gamma
\end{align*}
unter der Nebenbedingung $\bs n \cdot \bs u - g \le 0$ auf $\Gamma_C$  (siehe Vug Skript), bzw. $(\bs u\circ \chi - \bs u)\cdot \bs n_c + g \ge 0$ auf $\Gamma_C$ (etwas allgemeiner, vgl. Wriggers Paper).
\item Herleitung auch über starke Formulierung möglich, vgl. Stephan – Kontaktprobleme.
\item Herleitung der Variationsformulierung: Finde $\bs u \in K$: $a(\bs u,\bs v-\bs u) \ge f(\bs v-\bs u) \, \forall \, \bs v \in K$ (s. auch Wriggers Paper) analog zum Hindernisproblem (nicht mehr ausführlich, wenn oben schon ausführlich).
\item \cite{KikOden} Seite 113 für Bedingung für die Eindeutigkeit und Existenz der Lösung des Problems (hierfür wird Korn's Ungleichung benötigt $\Ra$ vielleicht Anhang?).
\end{itemize}

\subsection{Lösung des Kontaktproblems mittels FEM}

\begin{itemize}
\item Beschreibe das diskrete Problem, was man bekommt mit: Finde $\bs x^* \in \R^N$ mit $B\bs x^* \ge c$, so dass
\begin{align*}
	(A\bs x^* - \bs b)^T (\bs x - \bs x^*) \ge 0 \, \forall \bs x\in \R^N \text{ mit } B\bs x \ge \bs c \, ,
\end{align*}
wobei
\begin{align*}
A &= \left[\int_\Omega \mcal C \bs \eps (\bs \Psi_j):\bs \eps(\bs \Psi_i) \, d\Omega\right]_{1 \le i,j\le N} , \,  \bs b = \left[ \int_\Omega \bs b \cdot \bs\Psi_i \, d\Omega + \int_{\Gamma_N} \bs t \cdot \bs\Psi_i \, ds\right]_{1\le i \le N} \\
B & = [(\bs \Psi_j(\chi(\bs x_i))-\bs \Psi_j(\bs x_i))\cdot \bs n_c(\bs x_i)]_{\bs x_i \in \Gamma_c, 1\le j \le N} , \, c = [-g(\bs x_i)]_{\bs x_i \in \Gamma_c}
\end{align*}
Dieses Problem ist (wie vorher schon gezeigt) äquivalent zu einem quadratischen Problem
\begin{align*}
\min_{\bs x\in \R^N} \frac 1 2 \bs x^T A \bs x - \bs b^T \bs x \text{ s.t. } B\bs x \ge \bs c \, ,
\end{align*}
d.h. Lösbarkeit des quadratischen Programms sollte auch gezeigt sein (vgl. Vug Skript oder auch nichtlineare Optimierung).
\end{itemize}


\newpage

%%% Local Variables: 
%%% mode: latex
%%% TeX-master: "Skript"
%%% End: 
