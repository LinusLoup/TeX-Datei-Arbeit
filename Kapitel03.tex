\newchapter{Variationsungleichungen}

\section{Ein Hindernisproblem}

\begin{itemize}
\item Hindernisproblem: Auslenkung $u$ einer Membran $\Omega$ unter Krafteinwirkung $f$, wobei die Membran durch ein Hindernis $\psi$ behindert wird. Mathematische modelliert bedeutet dies:
\begin{align}\label{eq:Hindernis}
\min_{v\in K} J(v) = \frac 1 2 a(v,v)-(f,v)
\end{align}
mit $K := \{v \in H^1_0(\Omega) \with v\ge \psi$ fast überall in $\Omega\}$. 

\item $J$ gibt wieder die in der Membran gespeicherte Energie an.
\item wobei jetzt die Lösung nicht auf ganz $H^1_0(\Omega)$ gesucht ist, sondern in einer Teilmenge $K \subset H^1_0(\Omega)$.

\item wir können auch hier eine Variationsformulierung, die äquivalent zu \eqref{eq:Hindernis} ist, herleiten

\item zu Beginn noch eine Skizze von einem Hindernisproblem

%\item hierfür zunächst wieder etwas allgemeineres Energiefunktional $J: H \ra \R, J(v) = \frac 1 2 a(v,v)-F(v)$, wobei $H$ Hilbertraum
\end{itemize}


\subsection{Variationsformulierung für das Hindernisproblem}

\begin{itemize}
\item zeigen zunächst, dass $K$ konvex und abgeschlossen ist.
\begin{lemma}\label{lem:3.1}
Die Menge $K =  \{v \in H^1_0(\Omega) \with v\ge \psi\text{ fast überall in }\Omega\}$ ist eine konvexe abgeschlossene Teilmenge von $H^1_0(\Omega)$.
\end{lemma}

\begin{proof}
(i) Es seien $u,v \in K$, d.h. $u \ge \psi$ und $v \ge \psi$ fast überall in $\Omega$. Dann gilt für $t \in [0,1]$
\begin{align*}
	(1-t) u + tv \ge (1-t)\psi + t \psi = \psi  \, ,
\end{align*}
somit ist $(1-t)u + tv \in K$, also $K$ konvex.

(ii) Es sei $(v_n)_{n\in \N}\subset K$ eine konvergente Folge mit $v_n \ra v$ für $n\ra \infty$. Da $H^1_0(\Omega)$ ein abgeschlossener Unterraum von $H^1(\Omega)$ (vgl. auch \cite{Walker} Bemerkung 6.7) ist, folgt direkt $v \in H^1_0(\Omega)$. Da weiter $v_n \ge \psi$ für alle $n \in \N$ gilt, muss auch $v \ge \psi$ gelten und damit folgt $v \in K$, d.h. $K$ ist abgeschlossen.
\end{proof}

\item \begin{satz}\label{satz:3.2}
Es sei $K =   \{v \in H^1_0(\Omega) \with v\ge \psi\text{ fast überall in }\Omega\}$. Das Minimierungsproblem
\begin{align}\label{eq:3.2}
	\min_{v\in K} J(v) = \frac 1 2 a(v,v)-(f,v)
\end{align}
ist äquivalent zur Variationsungleichung: Finde $u \in K$, so dass
\begin{align}\label{eq:3.3}
	a(u,v-u) \ge (f,v-u) \quad \forall \, v \in K \, .
\end{align}
\end{satz}

\begin{proof}
Aus Lemma \ref{lem:2.3} folgt, dass $J$ konvex ist und damit gilt mit Satz \ref{satz:A.10}, dass $u \in K$ genau dann eine Lösung von \eqref{eq:3.2} ist, wenn
\begin{align}\label{eq:3.4}
	\mscr D_{v-u} J(u) \ge 0 \quad \forall \, v \in K 
\end{align}
gilt. Analog zu der berechneten Gâteaux-Ableitung von $J$ in Lemma \ref{lem:2.4}, gilt
\[
	\mscr D_{v-u} J(u) = \frac d{dt} J(u+t(v-u))\Big|_{t=0} = a(u,v-u) -(f,v-u)
\]
und damit folgt mit \eqref{eq:3.4} die Behauptung.
\end{proof}

\item \begin{bem}
Wie man mit Satz \ref{satz:A.10} sehen kann, gilt analog zu Satz \ref{satz:3.2} auch allgemeiner: Es sei $K\subset H$ eine konvexe Teilmenge eines Hilbertraumes $H$. Dann ist
\begin{align*}
	\min_{v\in K} J(v) = \frac 1 2 a(v,v)-F(v)
\end{align*}
äquivalent zur Variationsungleichung: Finde $u \in K$, so dass
\begin{align*}
	a(u,v-u) \ge F(v-u) \quad \forall \, v \in K \, ,
\end{align*}
wobei $F :H\ra \R$ eine lineare stetige Abbildung ist.
\end{bem}

\item auch für das Hindernisproblem gibt es analog zum homogenen Dirichlet-Problem \eqref{eq:2.2} eine äquivalente starke Formulierung

\item \begin{satz}[Starke Formulierung des Hindernisproblems]\label{satz:3.4} Jede Lösung $u \in H^2(\Omega) \cap H^1_0(\Omega)$ des Problems
\begin{align}\label{eq:3.5}
\begin{aligned}
	-\Delta u -f&\ge 0 \\
	u-\psi &\ge 0 \\
	(u-\psi) (-\Delta u &- f) = 0
\end{aligned}
\end{align}
mit $\psi \in H^1(\Omega)$ erfüllt die Variationsungleichung \eqref{eq:3.3}. Umgekehrt ist jede Lösung $u\in H^2(\Omega) \cap K$ von \eqref{eq:3.3} auch eine Lösung von \eqref{eq:3.5}.
\end{satz}

\begin{proof}
"`$\Ra$"' Sei $u \in H^2(\Omega) \cap H^1_0(\Omega)$ eine Lösung von \eqref{eq:3.5}, dann gilt für ein beliebiges $v \in K$
\begin{align*}
	\int_\Omega (-\Delta u - f) (v-u) \, dx & = \underbrace{- \int_\Omega \Delta u  (v-u) \, dx}_{\parbox{3.5cm}{\scriptsize$\stackrel{\text{Green}}= \int_\Omega \nabla u \nabla (v-u) \, dx$ \\ \text{ }\text{ } \text{ }\text{ } $ -\int_\Gamma \underbrace{(v-u)}_{=0} \partial_\nu u \, ds$}} - \int_\Omega  f (v-u) \, dx  \\
	& = \int_\Omega \nabla u \nabla(v-u) \, dx - \int_\Omega f(v-u) \\
	& = a(u,v-u) - (f,v-u) \, .
\end{align*}
Mit $\Omega_0 := \{ x \in \Omega \with u = \psi\}$ folgt, dass $-\Delta u = f$ auf $\Omega_1 := \Omega \setminus \bar\Omega_0$ gelten muss.
\begin{align*}
	 \lra\,   \int_{\Omega = \Omega_0 \cup \Omega_1} \underbrace{(-\Delta u - f)}_{=0 \text{ auf } \Omega_1} (v-u) \, dx  = \int_{\Omega_0}\underbrace{ (-\Delta u - f)}_{\ge0} \underbrace{(v-\psi)}_{\ge 0} \, dx \ge 0
\end{align*}
Damit ist $u$ eine Lösung von \eqref{eq:3.3}
\[
	a(u,v-u) \ge (f,v-u) \quad \forall \, v \in K\, .
\]
"`$\La$"' Es sei $u \in H^2(\Omega) \cap K$ Lösung von \eqref{eq:3.3}. Weiter sei $v \in K$  beliebig, dann gilt
\begin{align}\label{eq:3.6}
\begin{aligned}
	0 & \le a(u,v-u) - (f,v-u) \\
	&= \int_\Omega \nabla u \nabla(v-u) \, dx - \int_\Omega f(v-u) \, dx \\
	& \stackrel{\scriptsize\text{Green}}= \int_\Omega -\Delta u (v-u) \, dx - \int_\Omega f(v-u) \, dx \\
	& = \int_\Omega (-\Delta u - f) (v-u) \, dx \, .
\end{aligned}
\end{align}
Wir nehmen an, dass $-\Delta u -f < 0$ in einem Ball $B_{r_0} := B_{r_0} (x_0)\subset \Omega$ mit Radius $r_0$ um $x_0 \in \Omega$ gilt. Sei weiter $\chi \in C^\infty(\Omega)$ mit $\chi = 0$ auf $\Omega \setminus \bar B_{r_0}, \rho(r) := \(1-\frac r{r_0}\)^2 \chi >0$ und $v := u + \rho (r) \in K$, da $u\in K$ und $\rho (r) >0$. Dann gilt
\[
	\int_\Omega (-\Delta u - f) (v-u) \, dx = \int_{B_{r_0}} \underbrace{(-\Delta u - f)}_{< 0} \underbrace{\rho(r)}_{>0} \, dx < 0 \, ,
\]
was im Widerspruch zu \eqref{eq:3.6} steht. Also muss $-\Delta u - f \ge 0$ gelten.

Nun nehmen wir an, dass $-\Delta u -f > 0$ und $u > \psi$ fast überall in einem Ball $B_{r_0}$ gilt. Wir betrachten $v:= u  + \eps \rho(r) (\psi - u) \in K$ mit $0< \eps\le 1$, dann folgt
\[
	\int_\Omega (-\Delta u - f) (v-u) \, dx = \eps \int_{B_{r_0}} \underbrace{(-\Delta u -f)}_{>0} \underbrace{\rho(r)}_{>0} \underbrace{(\psi-u)}_{<0} \, dx < 0 \, ,
\]
was wiederum im Widerspruch zu \eqref{eq:3.6} steht. Damit muss $u = \psi$ gelten, wenn $-\Delta u = f$ ist. Es folgt, dass $u \in H^2(\Omega) \cap K$ eine Lösung von \eqref{eq:3.5} ist.
\end{proof}
\end{itemize}

\subsection{Existenz und Eindeutigkeit der Lösung}

\begin{itemize}
\item Kapitel 3 in \cite{KikOden} mit Theorem 3.1-3.4 (\textbf{Beweis vgl. NPDE I von Stephan Seite 39}, auch in Solution of Variational Inequalities in Mechanics (Theorem 1.1 Seite 4))

\item für die Existenz und Eindeutigkeit der Lösung des Problems betrachten wir zunächst wieder das allgemeine reelle quadratische Funktional $J: H \ra \R, J(v) = \frac 1 2 a(v,v) - F(v)$.

\item \begin{vor}
Sei $H$ ein reeller Hilbertraum mit Skalarprodukt $(\cdot,\cdot)_H$ und der damit induzierten Norm $\norm\cdot_H$. Mit $H'$ bezeichnen wir den Dualraum zu $H$. Weiter sei vorausgesetzt:
\begin{enumerate}[(a)]
\item $a: H\times H \ra \R$ ist eine stetige koerzive Bilinearform,
\item $F:H\ra\R$ ist ein stetiges lineares Funktional,
\item $K\not = \emptyset$ ist eine abgeschlossene konvexe Teilmenge von $H$.
\end{enumerate}
\end{vor}

\item \begin{theorem}[Existenz und Eindeutigkeit]\label{theorem:3.5}
Unter den obigen Voraussetzungen hat die Variationsungleichung, finde $u\in K$, so dass
\begin{align}\label{eq:3.7}
	a(u,v-u) \ge F(v-u) \quad \forall \, v \in K
\end{align}
ist, genau eine Lösung.
\end{theorem}

\begin{proof}
(i) Eindeutigkeit: Es seien $u_1,u_2 \in K$ zwei Lösungen der Variationsungleichung \eqref{eq:3.7}, d.h.
\begin{align}\label{eq:3.8}
	a(u_1,v-u_1) \ge F(v-u_1) \quad \forall \, v \in K\, , \\
	a(u_2,v-u_2) \ge F(v-u_2) \quad \forall \, v \in K\, . \label{eq:3.9}
\end{align}
Addieren wir \eqref{eq:3.8} und \eqref{eq:3.9} miteinander und setzen zuvor $v = u_2$ in \eqref{eq:3.8} und $v = u_1$ in \eqref{eq:3.9}, so erhalten wir
\begin{align*}
	0 & \le a(u_1,u_2-u_1) - F(u_2-u_1) + a(u_2,u_1-u_2) \underbrace{- F(u_1-u_2)}_{=F(u_2-u_1)}  \\
	& = a(u_1,u_2-u_1)-a(u_2,u_2-u_1) = -a(u_2-u_1,u_2-u_1) \\
	& \le -\alpha \norm{u_2-u_1}_H^2 \, .
\end{align*}
Also gilt $\norm{u_2-u_1}_H^2 \le 0 \Ra \norm{u_2-u_1}_H^2 = 0$ und damit folgt $u_1 = u_2$.

(ii) Existenz: Aus dem Darstellungssatz von Riesz bzw. das Korollar \ref{kor:2.14} folgt, dass ein $A \in \mcal L(H,H), l \in H$ existiert, so dass
\begin{align*}
	a(u,v) &= (Au,v)_H \quad \forall \, u,v \in H\, , \\
	F(v) &= (l,v)_H \qquad \forall \, v \in H \, .
\end{align*}
Damit gilt
\begin{align*}
	  F(v-u) - a(u,v-u) &= (l,v-u)_H - (Au,v-u)_H \\
	 &=   (l-Au,v-u)_H \le 0 \, .
\end{align*}
Durch Multiplikation mit $\varrho > 0$ und Addition der Null erhalten wir das äquivalente Problem: Finde $u \in K$, so dass
\begin{align}
	(u-\varrho(Au-l)-u,v-u)_H \le 0 \quad \forall \, v \in K \, .
\end{align}
Nach Satz \ref{satz:2.3} ist $u$ damit das Bild der Projektion von $u-\varrho (Au-l)$ auf $K$, d.h.
\[
	u = P_K (u-\varrho (Au-l)) \,.
\]
Es bleibt zu zeigen, dass $W_\varrho : H \ra K, W_\varrho (v) \coloneqq P_K(v-\varrho (Av-l))$ einen Fixpunkt besitzt. Mit Anwendung von Satz \ref{satz:2.4} und der Koerzivität von $a$ rechnen wir nach, dass
\begin{align*}
	\norm{W_\varrho (v_1) - W_\varrho (v_2)}_H^2  & = \norm{ P_K (v_1-\varrho (Av_1-l))- P_K (v_2-\varrho (Av_2-l))  }_H^2 \\
	& \le \norm{v_1-\varrho (Av_1-l)- (v_2-\varrho (Av_2-l))  }_H^2  \\
	& = \norm{(v_1-v_2)-\varrho \, A(v_1-v_2)  }_H^2  \\
	& = \norm{v_1-v_2}_H^2 + \varrho^2 \norm{A(v_1-v_2)}^2_H \\
	& \ \ \, - \underbrace{ \varrho\, (A(v_1-v_2),v_1-v_2)_H - \varrho\, (v_1-v_2,A(v_1-v_2))_H}_{=2\varrho\, (A(v_1-v_2),v_1-v_2)_H  =2\varrho\, a(v_1-v_2,v_1-v_2)} \\
	& \le \norm{v_1-v_2}_H^2 + \varrho^2\, \norm A^2 \norm{v_1-v_2}_H^2 - 2\varrho\alpha \, \norm{v_1-v_2}_H^2 \\
	& = (1-2\varrho \alpha+\varrho^2 \, \norm A^2) \,\norm{v_1-v_2}_H^2
\end{align*}
mit $\norm A := \sup_{v \in H} \frac{\norm{Av}_H}{\norm v_H}$. Also ist die Abbildung $W_\varrho$  eine Kontraktion, wenn gilt
\begin{align*}
	1-2\varrho \alpha+\varrho^2 \, \norm A^2 < 1 \, \lra\, 0 < \varrho < \frac {2\alpha}{\norm A^2} \, .
\end{align*}
Nach dem Banach'scher Fixpunktsatz (vgl. \cite{Stoer} Satz 5.2.3) existiert für solch ein $\varrho$ ein $u \in H$ mit $u = W_\varrho (u) = P_K(u-\varrho (Au-l))$.

Insgesamt gibt es also für das Problem \eqref{eq:3.7} genau eine  Lösung.
\end{proof}

\item \begin{kor}
Das Problem \eqref{eq:Hindernis} hat eine eindeutige Lösung.
\end{kor}

\begin{proof}
Da laut Lemma \ref{lem:3.1} die Menge 
\[
	K=\{v \in H^1_0(\Omega) \with v \ge \psi \text{ fast überall in }\Omega\}
\]
abgeschlossen und konvex ist, $F(v) = (f,v)$ ein stetiges lineares Funktional und 
\[
	a(u,v) = \int_\Omega \nabla u \nabla v \, dx
\]
stetig bilinear und koerziv, sind die Voraussetzungen für Theorem \ref{theorem:3.5} erfüllt. Damit hat das Problem, finde $u \in K$, so dass
\begin{align}\label{eq:3.11}
	a(u,v-u) \ge (f,v-u) \quad \forall \, v \in K \, ,
\end{align}
genau eine Lösung. Nach Satz \ref{satz:3.2} ist \eqref{eq:Hindernis} äquivalent zu \eqref{eq:3.11} und damit folgt die Behauptung.
\end{proof}

\item \begin{bem}
Insbesondere hat auch das Problem \eqref{eq:3.5} nach Satz \ref{satz:3.4} und Theorem \ref{theorem:3.5} eine eindeutige Lösung, wenn $u \in H^2(\Omega) \cap H^1_0(\Omega)$ ist.
\end{bem}
\end{itemize}







\subsection{Lösung des Hindernisproblems mittels FEM}

\begin{itemize}
\item Analog zum vorherigen Kapitel kann man auch im $\R^n$ Existenz und Eindeutigkeit der Lösung unter bestimmten Voraussetzungen zeigen. (vgl. Vug Skript Kapitel 2) $\Ra$ Beachte hierfür auch den Fixpunktsatz von Brouwer.
\end{itemize}


\section{Kontaktprobleme}

\subsection{Mathematische Modellierung von Kontaktproblemen}

\begin{itemize}
\item Starke Formulierung (s. Wriggers Paper) für Kontaktproblem mit Signorini-Kontakt (ohne Reibung).
\begin{align}
\div \bs \sigma + \bs b &= \bs 0 \text{ in } \Omega\\
\bs \sigma  - \mcal C \bs \eps & = \bs 0 \text{ in } \Omega\\
\bs \sigma \cdot \bs n &= \bs t  \text{ auf } \Gamma_N \\
\bs u &= \bs 0 \text{ auf } \Gamma_D \\
(\bs u \circ \chi - \bs u) \cdot \bs n_c + g& \ge 0 \text{ auf } \Gamma_C
\end{align}
sowie auf $\Gamma_C$ muss $\sigma_n \le 0$ (Normalenkraft $\sigma_n = \bs n\cdot ( \bs \sigma \cdot \bs n)$), $\bs \sigma_t = \bs 0$ (keine Tangentialkraft, da keine Reibung – $\bs \sigma_t = \bs \sigma \cdot \bs n - \sigma_n \bs n$) und $((\bs u \circ \chi - \bs u) \cdot \bs n_c + g)\sigma_n = 0$, d.h. wenn kein Kontakt ist, ist die Normalkraft in den Punkten Null, also herrscht Kräftegleichgewicht.
\item Anreißen von Kontaktproblem mit Tresca-Reibung (vgl. Numerik für Kontaktmechanik von Stephan und Vug von Starke) $\Ra$ Herleitung der Variationsungleichung durch Ableitung nicht mehr möglich, da Reibungspotential nicht mehr differenzierbar.
\end{itemize}

\subsection{Variationsformulierung für Kontaktprobleme}

\begin{itemize}
\item Minimierung von Energiefunktional (vgl. \cite{KikOden} Seite 112 unten) mit $\boldsymbol{u}: \Omega\ra \R^3$:
\begin{align*}
	E(u)& = \frac 1 2 a(u,u)-f(u) \text{ mit } \\
	 a(u,u) &= \int_\Omega {\mcal C} \bs\eps (\bs u): \bs\eps(\bs u) \, d\Omega , \, f(u) = \int_\Omega \bs b \cdot \bs u \, d\Omega + \int_{\Gamma_N} \bs t \cdot \bs u \, d\Gamma
\end{align*}
unter der Nebenbedingung $\bs n \cdot \bs u - g \le 0$ auf $\Gamma_C$  (siehe Vug Skript), bzw. $(\bs u\circ \chi - \bs u)\cdot \bs n_c + g \ge 0$ auf $\Gamma_C$ (etwas allgemeiner, vgl. Wriggers Paper).
\item Herleitung auch über starke Formulierung möglich, vgl. Stephan – Kontaktprobleme.
\item Herleitung der Variationsformulierung: Finde $\bs u \in K$: $a(\bs u,\bs v-\bs u) \ge f(\bs v-\bs u) \, \forall \, \bs v \in K$ (s. auch Wriggers Paper) analog zum Hindernisproblem (nicht mehr ausführlich, wenn oben schon ausführlich).
\item \cite{KikOden} Seite 113 für Bedingung für die Eindeutigkeit und Existenz der Lösung des Problems (hierfür wird Korn's Ungleichung benötigt $\Ra$ vielleicht Anhang?).
\end{itemize}

\subsection{Lösung des Kontaktproblems mittels FEM}

\begin{itemize}
\item Beschreibe das diskrete Problem, was man bekommt mit: Finde $\bs x^* \in \R^N$ mit $B\bs x^* \ge c$, so dass
\begin{align*}
	(A\bs x^* - \bs b)^T (\bs x - \bs x^*) \ge 0 \, \forall \bs x\in \R^N \text{ mit } B\bs x \ge \bs c \, ,
\end{align*}
wobei
\begin{align*}
A &= \left[\int_\Omega \mcal C \bs \eps (\bs \Psi_j):\bs \eps(\bs \Psi_i) \, d\Omega\right]_{1 \le i,j\le N} , \,  \bs b = \left[ \int_\Omega \bs b \cdot \bs\Psi_i \, d\Omega + \int_{\Gamma_N} \bs t \cdot \bs\Psi_i \, ds\right]_{1\le i \le N} \\
B & = [(\bs \Psi_j(\chi(\bs x_i))-\bs \Psi_j(\bs x_i))\cdot \bs n_c(\bs x_i)]_{\bs x_i \in \Gamma_c, 1\le j \le N} , \, c = [-g(\bs x_i)]_{\bs x_i \in \Gamma_c}
\end{align*}
Dieses Problem ist (wie vorher schon gezeigt) äquivalent zu einem quadratischen Problem
\begin{align*}
\min_{\bs x\in \R^N} \frac 1 2 \bs x^T A \bs x - \bs b^T \bs x \text{ s.t. } B\bs x \ge \bs c \, ,
\end{align*}
d.h. Lösbarkeit des quadratischen Programms sollte auch gezeigt sein (vgl. Vug Skript oder auch nichtlineare Optimierung).
\end{itemize}


\newpage

%%% Local Variables: 
%%% mode: latex
%%% TeX-master: "Skript"
%%% End: 
