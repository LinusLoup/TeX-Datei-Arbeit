\newchapter{Variationsungleichungen}
\label{kap:3}

Hindernis- und Kontaktprobleme basieren auf der Minimierung von Energiefunktionalen, wie im Theorem von Lax-Milgram\index{Lax-Milgram} (Theorem \ref{theorem:2.12}) auch schon verwendet wurde. Allerdings minimieren wir diese jetzt nur auf einer Teilmenge der Funktionen aus dem betrachteten Hilbertraum, d.h. wir werden eine Nebenbedingung einführen. Wir wollen in diesem Kapitel die theoretische Grundlage für die Lösung solcher Probleme bilden und eine a priori Fehlerschätzung zwischen exakter und mit der Finiten-Elemente-Methode approximierter Lösung durchführen.

Dieses Kapitel basiert auf \cite{KikOden}, \cite{StarkeVar}, \cite{EPS}, \cite{EPSContact}, \cite{WriggersFEM}, \cite{WriggersContact}, \cite{HlaHas}, \cite{Glow}, \cite{Falk}.

\section{Ein Hindernisproblem}
\label{kap:3.1}


Als Modellproblem wollen wir wieder die Auslenkung $u:\Omega \ra \R$ einer in $\Omega$ eingespannten Membran betrachten, die mit der Flächenlast $f$ belastet wird (vgl. auch Abbildung \ref{abb:2.1}). Nun wollen wir jedoch die Auslenkung $u$ durch ein Hindernis $\psi$ in $\Omega$ behindern (s. Abbildung \ref{abb:3.1} -- die Auslenkung $u$ der Membran liegt bezogen auf die angedeutete Kurve in der Skizze rotationssymmetrisch zur $z$-Achse). Dies führt auf die Minimierung des quadratischen Energiefunktionals
\begin{align}\label{eq:3.1}
\min_{v\in K} J(v) = \frac 1 2 a(v,v)-(f,v)
\end{align}
mit $K := \{v \in H^1_0(\Omega) \mid v\ge \psi$ fast überall in $\Omega\}$, wobei $a(\cdot,\cdot), (f,\cdot)$ wie in Kapitel \ref{kap:2.2} definiert sind. Das Funktional $J$ gibt hierbei anschaulich wieder die in der Membran gespeicherte Energie an.


Der Unterschied zur Minimierung von $J$ aus dem Theorem von Lax-Milgram\index{Lax-Milgram} liegt nun also darin, dass wir $J$ nicht über ganz $H^1_0(\Omega)$ minimieren, sondern nur über eine Teilmenge $K\subset H^1_0(\Omega)$ (anschaulich alle Auslenkungen $u$, die oberhalb des Hindernisses $\psi$ liegen).



\begin{figure}[h]
\begin{center}
	\begin{pspicture}(-2,-1)(2,2)
		% Farbendefinition:
		\newgray{mygray}{0.75}
		
		% Hindernis:
		\psellipticarc[fillstyle=shape,fillcolor=mygray](0,0)(1.3,1.5){0}{180}
		\psellipticarc[fillstyle=shape,fillcolor=mygray](0,0)(1.3,0.35){180}{360}
		\psellipticarc[linestyle=dotted](0,0)(1.3,0.35){0}{180}
		
		% Omega:
		\psellipticarc(0,0)(3,0.8){147}{393}
		\psellipticarc[linestyle=dotted](0,0)(3,0.8){33}{147}
		
		% Lösung:
		\pscurve(3,0)(1.7,0.3)(0.95,1)
		\pscurve(-3,0)(-1.7,0.3)(-0.95,1)
%		\pscurve(0,1.5)(0.5,1.1)(1,0)(1.66,-0.66)
%		\pscurve(0,1.5)(-0.5,1.1)(-1,0)(-1.66,-0.66)
%		\pscurve(0,1.5)(0.5,1.1)(1,0)(1.66,-0.66)
		
		% Kraft f:
		\pscurve[linewidth=0.3pt](3,0.8)(1.7,1.1)(0.95,1.8)
		\pscurve[linewidth=0.3pt](-3,0.8)(-1.7,1.1)(-0.95,1.8)
		\psellipticarc[linewidth=0.3pt](0,0.8)(1.3,1.5){45}{135}
%		\psline{->}(1.6,0.2)(1.6,-0.6)
%		\psline{->}(-1.6,0.2)(-1.6,-0.6)

		\psline{->}(3,0.8)(3,0)
		\psline{->}(-3,0.8)(-3,0)
		\psline{->}(0,2.3)(0,1.5)
		\psline{->}(1,1.75)(1,0.95)
		\psline{->}(-1,1.75)(-1,0.95)
		\psline{->}(2,0.98)(2,0.18)
		\psline{->}(-2,0.98)(-2,0.18)
		
		% Beschriftungen:
		\rput(2,0){$u$}
%		\rput(-3.2,0.4){$f$}
		\rput(-0.2,1.85){$f$}
		\rput(-0.95,0.45){$\psi$}
		\rput(-1.95,-0.35){$\Omega$}
	\end{pspicture}
\end{center}
\caption{Ein Hindernisproblem mit Hindernis $\psi$, konstanter Streckenlast $f$ und Lösung $u$\label{abb:3.1}}
\end{figure}


Wir können nun analog zum Kapitel \ref{kap:2.2} auch für diese Art von Problemen eine Variationsformulierung herleiten.




\subsection{Variationsformulierung für das Hindernisproblem}
\label{kap:3.1.1}


Um für das Hindernisproblem \eqref{eq:3.1} eine äquivalente Variationsformulierung zu erhalten, wollen wir die in Anhang \ref{anhang:A.2} aufgeführten Optimalitätskriterien verwenden. Hierfür müssen wir zunächst zeigen, dass die oben aufgeführte Menge $K$ konvex und abgeschlossen in $H^1_0(\Omega)$ ist.


\begin{lemma}\label{lem:3.1}
Die Menge $K =  \{v \in H^1_0(\Omega) \mid v\ge \psi\text{ fast überall in }\Omega\}$ ist eine konvexe abgeschlossene Teilmenge von $H^1_0(\Omega)$.
\end{lemma}

\begin{proof}
(i) Es seien $u,v \in K$, d.h. $u \ge \psi$ und $v \ge \psi$ fast überall in $\Omega$. Dann gilt für $t \in [0,1]$
\begin{align*}
	(1-t) u + tv \ge (1-t)\psi + t \psi = \psi  \, ,
\end{align*}
somit ist $(1-t)u + tv \in K$, also $K$ konvex.

(ii) Es sei $(v_n)_{n\in \N}\subset K$ eine konvergente Folge mit $v_n \ra v$ für $n\ra \infty$. Da $H^1_0(\Omega)$ laut Bemerkung \ref{bem:A.8} ein abgeschlossener Unterraum von $H^1(\Omega)$ ist, folgt direkt $v \in H^1_0(\Omega)$. Da weiter $v_n \ge \psi$ für alle $n \in \N$ gilt, folgt aus dem Spursatz (vgl. \cite{BraeFEM} Kapitel II, \S3, Satz 3.1), dass auch $v \ge \psi$ fast überall in $\Omega$ gilt und damit ist $v \in K$, d.h. $K$ ist abgeschlossen.
\end{proof}


\begin{satz}\label{satz:3.2}
Es sei $K =   \{v \in H^1_0(\Omega) \mid v\ge \psi\text{ fast überall in }\Omega\}$. Das Minimierungsproblem
\begin{align}\label{eq:3.2}
	\min_{v\in K} J(v) = \frac 1 2 a(v,v)-(f,v)
\end{align}
ist äquivalent zur Variationsungleichung: Finde $u \in K$, so dass
\begin{align}\label{eq:3.3}
	a(u,v-u) \ge (f,v-u) \quad \forall \, v \in K \, .
\end{align}
\end{satz}

\begin{proof}
Aus Lemma \ref{lem:2.10} folgt, dass $J$ konvex ist und damit gilt mit Satz \ref{satz:A.10}, dass $u \in K$ genau dann eine Lösung von \eqref{eq:3.2} ist, wenn
\begin{align}\label{eq:3.4}
	\mscr D_{v-u} J(u) \ge 0 \quad \forall \, v \in K 
\end{align}
gilt. Analog zu der berechneten Gâteaux-Ableitung von $J$ in Lemma \ref{lem:2.11}, gilt
\[
	\mscr D_{v-u} J(u) = \frac d{dt} J(u+t(v-u))\Big|_{t=0} = a(u,v-u) -(f,v-u)
\]
und damit folgt mit \eqref{eq:3.4} die Behauptung.
\end{proof}


\begin{bem}\label{bem:3.3}
Wie man mit Satz \ref{satz:A.10} nachrechnen kann, gilt analog zu Satz \ref{satz:3.2} auch allgemeiner: Es sei $K\subset H$ eine konvexe Teilmenge eines Hilbertraumes $H$. Dann ist
\begin{align*}
	\min_{v\in K} J(v) = \frac 1 2 a(v,v)-F(v)
\end{align*}
äquivalent zur Variationsungleichung: Finde $u \in K$, so dass
\begin{align*}
	a(u,v-u) \ge F(v-u) \quad \forall \, v \in K \, ,
\end{align*}
wobei $F :H\ra \R$ eine lineare stetige Abbildung ist.
\end{bem}


Wie wir sehen, erhalten wir durch die Einführung einer Nebenbedingung keine Variationsgleichung mehr wie in Kapitel \ref{kap:2.2}, sondern eine Variationsungleichung. Anschaulich liegt dies daran, dass die Veränderung der auf die Membran wirkenden Flächenlast $f$ nicht zwangsläufig eine Änderung in der Auslenkung $u$ hervorrufen muss, da diese durch $\psi$ behindert werden könnte. 

Diese Ungleichunsrelation überträgt sich auch auf die äquivalente starke Formulierung des Problems \eqref{eq:3.1}, die wir auch für das Hindernisproblem, analog zum homogenen Dirichlet-Problem\index{Dirichlet-Problem!homogenes}, finden können.


\begin{satz}[Starke Formulierung des Hindernisproblems]\label{satz:3.4} Jede Lösung $u \in H^2(\Omega) \cap H^1_0(\Omega)$ des Problems
\begin{align}\label{eq:3.5}
\begin{aligned}
	-\Delta u -f&\ge 0\, , \\
	u-\psi &\ge 0\, ,  \\
	(u-\psi) (-\Delta u &- f) = 0
\end{aligned}
\end{align}
mit $\psi \in H^1(\Omega)$ erfüllt die Variationsungleichung \eqref{eq:3.3}. Umgekehrt ist jede Lösung $u\in H^2(\Omega) \cap K$ von \eqref{eq:3.3} auch eine Lösung von \eqref{eq:3.5}.
\end{satz}

\begin{proof}
"`$\Ra$"' Sei $u \in H^2(\Omega) \cap H^1_0(\Omega)$ eine Lösung von \eqref{eq:3.5}, dann gilt für ein beliebiges $v \in K$
\begin{align*}
	\int_\Omega (-\Delta u - f) (v-u) \, dx & = \underbrace{- \int_\Omega \Delta u  (v-u) \, dx}_{\parbox{3.5cm}{\scriptsize$\stackrel{\text{Green}}= \int_\Omega \nabla u \nabla (v-u) \, dx$ \\ \text{ }\text{ } \text{ }\text{ } $ -\int_\Gamma \underbrace{(v-u)}_{=0} \partial_\nu u \, ds$}} - \int_\Omega  f (v-u) \, dx  \\
	& = \int_\Omega \nabla u \nabla(v-u) \, dx - \int_\Omega f(v-u) \\
	& = a(u,v-u) - (f,v-u) \, .
\end{align*}
Mit $\Omega_0 := \{ x \in \Omega \mid u = \psi\}$ folgt, dass $-\Delta u = f$ auf $\Omega_1 := \Omega \setminus \bar\Omega_0$ gelten muss.
\begin{align*}
	 \lra\,   \int_{\Omega = \Omega_0 \cup \Omega_1} \underbrace{(-\Delta u - f)}_{=0 \text{ auf } \Omega_1} (v-u) \, dx  = \int_{\Omega_0}\underbrace{ (-\Delta u - f)}_{\ge0} \underbrace{(v-\psi)}_{\ge 0} \, dx \ge 0
\end{align*}
Damit ist $u$ eine Lösung von \eqref{eq:3.3}
\[
	a(u,v-u) \ge (f,v-u) \quad \forall \, v \in K\, .
\]
"`$\La$"' Es sei $u \in H^2(\Omega) \cap K$ Lösung von \eqref{eq:3.3}. Weiter sei $v \in K$  beliebig, dann gilt
\begin{align}\label{eq:3.6}
\begin{aligned}
	0 & \le a(u,v-u) - (f,v-u) \\
	&= \int_\Omega \nabla u \nabla(v-u) \, dx - \int_\Omega f(v-u) \, dx \\
	&\!\!\!\!\stackrel{\scriptsize\text{Green}}= \int_\Omega -\Delta u (v-u) \, dx - \int_\Omega f(v-u) \, dx \\
	& = \int_\Omega (-\Delta u - f) (v-u) \, dx \, .
\end{aligned}
\end{align}
Wir nehmen an, dass $-\Delta u -f < 0$ in einem Ball $B_{r_0} := B_{r_0} (x_0)\subset \Omega$ mit Radius $r_0$ um $x_0 \in \Omega$ gilt. Sei weiter $\chi \in C^\infty(\Omega)$ mit $\chi = 0$ auf $\Omega \setminus \bar B_{r_0}, \rho(r) := \(1-\frac r{r_0}\)^2 \chi >0$ und $v := u + \rho (r) \in K$, da $u\in K$ und $\rho (r) >0$. Dann gilt
\[
	\int_\Omega (-\Delta u - f) (v-u) \, dx = \int_{B_{r_0}} \underbrace{(-\Delta u - f)}_{< 0} \underbrace{\rho(r)}_{>0} \, dx < 0 \, ,
\]
was im Widerspruch zu \eqref{eq:3.6} steht. Also muss $-\Delta u - f \ge 0$ gelten.

Nun nehmen wir an, dass $-\Delta u -f > 0$ und $u > \psi$ fast überall in einem Ball $B_{r_0}$ gilt. Wir betrachten $v:= u  + \eps \rho(r) (\psi - u) \in K$ mit $0< \eps\le 1$, dann folgt
\[
	\int_\Omega (-\Delta u - f) (v-u) \, dx = \eps \int_{B_{r_0}} \underbrace{(-\Delta u -f)}_{>0} \underbrace{\rho(r)}_{>0} \underbrace{(\psi-u)}_{<0} \, dx < 0 \, ,
\]
was wiederum im Widerspruch zu \eqref{eq:3.6} steht. Damit muss $u = \psi$ gelten, wenn $-\Delta u = f$ ist. Es folgt, dass $u \in H^2(\Omega) \cap K$ eine Lösung von \eqref{eq:3.5} ist.
\end{proof}




\subsection{Existenz und Eindeutigkeit der Lösung}
\label{kap:3.1.2}


Wir wollen nun die Existenz und Eindeutigkeit der Lösung des Hindernisproblems \eqref{eq:3.2} bzw. der Variationsungleichung \eqref{eq:3.3} überprüfen. Hierzu betrachten wir zunächst wieder das allgemeine reelle quadratische Funktional
\[
	J: H \ra \R \, ,\quad J(v) = \frac 1 2 a(v,v) - F(v) \, .
\]
Folgende Voraussetzungen wollen wir für den Beweis der Existenz und Eindeutigkeit der Lösung stellen.

\begin{vor}\label{vor:3.5}
Sei $H$ ein reeller Hilbertraum mit Skalarprodukt $(\cdot,\cdot)_H$ und der damit induzierten Norm $\norm\cdot_H$. Mit $H'$ bezeichnen wir den Dualraum zu $H$. Weiter sei vorausgesetzt:
\begin{enumerate}[(a)]
\item $a: H\times H \ra \R$ ist eine stetige koerzive Bilinearform,
\item $F:H\ra\R$ ist ein stetiges lineares Funktional,
\item $K\not = \emptyset$ ist eine abgeschlossene konvexe Teilmenge von $H$.
\end{enumerate}
\end{vor}


\begin{theorem}[Existenz und Eindeutigkeit]\label{theorem:3.6}
Unter den obigen Voraussetzungen \ref{vor:3.5} hat die Variationsungleichung, finde $u\in K$, so dass
\begin{align}\label{eq:3.7}
	a(u,v-u) \ge F(v-u) \quad \forall \, v \in K
\end{align}
ist, genau eine Lösung.
\end{theorem}

\begin{proof}
(i) Eindeutigkeit: Es seien $u_1,u_2 \in K$ zwei Lösungen der Variationsungleichung \eqref{eq:3.7}, d.h.
\begin{align}\label{eq:3.8}
	a(u_1,v-u_1) \ge F(v-u_1) \quad \forall \, v \in K\, , \\
	a(u_2,v-u_2) \ge F(v-u_2) \quad \forall \, v \in K\, . \label{eq:3.9}
\end{align}
Addieren wir \eqref{eq:3.8} und \eqref{eq:3.9} miteinander und setzen zuvor $v = u_2$ in \eqref{eq:3.8} und $v = u_1$ in \eqref{eq:3.9}, so erhalten wir
\begin{align*}
	0 & \le a(u_1,u_2-u_1) - F(u_2-u_1) + a(u_2,u_1-u_2) \underbrace{- F(u_1-u_2)}_{=F(u_2-u_1)}  \\
	& = a(u_1,u_2-u_1)-a(u_2,u_2-u_1) = -a(u_2-u_1,u_2-u_1) \\
	& \le -\alpha \norm{u_2-u_1}_H^2 \, .
\end{align*}
Also gilt $\norm{u_2-u_1}_H^2 \le 0 \Ra \norm{u_2-u_1}_H^2 = 0$ und damit folgt $u_1 = u_2$.

(ii) Existenz: Aus dem Darstellungssatz von Riesz bzw. das Lemma \ref{lem:2.16} folgt, dass ein $A \in \mcal L(H,H), l \in H$ existiert, so dass
\begin{align*}
	a(u,v) &= (Au,v)_H \quad \forall \, u,v \in H\, , \\
	F(v) &= (l,v)_H \qquad \forall \, v \in H \, .
\end{align*}
Damit gilt
\begin{align*}
	  F(v-u) - a(u,v-u) &= (l,v-u)_H - (Au,v-u)_H \\
	 &=   (l-Au,v-u)_H \le 0 \, .
\end{align*}
Durch Multiplikation mit $\varrho > 0$ und Addition der Null erhalten wir das äquivalente Problem: Finde $u \in K$, so dass
\begin{align}
	(u-\varrho(Au-l)-u,v-u)_H \le 0 \quad \forall \, v \in K \, .
\end{align}
Nach Satz \ref{satz:2.3} ist $u$ damit das Bild der Projektion von $u-\varrho (Au-l)$ auf $K$, d.h.
\[
	u = P_K (u-\varrho (Au-l)) \,.
\]
Es bleibt zu zeigen, dass $W_\varrho : H \ra K, W_\varrho (v) \coloneqq P_K(v-\varrho (Av-l))$ einen Fixpunkt besitzt. Mit Anwendung von Satz \ref{satz:2.4} und der Koerzivität von $a$ rechnen wir nach, dass
\begin{align*}
	\norm{W_\varrho (v_1) - W_\varrho (v_2)}_H^2  & = \norm{ P_K (v_1-\varrho (Av_1-l))- P_K (v_2-\varrho (Av_2-l))  }_H^2 \\
	& \le \norm{v_1-\varrho (Av_1-l)- (v_2-\varrho (Av_2-l))  }_H^2  \\
	& = \norm{(v_1-v_2)-\varrho \, A(v_1-v_2)  }_H^2  \\
	& = \norm{v_1-v_2}_H^2 + \varrho^2 \norm{A(v_1-v_2)}^2_H \\
	& \ \ \, - \underbrace{ \varrho\, (A(v_1-v_2),v_1-v_2)_H - \varrho\, (v_1-v_2,A(v_1-v_2))_H}_{=2\varrho\, (A(v_1-v_2),v_1-v_2)_H  =2\varrho\, a(v_1-v_2,v_1-v_2)} \\
	& \le \norm{v_1-v_2}_H^2 + \varrho^2\, \norm A^2 \norm{v_1-v_2}_H^2 - 2\varrho\alpha \, \norm{v_1-v_2}_H^2 \\
	& = (1-2\varrho \alpha+\varrho^2 \, \norm A^2) \,\norm{v_1-v_2}_H^2
\end{align*}
mit $\norm A := \sup_{v \in H} \frac{\norm{Av}_H}{\norm v_H}$. Also ist die Abbildung $W_\varrho$  eine Kontraktion, wenn gilt
\begin{align*}
	1-2\varrho \alpha+\varrho^2 \, \norm A^2 < 1 \, \lra\, 0 < \varrho < \frac {2\alpha}{\norm A^2} \, .
\end{align*}
Nach dem Banach'scher Fixpunktsatz (vgl. \cite{Stoer} Satz 5.2.3) existiert für solch ein $\varrho$ ein $u \in H$ mit $u = W_\varrho (u) = P_K(u-\varrho (Au-l))$.

Insgesamt gibt es also für das Problem \eqref{eq:3.7} genau eine  Lösung.
\end{proof}

\newpage

\begin{kor}\label{kor:3.7}
Das Problem \eqref{eq:3.1} hat eine eindeutige Lösung.
\end{kor}

\begin{proof}
Da laut Lemma \ref{lem:3.1} die Menge 
\[
	K=\{v \in H^1_0(\Omega) \mid v \ge \psi \text{ fast überall in }\Omega\}
\]
abgeschlossen und konvex ist, $F(v) = (f,v)$ ein stetiges lineares Funktional und 
\[
	a(u,v) = \int_\Omega \nabla u \nabla v \, dx
\]
stetig bilinear und koerziv, sind die Voraussetzungen für Theorem \ref{theorem:3.6} erfüllt. Damit hat das Problem, finde $u \in K$, so dass
\begin{align}\label{eq:3.11}
	a(u,v-u) \ge (f,v-u) \quad \forall \, v \in K \, ,
\end{align}
genau eine Lösung. Nach Satz \ref{satz:3.2} ist \eqref{eq:3.1} äquivalent zu \eqref{eq:3.11} und damit folgt die Behauptung.
\end{proof}


\begin{bem}\label{bem:3.8}
Insbesondere hat auch das Problem \eqref{eq:3.5} nach Satz \ref{satz:3.4} und Theorem \ref{theorem:3.6} eine eindeutige Lösung, wenn $u \in H^2(\Omega) \cap H^1_0(\Omega)$ ist.
\end{bem}







\subsection{Lösung des Hindernisproblems mittels FEM}
\label{kap:3.1.3}

Analog zu Kapitel \ref{kap:2.3} können wir nun auch die Variationsungleichung \eqref{eq:3.11} mittels FEM lösen. Hierzu betrachten wir \eqref{eq:3.11} bzgl. eines endlich dimensionalen Unterraum
\[
	K_h \coloneqq \{v_h \in V_h \mid v_h (p) \ge \psi(p) \, \forall \, p \in \mcal N \cap \Omega\}\, ,
\]
wobei $\mcal N$ die Knotenmenge bzgl. der Triangulierung $\mcal T_h$ bezeichne und $V_h$ wie oben ein endlich dimensionaler Unterraum eines Hilbertraumes $H$ ist. Damit ist \eqref{eq:3.11} in diskreter Form: Finde $u_h \in K_h$, so dass
\begin{align}\label{eq:3.12}
	a(u_h,v_h-u_h) \ge (f,v_h-u_h) \quad \forall \, v_h \in K_h \, .
\end{align}
Auch für das diskrete Problem existiert eine eindeutige Lösung, wie sich mithilfe des nächsten Sachtes zeigen lässt.

\begin{satz}[\idx{Fixpunktsatz von Brouwer}]\label{satz:3.9}
Es sei $K \not= \emptyset$ eine kompakte konvexe Teilmenge eines endlich dimensionalen normierten Raumes $H$ und $F: K \ra K$ sei stetig. Dann besitzt $F$ einen Fixpunkt $v \in K$.
\end{satz}

\begin{proof}
Der Beweis ist in \cite{Werner} Kapitel 4 Satz 7.15 zu finden.
\end{proof}

\begin{theorem}[Existenz und Eindeutigkeit]\label{theorem:3.10}
Es gelten die Voraussetzungen \ref{vor:3.5}. Das Problem \eqref{eq:3.12} hat eine eindeutige Lösung $u_h \in K_h$.
\end{theorem}

\begin{proof}
Der Beweis ist analog zu Theorem \ref{theorem:3.6} zu führen. Wir ersetzen lediglich $H$ durch $V_h$ und $K$ durch $K_h$ und verwenden im endlich dimensionalen Raum $V_h$ den Fixpunktsatz von Brouwer.
\end{proof}


 \begin{bem*}
In Kapitel 2.2 von \cite{StarkePDE} sind die Argumente bzgl. der Existenz und Eindeutigkeit einer Lösung von \eqref{eq:3.11} für den endlich dimensionalen Fall $K_h$ auch noch einmal im Einzelnen aufgeführt.
\end{bem*}

Es sei $\mcal B_h = \{\phi_1,\ldots,\phi_N\}$ eine nodale Basis von $V_h$, d.h. analog zu \eqref{eq:2.10} können wir $u_h$ und $v_h$ mit Koordinaten $\mu_i,\nu_i, i = 1,\ldots,N$ bzgl. $\mcal B_h$ ausdrücken. Dann schreiben wir \eqref{eq:3.12} als
\begin{align*}
	\sum_{i = 1}^N\sum_{j=1}^N \mu_i \, a(\phi_i,\phi_j)\, (\nu_j-\mu_j) & \ge \sum_{j=1}^N  (f,\phi_j) \, (\nu_j-\mu_j)  \\
	\Llra\quad \qquad  \qquad \qquad \bs \mu^T A (\bs \nu- \bs\mu) &  \ge \bs f^T (\bs \nu-\bs \mu)
\end{align*}
mit $A = [a(\phi_j,\phi_i)]_{i,j=1}^N, \bs \mu = [\mu_i]_{i=1}^N,\bs \nu = [\nu_i]_{i=1}^N$ und $\bs f = [(f,\phi_i)]_{i=1}^N$. Damit lässt sich die Menge $K_h$ auch eindeutig durch die Koordinatenvektoren bzgl. $\mcal B_h$ ausdrücken. Die Menge $K_h$ ist bzgl. $V_h$ äquivalent zu
\begin{align}\label{eq:3.13}
	K_{\mcal S} \coloneqq \{\bs \nu \in \R^N \mid \nu_i \ge \psi(p_i) , p_i \in \mcal N \cap \Omega, i = 1,\ldots, N \} \, .
\end{align}
Im Folgenden schreiben wir $\bs \psi \coloneqq [\psi(p_i)]_{i=1}^N$ mit $p_i \in \mcal N\cap \Omega$.


\begin{bem*}
$K_h$ bzw. $K_{\mcal S}$ sind analog zu $K$ konvex und abgeschlossen.
\end{bem*}


Damit erhalten wir aus \eqref{eq:3.12} die diskrete Variationsungleichung: Finde $\bs\mu \in K_{\mcal S}$, so dass
\begin{align}\label{eq:3.14}
	(A\bs \mu-\bs f)^T (\bs \nu- \bs\mu) &  \ge 0 \quad \forall \, \bs \nu \in K_{\mcal S} \, .
\end{align}
Es fehlt uns nun jedoch die Möglichkeit solch eine Ungleichung, die offensichtlich nicht mehr linear ist, zu lösen. Wir wollen also \eqref{eq:3.14} mathematisch äquivalent formulieren, um es das erzeugte Problem lösen zu können.


 \begin{satz}\label{satz:3.11}
Das Problem \eqref{eq:3.14} ist äquivalent zum linearen Komplementaritätsproblem\index{lineares Komplementaritätsproblem}: Bestimme $\bs\mu \in K_{\mcal S}$, so dass
\begin{align}\label{eq:3.15}
	A\bs\mu - \bs f \ge \bs 0\quad\text{ und }\quad (A\bs\mu-\bs f)^T(\bs \mu - \bs\psi) = 0
\end{align}
gilt.
\end{satz}

\begin{proof}
"`$\Ra$"' Sei $\bs \mu \in K_{\mcal S}$ Lösung von \eqref{eq:3.14}. Wir setzen $\bs\nu = \bs \mu +\bs e_i \ge \bs\psi$ mit einem beliebigen $i \in \{1,\ldots,N\}$, wobei $\bs e_i $ den $i$-te Einheitsvektor bezeichne. Dann gilt
\[
	0 \le (A\bs\mu - \bs f)^T (\bs\nu - \bs\mu) = (A\bs\mu - \bs f)^T \bs e_i = (A\bs \mu-\bs f)_i \, .
\]
Da $i$ beliebig war, folgt $A\bs\mu - \bs f\ge \bs 0$.

Wir nun nehmen an, dass ein $i \in \{1,\ldots,N\}$ existiert, so dass $(A\bs\mu - \bs f)_i (\bs\mu-\bs\psi)_i >0$ ist. Weiter wählen wir
\[
	\bs\nu = \begin{pmatrix}
				\mu_1 \\
				\vdots \\
				\mu_{i-1} \\
				0 \\
				\mu_{i+1}\\
				\vdots \\
				\mu_N
			\end{pmatrix} + 
			\begin{pmatrix}
				0 \\
				\vdots \\
				0\\
				\psi_i \\
				0\\
				\vdots \\
				0
			\end{pmatrix} \ge \bs\psi
\]
und damit folgt
\begin{align*}
	0 > (A\bs \mu - \bs f)_i (\bs\psi - \bs\mu)_i  = (A\bs\mu-\bs f)^T (\bs \nu-\bs \mu) \ge 0 \, ,
\end{align*}
was im Widerspruch zu \eqref{eq:3.14} steht, daraus folgt die Behauptung.

"`$\La$"' Es sei $\bs \mu \in K_{\mcal S}$ Lösung von \eqref{eq:3.15}. Dann rechnen wir nach, dass für ein beliebiges $\bs\nu \in K_{\mcal S}$ gilt
\begin{align*}
	(A\bs \mu-\bs f)^T (\bs \nu- \bs\mu) & = (A\bs \mu-\bs f)^T (\bs \nu -\bs \psi + \bs\psi- \bs\mu) \\
	& = \underbrace{(A\bs \mu-\bs f)^T}_{\ge \bs 0} \underbrace{(\bs \nu -\bs \psi)}_{\ge \bs 0} -\underbrace{(A\bs \mu-\bs f)^T (\bs\mu-\bs\psi)}_{=0} \\
	& \ge 0\, . \qedhere
\end{align*}
\end{proof}


\begin{satz}[Äquivalenz zu quadratischem Programm]\label{satz:3.12}
Das Problem \eqref{eq:3.14} ist äquivalent zum quadratischen Programm\index{quadratisches Programm}
\begin{align}\label{eq:3.16}
	\min_{\bs\nu \in \R^N} J(\bs\nu) = \frac 1 2 \bs \nu^T A \bs \nu - \bs f^T \bs \nu\quad \text{s.t.} \quad \bs \nu \ge \bs \psi \, .
\end{align}
\end{satz}

\begin{proof}
Wir zeigen zunächst die Äquivalenz von \eqref{eq:3.15} zu \eqref{eq:3.16} und dann folgt mit Satz \ref{satz:3.11} die Behauptung.

"`$\Ra$"' Es sei $\bs\mu \in \R^N$ Lösung vom Problem \eqref{eq:3.15}. Dann folgt mit einem beliebigen $\bs\nu \in K_{\mcal S}$
\begin{align*}
	J(\bs\nu) - J(\bs\mu) & =  \frac 1 2 \bs \nu^T A \bs \nu - \bs f^T \bs \nu -  \frac 1 2 \bs \mu^T A \bs \mu + \bs f^T \bs \mu \\
	& = \frac 1 2\underbrace{ (\bs\nu-\bs \mu)^T A(\bs\nu-\bs\mu) }_{\ge 0 \text{ wegen Bem. \ref{bem:2.18}}}+ \bs \mu^T A \bs \nu - \bs \mu ^TA \bs\mu - \bs f^T (\bs\nu-\bs\mu) \\
	%& \ge (A\bs \mu - \bs f)^T(\bs\nu-\bs\mu) \\
	& \ge (A\bs \mu - \bs f)^T(\bs\nu-\bs\psi+\bs\psi-\bs\mu) \\
	& = \underbrace{(A\bs \mu - \bs f)^T}_{\ge 0} \underbrace{(\bs\nu-\bs\psi)}_{\ge 0}- \underbrace{(A\bs \mu - \bs f)^T(\bs\mu-\bs\psi) }_{=0}\\
	&  \ge 0\, .
\end{align*}
Somit ist $\bs\mu \in K_{\mcal S}$ auch Lösung des quadratischen Programms \eqref{eq:3.16}.

"`$\La$"' Sei $\bs\mu \in K_{\mcal S}$ Lösung von \eqref{eq:3.16}, dann gelten nach \cite{NocWri} Kapitel 12, Theorem 12.1 für die Lagrange-Funktion
\[
	\mcal L(\bs\nu,\bs\lambda) = J(\bs\nu)-\bs\lambda^T(\bs\nu-\bs\psi)
\]
die Karush-Kuhn-Tucker Bedingungen für den Optimalpunkt $(\bs \mu,\bs\lambda^*)$
\begin{subequations}
\begin{align}\label{eq:3.17}
	\nabla_{\bs\nu} \mcal L(\bs\mu,\bs\lambda^*) = \nabla J(\bs\mu)-\lambda^* & = A\bs\mu-\bs f - \bs \lambda^* \stackrel != \bs 0 \, ,\\
	\label{eq:3.18}
	\bs\mu - \bs \psi & \ge \bs 0 \, , \\
	\label{eq:3.19}
	\bs \lambda^* & \ge \bs 0 \, , \\
	\label{eq:3.20}
	\lambda_i^* (\mu_i-\psi_i) & = 0 \qquad \forall \, i =1,\ldots,N \, .
\end{align}
\end{subequations}
Mit \eqref{eq:3.17} gilt also $\bs\lambda^* = A\bs\mu-\bs f$ und daher folgt aus \eqref{eq:3.19}
\[
	A\bs\mu - \bs f \ge\bs 0 \, .
\]
Aus \eqref{eq:3.20} folgt wegen $(A\bs\mu-\bs f)_i (\mu_i-\psi_i) = 0$ für alle $i = 1,\ldots,N$ direkt
\[
	(A\bs\mu - \bs f)^T (\bs \mu-\bs\psi) = 0\, .
\]
Also ist $\mu \in K_{\mcal S}$ auch Lösung von \eqref{eq:3.15}.
\end{proof}


\begin{bem}\label{bem:3.13}
Analog zu Satz \ref{satz:3.11} und \ref{satz:3.12} können wir durch leichte Abwandlung der Beweise zeigen, dass das quadratische Programm
\begin{align}\label{eq:3.21}
	\min_{\bs\nu \in \R^N} J(\bs\nu) = \frac 1 2 \bs \nu^T A \bs \nu - \bs f^T \bs \nu \quad \text{s.t.} \quad B\bs\nu \ge \bs \psi
\end{align}
mit $B \in \R^{M\times N}$ äquivalent ist zur Variationsungleichung: Finde $\bs\mu \in \R^N$ mit $B\bs\mu \ge \bs \psi$, so dass
\begin{align}\label{eq:3.22}
	(A\bs\mu - \bs f)^T (\bs\nu-\bs\mu) \ge 0 \quad \forall\, \bs\nu \in \R^N \text{ mit } B \bs \nu \ge \bs\psi \, .
\end{align}
Diese Formulierung werden wir später in Kapitel \ref{kap:3.2.3} wiederfinden, wenn wir Kontaktprobleme mittels der Finiten-Elemente-Methode lösen werden.
\end{bem}


\begin{bem}\label{bem:3.14}
Die quadratischen Programme \eqref{eq:3.16} und \eqref{eq:3.21} haben mit den Voraussetzungen aus Bemerkung \ref{bem:2.18} eine globale Lösung $\bs\mu$, wenn diese die KKT-Bedingungen \eqref{eq:KKT} erfüllt (vgl. Theorem \ref{theorem:B.1}).
\end{bem}


Da wir wegen Bemerkung \ref{bem:2.18} konvexe quadratischen Programme vorliegen haben, können wir diese mit dem in Anhang \ref{anhang:B.2} vorgestellten Active-Set-Algorithmus lösen. Wie wir später sehen werden, ist es sinnvoll bei der Implementierung auf eine Innere-Punkte-Methode (kurz: IPM) zurückzugreifen.

Es bleibt noch zu prüfen, ob die Lösung der Variationsungleichung für Netzverfeinerung an die exakte Lösung konvergiert. Hierfür wollen wir folgende Voraussetzungen an unsere Ansatzräume stellen.


\begin{vor}\label{vor:3.15}
Gegeben sei ein Parameter $h \ra 0$. Weiter seien $(V_h)_h$ eine Familie aus abgeschlossenen Teilmengen von einem Hilbertraum $H, \emptyset \not= K \subset H$ eine konvexe abgeschlossene Teilmenge und $(K_h)_h$ eine Familie von abgeschlossenen konvexen nichtleeren Teilmengen von $H$, so dass $K_h \subset V_h$ für alle $h$.

Dabei sei $K_h$ eine Approximation von $K$ im folgenden Sinne:
\begin{enumerate}[(i)]
\item wenn $(v_h)_h$ eine in $H$ beschränkte Folge mit $v_h \in K_h$ ist, dann folgt $v_h \ra v \in K$,
\item es existiert ein $W \subset H$ mit $\overline W = K$ and ein $I_h: W \ra K_h$, so dass
\[
	\lim_{h\ra 0} I_h v = v
\]
stark in $H$ für alle $v \in W$ konvergiert.
\end{enumerate}
\end{vor}


\begin{theorem}[a priori Konvergenz]\label{theorem:3.16}
Mit den obigen Voraussetzungen für $K$ und $(K_h)_h$ gilt für die Lösungen $u$ von \eqref{eq:3.7} und $u_h$ vom approximierten Problem: Finde $u_h \in K_h$, so dass
\begin{align}\label{eq:3.23}
	a(u_h,v_h-u_h) \ge F(v_h-u_h) \quad \forall \, v_h \in K_h \, , 
\end{align}
der Zusammenhang
\begin{align*}
	\lim_{h\ra 0} \norm{u_h-u}_H  = 0 \, .
\end{align*}
\end{theorem}

\begin{proof}
(i) \underline{Abschätzung von $u_h$:} Es sei $u_h$ Lösung von \eqref{eq:3.23}, dann gilt nach einer Umformung für alle $v_h \in K_h$
\begin{align*}
	a(u_h,u_h) & \le a(u_h,v_h)-F(v_h-u_h) \\ 
	& = (Au_h,v_h)_H - (f,v_h-u_h)_H \\
	& \stackrel{\scriptsize\textnormal{CS}}\le \underbrace{\norm{Au_h}_H}_{\le\norm{A} \norm{u_h}_H} \, \norm{v_h}_H + \norm{f}_H\underbrace{\norm{v_h-u_h}_H}_{\le \norm{v_h}_H+\norm{u_h}_H} \\
	& \le \norm A \, \norm{u_h}_H  \norm{v_h}_H + \norm f_H \, (\norm{v_h}_H+\norm{u_h}_H)  \, .
\end{align*}
Zusammen mit der Koerzivität folgt dann
\begin{align}\label{eq:3.24}
	\alpha \, \norm{u_h}^2_H \le \norm A \, \norm{u_h}_H  \norm{v_h}_H + \norm f_H \, (\norm{v_h}_H+\norm{u_h}_H) \, .
\end{align}
Wähle ein festes $v_0 \in W$, sodass $I_h v_0 = v_h \in K_h$ gilt. Aus Voraussetzungen (ii) folgt dann
\[
	\lim_{h\ra 0} I_h v_0 = v_0
\]
und daher muss $v_h$ beschränkt sein, d.h. es existiert ein $m \in \R: \norm{v_h}_H \le m$ für alle $h$. Zusammen mit \eqref{eq:3.24} gilt dann
\begin{align*}
	\norm{u_h}^2_H & \le \frac 1 \alpha (m \, \norm A \, \norm{u_h}_H + \norm f_H (m+\norm{u_h}_H)) \\
	& = \underbrace{\(\frac m \alpha\, \norm A +  \norm f_H \)}_{\eqqcolon c_1} \norm{u_h}_H + \underbrace{\frac m\alpha\norm f_H}_{ \eqqcolon c_2} \\
	& = c_1 \, \norm{u_h}_H + c_2
\end{align*}
und damit können wir durch quadratischer Ergänzung folgern, dass es ein $c \in \R$ gibt mit $\norm{u_h} \le c$ für alle $h$, d.h. $(u_h)_h$ ist gleichmäßig beschränkt.

(ii) \underline{schwache Konvergenz:} Da $(u_h)_h$ in $H$ gleichmäßig beschränkt ist, folgt mit Bemerkung \ref{bem:A.13} (b), dass es eine schwach konvergente Teilfolge $(u_{h_j})_{h_j} \in K_{h_j}$ mit einem Grenzwert $u^*$ in $H$ gibt, d.h.
\[
	u_{h_j} \rightharpoonup u^* \in H \, .
\]
Mit den Voraussetzungen (i) für $(K_h)_h$ folgt direkt $u^* \in K$, außerdem ist $u^*$ nach Bemerkung \ref{bem:A.13} (e) eindeutig.

Wir zeigen nun, dass $u^*$ eine Lösung von \eqref{eq:3.7} ist. Für die oben betrachtete Teilfolge gilt
\begin{align}\label{eq:3.25}
	a(u_{h_j},v_{h_j}-u_{h_j})\ge F(v_{h_j}-u_{h_j}) \quad \forall \, v_{h_j} \in K_{h_j} \, .
\end{align}
Sei $v \in W$ mit $v_{h_j} = I_{h_j} v$. Dann gilt $v_{h_j} = I_{h_j} v \ra v \in W$ für $h_j \ra 0$. Mit \eqref{eq:3.25} folgt
\begin{align}
	\notag a(u_{h_j},u_{h_j}) &\le a(u_{h_j},v_{h_j})-F(v_{h_j}-u_{h_j}) \\
	\notag & = a(u_{h_j},I_{h_j} v)-F(v_{h_j}-u_{h_j}) \\
	\label{eq:3.26} \lra \liminf_{h_j\ra 0} a(u_{h_j},u_{h_j}) & \le  a(u^*,v)-F(v-u^*) \, .
\end{align}
Weiter schätzen wir durch Bemerkung \ref{bem:A.13} (f) nach unten ab
\begin{align}\label{eq:3.27}
	\liminf_{h_j\ra 0} a(u_{h_j},u_{h_j}) = \liminf_{h_j\ra 0} \norm{u_{h_j}}_H^2 \ge \norm{u^*}_H^2 = a(u^*,u^*) \, .
\end{align}
Insgesamt folgt also mit \eqref{eq:3.26} und \eqref{eq:3.27}
\[
	a(u^*,u^*) \le \liminf_{h_j\ra 0} a(u_{h_j},u_{h_j}) \le   a(u^*,v)-F(v-u^*)
\]
und damit nach Umformung
\[
	a(u^*,v-u^*)\ge F(v-u^*) \quad \forall \, v \in W \, .
\]
Da $W$ dicht in $K$ liegt, d.h. $\overline W = K$, und $a, F$ stetig sind, erhalten wir
\[
	a(u^*,v-u^*)\ge F(v-u^*) \quad \forall \, v \in K 
\]
mit $u^*\in K$, also ist $u^*\eqqcolon u$ Lösung von \eqref{eq:3.7}. Da $u$ ein Häufungspunkt von $(u_h)_h$ bzgl. der schwachen Topologie von $H$ ist, konvergiert auch die Folge $(u_h)_h$ schwach gegen $u$.

(iii) \underline{starke Konvergenz:} Aus der Koerzivität von $a$ folgt
\begin{align}\label{eq:3.28}
	\begin{aligned}
		0 &\le \alpha \, \norm{u_h-u}_H^2 \le a(u_h-u,u_h-u) \\
		& \le a(u_h,u_h)-a(u_h,u)-a(u,u_h)+a(u,u) \, ,
	\end{aligned}
\end{align}
wobei $u_h$ Lösung vom approximierten Problem \eqref{eq:3.23} und $u$ Lösung vom exakten Problem \eqref{eq:3.7} ist. Es sei $v \in W$ mit $I_h v = v_h \in K_h$, dann folgt aus \eqref{eq:3.23}
\begin{align}\label{eq:3.29}
	a(u_h,u_h) \le a(u_h,I_h v) - F(I_h v-u_h) \quad \forall \, v \in W \, .
\end{align}
Da $u_h \rightharpoonup u$ in $H$ und $I_h v \ra v$ in $H$ für $h \ra 0$, folgt aus \eqref{eq:3.28} und \eqref{eq:3.29} unter Verwendung von Voraussetzungen (ii)
\begin{align}\label{eq:3.30}
	0 \le \alpha \lim_{h\ra 0} \norm{u_h-u}^2_H \le a(u,v-u)-F(v-u) \quad \forall \, v \in W \, .
\end{align}
Da $a$ und $F$ stetig sind und $W$ dicht in $K$ liegt, gilt \eqref{eq:3.30} auch für alle $v \in K$. Setzen wir dann also $v = u$ in \eqref{eq:3.30}, dann folgt die Behauptung $\lim_{h\ra 0} \norm{u_h - u }_H^2 = 0$. 
\end{proof}


Es sei $V_h \subset H^1_0(\Omega)$ ein beliebiger endlich dimensionaler Unterraum von $H^1_0(\Omega)$ und $K_h$ wie zu Beginn dieses Kapitels definiert. Dann ist $K_h \subset V_h$ für alle Parameter $h$. Außerdem folgt Voraussetzung \ref{vor:3.15} (i) wegen der Abgeschlossenheit von $K_h$. Betrachten wir
\begin{align*}
	W \coloneqq \{v \in H^1_0(\Omega) \mid v > \psi \text{ f.ü. in } \Omega\} \, ,
\end{align*}
so gilt $\overline W = K$ und mit der bekannten \idx{Spline-Interpolation} folgt dann auch Voraussetzung \ref{vor:3.15} (ii). Damit gilt
\[
	\lim_{h\ra 0} \norm{u_h-u}_1 = 0\, .
\]


Es lässt sich folglich auch eine a priori Abschätzung für den Fehler von $u$ und $u_h$ machen. Die Herleitung ist detailliert beispielsweise in \cite{Falk} wiederzufinden.


\begin{theorem}[a priori Fehlerabschätzung]\label{theorem:3.17}
Seien $u$ und $u_h$ die Lösungen von \eqref{eq:3.7} und \eqref{eq:3.23}. Dann existiert eine Konstante $C \coloneqq C(\Omega, f, \psi)$ unabhängig von $u$, so dass
\[
	\norm{u_h-u}_1 \le C h \, .
\]
\end{theorem}

\begin{proof}
Vgl. \cite{Falk} Theorem 2 bzw. \cite{EPS} Theorem 6.4.
\end{proof}


Damit führt die Netzverfeinerung also zur exakten Lösung der Variationsungleichung \eqref{eq:3.11}. Inwiefern adaptive Netzverfeinerung hier sinnvoll ist, wollen wir in Kapitel \ref{kap:4} betrachten.








\section{Kontaktprobleme}
\label{kap:3.2}


Das in Kapitel \ref{kap:3.1} vorgestellte \idx{Hindernisproblem} ist ein Modellproblem zur Minimierung eines Energiefunktionals unter Nebenbedingung. Kontaktprobleme sind Probleme aus der Strukturmechanik, die auch auf eine solche Problemstellung führen. Daher können wir die Ideen zur Lösung eines Hindernisproblems auf diese Problemstellung übertragen.


\subsection{Mathematische Modellierung eines Kontaktproblems}
\label{kap:3.2.1}

Zunächst wollen wir mithilfe der in Kapitel \ref{kap:2.5} eingeführten mechanischen Gleichungen ein \idx{Kontaktproblem} mathematisch modellieren.

\begin{vor}\label{vor:3.18}
Wir treffen folgende Annahmen für unser Kontaktmodell:
\begin{enumerate}[(a)]
\item Die in Kontakt stehenden Körper sind beschränkt.
\item Es liegen kleine Deformationen und linear elastische Materialien vor.
\item Wir betrachten ein konstantes Temperaturfeld, d.h. thermodynamische Prozesse werden ausgeschlossen.
\item Zu Beginn, also in der \idx{Ausgangskonfiguration}, gilt für die \idx{Spannung} und \idx{Verzerrung}: 
\[
	\bs \sigma = \bs 0\, , \quad \bs \eps =\bs 0 \, .
\]
\item Wir gehen von einem reibungslosen Kontakt aus. Dieses Kontaktproblem wird als \textit{\idx{Signorini-Kontakt}-Problem} bezeichnet.
\item Wir gehen von ebenen Problemen aus, d.h. $\Omega \subset \R^2$. Im $\R^3$ sind alle Resultate analog.
\end{enumerate}
\end{vor}


Zur Herleitung der starken Kontaktformulierung wollen wir zwei Körper $\mscr B^1, \mscr B^2$ betrachten, welche wegen Voraussetzung \ref{vor:3.18} (a) durch zwei beschränkte Gebiete $\Omega^1, \Omega^2$ mathematisch beschrieben werden können. Diese Voraussetzung lässt sich auch auf beliebig viele Körper verallgemeinern (vgl. hierfür \cite{CarWri}).

Weiter lassen sich die Ränder $\Gamma^i = \partial \Omega^i$ von $\Omega^i, i = 1,2$, in drei disjunkte Teile unterteilen (s. Abbildung \ref{abb:3.2}):
\begin{enumerate}
\item[$\Gamma^i_u$:] Der \textit{Dirichlet-Rand}\index{Randbedingungen!Dirichlet}, oder auch \textit{\idx{Verschiebungsrand}}, auf dem die Werte von der Verschiebung $\bs u$ vorgegeben sind.
\item[$\Gamma^i_\sigma$:] Der \textit{Neumann-Rand}\index{Randbedingungen!Neumann}, oder auch \textit{\idx{Spannungsrand}}, auf dem die Oberflächenlast bzw. -spannung $\bar{\bs t}$ vorgegeben ist.
\item[$\Gamma^i_c$:] Der \textit{Kontaktrand}\index{Randbedingungen!Kontakt}, auf dem die Kontaktbedingungen definiert sind.
\end{enumerate}



\begin{figure}[h!]
\begin{center}
	\begin{pspicture}(0,0)(4,4.5)
		% Die Körper B_1,B_2:
		\psccurve(-2,3)(0,4)(2,3)(2,1)(1,0)(-0.5,1.5)
		\psccurve(3,1)(5,4)(6,1)(5,0.7)(3,0)
		\rput(-1,3){$\Omega^1$}
		\rput(5.5,1.5){$\Omega^2$}
		
		% Grenzen der Ränder:
		% B_1:
		\psline(-1.85,2.95)(-2.15,3)
		\psline(2,3.3)(1.7,3.05)
		\psline(1,0.15)(1,-0.15)
		% B_2:
		\psline(3.1,0.12)(2.9,-0.12)
		\psline(5,3.8)(5.05,4.15)
		\psline(5.8,1)(6.05,0.8)
		
		% Beschriftung:
		\rput(2.3,0.8){$\Gamma_c^1$}
		\rput(-1,1.3){$\Gamma_\sigma^1$}
		\rput(-0.3,4.3){$\Gamma_u^1$}
		
		\rput(5,0.3){$\Gamma_\sigma^2$}
		\rput(6.5,2.5){$\Gamma_u^2$}
		\rput(3,2.2){$\Gamma_c^2$}
	\end{pspicture}
\end{center}
\caption{Körper $\mcal B^1$ und $\mcal B^2$ mit Randbezeichnungen\label{abb:3.2}}
\end{figure}


\subsubsection{Kontaktkinematik}

 Wir wollen zunächst die Kontaktkinematik (wie in \cite{CarWri}) etwas allgemeiner als in $\cite{WriggersFEM}$ oder $\cite{WriggersContact}$ beschrieben einführen. Für die Formulierung der Kontaktbedingungen werden den Körpern $\mscr B^1, \mscr B^2$ die Bezeichnungen \textit{\idx{master}} und \textit{\idx{slave}} zugeordnet. Mit slave bezeichnen wir dabei die Menge an Punkten, für die überprüft wird, ob sie in die master-Fläche eindringen. Die Zuordnung von master und slave ist jedoch für das Ergebnis der Kontaktbedingung vollkommen unabhängig. Es sei daher o.B.d.A. $\mscr B^1$ der slave.



\begin{figure}[h!]
\begin{center}
	\begin{pspicture}(-1,-0.5)(4,4)
		% B_1 (slave):
		\pscurve(-1,-1)(0,-0.2)(2,0.3)(5,-0.8)
		\rput(2,-0.2){$\Omega^1$}
		
		% B_2 (master)
		\psline(0,4)(1,2.5)
		\psline(1,2.5)(3,3)
		\psline(3,3)(4,4)
		\rput(2,3.5){$\Omega^2$}
		\rput(3,2.7){$\Gamma_c^2$}
		\rput(0.2,3){$\Gamma_c^2$}
		
		% Punkt und kritische/r Punkt/Richtung:
		\psdot(0,-0.2)
		\psdot(1,2.5)
		\psline{->}(0,-0.2)(0.5,1.15)
		\rput(-0.3,-0.1){$\bs X$}
		\rput(0.6,1.35){$\bs n_c(\bs X)$}
		\rput(1.5,2.35){$\chi (\bs X)$}
	\end{pspicture}
\end{center}
\caption{Kontaktformulierung zwischen zwei Körpern\label{abb:3.3}}
\end{figure}

Für einen gegebenen Punkt $\bs X \in \Omega^1$, bzw. $\bs X \in \Gamma^1_c$, in der Ausgangskonfiguration ist $\bar{\bs X} \coloneqq \chi(\bs X)$ derjenige Punkt aus $\Gamma^2$, der minimalsten Abstand zu $\bs X$ hat, d.h.
\[
	\norm{\bs X-\bar{\bs X}} = \min \{\norm{\bs X-\bs Y} \mid \bs Y \in \Gamma^2\} \, ,
\]
also ist $\chi: \Gamma_c^1\cup \Gamma_c^2 \ra \Gamma^1 \cup \Gamma^2$ eine Abbildung der kleinsten Distanz (s. Abbildung \ref{abb:3.3}). Damit definieren wir entsprechend die \textit{\idx{kritische Richtung}} mit Länge 1 als
\begin{align}
	\bs n_c(\bs X) \coloneqq \frac{\chi(\bs X) - \bs X}{\norm{\chi (\bs X)-\bs X}} \, ,
\end{align}
wobei im Falle $\bs X = \chi(\bs X)$, d.h. im Falle des Kontaktes, eine beliebige normierte Richtung für $\bs n_c(\bs X)$ gesetzt wird. Den Punkt $\bar{\bs X}$ nennen wir analog \textit{kritischen Punkt}\index{kritischer Punkt}.


\begin{bem}\label{bem:3.19}
Der Punkt $\bar{\bs X}$ heißt deshalb kritischer Punkt, da er wegen des kleinsten Abstandes zu $\bs X$ der wohlmöglich nächste Punkt ist, der mit $\bs X$ in Kontakt tritt. Der Vorteil dieser Formulierung ist, dass die kritische Richtung $\bs n_c(\bs X)$ auch existiert, wenn der Rand eines Körpers nicht hinreichend glatt ist.
\end{bem}


In den Koordinaten der \idx{Momentankonfiguration} gilt $\bs x = \bs X + \bs u$ für das Verschiebungsfeld $\bs u$ und damit erhalten wir die \textit{\idx{Nichtdurchdringungsbedingung}}
\begin{align}\label{eq:3.32}
	(\bar{\bs x} - \bs x) \, \bs n_c (\bs X) \ge 0 \, , 
\end{align}
wobei $\bar{\bs x} \coloneqq \bar{\bs X} + u(\bar{\bs X})$ ist. Dies bedeutet, dass die Verbindung der Punkte in der Momentankonfiguration mit der kritischen Richtung einen Winkel $\alpha \in [-\frac \pi2,\frac \pi2]$ einschließen muss, andernfalls läge $\bar{\bs x}$ "`hinter"' $\bs x$, d.h. $\mscr B^1$ wäre in $\mscr B^2$ eingedrungen. Aus \eqref{eq:3.32} folgt
\begin{align*}
	0 &  \le (\bar{\bs x} - \bs x) \, \bs n_c (\bs X) = (\bar{\bs X} + u(\bar{\bs X}) -{\bs X} - u({\bs X})) \, \bs n_c(\bs X) \\
	& = \underbrace{(\bs{\bar X} - \bs X) \frac{\bar{\bs X}-\bs X}{\norm{\bar{\bs X}-\bs X}}}_{ = \norm{\bar{\bs X}-\bs X}\eqqcolon g} + ( u(\bar{\bs X})  - u({\bs X})) \, \bs n_c(\bs X) \\
	& = g + ( u(\chi(\bs X))  - u({\bs X})) \, \bs n_c(\bs X) \, ,
\end{align*}
was uns \idx{Nichtdurchdringungsbedingung} bzgl. der \idx{Ausgangskonfiguration} liefert, wobei wir die Funktion $g$ auch als \textit{\idx{Gap-Funktion}} bezeichnen, da sie die Lücke zwischen den Körpern beschreibt.

Aufgrund von Voraussetzung \ref{vor:3.18} (b) gehen wir von kleinen Deformationen aus. Damit gilt unter anderem  $\bs X \approx \bs x,  \nabla_{\bs X} =\Grad(\cdot)\approx \grad(\cdot) =  \nabla_{\bs x}$ (vgl. \cite{AltKonti} S. 122f). Daher schreiben wir im folgenden immer $\bs x$ statt $\bs X$. Insgesamt lässt sich also die \idx{Nichtdurchdringungsbedingung} schreiben als
\begin{align}\label{eq:3.33}
	(\bs u \circ \chi -\bs u) \cdot \bs n_c + g \ge 0 \quad \forall \, \bs x \in \Gamma_c \, ,
\end{align}
wobei $\Gamma_c \coloneqq \Gamma_c^1 \cup \Gamma_c^2$ ist.


Der Einfachheit halber wollen wir in dieser Arbeit noch weitere Forderungen an die beiden Körper $\mscr B^1$ und $\mscr B^2$ stellen. Wir fordern, dass die Ränder $\Gamma^i$ hinreichend glatt sind. Daraus folgt, dass \eqref{eq:3.33}  auch für $\bs n_c$ als Einheitsnormale von $\mscr B^1$ gilt. Weiter soll $\bs u(\bar{\bs x})\equiv \bs 0$ gelten, d.h. falls $\mscr B^2$ ein Verschiebungsfeld ungleich Null hat, können wir $\bs u$ bzgl. $\Omega = \Omega^1\cup\Omega^2$ auch als Relativverschiebung interpretieren.


Im numerischen Beispiel wollen wir später $\mscr B^1$ als feste Ebene verwenden. Damit reduziert sich \eqref{eq:3.33} auf
\begin{align}\label{eq:3.34}
	\bs u \cdot \bs n - g \le 0 \quad \forall \, x \in \Gamma_c \, ,
\end{align}
wobei wir auch $u_n \coloneqq \bs u \cdot \bs n$ im Folgenden schreiben werden. Weiter muss auf dem \idx{Kontaktrand} $\Gamma_c$ die Normalkraft eine Druckkraft sein oder es herrscht Kräftegleichgewicht, d.h. für  $\sigma_n \coloneqq \bs n\cdot (\bs \sigma\cdot \bs n)$, also die \idx{Spannung} in Normalenrichtung, gilt
\begin{align}\label{eq:3.35}
	\sigma_n \le 0  \ \text{ auf }  \Gamma_c \, .
\end{align}
Damit gilt auch, wenn die Kontaktbedingung nicht aktiv ist (also nicht die Gleichheit gilt), so muss Kräftegleichgewicht herrschen, d.h. in \eqref{eq:3.35} gilt die Gleichheit. Zusammen erhält man die \textit{\idx{Komplementaritätsbedingung}}
\begin{align}\label{eq:3.36}
	(u_n - g )\,   \sigma_n = 0 \ \text{ auf } \Gamma_c \, .
\end{align}
Laut der Voraussetzung (e) betrachten wir \idx{Signorini-Kontakt} (also keine Reibung) und damit muss die Tangentialkraft auf dem Kontaktrand gleich Null sein, d.h.
\begin{align}\label{eq:3.37}
	\bs \sigma_t \coloneqq \bs \sigma \cdot \bs n - \sigma_n \bs n = 0 \ \text{ auf } \Gamma_c \, .
\end{align}



\subsubsection{Bilanzgleichungen, materialunabhängige Gleichungen}


Wie in Kapitel \ref{kap:2.5} eingeführt, gelten auch hier die Gleichung \eqref{eq:2.32} des Kräftegleichgewichts. Da wir laut Voraussetzung (b) von kleinen Deformationen ausgehen, gilt
\begin{align}\label{eq:3.38}
	\div\bs \sigma + \bar{\bs b} = \bs 0 \ \text{ in } \Omega\, .
\end{align}
Weiter gilt nach dem \idx{Cauchy-Theorem} \eqref{eq:2.29}, dass die Spannung in Normalenrichtung auf der Oberfläche $\Gamma$ von $\Omega$ gleich der von außen angebrachten Spannung $\bar{\bs t}$ ist, d.h.
\begin{align}\label{eq:3.39}
	\bs \sigma \cdot \bs n = \bar{\bs t} \ \text{ auf } \Gamma_\sigma \, ,
\end{align}
also auf dem Neumann-Rand\index{Randbedingungen!Neumann}.



\subsubsection{Konstitutive Gleichungen}



Da wir laut Voraussetzung (b) von einem linear elastischen Material und kleinen Deformationen ausgehen, gilt ein linearer Zusammenhang bzgl. der Spannung $\bs \sigma$ und Verzerrung $\bs \eps$, d.h. das \idx{Hooke'sche-Gesetz} \eqref{eq:2.33} und wir können  den linearisierten Verzerrungstensor $\bs \eps$ (vgl. \eqref{eq:2.28}) verwenden, d.h. mit einem 4 stufigem Materialtensor $\mcal C = (c_{ijkl})$ gilt
\begin{align}\label{eq:3.40}
	\bs \sigma - \mcal C : \bs \eps = \bs 0 \ \text{ in } \Omega \, .
\end{align}


Zusammenfassend lässt sich das \idx{Signorini-Kontakt}-Problem mit \eqref{eq:3.34} bis \eqref{eq:3.40} in der starken Formulierung beschreiben durch:
\begin{subequations}\label{eq:3.41}
\begin{align}\label{eq:3.41a}
	\div \bs \sigma + \bar{\bs b} = \bs 0&\ \ \text{ in } \Omega\\
	\label{eq:3.41b}
	\bs \sigma  - \mcal C :\bs \eps  = \bs 0&\ \ \text{ in } \Omega\\
	\label{eq:3.41c}
	\bs \sigma \cdot \bs n = \bar{\bs t} &\ \ \text{ auf } \Gamma_\sigma \\
	\label{eq:3.41d}
	\bs u = \bs 0 &\ \ \text{ auf }\Gamma_u \\
	\label{eq:3.41e}
	\begin{aligned}
		 u_n  - g& \le 0 \\
		\sigma_n &\le 0 \\
		(u_n  - g)\,  \sigma_n &= 0  \\
		\bs \sigma_t &= \bs 0
	\end{aligned}&\Bigg\}
\text{ auf } \Gamma_c 
\end{align}
\end{subequations}



An dieser Stelle sei kurz angemerkt, wie sich die Problemstellung \eqref{eq:3.41} ändern würde, wenn wir ein Modell mit Reibung betrachten.

\begin{bem}
Ein Kontaktmodell mit $\bs \sigma_t \not = \bs 0$ ist beispielsweise das Modell mit \textit{\idx{Tresca-Reibung}}. Für dieses Problem wird die letzte Bedingung aus \eqref{eq:3.41e} durch die Bedingungen
\begin{align}\label{eq:3.42}
	\norm{\bs \sigma_t} \le \mscr F \, , \quad \bs \sigma_t \bs u_t + \mscr F\,  \norm{\bs u_t} = 0
\end{align}
mit $\bs u_t \coloneqq \bs u - u_n \bs n$, dem tangentialen Anteil des Verschiebungsfeldes $\bs u$, ersetzt. Hierbei ist $\mscr F \ge 0$ eine Schranke für die Reibung. Gilt $\norm{\bs \sigma_t} < \mscr F$, so folgt aus der zweiten Gleichung von \eqref{eq:3.42}, dass $\bs u_t = \bs 0$ ist. Also kann $\bs u_t \not=\bs 0$ nur gelten, wenn $\norm{\bs \sigma_t} = \mscr F$ ist.

Mit $\mscr F \coloneqq \mu \sigma_n$ erhalten wir das \textit{Reibungsgesetz von Coulomb}\index{Coulomb-Reibung}, wobei $\mu$ den aus der Mechanik bekannten Reibungskoeffizienten darstellt.

Da die Herleitung der zu diesem Problem äquivalenten Variationsungleichung zusätzliche mathematische Resultate erfordert, werden wir uns in der weiteren Herleitung auf das \idx{Signorini-Kontakt}-Problem beziehen. Außerdem führt solch ein Problem auf eine sogenannte Variationsungleichung zweiter Art, deren Lösung wir in dieser Arbeit nicht ausführen wollen.
\end{bem}





\subsection{Variationsformulierung des Signorini-Kontaktproblems}
\label{kap:3.2.2}

\begin{itemize}
\item Sei $\Omega \subset \R^2$ (Voraussetzung (f))

\item wir betrachten das \idx{Signorini-Kontakt}-Problem \eqref{eq:3.41}

\item Es seien
$H^1_{\Gamma_u}(\Omega) \coloneqq \{\bs v \in (H^1(\Omega))^2 \mid \bs v = \bs 0 \text{ auf } \Gamma_u\}$ und
$\mscr K \coloneqq \{\bs v \in H^1_{\Gamma_u}(\Omega) \mid v_n - g \le 0 \text{ auf } \Gamma_c\}$ $\Ra$ $\mscr K$ ist analog zu Lemma \ref{lem:3.1} konvex

\item weiter seien $\bs u, \bs v \in K$, wobei $\bs u$ die Lösung des Signorini-Kontaktproblems darstellt und $\bs v$ (häufig in den Ingenieurswissenschaften als \textit{\idx{virtuelle Verschiebung}} bezeichnet) eine beliebige Testfunktion ist. Dann gilt $\bs w = \bs v - \bs u \in (H^1_{\Gamma_u}(\Omega))^2$ eine Testfunktion, die wir mit \eqref{eq:3.41a} multiplizieren und über $\Omega$ integrieren.
\begin{align}\notag
	 0 & =  \int_\Omega (\div \bs\sigma + \bar{\bs b}) \cdot \bs w \, d \Omega = \int_\Omega \div \bs \sigma \cdot \bs w + \bar{\bs b} \cdot \bs w \, d\Omega \\
	 \notag
	& = \int_\Omega \div(\bs w \cdot \bs\sigma) - \grad \bs w : \bs \sigma + \bar{\bs b} \cdot \bs w \, d\Omega \\
	\notag
	& = \int_\Gamma \underbrace{\bs w \cdot \bs \sigma \cdot \bs n}_{=\bs w \cdot \norm{\bs n}^2 \cdot (\bs \sigma \cdot \bs n)} \, d\Gamma - \int_\Omega \underbrace{\frac 1 2 (\grad \bs w + \grad^T \bs w)}_{= \bs \eps (\bs w)} : \bs \sigma \, d\Omega + \int_\Omega \bar{\bs b} \cdot \bs w \, d \Omega \\
	\label{eq:3.43}
	& = \int_{\Gamma_c} w_n \, \sigma_n \, d \Gamma + \int_{\Gamma_\sigma} \bar{\bs t} \cdot \bs w \, d\Gamma-\int_\Omega \bs\sigma: \bs \eps(\bs w) \, d\Omega + \int_\Omega \bar{\bs b} \cdot \bs w \, d\Omega
\end{align}
(zweite Zeile ist Produktregel für die Divergenz, dritte Zeile Gauß, vierte Zeile die Aufteilung von $\Gamma = \Gamma_u \cup \Gamma_\sigma \cup \Gamma_c$, wobei $\bs w = \bs 0$ auf $\Gamma_u$ und $\bs \sigma \bs n = \bar{\bs t}$ auf $\Gamma_\sigma$).

\item betrachte das Integral über $\Gamma_c$, dann gilt für den Integranden
\begin{align*}
	w_n \, \sigma_n &= (v_n - u_n) \, \sigma_n \stackrel{\scriptsize\text{"`}+0\text{"'}}= (v_n-g+g-u_n) \, \sigma_n  \\
	& =(v_n - g) \, \sigma_n - \underbrace{(u_n-g) \, \sigma_n}_{=0 \text{ auf } \Gamma_c} \\
	&  =\underbrace{ (v_n-g)}_{\le 0} \, \underbrace{\sigma_n}_{\le 0} \ge 0
\end{align*}

\item damit ist das Integral $\ge 0$ und aus \eqref{eq:3.43} folgt
\begin{align}\notag
	& 0 \ge \int_{\Gamma_\sigma} \bar{\bs t} \cdot \bs w \, d\Gamma-\int_\Omega \bs\sigma: \bs \eps(\bs w) \, d\Omega + \int_\Omega \bar{\bs b} \cdot \bs w \, d\Omega \\
	\label{eq:3.44}
	\Llra& \int_\Omega \bs\sigma(\bs u): \bs \eps(\bs v - \bs u) \, d\Omega \ge\int_\Omega \bar{\bs b} \cdot (\bs v-\bs u) \, d\Omega+ \int_{\Gamma_\sigma} \bar{\bs t} \cdot (\bs v-\bs u) \, d\Gamma 
\end{align}

\item mit \eqref{eq:3.41b} kann die Spannung aus \eqref{eq:3.44} bzgl. der Verzerrung mit $\bs\sigma = \mcal C : \bs\eps$ ausgedrückt werden

\item mit der Bilinearform $a: H^1_{\Gamma_u} \times H^1_{\Gamma_u} \ra \R$ und Linearform $F: H^1_{\Gamma_u} \ra \R$ mit
\begin{align*}
	a(\bs u,\bs v) &\coloneqq \int_\Omega \bs\eps (\bs u) : \mcal C : \bs\eps(\bs v) \, d\Omega \, ,\\ 
	 F(\bs v)& \coloneqq \int_\Omega \bar{\bs b} \cdot \bs v \, d\Omega + \int_{\Gamma_\sigma} \bar{\bs t} \cdot \bs v \, d\Gamma 
\end{align*}
lässt sich \eqref{eq:3.44} in der altbekannten Form: Finde $\bs u \in \mscr K$, so dass gilt
\begin{align}\label{eq:3.45}
	a(\bs u, \bs v - \bs u) \ge F(\bs v- \bs u) \quad \forall \, \bs v \in \mscr K \, .
\end{align}

\item[noch zu zeigen:] Kornsche Ungleichung (vgl. \cite{BraeFEM}) auch für $a(\cdot,\cdot)$ unter der Bedingung, dass $\max \norm{C_{ijkl}} \le M$ für ein $M\ge 0$ ist. $\Ra$ damit folgt die Koerzivität von $a$. Und noch Stetigkeit zeigen

\item vgl. \cite{KikOden} S. 112, zeige: $\abs{F(\bs v)} \le C(\norm{\bar{\bs b}}_0+\norm{\bar{\bs t}}_{0,\Gamma_c})\, \norm{\bs v}_1$, also mit $\bar{\bs b} \in (L^2(\Omega))^2$ und $\bar{\bs t} \in (L^2(\Gamma_c))^2$.

\item als letztes: $K$ ist analog zu Lemma \ref{lem:3.1} abgeschlossen und konvex.

\item also hat das Problem \eqref{eq:3.45} eine eindeutige Lösung. als Theorem
\begin{theorem}

\end{theorem}

\begin{proof}

\end{proof}

\item 
\begin{theorem}
Es sei $\bs v : \Omega \ra \R^2$ und $K$ wie oben definiert. Die Variationsungleichung \eqref{eq:3.45} ist äquivalent zum Minimierungsproblem:
\begin{align}
	\min_{\bs v \in K} J(\bs v)& = \frac 1 2 a(\bs v, \bs v)-F(\bs v)
\end{align}
\end{theorem}

\begin{proof}
zeige die Voraussetzungen von Satz \ref{satz:A.10}, also Konvexität von $J$ z.B.
\end{proof}

\item 
\begin{bem}
für das Kontaktproblem mit Tresca-Reibung gibt es ein analoges Funktional. dieses aufführen!! dieses ist nicht G-differenzierbar $\Ra$ anders herleiten. dies führt auf eine sogenannte  Variationsungleichung 2. Art (vgl. \cite{EPSContact})
\[
	a(\bs u, \bs v-\bs u) + j(\bs v) -  j(\bs u) \ge F(\bs v-\bs u)
\]
\end{bem}
mit dem Reibungsfunktional $j$.
\end{itemize}






\subsection{Lösung des Kontaktproblems mittels FEM}
\label{kap:3.2.3}


\begin{itemize}
\item betrachte analog zu $\mcal S_h$ den Raum der linearen mehrdimensionalen Ansatzfunktionen bzgl. einer quasi-uniformen Triangulierung $\mcal T_h$
\begin{align}
	 \mscr S_h \coloneqq \{\bs v \in (C^0(\Omega))^2 \mid \bs v|_T \in \mcal P_1^2 \text{ für } T \in \mcal T_h, \bs v|_{\Gamma_u} = \bs 0\} \subset H^1_{\Gamma_u} (\Omega)
\end{align}

\item mit einer Basis $\mscr B_h \coloneqq \{\bs\psi_1,\ldots,\bs \psi_N\}$ von $\mscr S_h$ lässt sich jedes Element $\bs v_h \in \mscr S_h$ als Linearkombination schreiben
\begin{align}
	\bs v_h(\bar x) = \sum_{i = 1}^N x_i \, \bs \psi_i(\bar x) \quad \forall \, \bar x \in \Omega
\end{align}
für genau ein $(x_1,\ldots,x_N)^T \eqqcolon \bs x \in \R^N$.

\item betrachten wir analog zu oben \eqref{eq:3.45} diskret, so ergibt sich: Finde $\bs u_h \in \mscr S_h$, so dass
\begin{align}
	a(\bs u_h, \bs v_h - \bs u_h) \ge F(\bs v_h- \bs u_h) \quad \forall \, \bs v_h \in K_h \, .
\end{align}
mit $\mscr K_h \coloneqq \{  \bs v_h \in \mscr S_h \mid \bs v_h(\bar x_i) \cdot \bs n - g(\bar x_i) \le 0 \text{ mit } \bar x_i \in \mcal N \cap \Gamma_c\}$, d.h. die punktuelle Form (der Nebenbedingung) von $\mscr K$.

\item analog zu $K_{\mcal S}$ können wir auch hier bzgl. einer Basis $\mscr B_h$ die Menge $\mscr K_h$ äquivalent durch den Koordinatenvektor $\bs x \in \R^N$ ausdrücken, d.h.
\begin{align*}
	\mscr K_{\mscr S}  \coloneqq& \bigg\{ \bs x \in \R^N \,\bigg| \, \sum_{j=1}^N x_j \, \bs \psi_j(\bar x_i) \cdot \bs n- g(x_i) \ge 0 \text{ für } \bar x_i \in \mcal N \cap \Gamma_c\bigg\} \\
	=& \{ \bs x \in \R^N \mid B\bs x \ge \bs c , B =  [-\bs \psi_j(\bar x_i)\cdot \bs n(\bar x_i)]_{\bar x_i \in \mcal N \cap \Gamma_c, 1\le j \le N} ,\bs c = [-g(\bar x_i)]_{\bar x_i \in\mcal N \cap \Gamma_c}\}
\end{align*}

\item damit ist das diskrete Problem: Finde $\bs x^* \in \mscr K_{\mscr S}$, so dass
\begin{align}\label{eq:3.50}
	(A\bs x^* - \bs b)^T (\bs x - \bs x^*) \ge 0 \quad \forall \bs x\in \mscr K_{\mscr S} \, ,
\end{align}
wobei
\begin{align*}
	A &= \left[\int_\Omega \bs \eps (\bs \psi_j):\mcal C:\bs \eps(\bs \psi_i) \, d\Omega\right]_{1 \le i,j\le N} , \,  \bs b = \left[ \int_\Omega \bar{\bs b} \cdot \bs\psi_i \, d\Omega + \int_{\Gamma_N} \bar{\bs 	t} \cdot \bs\psi_i \, ds\right]_{1\le i \le N}
\end{align*}

\item Aus Bemerkung \ref{bem:3.12} folgt, dass die Variationsungleichung \eqref{eq:3.50} äquivalent zu folgendem quadratischen Programm ist:
\begin{align*}
\min_{\bs x\in \R^N} \frac 1 2 \bs x^T A \bs x - \bs b^T \bs x \ \text{ s.t. } \ B\bs x \ge \bs c \, ,
\end{align*}
d.h. Lösbarkeit des quadratischen Programms sollte auch gezeigt sein (vgl. Vug Skript oder auch nichtlineare Optimierung).
\end{itemize}


\newpage

%%% Local Variables: 
%%% mode: latex
%%% TeX-master: "Skript"
%%% End: 
