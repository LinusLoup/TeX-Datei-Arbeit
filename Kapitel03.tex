\newchapter{Ein hierarchischer Fehlerschätzer für Hindernisprobleme}

\begin{itemize}
\item Herleitung des Fehlerschätzers bei Hindernisproblemen (s. Mainpaper)
\item Vergleich Hindernisprobleme zu Kontaktproblemen $\ra$ warum gerade dieser Fehlerschätzer bei Hindernis- bzw. Kontaktproblemen
\end{itemize}

\section{Herleitung eines a posteriori hierarchischen Fehlerschätzers}

\subsection{Diskretisierung}

\subsection{Lokaler Anteil des Fehlerschätzers}

\subsection{Oszillationsterme}

\subsection{Zuverlässigkeit des Fehlerschätzers}

\subsection{Effektivität des Fehlerschätzers}

\section{Ein adaptiver Algorithmus}

\section{Erfüllung einer Saturationseigenschaft}

\newpage

%%% Local Variables: 
%%% mode: latex
%%% TeX-master: "Skript"
%%% End: 
