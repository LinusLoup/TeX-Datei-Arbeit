\newchapter{Grundlagen}
\label{sec:Grundlagen}

\begin{itemize}
\item FEM $\ra$ einleitend ansprechen, dass analytische nicht immer lösbar
\item Fehlerschätzer $\ra$ alle aufführen (s. Braess) $\ra$ damit verbundene adaptive Verfeinerungsstrategien (wie arbeitet Matlab mit Verfeinerung und welche Verfeinerungen gibt es?)
\item mathematisches Modell für Hindernisprobleme / Kontaktprobleme
\end{itemize}

\section{Variationsformulierung}

\section{Finite Elemente Methode}

\section{Adaptive Verfeinerungsstrategien}

\subsection{A posteriori Fehlerschätzer}

\section{Ein Hindernisproblem}

\subsection{Variationsformulierung für das Hindernisprobleme}

\subsection{Lösung des Hindernisproblems mittels FEM}

\section{Kontaktprobleme}

\subsection{Mathematische Modellierung von Kontaktproblemen}

\subsection{Variationsformulierung für Kontaktprobleme}

\subsection{Lösung des Kontaktproblems mittels FEM}

\newpage

%%% Local Variables: 
%%% mode: latex
%%% TeX-master: "Skript"
%%% End: 
