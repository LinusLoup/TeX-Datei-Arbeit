\newchapter{Grundlagen}
\label{sec:Grundlagen}

In diesem Kapitel wollen wir uns mit grundlegender Theorie beschäftigen, die nicht im Anhang aufgeführt ist, zum Verständnis von den darauffolgenden Kapiteln jedoch notwendig ist.

\section{Hilberträume}

\begin{itemize}
\item benötigen in den Variationsformulierungen immer wieder Hilberträume, daher werden Eigenschaften dieser hier nochmal eingeführt
\item \begin{defi}
Ein \textit{\idx{Hilbertraum}} ist ein reeller oder komplexer Vektorraum $H$ mit Skalarprodukt $(\cdot, \cdot)_H$, der vollständig bzgl. der durch das Skalarprodukt induzierten Norm, $\norm v_H^2 \coloneqq{(v,v)_H}$ für alle $v \in H$, ist, d.h. in dem jede Cauchy-Folge konvergiert.
\end{defi}

\item Es sei in diesem Kapitel $H$ ein reeller Hlbertraum mit Skalarprodukt $(\cdot,\cdot)_H$ und der dazu induzierten Norm $\norm v_H^2 = (v,v)_H$ für alle $v \in H$.

\item \begin{bem*}
Für alle $v,w \in H$ gilt die \idx{Cauchy-Schwarz'sche Ungleichung}
\[
	(v,w)_H \le \norm v_H \, \norm w_H \, .
\]
\end{bem*}

\item \begin{satz}[\idx{Approximationssatz}]
Es sei $\emptyset\neq M\subset H$ konvex und abgeschlossen. Dann existiert für alle $v\in H$ ein $m_v\in M$ mit
\[ 
  	\norm{v-m_v}=\dist(v,M)\coloneqq \inf_{w\in M}\norm{v-w}\, .
\]
Wir nennen $P_M:H\ra M$ mit $v\mapsto m_v$ die \idx{Projektionen} auf $M$.
\end{satz}

\begin{proof}
Der Beweis ist in \cite{Walker} Kapitel 7.1 Satz 7.2 zu finden.
\end{proof}

\item \begin{satz}[Charakterisierung der Projektionen]\label{satz:2.3}
$\emptyset\neq M\subset H$ sei abgeschlossen und konvex und $v\in H$. Dann gilt:
\[ 
  	m_0=P_M(v)\quad\Longleftrightarrow\quad (m-m_0, v-m_0)_H\leq0 
\]
für alle $m\in M$.
\end{satz}

\begin{proof}
 Es sei o.B.d.A. $0\in M$ und $m_0=0$.
 
  "`$\Rightarrow$"' Wegen $0=P_M(x)$ muss $\norm{v-tm}_H\geq\norm v_H$ für $m\in M$ und $0\leq t\leq1$ sein. Dann ist
\begin{align*}
    	 \norm v^2_H\leq\norm v^2_H-2t(v, m)_H+t^2\norm m^2_H
	\ \lra \ 0 \le - 2t (v,m)_H + \underbrace{t^2 \norm m_H^2}_{\ge 0} \, .
 \end{align*}
Damit ist $2(v, m)_H\leq0$.
 
 "`$\La$"' Für alle $m\in M$ ist $(v, m)_H\leq0$. Es folgt
\[ 
	\norm v^2_H\leq\norm v^2_H+\norm m^2-2(v,m)_H=\norm{v-m}^2_H\, . 
\]
Wegen $0\in M$ ist $\dist(v,M)=\norm v^2_H$ und damit $0=P_M(v)$.
\end{proof}

\item \begin{satz}\label{satz:2.4}
Es sei $\emptyset \not = M \subset H$ konvex und abgeschlossen. Dann gilt:
\[
	\norm{P_M(v)-P_M(w)}_H \le \norm{v-w}_H \quad \forall \, v,w \in H \, .
\]
\end{satz}

\begin{proof}
Da $P_M(v), P_M(w) \in M$ für alle $v,w \in H$ ist, folgt aus Satz \ref{satz:2.3}
\begin{align}\label{eq:2.1}
	(P_M(w)-P_M(v),v-P_M(v))_H  \le 0 \, , \\
	(P_M(v)-P_M(w),w-P_M(w))_H \le 0 \, .\label{eq:2.2}
\end{align}
Addieren wir \eqref{eq:2.1} und \eqref{eq:2.2}, so erhalten wir
\begin{align*}
	0 &\ge (P_M(w)-P_M(v),v-P_M(v))_H + (P_M(v)-P_M(w),w-P_M(w))_H \\
	& = (P_M(w)-P_M(v),v-w+P_M(w)-P_M(v))_H \\
	& = \norm{P_M(w)-P_M(v)}_H^2-(P_M(w)-P_M(v),w-v)_H \\
	&\stackrel{\scriptsize \text{CS}}\ge \norm{P_M(w)-P_M(v)}_H^2 - \norm{P_M(w)-P_M(v)}_H \, \norm{w-v}_H \, .
\end{align*}
Nach Umstellen der Ungleichung folgt die Behauptung.
\end{proof}

\item \begin{defi}
Es sei $\emptyset\neq M\subset H$ und wir definieren das \textit{{orthogonale Komplement}}\index{orthogonales~Komplement} von $M$ durch
\[
	M^\perp\coloneqq\{v\in H\with v\perp M\}\coloneqq\{v\in H\with (v, m)_H=0\;\fa\,  m\in M\}\, .
\]
\end{defi}

\item \begin{satz}\label{satz:2.6}
 Es sei $M$ ein abgeschlossener Untervektorraum von $H$. Dann ist
  \[
  	H=M\oplus M^\perp\, , 
  \]
  d.h. jedes $v\in M$ hat eine eindeutige Zerlegung $v=v_M+v_{M^\perp}$ mit $v_M\in M$ und $v_{M^\perp}\in M^\perp$.
\end{satz}

\begin{proof}
Der Beweis findet sich in \cite{Walker} Kapitel 7.1 Theorem 7.6.
\end{proof}

\item \begin{kor}\label{kor:2.6}
Es sei $\emptyset \not = M \subset H$   ein Untervektorraum. Dann ist $\bar M = H$ genau dann, wenn $M^\perp = \{0\}$ ist.
\end{kor}

\begin{proof}
Man kann zeigen, dass $\overline{\spn M} = (M^\perp)^\perp =: M^{\perp\perp}$ ist und dann unter Verwendung von Satz \ref{satz:2.6} die Behauptung folgern. Den kompletten Beweis können wir in \cite{Walker} Kapitel 7.1 Korollar 7.7 (iii) einsehen.
\end{proof}
\end{itemize}
 

\section{Variationsformulierung}


Stichpunkte für die Formulierung:

\begin{itemize}
\item Betrachte als Modellproblem Ausrenkung $u: \Omega \ra \R$ einer in $\Omega \subset \R^d$ eingespannten Membran unter Kraft $f$
\item mathematisch beschrieben wird dies durch das \textit{\idx{Dirichlet-Problem}}
\begin{align}\label{eq:2.1}
\begin{aligned}
	-\Delta u &= f \text{ in } \Omega \, ,\\
	u & = g \text{ auf } \partial \Omega \, ,
\end{aligned}
\end{align}
\item in der Praxis $d = 2,3$ übliche Dimensionen
\item Richtiger Punkt: \begin{notation} 
der Einfachheit halber sei im Folgenden $d = 2$ und $\Omega \subset \R^2$ ein durch ein Polygonzug berandetes Gebiet, den Rand $\partial \Omega$ bezeichnen wir mit $\Gamma$.
\end{notation}
\item allgemeiner berandete Gebiete können durch polygonale beliebig genau approximiert werden
\item Transformation: Sei $u_0: \Omega \ra \R$ eine zulässige Funktion, d.h. deren Regularität für \eqref{eq:2.1} ausreichend ist, und für die $u_0 = g$ auf $\Gamma$ gilt. Dann gilt für 
$\tilde u = u-u_0$
\begin{align}\label{eq:2.2}
\begin{aligned}
	-\Delta \tilde u &= \tilde f \text{ in } \Omega \, ,\\
	\tilde u & = 0 \text{ auf } \Gamma 
\end{aligned}
\end{align}
mit $\tilde f = f-\Delta u_0$.
\item $\Ra$ wir beschränken uns auf das \textit{homogene Dirichlet-Problem}\index{Dirichlet-Problem!homogenes} \eqref{eq:2.2}, d.h. sei $g\equiv 0$ in \eqref{eq:2.1}
\item Sei im Folgenden $H^1_0 (\Omega)$ wie in Bemerkung \ref{bem:A.8} der Raum der schwach differenzierbaren Funktionen, die am Rand $\Gamma$ verschwinden im Sinne der Spur.
\item für $v \in H^1_0(\Omega)$ gilt dann mit \eqref{eq:2.1}
\begin{align*}
	\int_\Omega -\Delta u \cdot v \, dx = \int_\Omega f v \, dx \, .
\end{align*}
Betrachte also \eqref{eq:2.1} im Mittel über das ganze Gebiet $\Omega$. Durch Anwenden der 1. Green'schen Formel (bzw. Satz von Gauß) ergibt sich
\begin{align}
\notag	& \int_\Omega \nabla u \cdot \nabla v \, dx -\underbrace{\int_\Gamma v \partial_\nu u \, ds}_{=0, \text{ da } v|_\Gamma = 0} = \int_\Omega f v \, dx \, \\
\label{eq:2.3}	\Llra & \quad \qquad \int_\Omega \nabla u \cdot \nabla v \, dx =\int_\Omega f v \, dx
\end{align}
\item kurz geschrieben ist \eqref{eq:2.3} mit der Notation aus Satz \ref{satz:A.5} (b) 
\[
	(\nabla u, \nabla v)_0 = (f,v)_0 \, .
\]
\item wir definieren die Bilinearform $a: (H^1_0(\Omega))^2 \ra \R, a(u,v) := (\nabla u, \nabla v)_0$ und $(f,v):=(f,v)_0$.
\begin{defi}
Eine Funktion $u \in H^1_0(\Omega)$ heißt \textit{\idx{schwache Lösung}} vom \idx{homogenen Dirichlet-Problem}
\begin{align}\label{eq:DP}\tag{DP}
\begin{aligned}
	-\Delta  u &=  f \text{ in } \Omega \, ,\\
	 u & = 0 \text{ auf } \Gamma \, ,
\end{aligned}
\end{align}
wenn die Gleichung
\begin{align}\label{eq:2.4}
	a(u,v) = (f,v)\quad \forall \, v \in H^1_0(\Omega) 
\end{align}
gilt.
\end{defi}
\item Wir betrachten im folgenden alle Hilberträume über $\R$.
\item Frage nach der Existenz und Eindeutigkeit einer schwachen Lösung für \eqref{eq:DP} $\Ra$ hierfür wird ein Hilbertraum benötigt (nachher im Beweis ersichtlich) $\ra$ Lösung liefert der Satz von Lax-Milgram.
\item zuvor noch eine Definition.
\begin{defi}
Sei $H$ ein Hilbertraum. Die Bilinearform  $a : H\times H \ra \R$ heißt \textit{stetig}\index{Bilinearform!stetig}, falls mit einem $c>0$
\[
	\abs{a(u,v)} \le c \, \norm{u}_H   \norm{v}_H \quad \forall \, u,v \in H
\]
gilt. Sie heißt $H$-\textit{elliptisch} (oder kurz \textit{elliptisch} oder \textit{koerziv})\index{Bilinearform!koerziv}\index{Bilinearform!elliptisch}, falls es ein $\alpha > 0$ gibt, so dass
\[
	a(v,v) \ge \alpha \, \norm{v}_H^2 \quad \forall \, v \in H 
\]
gilt.
\end{defi}

\item Bevor Existenz der Lösung gezeigt, betrachte Funktional $J(v) = \frac 1 2 a(v,v)-F(v)$ genauer

\item 
\begin{lemma}\label{lem:2.3}
Es sei $H$ ein Hilbertraum. Das Funktional
\[
	J: H \ra \R \, , \quad J(v) := \frac 1 2 a(v,v) - F(v) \, ,
\]
wobei $a: H\times H \ra \R$ eine stetige bilineare koerzive und $F: H\ra \R$ eine lineare Abbildung ist, ist konvex.
\end{lemma}

\begin{proof}
Es seien $u,v \in H$, dann gilt $u + t(v-u) = (1-t)u + tv \in H$ (dies gilt auch, wenn wir den Satz auf eine konvexe Teilmenge $M \subset H$ beschränken). Damit folgt mit $t \in [0,1]$
\begin{align*}
	J((1-t)u+tv)  = & \frac 1 2 a((1-t)u+tv,(1-t)u+tv) - F((1-t)u+tv) \\
	= &(1-t) \, J(u) + t \, J(v) +  \frac 1 2 a((1-t)u+tv,(1-t)u+tv) \\
	 & - \frac 1 2(1-t) \, a(u,u)-\frac 1 2 t \, a(v,v) \\
	= & (1-t) \, J(u) + t \, J(v)  + \frac 1 2 a(u,u) + t \, a(u,v-u)  \\
	&+ \frac {t^2} 2 a(v-u,v-u) - \frac 12 (1-t)\, a(u,u) -\frac 1 2 t \, a(v,v) \\
	= & (1-t) \, J(u) + t \, J(v) + \frac {t^2} 2 a(v-u,v-u)  \\
	 &\underbrace{+ t \, a(u,v)  - \frac 12 t\, a(u,u) -\frac 1 2 t \, a(v,v) }_{=  -\frac 1 2 t\, a(v-u,v-u)}\\
	= &  (1-t) \, J(u) + t \, J(v) - \frac {1} 2 \underbrace{t \, (1-t)}_{\ge 0} \,\underbrace{ a(v-u,v-u) }_{\ge \alpha  \norm{v-u}_H^2 \ge 0} \\
	\le &   (1-t) \, J(u) + t \, J(v) \, .% \qedhere
\end{align*}
Daraus folgt die Behauptung.
\end{proof}
\item \begin{lemma} \label{lem:2.4}
Sei $H$ ein Hilbertraum. Das Funktional $J: H \ra \R, J(v) =\frac 1 2 a(v,v)-F(v)$ aus Lemma \ref{lem:2.3} ist Gâteaux-differenzierbar $($s. Definition \ref{def:Gateaux-Ableitung}$)$.
\end{lemma}

\begin{proof}
Wir rechnen einfach nach, dass der Grenzwert des Differenzenquotienten existiert und verwenden dabei die Bilinearität von $a$ und Linearität von $F$. Seien $u,v \in H$, dann gilt
\begin{align*}
	\mscr D_v J(u) & = \lim_{t\ra 0} \frac{J(u+tv)-J(u)}t \\ 
	&= \lim_{t\ra 0} \frac{J(u) + t \, (a(u,v)-F(v)) + \frac {t^2}2 a(v,v)-J(u)}t \\
	& =  \lim_{t\ra 0}  (a(u,v)-F(v)) + \frac {t}2 a(v,v) \\
	& = a(u,v)-F(v) < \infty\, ,
\end{align*}
da $a$ und $F$ jeweils stetig sind und daher durch $\norm u_H,\norm v_H$ beschränkt sind. Damit folgt die Behauptung.
\end{proof}
\item 
\begin{theorem}\label{theorem:Lax-Milgram}\textnormal{(Lax-Milgram)}
Es sei $H$ ein Hilbertraum und  $a : H \times H \ra \R$ eine symmetrische, in $H$ stetige, koerzive Bilinearform. Weiter sei $F:H\ra \R$ ein stetiges lineares Funktional, d.h.
\[
	\abs{F(v)} \le c \, \norm{v}_H \quad \forall \, v \in H
\]
mit einer Konstante $c >0$. Dann gibt es eine eindeutige Lösung $u \in H$, für die
\[
	a(u,v) = F(v) \quad \forall \, v \in H \, .
\]
gilt. Diese minimiert den Ausdruck
\[
	J(v) = \frac 1 2 a(v,v) - F(v)
\]
unter allen $v \in H$.
\end{theorem}

\begin{proof}
(i) Zunächst zeigen wir die Äquivalenz der beiden oberen Probleme.

"`$\Ra$"' Es sei $u\in H$, so dass $a(u,v) = F(v) \, \forall \, v \in H$. Für $t>0$ und $v\in H$ gilt dann
\begin{align*}
	J(u+tv) & = \frac 1 2 a(u+tv,u+tv) -F(u+tv) \\
	& = \frac 1 2 a(u,u) + t \, a(u,v) + \frac {t^2} 2 a(v,v)-F(u)-t \, F(v) \\
	& = \frac 1 2 a(u,u)-F(u) + t\, (\underbrace{a(u,v)-F(v)}_{=0}) + \frac{t^2}2 \underbrace{a(v,v)}_{\parbox{1.2cm}{\scriptsize$\ge 0$, da $a$ koerziv}} \\
	& > \frac 1 2 a(u,u) - F(u) = J(u) \, ,
\end{align*}
also ist $u = \arg\min\limits_{v\in H} J(v)$.

"`$\La$"' Es sei $u \in H$ das Minimum von dem Problem
\[
	\min_{v\in H} J(v) = \frac 1 2 a(v,v) -F(v) \, .
\]
Da $J:H\ra \R$ nach Lemma \ref{lem:2.3} ein konvexes Funktional ist und $J$ nach Lemma \ref{lem:2.4} Gâteaux-differenzierbar, gilt mit Satz \ref{satz:A.10} für alle $v \in H$
\begin{align*}
	0& = \mscr D_vJ(u) = \frac d{dt} J(u+tv)\Big|_{t=0} \\
	& = \frac d{dt}(J(u) + t \, (a(u,v)-F(v))+\frac{t^2}2 a(v,v))\Big|_{t=0} \\
	& = a(u,v)-F(v) + t \, a(v,v) \Big|_{t=0} = a(u,v)-F(v)
\end{align*}
(ii) Eindeutigkeit: Es seien $u,\tilde u \in H$ Lösungen der Variationsungleichung, d.h.
\begin{align*}
	a(u,v) = F(v) \, \wedge \, a(\tilde u,v) = F(v) \quad \forall \, v \in H \, .
\end{align*}
Damit folgt durch Subtraktion der beiden Gleichungen für alle $v \in H$
\begin{align}\label{eq:2.5}
	a(u,v) = a(\tilde u,v) \Llra a(u-\tilde u,v) = 0 \, .
\end{align}
Da $H$ ein Vektorraum ist, gilt auch $u-\tilde u \in H$. Ersetzen wir also in \eqref{eq:2.5} $v = u-\tilde u$, dann ergibt sich
\begin{align*}
	&0 = a(u-\tilde u,u-\tilde u) \stackrel{\scriptsize a\text{ koerziv}}\ge \underbrace{\alpha}_{>0} \norm{u-\tilde u}_H^2 \ge 0 
	\lra\norm{u-\tilde u}_H^2 = 0 \, ,
\end{align*}
also folgt $u = \tilde u$.

(iii) Existenz: Die Existenz einer Lösung weisen wir über das Funktional nach.
\begin{align*}
	J(v) & = \frac 1 2 a(v,v)-F(v) \stackrel[\scriptsize F \text{ linear}]{\scriptsize a \text{ koerziv}}\ge \frac 1 2 \alpha \norm v_H^2 - c \norm v_H \\
	& = \frac 1 2 \alpha \(\norm v_H^2 - \frac{2c}\alpha \norm v_H\) = \frac 1 2 \alpha\(\norm v_H - \frac c\alpha\)^2 - \frac {c^2}{2\alpha} \\
	& \ge - \frac{c^2}{2\alpha}
\end{align*}
Folglich ist $J$ nach unten beschränkt. Sei $\eta := \inf \{J(v)\with v \in H\}$ und $(v_n)_{n\in\N}$ eine Folge mit $J(v_n) \ra\eta$ für $n\ra \infty$. Dann folgt mit der Koerzivität von $a$
\begin{align*}
	\alpha \norm{v_n-v_m}^2_H  \le & a(v_n-v_m,v_n-v_m) \\
	 = &a(v_n,v_n)+a(v_m,v_m)-a(v_n,v_m)-a(v_m,v_n) \\
	=& 2a(v_n,v_n)+2a(v_m,v_m) \underbrace{-a(v_n,v_n+v_m)-a(v_m,v_n+v_m)}_{=-a(v_n+v_m,v_n+v_m)} \\
	=& 2a(v_n,v_n)-4F(v_n)+2a(v_m,v_m)-4F(v_m) \\
	& -a(v_n+v_m,v_n+v_m)+4F(v_n+v_m) \\
	= & 4 J(v_n) + 4J(v_m) - 4 a\(\frac{v_n+v_m}2,\frac{v_n+v_m}2\)+8F\(\frac{v_n+v_m}2\) \\
	= & 4 J(v_n) + 4J(v_m) - 8 J\(\frac{v_n+v_m}2\) \\
	\le &4 J(v_n) + 4J(v_m) - 8\eta  \xrightarrow[n,m\ra\infty]{} 4\eta+4\eta-8\eta = 0 \, ,
\end{align*}
d.h. $(v_n)_{n\in \N}$ ist eine Cauchy-Folge. Da $H$ ein Hilbertraum ist, gilt somit: $\exists \, u \in H : v_n \xrightarrow[n\ra \infty]{} u$ mit $J(u) = \eta$.
\end{proof}

\item
\begin{satz}\textnormal{(\idx{Poincaré-Friedrich-Ungleichung})}\label{satz:2.13}
Es sei $\Omega$ in einem $d$-dimensionalen Würfel der Kantenlänge $s>0$ enthalten. Dann gilt
\[
	\norm v_0 \le s \norm{\nabla v}_0 \quad \forall \, v \in H^1_0(\Omega) \, ,
\]
wobei $\norm\cdot_0$ die durch das Skalarprodukt $(\cdot,\cdot)_0$ induzierte Norm ist.
\end{satz}

\begin{proof}
Der Beweis ist in \cite{BraeFEM} Kapitel II, \S1 Sobolev-Räume, Satz 1.5 oder \cite{StarkePDE} Satz 1.5 zu finden.
\end{proof}

\item Greifen wieder die Frage auf, ob das Problem \eqref{eq:2.4} mit $a:(H^1_0(\Omega))^2\ra \R, a(u,v) = (\nabla u,\nabla v)_0$ und $F:H^1_0(\Omega) \ra \R, F(v) := (f,v)$ eine eindeutige Lösung hat.
\item Kann nun mit Theorem \ref{theorem:Lax-Milgram} beantwortet werden. Es seien $u,v \in H^1_0(\Omega)$, dann gilt
\begin{align*}
	 a(v,v) = & \int_\Omega \nabla v \nabla v \, dx = \norm{\nabla v}_0^2  \\
	\ge& \frac{s^2+1}{(1+s)^2}\norm{\nabla v}_0^2  \stackrel{\scriptsize \text{Satz \ref{satz:2.13}}}\ge \frac 1{(1+s)^2} (\norm v_0^2 + \norm{\nabla v}_0^2) \\
	= & \frac 1{(1+s)^2} \norm v_1^2 \, .
\end{align*}
Damit ist $a$ mit $\alpha :=  \frac 1{(1+s)^2}$ koerziv. Weiter rechnen wir nach:
\begin{align*}
	\abs{a(u,v)} = & \Abs{\int_\Omega \nabla u \nabla v \, dx} \le \sum_{i = 1}^d \int_\Omega\abs{\partial_i u}\abs{\partial_i v} \, dx \\
	\stackrel{\scriptsize\text{CS}}\le & \sum_{i = 1}^d \(\int_\Omega \abs{\partial_i u}^2 \, dx\)^{\frac 12} \(\int_\Omega \abs{\partial_i v}^2 \, dx\)^{\frac 12} \\
	\le & \(\sum_{i = 1}^d \int_\Omega \abs{\partial_i u}^2 \, dx\)^{\frac 12} \(\sum_{i=1}^d\int_\Omega \abs{\partial_i v}^2 \, dx\)^{\frac 12} \\
	\le & \( \int_\Omega \abs{\nabla u}^2 \, dx + \int_\Omega u^2 \, dx\)^{\frac 12} \(\int_\Omega \abs{\nabla v}^2 \, dx+\int_\Omega v^2\, dx\)^{\frac 12} \\
	= & \norm u_1 \, \norm v_1 \, ,
\end{align*}
d.h.  $a$ ist stetig mit $c := 1$. Die Symmetrie von $a$ ist trivial, also bleibt nur noch die Stetigkeit von $F$ zu zeigen. Es sei $v \in H^1_0(\Omega)$, dann gilt
\begin{align*}
	\abs{F(v)} &= \abs{(f,v)} =  \Abs{\int_\Omega fv \, dx} \stackrel{\scriptsize\text{CS}}\le\( \int_\Omega \abs{f}^2 \,dx\)^{\frac 12} \( \int_\Omega \abs v^2 \, dx\)^{\frac 12} \\
	&\le  c \, \( \int_\Omega \abs{\nabla v}^2 +  \abs v^2 \, dx\)^{\frac 12} = c \, \norm v_1
\end{align*}
mit $0<c := \int_\Omega \abs f^2 \, dx < \infty$, wenn $f \in L_2(\Omega)$ ist. Damit ist $F$ ein stetiges lineares Funktional und somit existiert nach Theorem \ref{theorem:Lax-Milgram} eine eindeutige Lösung $u \in H^1_0(\Omega)$ für die schwache Formulierung des homogenen Dirichlet-Problems. 

\item Weiter minimiert die Lösung $u \in H^1_0(\Omega)$ das Funktional
\begin{align*}
	J(v) = \frac 12 \int_\Omega \nabla v\nabla v \, dx - \int_\Omega fv \, dx \, ,
\end{align*}
welches die gespeicherte Energie der durch die Kraft $f$ belasteten Membran $\Omega$ beschreibt.

\item \begin{bem*}
Die Stetigkeit vom Funktional $F$ zeigt, welche Eigenschaft die Kraft $f$ aus dem Dirichlet-Problem wenigstens quadratisch integrierbar sein muss, damit es eine schwache Lösung geben kann.
\end{bem*}

\item \begin{bem*}
\begin{enumerate}[(a)]
\item Mit $H'$ bezeichnen wir den Dualraum zu einem Hilbertraum $H$.
\item Den Dualraum zu $H^1(\Omega)$ bezeichnen wir mit $H^{-1}(\Omega)$.
\end{enumerate}
\end{bem*}
\item Hier noch eine Folgerung aus dem Satz von Lax-Milgram:
\begin{satz}\textnormal{(Riesz'scher Darstellungssatz)}
Es sei $H$ ein Hilbertraum mit einem Skalarprodukt $(\cdot,\cdot)_H$. Es sei $F \in H'$, dann existiert genau ein $u \in H$, so dass
\[
	(u,v)_H = F(v) \quad \forall \, v \in H \, .
\]
\end{satz}

\begin{proof}
Dies ist eine direkte Folgerung aus dem Theorem \ref{theorem:Lax-Milgram}. Die Abbildung $(\cdot,\cdot)_H:H\times H \ra \R$ ist als Skalarprodukt bilinear, symmetrisch und positiv definit, damit auch bzgl. der auf $H$ durch das Skalarprodukt induzierten Norm $\norm v_H := \sqrt{(v,v)_H}$, koerziv. $F$ ist als Element des Dualraumes $H'$ eine lineare stetige Abbildung $F:H\ra\R$ und damit folgt mit $a(\cdot,\cdot) :=(\cdot,\cdot)_H$ aus dem Theorem von Lax-Milgram die Behauptung.
\end{proof}

\item \begin{kor}\label{kor:2.14}
Es sei $H$ ein Hilbertraum mit Skalarprodukt $(\cdot,\cdot)_H$ und $a:H\times H\ra \R$ eine stetige koerzive Bilinearform. Dann existiert genau ein linearer Operator $A : H \ra H$, so dass gilt:
\[
	a(u,v)  = (Au,v)_H \quad \forall \, u , v \in H \, .
\]
\end{kor}

\begin{proof}
Es sei $u\in H$ fest, dann ist $L: H\ra \R, L(v) := a(u,v)$ eine lineare Abbildung, die stetig ist, da
\[
	\abs{L(v)} = \abs{a(u,v)} \stackrel{\scriptsize\text{stetig}}\le c \,  \norm{u}_H \norm v_H  = \tilde c\, \norm v_H
\]
mit $0<\tilde c := c \, \norm u_H$ gilt. Damit folgt nach dem Darstellungssatz von Riesz, dass es ein eindeutiges $l \in H$ gibt, so dass
\[
	a(u,v) = L(v) = (l,v)_H \quad \forall \, v \in H
\]
gilt. Da $u \in H$ jedoch beliebig ist, bleibt zu zeigen, dass es ein eindeutiges $A:H\ra H$ gibt, so dass $Au = l$ ist.

Wir zeigen zunächst mithilfe der Bilinearform $a$, dass $A$ linear ist. Es gilt für $\lambda,\mu \in \R$ und $u,v \in H$
\begin{align*}
	(A(\lambda u + \mu v),w)_H &= a(\lambda u + \mu v, w) = \lambda a(u,w) + \mu a(v,w) \\
	& = \lambda (Au,w)_H + \mu(Av,w)_H \\
	& = (\lambda \, Au+\mu \, Av,w)_H
\end{align*}
für alle $w \in H$. Weiter gilt
\[
	\norm{Au}_H^2 = (Au,Au)_H = a(u,Au) \stackrel{\scriptsize \text{stetig}}\le c \, \norm u_H \norm{Au}_H \, ,
\]
d.h. $\norm{Au}_H \le c \, \norm u_H$ und damit ist nach \cite{Werner} Satz II.1.2 der Operator $A$ stetig.

Betrachten wir den Kern von $A$, so ergibt sich
\begin{align}\label{eq:2.6}
	\ker A := \{ v \in H \with Av = 0\} = \{0\} \, ,
\end{align}
denn
\[
	\alpha \norm v_H^2 \stackrel{\scriptsize \text{koerziv}}\le a(v,v)  = (Av,v)_H \stackrel{\scriptsize \text{CS}}\le \norm{Av}_H \norm v_H 
\]
und damit gilt $\norm{Av}_H \ge \alpha \norm v_H$, d.h. $Av = 0 \Lra v = 0$. Dies impliziert, dass $A$ injektiv ist, denn mit $v_1,v_2\in H, Av_1 = Av_2$ folgt
\[
	0 = Av_1 - Av_2 = A(v_1-v_2) \ \stackrel{\scriptsize\eqref{eq:2.6}}\lra \ v_1 = v_2 \, .
\]
Weiter betrachten wir das Bild von $A$, d.h.
\[
	\im A := \{ v \in H \with \exists \, u \in H : Au = v \} \subset H\, .
\]
Sei $(v_n)_{n \in \N}$ eine Folge mit $v_k \in \im A $ für alle $k \in \N$. Dann folgt, dass für jedes $v_k$ ein $u_k \in H$ existiert mit $A u_k = v_k$. Es gelte, dass $Au_k = v_k \ra v \in H$ geht, dann folgt
\begin{align*}
	\alpha \, \norm{u_n -u_m}_H & \le \norm{A(u_n-u_m)}_H = \norm{Au_n-Au_m}_H \\
	& = \norm{v_n-v_m}_H \xrightarrow[n,m\ra \infty]{} 0 \, ,
\end{align*}
d.h. $(u_n)_{n\in \N} \subset H$ ist eine Cauchy-Folge und konvergiert daher in $H$. Also existiert ein $u \in H$ mit $u_n \ra u$. Mit der Stetigkeit von $A$ folgt dann
\[
	v_n = A u_n \xrightarrow[n\ra \infty]{} Au = v\, ,
\]
d.h. $v \in \im A$ und damit ist $\im A$ abgeschlossen. Wir betrachten nun ein $v \in H$ mit $v \perp \im A \subset H$, dann gilt
\[
	(Au,v)_H = 0 \quad \forall \, u \in H \, .
\]
Damit folgt mit $u = v \in H$ oben eingesetzt
\[
	 0 = (Av,v)_H = a(v,v) \ge \alpha \, \norm v_H^2 \, \lra \, v = 0 \, .
\]
Also besteht der zu $\im A$ orthogonale Raum nur aus dem Nullelement und mit Korollar \ref{kor:2.6} gilt dann $\im A = \overline{\im A} = H$. Damit ist $A$ bijektiv.

Es seien nun $0 \not = l\in H$ sowie $A_1,A_2 \in \mcal L(H,H)$ zwei lineare Operatoren mit $A_1u = l$ und $A_2 u = l$, die nach der obigen Weise konstruiert sind. Dann gilt
\[
	0 = A_1 u - A_2 u = (A_1 - A_2)u \, \lra \, A_1 = A_2 \, , 
\]
da $u \not = 0$ und die Summe zweier bijektiver linearer Operatoren wieder bijektiv ist, also ist ein so konstruierter Operator eindeutig.
\end{proof}
\end{itemize}






\section{Finite Elemente Methode}

\begin{itemize}
\item FEM $\ra$ einleitend ansprechen, dass analytische nicht immer lösbar
\item Der für uns verwendete Finite Element Raum wird eingeführt (lineare Funktionen).
\item Was ist eine Triangulierung (vgl. Braess auf Seite 58)?
\item local-global node ordering zur Effizienzsteigerung

\item Unter FEM verstehen wir das Galerkin-Verfahren
\item Galerkin-Verfahren bedeutet, wir wollen die Variationsgleichung
\[
	a(u,v) = F(v) \quad \forall \, v \in H
\]
auf einem endlich dimensionalen Unterraum $V_h \subset H$ lösen, d.h. finde $u_h \in V_h$, so dass
\[
	a(u_h,v_h) = F(v_h) \quad \forall \, v_h \in V_h \, .
\]
\item \begin{satz}
Das "`Galerkin-Problem"' hat eine eindeutige Lösung.
\end{satz}

da $V_h$ als Unterraum von $H$ auch ein Hilbertraum ist und die Eigenschaften von $a, F$ weiterhin erfüllt sind, gilt auch hier der Satz von Lax-Milgram, was die Eindeutigkeit und Existenz einer Lösung sichert.

\item 
\end{itemize}

\section{Adaptive Verfeinerungsstrategien}

\subsection{A posteriori Fehlerschätzer}

\begin{itemize}
\item Fehlerschätzer $\ra$ alle aufführen (s. Braess) $\ra$ damit verbundene adaptive Verfeinerungsstrategien (wie arbeitet Matlab mit Verfeinerung und welche Verfeinerungen gibt es?)
\end{itemize}

\section{Einführung in die Strukturmechanik}

\begin{itemize}
\item Beschreibung der Kinematik: Referenz- bzw. Ausgangskonfiguration, Deformationsgradient, Verzerrungsmaße (Konti-Buch)
\item Lineararisierung der Verzerrungsmaße für unseren Fall (kleine Deformationen) mittels "Taylor" (siehe auch Gateaux-Ableitung - Seite 24 Konti Skript):
\[
	\bs \eps = \frac 1 2 (\nabla \bs u + \nabla^T \bs u)
\]
\item Kinetik: Kräftegleichgewicht und äußere Kontaktlast
\item Konzepte für ebene Spannungs- bzw. Verzerrungszustände
\item Konstitutive Modelle (vor allem Materialgesetze) $\Ra$ Hier vor allem Hooke:
\[
	\bs \sigma = \mcal C \bs \eps = 2 \mu \bs \eps + \lambda (\tr \bs \eps) \bs I \, ,
\]
wobei $\lambda,\mu$ die Lamé-Konstanten sind (Materialabhängige Parameter). $\Ra$ Hier noch mal den Zusammengang von Konstanten zu $E,\nu$ aufzeigen.
\item falls Tensorrechnungen konkret benötigt werden, können diese im Anhang dargelegt werden
\end{itemize}


\newpage

%%% Local Variables: 
%%% mode: latex
%%% TeX-master: "Skript"
%%% End: 
