\newchapter{Grundlagen}
\label{sec:Grundlagen}

In diesem Kapitel wollen wir uns mit grundlegender Theorie beschäftigen, die nicht im Anhang aufgeführt ist, zum Verständnis von den darauffolgenden Kapiteln jedoch notwendig ist.

\section{Variationsformulierung}

\begin{itemize}
\item Was ist eine schwache Form einer PDE? Am Standardbeispiel $\Delta u = f$ in $\Omega, u = g$ auf $\partial \Omega$. (Herleitung auch über das Funktional $\ra$ auch für später beim Hindernisproblem wichtig)
\item Hierbei ist $u = 0$ als Nebenbedingung im weiteren gewählt, da man dieses Problem durch Wahl der richtigen Lösung auf $u = g$ (vgl. \cite{BraeFEM} S. 35). Analog kann dies später auch an der starken Formulierung für das Hindernisproblem demonstriert werden, weshalb im folgenden immer $u \in H^1_0(\Omega)$ verwendet wird.
\item Warum gibt es eine Lösung? (Lax-Milgram $\ra$ auch Riesz aufführen, da in dem Beweis der Existenz und Eindeutigkeit von $a(u,v-u)\ge f(v-u)$ erwähnt wird)
\end{itemize}

Stichpunkte für die Formulierung:

\begin{itemize}
\item Betrachte als Modellproblem Ausrenkung $u: \Omega \ra \R$ einer in $\Omega \subset \R^d$ eingespannten Membran unter Kraft $f$
\item mathematisch beschrieben wird dies durch das \textit{\idx{Dirichlet-Problem}}
\begin{align}\label{eq:2.1}
\begin{aligned}
	-\Delta u &= f \text{ in } \Omega \, ,\\
	u & = g \text{ auf } \partial \Omega \, ,
\end{aligned}
\end{align}
\item in der Praxis $d = 2,3$ übliche Dimensionen
\item Richtiger Punkt: \begin{notation} 
der Einfachheit halber sei im Folgenden $d = 2$ und $\Omega \subset \R^2$ ein durch ein Polygonzug berandetes Gebiet, den Rand $\partial \Omega$ bezeichnen wir mit $\Gamma$.
\end{notation}
\item allgemeiner berandete Gebiete können durch polygonale beliebig genau approximiert werden
\item Transformation: Sei $u_0: \Omega \ra \R$ eine zulässige Funktion, d.h. deren Regularität für \eqref{eq:2.1} ausreichend ist, und für die $u_0 = g$ auf $\Gamma$ gilt. Dann gilt für 
$\tilde u = u-u_0$
\begin{align}\label{eq:2.2}
\begin{aligned}
	-\Delta \tilde u &= \tilde f \text{ in } \Omega \, ,\\
	\tilde u & = 0 \text{ auf } \Gamma 
\end{aligned}
\end{align}
mit $\tilde f = f-\Delta u_0$.
\item $\Ra$ wir beschränken uns auf das \textit{homogene Dirichlet-Problem}\index{Dirichlet-Problem!homogenes} \eqref{eq:2.2}, d.h. sei $g\equiv 0$ in \eqref{eq:2.1}
\item Sei im Folgenden $H^1_0 (\Omega)$ wie in Bemerkung \ref{bem:A.8} der Raum der schwach differenzierbaren Funktionen, die am Rand $\Gamma$ verschwinden im Sinne der Spur.
\item für $v \in H^1_0(\Omega)$ gilt dann mit \eqref{eq:2.1}
\begin{align*}
	\int_\Omega -\Delta u \cdot v \, dx = \int_\Omega f v \, dx \, .
\end{align*}
Betrachte also \eqref{eq:2.1} im Mittel über das ganze Gebiet $\Omega$. Durch Anwenden der 1. Green'schen Formel (bzw. Satz von Gauß) ergibt sich
\begin{align}
\notag	& \int_\Omega \nabla u \cdot \nabla v \, dx -\underbrace{\int_\Gamma v \partial_\nu u \, ds}_{=0, \text{ da } v|_\Gamma = 0} = \int_\Omega f v \, dx \, \\
\label{eq:2.3}	\Llra & \qquad \int_\Omega \nabla u \cdot \nabla v \, dx =\int_\Omega f v \, dx
\end{align}
\end{itemize}



\section{Finite Elemente Methode}

\begin{itemize}
\item FEM $\ra$ einleitend ansprechen, dass analytische nicht immer lösbar
\item Was ist Galerkin-Approximation und warum gibt es eine Lösung (hier ist Lax-Milgram auch anwendbar (warum?))
\item Der für uns verwendete Finite Element Raum wird eingeführt (lineare Funktionen).
\item Was ist eine Triangulierung (vgl. Braess auf Seite 58)?
\item local-global node ordering zur Effizienzsteigerung
\end{itemize}

\section{Adaptive Verfeinerungsstrategien}

\subsection{A posteriori Fehlerschätzer}

\begin{itemize}
\item Fehlerschätzer $\ra$ alle aufführen (s. Braess) $\ra$ damit verbundene adaptive Verfeinerungsstrategien (wie arbeitet Matlab mit Verfeinerung und welche Verfeinerungen gibt es?)
\end{itemize}

\section{Einführung in die Strukturmechanik}

\begin{itemize}
\item Beschreibung der Kinematik: Referenz- bzw. Ausgangskonfiguration, Deformationsgradient, Verzerrungsmaße (Konti-Buch)
\item Lineararisierung der Verzerrungsmaße für unseren Fall (kleine Deformationen) mittels "Taylor" (siehe auch Gateaux-Ableitung - Seite 24 Konti Skript):
\[
	\bs \eps = \frac 1 2 (\nabla \bs u + \nabla^T \bs u)
\]
\item Kinetik: Kräftegleichgewicht und äußere Kontaktlast
\item Konzepte für ebene Spannungs- bzw. Verzerrungszustände
\item Konstitutive Modelle (vor allem Materialgesetze) $\Ra$ Hier vor allem Hooke:
\[
	\bs \sigma = \mcal C \bs \eps = 2 \mu \bs \eps + \lambda (\tr \bs \eps) \bs I \, ,
\]
wobei $\lambda,\mu$ die Lamé-Konstanten sind (Materialabhängige Parameter). $\Ra$ Hier noch mal den Zusammengang von Konstanten zu $E,\nu$ aufzeigen.
\item falls Tensorrechnungen konkret benötigt werden, können diese im Anhang dargelegt werden
\end{itemize}


\newpage

%%% Local Variables: 
%%% mode: latex
%%% TeX-master: "Skript"
%%% End: 
