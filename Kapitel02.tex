\newchapter{Grundlagen}
\label{kap:2}

In diesem Kapitel wollen wir uns mit grundlegender Theorie beschäftigen, die nicht im Anhang aufgeführt, zum Verständnis von den darauffolgenden Kapiteln jedoch notwendig ist.

Dieses Kapitel basiert auf \cite{BraeFEM}, \cite{StarkePDE}, \cite{EPS}, \cite{Walker}, \cite{AltKonti}.


\section{Hilberträume}
\label{kap:2.1}


Wir benötigen für die Variationsrechnung Hilberträume und wollen uns daher in diesem Kapitel mit wichtigen Eigenschaften solcher Räume im Allgemeinen beschäftigen. Zunächst führen wir ein, was wir unter einem Hilbertraum verstehen.


\begin{defi}\label{def:2.1}
Ein \textit{\idx{Hilbertraum}} ist ein reeller oder komplexer Vektorraum $H$ mit Skalarprodukt $(\cdot, \cdot)_H$, der vollständig bzgl. der durch das Skalarprodukt induzierten Norm, $\norm v_H^2 \coloneqq{(v,v)_H}$ für alle $v \in H$, ist, d.h. in dem jede Cauchy-Folge konvergiert.
\end{defi}


\begin{bsp*}
Es sei $H = \R^n$ und $(\cdot,\cdot)_H : \R^n\times \R^n \ra \R$ definiert durch das Standartskalarprodukt. Dann konvergiert jede \idx{Cauchy-Folge} in $H$ bzgl. der durch $(\cdot,\cdot)_H$ induzierten (euklidischen) Norm (vgl. \cite{Ana2}, metrische Räume) und damit ist $H$ ein Hilbertraum.
\end{bsp*}


Wir wollen die im Folgenden aufgeführten Eigenschaften später auf weniger triviale Räume anwenden, vor allem den Funktionenraum $H^1_0(\Omega)$ (s. Anhang \ref{anhang:A} Sobolev-Räume\index{Sobolev-Raum}). Um alle Aussagen auch allgemein verwenden zu können, sei in diesem Kapitel $H$ ein reeller Hlbertraum mit Skalarprodukt $(\cdot,\cdot)_H$ und der dazu induzierten Norm $\norm v_H^2 = (v,v)_H$ für alle $v \in H$.


\begin{bem*}
Für alle $v,w \in H$ gilt die \idx{Cauchy-Schwarz'sche Ungleichung}
\[
	(v,w)_H \le \norm v_H \, \norm w_H \, .
\]
\end{bem*}


Da wir uns in dieser Arbeit mit Variationsproblemen über konvexen abgeschlossenen Mengen beschäftigen werden, sammeln wir zunächst einige Aussagen bzgl. dieser Mengen.


\begin{satz}[\idx{Approximationssatz}]\label{satz:2.2}
Es sei $\emptyset\neq M\subset H$ konvex und abgeschlossen. Dann existiert für alle $v\in H$ ein $m_v\in M$ mit
\[ 
  	\norm{v-m_v}=\dist(v,M)\coloneqq \inf_{w\in M}\norm{v-w}\, .
\]
Wir nennen $P_M:H\ra M$ mit $v\mapsto m_v$ die \idx{Projektionen} auf $M$.
\end{satz}

\begin{proof}
Der Beweis ist in \cite{Walker} Kapitel 7.1 Satz 7.2 zu finden.
\end{proof}


\begin{satz}[Charakterisierung der Projektionen]\label{satz:2.3}
$\emptyset\neq M\subset H$ sei abgeschlossen und konvex und $v\in H$. Dann gilt:
\[ 
  	m_0=P_M(v)\quad\Longleftrightarrow\quad (m-m_0, v-m_0)_H\leq0 
\]
für alle $m\in M$.
\end{satz}

\begin{proof}
 Es sei o.B.d.A. $0\in M$ und $m_0=0$.
 
  "`$\Rightarrow$"' Wegen $0=P_M(x)$ muss $\norm{v-tm}_H\geq\norm v_H$ für $m\in M$ und $0\leq t\leq1$ sein. Dann ist
\begin{align*}
    	 \norm v^2_H\leq\norm v^2_H-2t(v, m)_H+t^2\norm m^2_H
	\ \lra \ 0 \le - 2t (v,m)_H + \underbrace{t^2 \norm m_H^2}_{\ge 0} \, .
 \end{align*}
Damit ist $2(v, m)_H\leq0$.
 
 "`$\La$"' Für alle $m\in M$ ist $(v, m)_H\leq0$. Es folgt
\[ 
	\norm v^2_H\leq\norm v^2_H+\norm m^2-2(v,m)_H=\norm{v-m}^2_H\, . 
\]
Wegen $0\in M$ ist $\dist(v,M)=\norm v^2_H$ und damit $0=P_M(v)$.
\end{proof}


\begin{satz}\label{satz:2.4}
Es sei $\emptyset \not = M \subset H$ konvex und abgeschlossen. Dann gilt:
\[
	\norm{P_M(v)-P_M(w)}_H \le \norm{v-w}_H \quad \forall \, v,w \in H \, .
\]
\end{satz}

\begin{proof}
Da $P_M(v), P_M(w) \in M$ für alle $v,w \in H$ ist, folgt aus Satz \ref{satz:2.3}
\begin{align}\label{eq:2.1}
	(P_M(w)-P_M(v),v-P_M(v))_H  \le 0 \, , \\
	(P_M(v)-P_M(w),w-P_M(w))_H \le 0 \, .\label{eq:2.2}
\end{align}
Addieren wir \eqref{eq:2.1} und \eqref{eq:2.2}, so erhalten wir
\begin{align*}
	0 &\ge (P_M(w)-P_M(v),v-P_M(v))_H + (P_M(v)-P_M(w),w-P_M(w))_H \\
	& = (P_M(w)-P_M(v),v-w+P_M(w)-P_M(v))_H \\
	& = \norm{P_M(w)-P_M(v)}_H^2-(P_M(w)-P_M(v),w-v)_H \\
	&\!\!\!\: \stackrel{\scriptsize \text{C.S.}}\ge \norm{P_M(w)-P_M(v)}_H^2 - \norm{P_M(w)-P_M(v)}_H \, \norm{w-v}_H \, .
\end{align*}
Nach Umstellen der Ungleichung folgt die Behauptung.
\end{proof}


\begin{defi}\label{def:2.5}
Es sei $\emptyset\neq M\subset H$ und wir definieren das \textit{{orthogonale Komplement}}\index{orthogonales~Komplement} von $M$ durch
\[
	M^\perp\coloneqq\{v\in H\mid v\perp M\}\coloneqq\{v\in H\mid (v, m)_H=0\;\fa\,  m\in M\}\, .
\]
\end{defi}


\begin{satz}\label{satz:2.6}
 Es sei $M$ ein abgeschlossener Untervektorraum von $H$. Dann ist
  \[
  	H=M\oplus M^\perp\, , 
  \]
  d.h. jedes $v\in M$ hat eine eindeutige Zerlegung $v=v_M+v_{M^\perp}$ mit $v_M\in M$ und $v_{M^\perp}\in M^\perp$.
\end{satz}

\begin{proof}
Der Beweis findet sich in \cite{Walker} Kapitel 7.1 Theorem 7.6.
\end{proof}


\begin{kor}\label{kor:2.6}
Es sei $\emptyset \not = M \subset H$   ein Untervektorraum. Dann ist $\bar M = H$ genau dann, wenn $M^\perp = \{0\}$ ist.
\end{kor}

\begin{proof}
Man kann zeigen, dass $\overline{\spn M} = (M^\perp)^\perp =: M^{\perp\perp}$ ist und dann unter Verwendung von Satz \ref{satz:2.6} die Behauptung folgern. Den kompletten Beweis können wir in \cite{Walker} Kapitel 7.1 Korollar 7.7 (iii) einsehen.
\end{proof}



\section{Variationsformulierung}
\label{kap:2.2}


Bevor wir uns mit Variationsproblemen auf konvexen Teilmengen eines Hilbertraumes beschäftigen, wollen wir die Variationsrechnung an einem einfachen Modellproblem ohne Nebenbedingung beschreiben.

      \begin{figure}[ht!]
        \centering
        \begin{pspicture}(-3,-2.7)(3,2.6)
          \psset{Beta=15}
          
          % Membran
          \pstThreeDCircle[fillstyle=shape,fillcolor=lightgray](0,0,0)(2,2,0)(-2,2,0)
          \pstThreeDEllipse[beginAngle=0,endAngle=180](0,0,0)(2,-2,0)(0,0,-2)
          \pstThreeDLine[arrows=|-|](2.5,-2.5,0)(2.5,-2.5,-2)
          \pstThreeDPut[origin=rt](2.9,-2.9,-1){\large$u(x)$}
          \pstThreeDPut(-1.8,1.2,0){\large$\Omega$}

          % Kraft
          \pstThreeDLine[arrows=->,arrowscale=3](0,0,2)(0,0,0)
          \pstThreeDPut[origin=lt](-0.2,0.1,1.3){\large$f$}
        \end{pspicture}
        \caption{Membran $\Omega$ mit Flächenlast $f$ und Auslenkung $u(x)$}
      \end{figure}




Wir betrachten als Modellproblem die Auslenkung $u: \Omega \ra \R$ einer in $\Omega \subset \R^d$ eingespannten Membran unter Krafteinwirkung $f$. Mathematisch beschrieben wird dies durch das  \textit{\idx{Dirichlet-Problem}}
\begin{subequations}\label{eq:2.1a}
\begin{align}\label{eq:2.1aa}
%\begin{aligned}
	-\Delta u &= f \text{ in } \Omega \, ,\\
	\label{eq:2.1ab}
	u & = g \text{ auf } \partial \Omega \, ,
%\end{aligned}
\end{align}
\end{subequations}
dabei ist $g: \partial \Omega \ra \R$ eine für die Randwerte von $u$ gegebene Funktion.


\begin{notation} 
In der Praxis übliche Dimensionen sind $d = 2,3$. Der Einfachheit halber sei im Folgenden $d = 2$ und $\Omega \subset \R^2$ ein durch ein Polygonzug berandetes Gebiet, den Rand $\partial \Omega$ bezeichnen wir auch mit $\Gamma$.
\end{notation}


\begin{bem*}
Sollte $\Omega$ ein allgemeiner berandetes Gebiet sein, so können wir dieses beliebig genau durch ein polygonales Gebiet approximieren; hierbei entsteht schon bei der Gebietszerlegung ein Fehler.

Diesen Fehler kann man durch Verwendung von \textit{isoparametrischen Elementen}\index{isoparametrisches Element} (vgl. \cite{BraeFEM} Kapitel III, \S2, Isoparametrische Elemente) verringern. Dies soll in dieser Arbeit aber nicht weiter vertieft werden.
\end{bem*}


Es sei $u_0: \Omega \ra \R$ eine für das \idx{Dirichlet-Problem} zulässige Funktion, d.h. die für   \eqref{eq:2.1a} hinreichend regulär ist und für die $u_0 = g$ auf $\Gamma$ gilt. Dann gilt für 
$\tilde u = u-u_0$
\begin{subequations}\label{eq:2.2a}
\begin{align}\label{eq:2.2aa}
	-\Delta \tilde u &= \tilde f \text{ in } \Omega \, ,\\
	\label{eq:2.2ab}
	\tilde u & = 0 \text{ auf } \Gamma 
\end{align}
\end{subequations}
mit $\tilde f = f-\Delta u_0$. Also reicht es aus, sich auf das \textit{homogene Dirichlet-Problem}\index{Dirichlet-Problem!homogenes} \eqref{eq:2.2a} zu beschränken. Im Folgenden betrachten wir somit \eqref{eq:2.1a} mit $g \equiv 0$.

Mit $H^1_0 (\Omega)$ bezeichnen wir, wie in Bemerkung \ref{bem:A.8} beschrieben, den Raum der in $\Omega$ einmal schwach differenzierbaren Funktionen, die am Rand $\Gamma$ verschwinden im Sinne der Spur. Wählen wir nun ein beliebiges $v \in H^1_0(\Omega)$, dann folgt durch Multiplikation von \eqref{eq:2.1aa} mit $v$ und Integration über $\Omega$ die Beziehung
\begin{align*}
	\int_\Omega -\Delta u \cdot v \, dx = \int_\Omega f v \, dx \, .
\end{align*}
Wir betrachten nun \eqref{eq:2.1aa} also nicht mehr punktweise (lokal), sondern im gewichteten Mittel über ganz $\Omega$ (global). Durch Anwenden der 1. Green'schen Formel (bzw. dem Satz von Gauß) ergibt sich
\begin{align}\notag
	& \int_\Omega \nabla u \cdot \nabla v \, dx -\underbrace{\int_\Gamma v \partial_\nu u \, ds}_{=0, \text{ da } v|_\Gamma = 0} = \int_\Omega f v \, dx \, \\
	\label{eq:2.3}	
	\Llra \, & \quad \qquad \int_\Omega \nabla u \cdot \nabla v \, dx =\int_\Omega f v \, dx \, .
\end{align}
Die Gleichung \eqref{eq:2.3} wird als \textit{\idx{Variationsgleichung}} bezeichnet. Wenn wir die Notationen aus Satz \ref{satz:A.5} (b) verwenden, so können wir \eqref{eq:2.3} kurz schreiben als
\[
	(\nabla u, \nabla v)_0 = (f,v)_0 \, ,
\]
daher definieren wir die Bilinearform $a: (H^1_0(\Omega))^2 \ra \R, a(u,v) := (\nabla u, \nabla v)_0$ und $(f,v):=(f,v)_0$.


\begin{bem*}
Wir werden in dieser Arbeit oftmals auch $a(\cdot,\cdot)$ für eine beliebige Bilinearform $a: H\times H\ra \R$ verwenden.
\end{bem*}


\begin{defi}\label{def:2.8}
Eine Funktion $u \in H^1_0(\Omega)$ heißt \textit{\idx{schwache Lösung}} vom homogenen Dirichlet-Problem\index{homogenes Dirichlet-Problem}\index{Dirichlet-Problem!homogen}
\begin{align}\label{eq:DP}\tag{DP}
\begin{aligned}
	-\Delta  u &=  f \text{ in } \Omega \, ,\\
	 u & = 0 \text{ auf } \Gamma \, ,
\end{aligned}
\end{align}
wenn die Gleichung
\begin{align}\label{eq:2.4}
	a(u,v) = (f,v)\quad \forall \, v \in H^1_0(\Omega) 
\end{align}
gilt.
\end{defi}


Anschaulich ist eine solche Lösung deshalb schwach, da sie das Problem \eqref{eq:DP} nur im gewichteten Mittel löst. Eine schwache Lösung muss das \textit{starke Problem}\index{starkes Problem} nicht lösen, da sie beispielsweise die Regularitätsanforderungen an das Problem nicht erfüllen muss.

Wir wollen uns nun die Frage nach der Eindeutigkeit und Existenz einer Lösung für die Variationsgleichung \eqref{eq:2.4} stellen. Diese Frage wollen wir zunächst allgemein für einen beliebigen reellen Hilbertraum $H$ beantworten. Wie wir nachher im Beweis des zentralen Satzes von Lax-Milgram sehen werden, ist hierfür explizit ein Hilbertraum notwendig.

Zuvor benötigen wir allerdings noch ein paar Definitionen und Eigenschaften für Bilinearformen.


\begin{defi}\label{def:2.9}
Sei $H$ ein Hilbertraum. Die Bilinearform  $a : H\times H \ra \R$ heißt \textit{stetig}\index{Bilinearform!stetig}, falls mit einem $c>0$
\[
	\abs{a(u,v)} \le c \, \norm{u}_H   \norm{v}_H \quad \forall \, u,v \in H
\]
gilt. Sie heißt $H$-\textit{elliptisch} (oder kurz \textit{elliptisch} oder \textit{koerziv})\index{Bilinearform!koerziv}\index{Bilinearform!elliptisch}, falls es ein $\alpha > 0$ gibt, so dass
\[
	a(v,v) \ge \alpha \, \norm{v}_H^2 \quad \forall \, v \in H 
\]
gilt.
\end{defi}


Da man die Variationsgleichung \eqref{eq:2.4} auch aus der Minimierung eines quadratischen Energiefunktionals $J: (H^1_0(\Omega))^2\ra \R, J(v) \coloneqq \frac 12 a(v,v)-(f,v)$ herleiten kann, wollen wir für ein solches Funktional zuvor einige Eigenschaften sammeln.


\begin{lemma}\label{lem:2.10}
Es sei $H$ ein Hilbertraum. Das Funktional
\[
	J: H \ra \R \, , \quad J(v) := \frac 1 2 a(v,v) - F(v) \, ,
\]
wobei $a: H\times H \ra \R$ eine stetige bilineare koerzive und $F: H\ra \R$ eine lineare Abbildung ist, ist konvex.
\end{lemma}

\begin{proof}
Es seien $u,v \in H$, dann gilt $u + t(v-u) = (1-t)u + tv \in H$ (dies gilt auch, wenn wir den Satz auf eine konvexe Teilmenge $M \subset H$ beschränken). Damit folgt mit $t \in [0,1]$
\begin{align*}
	J((1-t)u+tv)  = & \frac 1 2 a((1-t)u+tv,(1-t)u+tv) - F((1-t)u+tv) \\
	= &(1-t) \, J(u) + t \, J(v) +  \frac 1 2 a((1-t)u+tv,(1-t)u+tv) \\
	 & - \frac 1 2(1-t) \, a(u,u)-\frac 1 2 t \, a(v,v) \\
	= & (1-t) \, J(u) + t \, J(v)  + \frac 1 2 a(u,u) + t \, a(u,v-u)  \\
	&+ \frac {t^2} 2 a(v-u,v-u) - \frac 12 (1-t)\, a(u,u) -\frac 1 2 t \, a(v,v) \\
	= & (1-t) \, J(u) + t \, J(v) + \frac {t^2} 2 a(v-u,v-u)  \\
	 &\underbrace{+ t \, a(u,v)  - \frac 12 t\, a(u,u) -\frac 1 2 t \, a(v,v) }_{=  -\frac 1 2 t\, a(v-u,v-u)}\\
	= &  (1-t) \, J(u) + t \, J(v) - \frac {1} 2 \underbrace{t \, (1-t)}_{\ge 0} \,\underbrace{ a(v-u,v-u) }_{\ge \alpha  \norm{v-u}_H^2 \ge 0} \\
	\le &   (1-t) \, J(u) + t \, J(v) \, % \qedhere
\end{align*}
Daraus folgt die Behauptung.
\end{proof}


\begin{lemma} \label{lem:2.11}
Sei $H$ ein Hilbertraum. Das Funktional $J: H \ra \R, J(v) =\frac 1 2 a(v,v)-F(v)$ aus Lemma \ref{lem:2.10} ist Gâteaux-differenzierbar $($s. Definition \ref{def:Gateaux-Ableitung}$)$.
\end{lemma}

\begin{proof}
Wir rechnen einfach nach, dass der Grenzwert des Differenzenquotienten existiert und verwenden dabei die Bilinearität von $a$ und Linearität von $F$. Seien $u,v \in H$, dann gilt
\begin{align*}
	\mscr D_v J(u) & = \lim_{t\ra 0} \frac{J(u+tv)-J(u)}t \\ 
	&= \lim_{t\ra 0} \frac{J(u) + t \, (a(u,v)-F(v)) + \frac {t^2}2 a(v,v)-J(u)}t \\
	& =  \lim_{t\ra 0}  (a(u,v)-F(v)) + \frac {t}2 a(v,v) \\
	& = a(u,v)-F(v) < \infty\, ,
\end{align*}
da $a$ und $F$ jeweils stetig sind und daher durch $\norm u_H,\norm v_H$ beschränkt sind. Damit folgt die Behauptung.
\end{proof}


Nun können wir die Existenz und Eindeutigkeit einer Lösung durch das folgende Theorem zeigen.

\begin{theorem}[\idx{Lax-Milgram}]\label{theorem:2.12}
Es sei $H$ ein Hilbertraum und  $a : H \times H \ra \R$ eine symmetrische, in $H$ stetige, koerzive Bilinearform. Weiter sei $F:H\ra \R$ ein stetiges lineares Funktional, d.h.
\[
	\abs{F(v)} \le c \, \norm{v}_H \quad \forall \, v \in H
\]
mit einer Konstante $c >0$. Dann gibt es eine eindeutige Lösung $u \in H$, für die
\[
	a(u,v) = F(v) \quad \forall \, v \in H \, .
\]
gilt. Diese minimiert den Ausdruck
\[
	J(v) = \frac 1 2 a(v,v) - F(v)
\]
unter allen $v \in H$.
\end{theorem}

\begin{proof}
(i) Zunächst zeigen wir die Äquivalenz der beiden oberen Probleme.

"`$\Ra$"' Es sei $u\in H$, so dass $a(u,v) = F(v) \, \forall \, v \in H$. Für $t>0$ und $v\in H$ gilt dann
\begin{align*}
	J(u+tv) & = \frac 1 2 a(u+tv,u+tv) -F(u+tv) \\
	& = \frac 1 2 a(u,u) + t \, a(u,v) + \frac {t^2} 2 a(v,v)-F(u)-t \, F(v) \\
	& = \frac 1 2 a(u,u)-F(u) + t\, (\underbrace{a(u,v)-F(v)}_{=0}) + \frac{t^2}2 \underbrace{a(v,v)}_{\parbox{1.2cm}{\scriptsize$\ge 0$, da $a$ koerziv}} \\
	& > \frac 1 2 a(u,u) - F(u) = J(u) \, ,
\end{align*}
also ist $u = \arg\min\limits_{v\in H} J(v)$.

"`$\La$"' Es sei $u \in H$ das Minimum von dem Problem
\[
	\min_{v\in H} J(v) = \frac 1 2 a(v,v) -F(v) \, .
\]
Da $J:H\ra \R$ nach Lemma \ref{lem:2.10} ein konvexes Funktional ist und $J$ nach Lemma \ref{lem:2.11} Gâteaux-differenzierbar, gilt mit Satz \ref{satz:A.10} für alle $v \in H$
\begin{align*}
	0& = \mscr D_vJ(u) = \frac d{dt} J(u+tv)\Big|_{t=0} \\
	& = \frac d{dt}(J(u) + t \, (a(u,v)-F(v))+\frac{t^2}2 a(v,v))\Big|_{t=0} \\
	& = a(u,v)-F(v) + t \, a(v,v) \Big|_{t=0} = a(u,v)-F(v)
\end{align*}
(ii) Eindeutigkeit: Es seien $u,\tilde u \in H$ Lösungen der Variationsungleichung, d.h.
\begin{align*}
	a(u,v) = F(v) \, \wedge \, a(\tilde u,v) = F(v) \quad \forall \, v \in H \, .
\end{align*}
Damit folgt durch Subtraktion der beiden Gleichungen für alle $v \in H$
\begin{align}\label{eq:2.5}
	a(u,v) = a(\tilde u,v) \Llra a(u-\tilde u,v) = 0 \, .
\end{align}
Da $H$ ein Vektorraum ist, gilt auch $u-\tilde u \in H$. Ersetzen wir also in \eqref{eq:2.5} $v = u-\tilde u$, dann ergibt sich
\begin{align*}
	&0 = a(u-\tilde u,u-\tilde u) \stackrel{\scriptsize a\text{ koerziv}}\ge \underbrace{\alpha}_{>0} \norm{u-\tilde u}_H^2 \ge 0 
	\lra\norm{u-\tilde u}_H^2 = 0 \, ,
\end{align*}
also folgt $u = \tilde u$.

(iii) Existenz: Die Existenz einer Lösung weisen wir über das Funktional nach.
\begin{align*}
	J(v) & = \frac 1 2 a(v,v)-F(v) \stackrel[\scriptsize F \text{ linear}]{\scriptsize a \text{ koerziv}}\ge \frac 1 2 \alpha \norm v_H^2 - c \norm v_H \\
	& = \frac 1 2 \alpha \(\norm v_H^2 - \frac{2c}\alpha \norm v_H\) = \frac 1 2 \alpha\(\norm v_H - \frac c\alpha\)^2 - \frac {c^2}{2\alpha} \\
	& \ge - \frac{c^2}{2\alpha}
\end{align*}
Folglich ist $J$ nach unten beschränkt. Sei $\eta := \inf \{J(v)\mid v \in H\}$ und $(v_n)_{n\in\N}$ eine Folge mit $J(v_n) \ra\eta$ für $n\ra \infty$. Dann folgt mit der Koerzivität von $a$
\begin{align*}
	\alpha \norm{v_n-v_m}^2_H  \le & a(v_n-v_m,v_n-v_m) \\
	 = &a(v_n,v_n)+a(v_m,v_m)-a(v_n,v_m)-a(v_m,v_n) \\
	=& 2a(v_n,v_n)+2a(v_m,v_m) \underbrace{-a(v_n,v_n+v_m)-a(v_m,v_n+v_m)}_{=-a(v_n+v_m,v_n+v_m)} \\
	=& 2a(v_n,v_n)-4F(v_n)+2a(v_m,v_m)-4F(v_m) \\
	& -a(v_n+v_m,v_n+v_m)+4F(v_n+v_m) \\
	= & 4 J(v_n) + 4J(v_m) - 4 a\(\frac{v_n+v_m}2,\frac{v_n+v_m}2\)+8F\(\frac{v_n+v_m}2\) \\
	= & 4 J(v_n) + 4J(v_m) - 8 J\(\frac{v_n+v_m}2\) \\
	\le &4 J(v_n) + 4J(v_m) - 8\eta  \xrightarrow[n,m\ra\infty]{} 4\eta+4\eta-8\eta = 0 \, ,
\end{align*}
d.h. $(v_n)_{n\in \N}$ ist eine Cauchy-Folge. Da $H$ ein Hilbertraum ist, gilt somit: $\exists \, u \in H : v_n \xrightarrow[n\ra \infty]{} u$ mit $J(u) = \eta$.
\end{proof}


Um die allgemeine Aussage aus dem Theorem von \idx{Lax-Milgram} auf unser Modellproblem \eqref{eq:2.4} übertragen zu können, benötigen wir die \textit{\idx{Poincaré-Friedrich-Ungleichung}}, die auch später noch eine zentrale Rolle für den hierarchischen Fehlerschätzer spielen wird.

\begin{satz}[\idx{Poincaré-Friedrich-Ungleichung}]\label{satz:2.13}
Es sei $\Omega$ in einem $d$-dimensionalen Würfel der Kantenlänge $s>0$ enthalten. Dann gilt
\[
	\norm v_0 \le s \norm{\nabla v}_0 \quad \forall \, v \in H^1_0(\Omega) \, ,
\]
wobei $\norm\cdot_0$ die durch das Skalarprodukt $(\cdot,\cdot)_0$ induzierte Norm ist.
\end{satz}

\begin{proof}
Der Beweis ist in \cite{BraeFEM} Kapitel II, \S1 Sobolev-Räume, Satz 1.5 oder \cite{StarkePDE} Satz 1.5 zu finden.
\end{proof}


\begin{bem}\label{bem:2.14}
Für die Gültigkeit der Poincaré-Friedrich-Ungleichung, muss $v$ nicht auf ganz $\Gamma$ gleich Null sein, sondern es reicht aus, dass
\[
	v \in H^1_{\Gamma_u} (\Omega)\coloneqq \{v \in H^1(\Omega) \mid v = 0 \text{ auf } \Gamma_u\}
\]
ist mit $\Gamma_u \subset \Gamma$, wobei mit einem Maß $\mu$ gilt: $\mu (\Gamma_u) \not = 0$, d.h. $\Gamma_u$ ist keine Nullmenge (vgl. \cite{BraeFEM} Kapitel II, \S1, Bemerkung 1.6).
\end{bem}


Jetzt können wir mittels Theorem \ref{theorem:2.12} überprüfen, ob  Problem \eqref{eq:2.4} mit $a:(H^1_0(\Omega))^2\ra \R, a(u,v) = (\nabla u,\nabla v)_0$ und $F:H^1_0(\Omega) \ra \R, F(v) := (f,v)$ eine eindeutige Lösung hat. Es seien $u,v \in H^1_0(\Omega)$, dann gilt
\begin{align*}
	 a(v,v) = & \int_\Omega \nabla v \nabla v \, dx = \norm{\nabla v}_0^2  \\
	\ge& \frac{s^2+1}{(1+s)^2}\norm{\nabla v}_0^2  \stackrel{\scriptsize \text{Satz \ref{satz:2.13}}}\ge \frac 1{(1+s)^2} (\norm v_0^2 + \norm{\nabla v}_0^2) \\
	= & \frac 1{(1+s)^2} \norm v_1^2 \, .
\end{align*}
Damit ist $a$ mit $\alpha :=  \frac 1{(1+s)^2}$ koerziv. Weiter rechnen wir unter Verwendung der Cauchy-Schwarz-Ungleichung\index{Cauchy-Schwarz'sche Ungleichung} nach:
\begin{align*}
	\abs{a(u,v)} &= \Abs{\int_\Omega \nabla u \nabla v \, dx} \le \sum_{i = 1}^d \int_\Omega\abs{\partial_i u}\abs{\partial_i v} \, dx \\
	&\!\!\:\! \stackrel{\scriptsize\text{C.S.}}\le  \sum_{i = 1}^d \(\int_\Omega \abs{\partial_i u}^2 \, dx\)^{\frac 12} \(\int_\Omega \abs{\partial_i v}^2 \, dx\)^{\frac 12} \\
	&\le \(\sum_{i = 1}^d \int_\Omega \abs{\partial_i u}^2 \, dx\)^{\frac 12} \(\sum_{i=1}^d\int_\Omega \abs{\partial_i v}^2 \, dx\)^{\frac 12} \\
	&\le \( \int_\Omega \abs{\nabla u}^2 \, dx + \int_\Omega u^2 \, dx\)^{\frac 12} \(\int_\Omega \abs{\nabla v}^2 \, dx+\int_\Omega v^2\, dx\)^{\frac 12} \\
	&= \norm u_1 \, \norm v_1 \, ,
\end{align*}
d.h.  $a$ ist stetig mit $c := 1$. Die Symmetrie von $a$ ist trivial, also bleibt nur noch die Stetigkeit von $F$ zu zeigen. Es sei $v \in H^1_0(\Omega)$, dann gilt
\begin{align*}
	\abs{F(v)} &= \abs{(f,v)} =  \Abs{\int_\Omega fv \, dx} \stackrel{\scriptsize\text{C.S.}}\le\( \int_\Omega \abs{f}^2 \,dx\)^{\frac 12} \( \int_\Omega \abs v^2 \, dx\)^{\frac 12} \\
	&\le  c \, \bigg( \int_\Omega \underbrace{\abs{\nabla v}^2}_{\ge 0} +  \abs v^2 \, dx\bigg)^{\frac 12} = c \, \norm v_1
\end{align*}
mit $0<c := \(\int_\Omega \abs f^2 \, dx\)^{\frac 12} < \infty$, wenn $f \in L_2(\Omega)$ ist. Damit ist $F$ ein stetiges lineares Funktional und somit existiert nach Theorem \ref{theorem:2.12} eine eindeutige Lösung $u \in H^1_0(\Omega)$ für die schwache Formulierung des homogenen Dirichlet-Problems. 

Weiter minimiert die Lösung $u \in H^1_0(\Omega)$ auch das Funktional
\begin{align*}
	J(v) = \frac 12 \int_\Omega \nabla v\nabla v \, dx - \int_\Omega fv \, dx \, ,
\end{align*}
welches die gespeicherte Energie der durch die Kraft $f$ belasteten in $\Omega$ eingespannten Membran beschreibt.


\begin{bem*}
Die Stetigkeit vom Funktional $F$ zeigt, dass  die Kraft $f$ aus dem Dirichlet-Problem wenigstens quadratisch integrierbar, also in $L^2(\Omega)$, sein muss, damit es eine schwache Lösung geben kann.
\end{bem*}


\begin{notation}
\begin{enumerate}[(a)]
\item Mit $H'$ bezeichnen wir den Dualraum zum Hilbertraum $H$.
\item Den Dualraum zu $H^1(\Omega)$ bezeichnen wir mit $H^{-1}(\Omega)$.
\end{enumerate}
\end{notation}


Als Folgerung aus dem Theorem von \idx{Lax-Milgram} betrachten wir den nächsten Satz.


\begin{satz}[\idx{Riesz'scher Darstellungssatz}]\label{satz:2.15}
Es sei $H$ ein Hilbertraum mit einem Skalarprodukt $(\cdot,\cdot)_H$. Es sei $F \in H'$, dann existiert genau ein $u \in H$, so dass
\[
	(u,v)_H = F(v) \quad \forall \, v \in H \, .
\]
\end{satz}

\begin{proof}
Dies ist eine direkte Folgerung aus dem Theorem \ref{theorem:2.12}. Die Abbildung $(\cdot,\cdot)_H:H\times H \ra \R$ ist als Skalarprodukt bilinear, symmetrisch und positiv definit, damit auch bzgl. der auf $H$ durch das Skalarprodukt induzierten Norm $\norm v_H := \sqrt{(v,v)_H}$, koerziv. $F$ ist als Element des Dualraumes $H'$ eine lineare stetige Abbildung $F:H\ra\R$ und damit folgt mit $a(\cdot,\cdot) :=(\cdot,\cdot)_H$ aus dem Theorem von Lax-Milgram die Behauptung.
\end{proof}


\begin{lemma}\label{lem:2.16}
Es sei $H$ ein Hilbertraum mit Skalarprodukt $(\cdot,\cdot)_H$ und $a:H\times H\ra \R$ eine stetige koerzive Bilinearform. Dann existiert genau ein linearer Operator $A : H \ra H$, so dass gilt:
\[
	a(u,v)  = (Au,v)_H \quad \forall \, u , v \in H \, .
\]
\end{lemma}

\begin{proof}
Es sei $u\in H$ fest, dann ist $L: H\ra \R, L(v) := a(u,v)$ eine lineare Abbildung, die stetig ist, da
\[
	\abs{L(v)} = \abs{a(u,v)} \stackrel{\scriptsize\text{stetig}}\le c \,  \norm{u}_H \norm v_H  = \tilde c\, \norm v_H
\]
mit $0<\tilde c := c \, \norm u_H$ gilt. Damit folgt nach dem Darstellungssatz von Riesz, dass es ein eindeutiges $l \in H$ gibt, so dass
\[
	a(u,v) = L(v) = (l,v)_H \quad \forall \, v \in H
\]
gilt. Da $u \in H$ jedoch beliebig ist, bleibt zu zeigen, dass es ein eindeutiges $A:H\ra H$ gibt, so dass $Au = l$ ist.

Wir zeigen zunächst mithilfe der Bilinearform $a$, dass $A$ linear ist. Es gilt für $\lambda,\mu \in \R$ und $u,v \in H$
\begin{align*}
	(A(\lambda u + \mu v),w)_H &= a(\lambda u + \mu v, w) = \lambda a(u,w) + \mu a(v,w) \\
	& = \lambda (Au,w)_H + \mu(Av,w)_H \\
	& = (\lambda \, Au+\mu \, Av,w)_H
\end{align*}
für alle $w \in H$. Weiter gilt
\[
	\norm{Au}_H^2 = (Au,Au)_H = a(u,Au) \stackrel{\scriptsize \text{stetig}}\le c \, \norm u_H \norm{Au}_H \, ,
\]
d.h. $\norm{Au}_H \le c \, \norm u_H$ und damit ist nach \cite{Werner} Satz II.1.2 der Operator $A$ stetig.

Betrachten wir den Kern von $A$, so ergibt sich
\begin{align}\label{eq:2.6}
	\ker A := \{ v \in H \mid Av = 0\} = \{0\} \, ,
\end{align}
denn
\[
	\alpha \norm v_H^2 \stackrel{\scriptsize \text{koerziv}}\le a(v,v)  = (Av,v)_H \stackrel{\scriptsize \text{CS}}\le \norm{Av}_H \norm v_H 
\]
und damit gilt $\norm{Av}_H \ge \alpha \norm v_H$, d.h. $Av = 0 \Lra v = 0$. Dies impliziert, dass $A$ injektiv ist, denn mit $v_1,v_2\in H, Av_1 = Av_2$ folgt
\[
	0 = Av_1 - Av_2 = A(v_1-v_2) \ \stackrel{\scriptsize\eqref{eq:2.6}}\lra \ v_1 = v_2 \, .
\]
Weiter betrachten wir das Bild von $A$, d.h.
\[
	\im A := \{ v \in H \mid \exists \, u \in H : Au = v \} \subset H\, .
\]
Sei $(v_n)_{n \in \N}$ eine Folge mit $v_k \in \im A $ für alle $k \in \N$. Dann folgt, dass für jedes $v_k$ ein $u_k \in H$ existiert mit $A u_k = v_k$. Es gelte, dass $Au_k = v_k \ra v \in H$ geht, dann folgt
\begin{align*}
	\alpha \, \norm{u_n -u_m}_H & \le \norm{A(u_n-u_m)}_H = \norm{Au_n-Au_m}_H \\
	& = \norm{v_n-v_m}_H \xrightarrow[n,m\ra \infty]{} 0 \, ,
\end{align*}
d.h. $(u_n)_{n\in \N} \subset H$ ist eine Cauchy-Folge und konvergiert daher in $H$. Also existiert ein $u \in H$ mit $u_n \ra u$. Mit der Stetigkeit von $A$ folgt dann
\[
	v_n = A u_n \xrightarrow[n\ra \infty]{} Au = v\, ,
\]
d.h. $v \in \im A$ und damit ist $\im A$ abgeschlossen. Wir betrachten nun ein $v \in H$ mit $v \perp \im A \subset H$, dann gilt
\[
	(Au,v)_H = 0 \quad \forall \, u \in H \, .
\]
Damit folgt mit $u = v \in H$ oben eingesetzt
\[
	 0 = (Av,v)_H = a(v,v) \ge \alpha \, \norm v_H^2 \, \lra \, v = 0 \, .
\]
Also besteht der zu $\im A$ orthogonale Raum nur aus dem Nullelement und mit Korollar \ref{kor:2.6} gilt dann $\im A = \overline{\im A} = H$. Damit ist $A$ bijektiv.

Es seien nun $0 \not = l\in H$ sowie $A_1,A_2 \in \mcal L(H,H)$ zwei lineare Operatoren mit $A_1u = l$ und $A_2 u = l$, die nach der obigen Weise konstruiert sind. Dann gilt
\[
	0 = A_1 u - A_2 u = (A_1 - A_2)u \, \lra \, A_1 = A_2 \, , 
\]
da $u \not = 0$ und die Summe zweier bijektiver linearer Operatoren wieder bijektiv ist, also ist ein so konstruierter Operator eindeutig.
\end{proof}






\section{Finite Elemente Methode}
\label{kap:2.3}


Für unser Modellproblem kann man zeigen, dass es für bestimmte Gebiete $\Omega$ eine exakte, d.h. analytische, Lösung gibt (vgl. \cite{Walker} Kapitel 5). Diese muss im Allgemeinen nicht für jedes Problem bekannt oder gar berechenbar sein. Daher wollen wir nicht mehr die exakte Lösung von unserer Variationsgleichung \eqref{eq:2.4} berechnen, sondern eine Approximation davon, die sogenannte \textit{\idx{Galerkin-Approximation}}.

Unter dem \textit{\idx{Galerkin-Verfahren}} verstehen wir, dass wir die Variationsgleichung
\begin{align}\label{eq:2.8}
	a(u,v) = F(v) \quad \forall \, v \in H
\end{align}
nur noch auf einem endlich dimensionalen Unterraum $V_h \subset H$ lösen wollen, d.h. finde $u_h \in V_h$, so dass
\begin{align}\label{eq:2.9}
	a(u_h,v_h) = F(v_h) \quad \forall \, v_h \in V_h \, .
\end{align}


\begin{satz}\label{satz:2.17}
Das "`Galerkin-Problem"' \eqref{eq:2.9} hat eine eindeutige Lösung.
\end{satz}

\begin{proof}
Da $V_h$ als Unterraum von $H$ auch ein Hilbertraum ist und die Eigenschaften von $a, F$ weiterhin erfüllt sind, gilt auch hier der Satz von Lax-Milgram, was die Eindeutigkeit und Existenz einer Lösung sichert.
\end{proof}


Da $V_h$ ein endlich dimensionaler Unterraum von $H$ ist, wird jener von einer endlichen Basis $\mcal B_h \coloneqq \{\phi_1,\ldots,\phi_N\}$ aufgespannt, d.h. für $u_h \in V_h$ gilt:
\begin{align}\label{eq:2.10}
	\exists! \, \bs\mu \in \R^N :  u_h(x) = \sum_{i = 1}^N \mu_i \,  \phi_i(x) \, .
\end{align}
Da $F(\cdot),a(u,\cdot)$ linear sind und alle $v_h \in V_h$ analog zu oben darstellbar sind, ist \eqref{eq:2.9} äquivalent zum Problem
\[
	a(u_h,\phi_i) = F(\phi_i) \quad \forall \, i = 1, \ldots,N \, ,
\]
mit $u_h$, wie in \eqref{eq:2.10} dargestellt, eingesetzt ergibt sich
\[
	a(u_h,\phi_i) = a \Big( \sum_{j = 1}^N \mu_j \,  \phi_j,\phi_i \Big) = \sum_{j = 1}^N \mu_j \, a(\phi_j,\phi_i) \, ,
\]
also
\[
	 \sum_{j = 1}^N \mu_j \, a(\phi_j,\phi_i) = F(\phi_i)\quad \forall \, i = 1, \ldots,N \, .
\]
Damit erhalten wir ein lineares Gleichungssystem
\[
	A \bs \mu = \bs f 
\]
mit $A = [a(\phi_j,\phi_i)]_{i,j=1}^N, \bs \mu = [\mu_i]_{i=1}^N$ und $\bs f = [F(\phi_i)]_{i=1}^N$. Dieses Gleichungssystem gilt es zu lösen, um die gesuchten Koordinaten $\mu_i, i = 1,\ldots,N,$ bzgl. der Basis $\mcal B_h$ für die approximierte Lösung $u_h$ zu finden.

In den Ingenieurswissenschaften, insbesondere bei kontinuumsmechanischen Problemen, wird $A$ als \textit{\idx{Steifigkeitsmatrix}} bezeichnet.

\begin{bem}\label{bem:2.18}
Ist die Bilinearform $a$ symmetrisch, so ist es auch die Matrix $A$, denn
\[
	a_{ij} = a(\phi_i,\phi_j) = a(\phi_j,\phi_i)= a_{ji} \, .
\]
Außerdem folgt aus der Koerzivität von $a$, dass  mit $\bs 0 \not = \bs \nu \in \R^N$ gilt
\begin{align*}
	\bs \nu^T A \bs \nu & = \sum_{i,j=1}^N \nu_i a_{ij} \nu_j  =  \sum_{i=1}^N \nu_i\sum_{j=1}^N \, a(\phi_i,\phi_j) \, \nu_j   \\
	& = \sum_{i=1}^N \nu_i \, a\Big(\phi_i,\sum_{j=1}^N\nu_j\phi_j\Big) =  a\Big(\sum_{i=1}^N \nu_i\phi_i,\sum_{j=1}^Nv_j\phi_j\Big) \\
	& = a(v_h,v_h) \ge \alpha \norm{v_h}^2_H > 0 \, ,
\end{align*}
da $\sum \nu_i \phi_i = v_h \not = 0$ wegen $\bs \nu \not = \bs 0$. Damit ist $A$ also positiv definit und es folgt nochmals, dass $A \bs \mu = \bs f$ eine eindeutige Lösung hat.
\end{bem}


Um eine Basis $\mcal B_h$ bzgl. $V_h$ beschreiben zu können, muss das Gebiet $\Omega\subset \R^2$ in endliche Elemente zerlegt werden. Die Basis $\mcal B_h$ und damit der Raum $V_h$ wird dann bzgl. einer Zerlegung $\mcal T_h$ beschrieben. Eine gebräuchliche Zerlegung $\mcal T_h$ kann durch Dreiecke oder auch Vierecke geschehen. Wir wollen in dieser Arbeit nur Zerlegungen durch Dreiecke betrachten, hierfür führen wir den folgenden Begriff ein (vgl. \cite{BraeFEM} Seite 58 oder \cite{StarkePDE} Seite 19).


\begin{defi}[\idx{Triangulierung}]\label{def:2.19}
Es sei $\Omega \subset \R^2$ ein durch einen Polygonzug berandetes Gebiet. Dann heißt eine Zerlegung aus Dreiecken
\[
	\mcal T = \{T_1,T_2,\ldots,T_M\}
\]
\textit{\idx{Triangulierung}}, wenn gilt:
\begin{enumerate}[(a)]
\item Für alle Dreiecke $T \in \mcal T$ gilt: $T$ ist abgeschlossen.
\item	Ganz $\Omega$ wird durch alle Dreiecke aus $\mcal T$ überdeckt, d.h. $\bar\Omega = \bigcup_{T\in \mcal T} T$.
\item Der Schnitt zweier Dreiecke $T_i\cap T_j$ mit $i \not = j$ überlappt sich nicht, d.h. $\operatorname{int}(T_i)\cap \operatorname{int}(T_j) = \emptyset$.
\end{enumerate}
Wir nennen eine Triangulierung \textit{konform}\index{Triangulierung!konform} oder \textit{zulässig}\index{Triangulierung!zulässig}, wenn zusätzlich gilt:
\begin{enumerate}[(d)]
\item Für jede Kante $k$ eines Dreiecks $T \in \mcal T$ gilt entweder $k \subset \partial \Omega$ oder $k = \tilde k$ für eine weitere Kante $\tilde k$ eines weiteren Dreiecks $\widetilde T \in \mcal T$. 
\end{enumerate}
Der Radius des Umkreis eines Dreieckes $T$ wird mit $h$ bezeichnet und beschreibt die Größe eines Dreiecks. Wenn jedes Dreieck $T \in \mcal T$ höchstens einen Radius von $h$ hat, so schreiben wir $\mcal T_h$ statt $\mcal T$.
\end{defi}


\begin{bem}\label{bem:2.20}
\begin{enumerate}[(a)]
\item Ein Dreieck $T\in \mcal T_h$ bezeichnen wir auch als (\textit{finites}) \textit{Element}.
\item Sollte $\Omega \subset \R^3$ sein, so können wir  analog zu Definition \ref{def:2.19} eine Zerlegung mit Tetraedern definieren.
\end{enumerate}
\end{bem}


Für die Netzverfeinerung führen wir zwei unterschiedliche Familien von Zerlegungen ein (vgl. \cite{BraeFEM} Seite 58).


\begin{defi}[\idx{(quasi-) uniforme Zerlegung}]\label{def:2.21}
Eine Familie von Zerlegungen $\{\mcal T_h\}$ heißt \textit{quasi-uniform}\index{Triangulierung!quasi-uniform}, wenn es eine Zahl $\kappa > 0$ gibt, so dass jedes $T \in \mcal T_h$ einen Kreis vom Radius
\[
	\rho_T \ge \frac{h_T}\kappa
\]
enthält, wobei $h_T$ der Radius des Dreiecks $T$ ist.

Eine Familie von Zerlegungen $\{\mcal T_h\}$ heißt \textit{uniform}\index{Triangulierung!uniform}, wenn es eine Zahl $\kappa > 0$ gibt, so dass jedes $T \in \mcal T_h$ einen Kreis vom Radius
\[
	\rho_T \ge \frac{h}\kappa
\]
enthält, wobei $h := \max_{T \in \mcal T_h} h_T$ ist.
\end{defi}



\begin{figure}[h]
\begin{center}
\begin{pspicture}(-3,0)(3,2)
	% zulässige Triangulierung:
	\psline(-3,0)(-3,2)
	\psline(-3,0)(-1,0)
	\psline(-1,0)(-3,2)
	\psline(-3,2)(-1,2)
	\psline(-1,2)(-1,0)
	\psline(-3,0)(-1,2)
	\psline(-3,1)(-2,1)
	\psline(-2,2)(-2,1)
	\psline(-2,2)(-3,1)
	
%	\psline(-2,2)(-1.5,1.5)
%	\psline(-1.5,1.5)(-1,0)

%	\psline(-2.5,1.5)(-3,1)
%	\psline(-2.5,1.5)(-1,2)
	
	% nichtzulässige Triangulierung
	\psline(1,2)(2,1)
	\psline(2,1)(1,0)
	\psline(1,0)(1,2)
	\psline(1,0)(3,0)
	\psline(3,0)(2,1)
	\psline(1,2)(3,2)
	\psline(3,2)(3,0)
	\psdot[dotstyle=o, dotsize=4pt](2,1)
\end{pspicture}
\end{center}
\caption{Zulässige und unzulässige Triangulierung (mit hängendem Knoten)\label{abb:2.2}}
\end{figure}


In der Abbildung \ref{abb:2.2} ist eine quasi-uniforme zulässige Zerlegung und eine unzulässige Zerlegung zu sehen; zweitere ist daher unzulässig, da es Kanten gibt, die weder am Rand liegen, noch mit einer Kante eines anderen Dreiecks übereinstimmt. Die Knoten, die zu diesem Phänomen führen, nennen wir \textit{hängende Knoten}\index{hängender Knoten}. Dies sind Knoten, die nicht Eckpunkt jedes angrenzenden Dreiecks sind.


\begin{bem}\label{bem:2.22}
Wie man leicht sehen kann, ist jede uniforme Zerlegung auch quasi-uniform. Umgekehrt gilt dies nicht (s. Abbildung oben).

Allerdings lassen uniforme Zerlegungen keine lokalen Verfeinerungen zu. Da dies für adaptive Verfeinerungsstrategien allerdings ausschlaggebend ist, gehen wir im Folgenden immer von einer quasi-uniformen Zerlegung $\mcal T_h$ aus.
\end{bem}


Nun wollen wir uns Gedanken über unseren Ansatzraum $V_h$ machen. Hierfür gibt es, abhängig von der Konstruktion des verwendeten Elements, viele Möglichkeiten -- vgl. hierzu auch \cite{BraeFEM} Kapitel II, \S5, Tabelle 2. Wir wollen uns weitestgehend aber nur auf ein Element konzentrieren. Zuvor betrachten wir hierfür ein wichtiges Resultat, wobei noch bemerkt sei, dass eine Funktion $u$ auf $\Omega$ bei gegebener Zerlegung $\mcal T_h$ eine Eigenschaft stückweise hat, wenn sie auf jedem Element diese Eigenschaft besitzt.


\begin{satz}\label{satz:2.23}
Sei $k \ge 1$ und $\Omega\subset \R^2$ ein polygonales Gebiet. Eine stückweise beliebig oft differenzierbare Funktion $v : \bar \Omega \ra \R$ liegt in $H^k(\Omega)$ genau dann, wenn $v \in C^{k-1}(\bar\Omega)$ ist.
\end{satz}

\begin{proof}
Der Beweis ist in \cite{BraeFEM} Kapitel II, \S5, Satz 5.2 zu finden.
\end{proof}


Der Satz \ref{satz:2.23} rechtfertigt, dass wir für das Modellproblem \eqref{eq:2.4} auf einer Triangulierung $\mcal T_h$ einen Ansatzraum $V_h$ mit stetigen Funktionen $v \in C^0(\Omega)$ verwendet, da dann auch $v \in H^1(\Omega)$ gilt. Daher wählen wir
\[
	V_h \coloneqq \{v \in C^0(\Omega) \mid v|_T \in \mcal P_m \text{ für } T\in \mcal T_h, v|_{\partial \Omega} = 0\} \, ,
\]
wobei $\mcal P_m$ der Raum der Polynome vom Grad $m$ ist. Es stellt sich nun die Frage, wie wir geschickt eine Basis wählen können, um $V_h$ aufzuspannen. Die einfachste Möglichkeit stellen \textit{\idx{nodale Basisfunktion}en} dar.

\begin{defi}[\idx{nodale Basisfunktion}]\label{def:2.24}
Zu einem Finiten Element Raum $V_h$ und einer gegebenen Zerlegung $\mcal T_h$ sei eine Menge von Punkten $P$ bekannt mit $\abs P = N$. Die Menge $\mcal B_h = \{\phi_1,\ldots,\phi_N\}$ mit $\phi_i \in \mcal P_m, i = 1,\ldots, N$, heißt \textit{\idx{nodale Basis}} (oder \textit{\idx{Lagrange-Basis}}), wenn
\[
	\phi_i (x_j) = \delta_{ij} = \begin{cases}
							1, &  i = j \\
							0 ,& i \not = j
						\end{cases} \, , %\qquad \forall \,\phi_i \in \mcal B_h, x_j \in P
\]
für alle $\phi_i \in \mcal B_h$ und $x_j \in P$ gilt.
\end{defi}


Mit der Menge der vorgegebenen Punkte $P$ kann durch die Anzahl $N$ der Grad des zur Interpolation verwendeten Polynoms gesteuert werden.


\begin{bem}
Sei $m \ge 0$. In einem Dreieck $T$ seien auf $m+1$ Linien $l = 1+2+\ldots+(m+1)$ Punkte $z_1,\ldots,z_l$ angeordnet (s. Abb. \ref{abb:2.3}). Dann gibt es zu jedem $C^0(T)$ genau ein Polynom $p$ vom Grad $m$ mit der Eigenschaft
\[
	p(z_i) = f(z_i) \quad \forall \, i = 1,\ldots,m \, .
\]
\end{bem}

\begin{proof}
Der Beweis steht in \cite{BraeFEM} Kapitel II, \S 5, Bemerkung 5.4.
\end{proof}


\begin{figure}[h]
\begin{center}
\begin{pspicture}(-3,0)(5,2)
	% lineares Element:
	\psline(-3,0)(-3,2)
	\psline(-3,0)(-1,0)
	\psline(-1,0)(-3,2)
	\psdots(-1,0)(-3,0)(-3,2)
	
	% quadratisches Element:
	\psline(0,0)(0,2)
	\psline(0,0)(2,0)
	\psline(2,0)(0,2)
	\psdots(0,0)(0,2)(2,0)(0,1)(1,1)(1,0)
	
	% kubisches Element:
	\psline(3,0)(3,2)
	\psline(3,0)(5,0)
	\psline(5,0)(3,2)
	\psdots(3,0)(5,0)(3,2)(3,1.3333)(3,0.6667)(3.6667,0)(4.3333,0)(3.6667,0.6667)(3.6667,1.3333)(4.3333,0.6667)
	
\end{pspicture}
\end{center}
\caption{Dreiecke für nodale Basen (linear, quadratisch, kubisch)\label{abb:2.3}}
\end{figure}


Damit lässt sich für $V_h$ mit einem beliebigen Polynomgrad $m$ eine eindeutige \idx{nodale Basis} finden, die den Raum aufspannt. Im weiteren wollen wir lineare Ansatzfunktionen verwenden. Wir bezeichnen, sofern nicht anders beschrieben, also im Folgenden $\mcal S_h$ mit
\[
	\mcal S_h \coloneqq \{v \in C^0(\Omega) \mid v|_T \in \mcal P_1 \text{ für } T\in \mcal T_h, v|_{\partial \Omega} = 0\} \, ,
\]
also sind in diesem Raum die Eckpunkte der Dreiecke vorgegeben bzw. später im LGS gesucht. Das \idx{Galerkin-Verfahren} mit dem Ansatzraum $\mcal S_h$ wird auch \idx{Finite-Elemente-Methode} (kurz: FEM) genannt. 


Wir wollen folgendes Beispiel zur Berechnung von den Matrixeinträgen der Matrix $A$ betrachten.

\begin{bsp}\label{bsp:2.26}
Wir betrachten das Variationsproblem \eqref{eq:2.9} auf $\Omega = [-1,1]^2$ mit $\mcal S_h$ wie oben eingeführt als den Raum der linearen Ansatzfunktionen auf einer Zerlegung $\mcal T_h$ aus 8 \textit{\idx{Courant-Element}en}, wie in Abbildung \ref{abb:2.4} zu sehen ist, wobei wir auf die rechte Seite $F(v_h)$ zunächst noch nicht genauer eingehen möchten.


\begin{figure}[h]
\begin{center}
\begin{pspicture}(-2,-2)(2,2.5)
	% Skalierung:
	\psset{xunit=2cm,yunit=2cm}

	% Die 8 Courant-Elemente:
	\psline(-1,-1)(-1,1)
	\psline(-1,1)(1,1)
	\psline(1,1)(1,-1)
	\psline(1,-1)(-1,-1)
	\psline(-1,0)(1,0)
	\psline(0,-1)(0,1)
	\psline(0,-1)(-1,0)
	\psline(0,1)(1,0)
	\psline(-1,1)(1,-1)
	
	% Beschriftung der Elemente:
	\rput(0.3,0.3){I}
	\rput(0.7,0.7){II}
	\rput(-0.3,-0.3){VI}
	\rput(-0.7,-0.7){V}
	\rput(-0.65,0.3){III}
	\rput(-0.3,0.7){IV}
	\rput(0.3,-0.7){VII}
	\rput(0.7,-0.3){VIII}
	
	% Beschriftung und Markierung der Punkte:
	\psdots(-1,-1)(-1,0)(-1,1)(0,-1)(0,0)(0,1)(1,-1)(1,0)(1,1)
	% links:
	\rput(-1.13,1.13){1}
	\pscircle[linewidth=0.5pt](-1.13,1.13){0.21}
	\rput(-1.15,0){4}
	\pscircle[linewidth=0.5pt](-1.15,0){0.21}
	\rput(-1.13,-1.13){7}
	\pscircle[linewidth=0.5pt](-1.13,-1.13){0.21}
	% rechts:
	\rput(1.13,1.13){3}
	\pscircle[linewidth=0.5pt](1.13,1.13){0.21}
	\rput(1.15,0){6}
	\pscircle[linewidth=0.5pt](1.15,0){0.21}
	\rput(1.13,-1.13){9}
	\pscircle[linewidth=0.5pt](1.13,-1.13){0.21}
	% mitte:
	\rput(0,1.15){2}
	\pscircle[linewidth=0.5pt](0,1.15){0.21}
	\rput(0.15,0.15){5}
	\pscircle[linewidth=0.5pt](0.15,0.15){0.21}
	\rput(0,-1.15){8}
	\pscircle[linewidth=0.5pt](0,-1.15){0.21}
\end{pspicture}
\end{center}
\caption{Triangulierung von $\Omega = [-1,1]^2$ in 8 Courant-Elemente\label{abb:2.4}}
\end{figure}

Wir stellen für die nodale Basisfunktion $\phi_5$ die Einträge in der Steifigkeitsmatrix $A$ auf. Man rechnet leicht nach, dass
\begin{align*}
	\phi_5(x,y) = \begin{cases}
					1-x-y , & \text{auf } \rz I \\
					1+x, & \text{auf } \rz{III} \\
					1-y, & \text{auf } \rz{IV} \\
					1+x+y, & \text{auf } \rz{VI} \\
					1+y, & \text{auf }\rz{VII}\\
					1-x, & \text{auf }\rz{VIII} \\
					0, & \text{sonst}
				\end{cases}
\end{align*}

\begin{table}[htpb]
\centering
\begin{tabular}[c]{|c|c|c|c|c|c|c|c|c|}
	\hline
      & I & II & III & IV & V & VI & VII & VIII\\
	\hline
     $\partial_x \phi_5$ &-1 &0 &1& 0&0 &1 &0 &-1  \\
     $\partial_y \phi_5$ & -1&0 & 0& -1& 0& 1& 1& 0\\
	\hline
\end{tabular}
\caption{\label{tab:2.1}Ableitungen der nodalen Basisfunktion $\phi_5$.}
\end{table}

ist und es ergeben sich die Ableitungen aus Tabelle \ref{tab:2.1}. Dann gilt
\begin{align*}
	a(\phi_5,\phi_5) & = \int_\Omega \nabla \phi_5 \nabla \phi_5 \, dx dy  = \int_{\rz I  \cup \ldots \cup \rz{VIII}} \underbrace{(\partial_x \phi_5)^2}_{\ge0}+ \underbrace{(\partial_y \phi_5)^2}_{\ge 0} \, dx dy \\
	& = 2 \int_{\rz I \cup \rz{III} \cup \rz{IV}} {(\partial_x \phi_5)^2}+ {(\partial_y \phi_5)^2} \, dx dy \\
	& = 2 \(\int_{\rz I \cup \rz{III}}\underbrace{(\partial_x \phi_5)^2}_{=1} \, dx dy + \int_{\rz I \cup \rz{IV}} \underbrace{(\partial_y \phi_5)^2}_{=1} \, dxdy\) \\
	& = 2 (\mscr A(\rz I) + \mscr A(\rz{III}) +\mscr A(\rz{I}) +\mscr A(\rz{IV})) \\
	& = 8 \cdot \mscr A(\rz I) = 8 \cdot \frac 1 2= 4 \, ,
\end{align*}
wobei verwendet wurde, dass die Dreiecke kongruent zueinander sind und $\mscr A(\cdot)$ den Flächeninhalt eines Dreiecks berechnet. Analog können wir auch die übrigen acht nodalen Basisfunktionen aufstellen und damit die Einträge der Steifigkeitsmatrix
\begin{align*}
	a(\phi_5,\phi_2) = a(\phi_5,\phi_4) = a(\phi_5,\phi_6) = a(\phi_5,\phi_8) &= -1 \, , \\
	a(\phi_5,\phi_1) = a(\phi_5,\phi_3) = a(\phi_5,\phi_7) = a(\phi_5,\phi_9)& = 0
\end{align*}
berechnen. Damit ist der Einteil der Basisfunktion $\phi_5$ an der Steifigkeitsmatrix $A$ von der Form
\[
	\widetilde A = \begin{pmatrix}
		0 & -1 & 0 \\
		-1 & 4 & -1 \\
		0 & -1 & 0
	\end{pmatrix} \, .
\]
Hierbei müssen die Einträge aus $\widetilde A$ in die Matrix $A \in \R^{9 \times 9}$ an die richtige Stelle zugeordnet werden, wie durch die Formel $a_{ij} = a(\phi_i,\phi_j)$ beschrieben wird. Daher nennen wir $\widetilde A$ \textit{\idx{lokale Steifigkeitsmatrix}} bzgl. des Knoten 5.

Dieses Vorgehen müssten wir noch für die übrigen Basisfunktion analog durchführen, um die vollständige Steifigkeitsmatrix $A$ zu erhalten. Dies soll hier aber nicht weiter ausgeführt werden.
\end{bsp}

Wie man sieht, ist das Vorgehen aus Beispiel \ref{bsp:2.26} sehr aufwendig. Außerdem ist es schwer jenes auf diese Weise zu verallgemeinern, damit man es gut implementieren kann, da die Ansatzfunktionen auf das Gitter bezogen von individueller Form sind.


\begin{figure}[h!]
\begin{center}
\begin{pspicture}(-0.5,0)(7,3)
	% Skalierung
	\psset{xunit=2cm,yunit=2cm}
	
	% lokales Koordinatensystem:
	\psline{->}(0,0)(1.3,0)
	\psline{->}(0,0)(0,1.3)
	\rput(1.3,-0.12){$\xi$}
	\rput(-0.1,1.3){$\eta$}
	
	% globales Koordinatensystem:
	\psline{->}(2,0)(2.5,0)
	\psline{->}(2,0)(2,0.5)
	\rput(2.5,-0.13){$x$}
	\rput(1.9,0.5){$y$}
	
	% Referenz-Dreieck:
	\psline(0,0)(1,0)
	\psline(0,0)(0,1)
	\psline(1,0)(0,1)
	\rput(0.3,0.3){$\widetilde T$}
	
	% Punktbeschriftung:
	\rput(0,-0.13){\small (0,0)}
	\rput(0.95,-0.13){\small (1,0)}
	\rput(-0.22,1){\small (0,1)}
	
	% allgemeines Dreieck:
	\psline(2.2,0.7)(3.2,1.2)
	\psline(3.2,1.2)(3.7,0.2)
	\psline(3.7,0.2)(2.2,0.7)
	\rput(3,0.8){$T$}
	
	% Punktbeschriftung:
	\rput(1.9,0.8){\small$(x_1,y_1)$}
	\rput(3.2,1.35){\small $(x_3,y_3)$}
	\rput(3.7,0.1){\small $(x_2,y_2)$}
	
	% affine Trafo:
	\pscurve{->}(0.7,0.7)(1.4,1.2)(2.2,1)
\end{pspicture}
\end{center}
\caption{Referenzelement $\widetilde T$ für ein allgemeines Dreieck $T \in \mcal T_h$\label{abb:2.5}}
\end{figure}


Um die Berechnung der Einträge der Steifigkeitsmatrix $A$ zu verallgemeinern, betrachten wir das Referenzelement
\[
	\widetilde T \coloneqq \{ (\xi,\eta) \in \R^2 \mid 0\le \xi \le 1, 0 \le \eta \le 1-\xi\} \, .
\]
Auf diesem können wir dann die Integrale für die Ansatzfunktionen lokal berechnen, um dann die berechneten Werte affin auf ein beliebiges Dreieck $T$ zu transformieren (s. Abbildung \ref{abb:2.5}). Diese auf das Element $T$ bezogene lokale Steifigkeitsmatrix müssen wir dann durch \textit{\idx{global-local node ordering}} in die globale Steifigkeitsmatrix $A$ assemblieren. Die genaue Berechnungsvorschrift für das oben beschriebene Vorgehen wird in Kapitel \ref{kap:5} noch mal genauer hergeleitet.







\subsection{A priori Fehlerabschätzung}
\label{kap:2.3.1}

Wir wollen nun zeigen, dass durch Netzverfeinerung, d.h. Verkleinern von $h$, der Fehler zwischen der exakten Lösung $u$ und der Galerkin-Approximation $u_h$ kleiner wird.

\begin{lemma}\label{lem:2.27}
Durch $\norm \cdot _E: H^1_0(\Omega) \ra \R, \norm v_E \coloneqq (a(v,v))^{\frac 1 2}$ mit einer stetigen koerziven Bilinearform $a$ wird eine Norm auf $H_0^1(\Omega)$ definiert.
\end{lemma}

\begin{proof}
Aus der Stetigkeit und Koerzivität von $a$ folgt direkt
\begin{align}\label{eq:2.12}
	\alpha \, \norm{v}_1^2 \le \underbrace{a(v,v)}_{= \norm v_E^2} \le c \, \norm v_1^2 \, .
\end{align}
Damit ist $\norm\cdot_E$ nach oben und unten durch die Norm auf $H^1_0(\Omega)$ beschränkt und somit eine zu dieser äquivalente Norm.
\end{proof}


\begin{bem*}
\begin{enumerate}[(a)]
\item Die Norm $\norm \cdot_E$ bezeichnen wir als \textit{\idx{Energie-Norm}}. Sie gibt für die von uns später in der Strukturmechanik betrachtete Bilinearform  die Verzerrungsenergie eines Kontinuums an.
\item Für die Bilinearform
\[
	a(u,v) = \int_\Omega \nabla u \nabla v \, dx
\]
mit $u,v \in H^1_0(\Omega)$ gilt dann $\norm\cdot_E = \abs\cdot_1$ (s. Bemerkung \ref{bem:A.6}).
\item Auf $H^1(\Omega)$ wäre $\norm{\cdot}_E$ also nur eine Halbnorm, da konstante Funktionen $v = c$ auch die Norm $\norm{v}_E = 0$ hätten.
\end{enumerate}
\end{bem*}


\begin{satz}\label{satz:2.28}
Die \idx{Galerkin-Approximation} $u_h$ ist die beste Approximation von $u$ bzgl. der \idx{Energie-Norm}, also
\[
	\norm{u-u_h}_E = \inf_{v \in V_h} \norm{u-v}_E \, .
\]
\end{satz}

\begin{proof}
Zunächst betrachten wir die exakte und approximierte Variationsgleichung \eqref{eq:2.8} und \eqref{eq:2.9}, d.h.
\begin{align}\label{eq:2.13}
	a(u,v) &= F(v) \quad \ \, \forall \, v \in H \, , \\
	\label{eq:2.14}
	a(u_h,v_h) &= F(v_h) \quad \forall \, v_h \in V_h \, .
\end{align}
Da $V_h \subset H$ ist, gilt \eqref{eq:2.13} auch für alle $v_h\in V_h$. Ersetzen wir dies  in \eqref{eq:2.13} und subtrahieren \eqref{eq:2.13} und \eqref{eq:2.14}, so erhalten wir
\begin{align}\label{eq:2.15}
	a(u-u_h,v_h ) = 0 \quad \forall \, v_h \in V_h \, .
\end{align}
Damit rechnen wir für ein beliebiges $v \in V_h$ einfach nach:
\begin{align*}
	\norm{u-u_h}_E^2 & = a(u-u_h,u-u_h) \\
	& = a(u-u_h, u-v+v-u_h) \\
	& = a(u-u_h,u-v)+\underbrace{a(u-u_h,\underbrace{v-u_h}_{\in V_h})}_{=0\text{ wegen \eqref{eq:2.14}}} \\
	& = a(u-u_h,u-v) \\
	&\!\!\!\:\!\:\!\:\!\:\! \stackrel{\scriptsize\text{C.S.}}\le \norm{u-u_h}_E \norm{u-v}_E 
\end{align*}
und damit folgt nach Division $\norm{u-u_h}_E \le \norm{u-v}_E$, was zu zeigen war.
\end{proof}


\begin{bem*}
Die Gleichung \eqref{eq:2.15} drückt aus, dass die Verbindung $u-u_h$ orthogonal zum Raum $V_h$ steht und wird daher auch \textit{\idx{Galerkin-Orthogonalität}} genannt. Diese wird bei Hindernisproblemen im Allgemeinen nicht mehr erfüllt, da diese, wie wir später sehen werden, nicht mehr auf Variationsgleichungen führen.
\end{bem*}


\begin{satz}[Céa]\label{satz:2.27}
Der Fehler der \idx{Galerkin-Approximation} $u_h$ hat in der $H^1$-Norm die Eigenschaft
\[
	\norm{u-u_h}_1 \le \tilde c \inf_{v\in V_h} \norm{u-v}_1 \, .
\]
\end{satz}

\begin{proof}
Aus \eqref{eq:2.12} und Satz \ref{satz:2.28} folgt
\[
	\norm{u-u_h}_1 \le \(\frac 1\alpha\)^{\frac 1 2} \norm{u-u_h}_E \le \(\frac 1\alpha\)^{\frac 1 2} \norm{u-v}_E \le \(\frac c\alpha\)^{\frac 1 2} \norm{u-v}_1  \, .
\]
Damit folgt die Behauptung mit $\tilde c :=  \sqrt{\frac c\alpha}$.
\end{proof}


Nun kommen wir zum zentralen Satz dieses Unterkapitels, mit dem man direkt die gewünschte Aussage folgern kann. Vergleiche hierzu auch \cite{BraeFEM} Kapitel II, \S6, Satz 6.4.

\begin{theorem}[\idx{Approximationssatz für Interpolationen}]\label{theorem:2.28}
Es sei $k \ge 2$ und $\mcal T_h$ eine quasi-uniforme Triangulierung\index{Triangulierung!quasi-uniform} von $\Omega$. Dann gilt für die Interpolation $I_h$ auf die stetigen, stückweise durch Polynome vom Grad $k-1$ gegebenen Funktionen mit einer von $\Omega, \kappa$ und $k$ abhängigen Kontanten $c$ die a priori Fehlerabschätzung
\[
	\norm{u-I_hu}_m \le c h^{k-m} \abs{u}_k
\]
für $u \in H^k(\Omega)$ und $0\le m\le k$.
\end{theorem}

\begin{proof}
Für den Beweis würden wir noch weitere Ausführungen über affine Transformationen benötigen, die wir hier nicht weiter aufführen wollen. Der komplette Beweis ist in \cite{BraeFEM} auf Seite 75ff einzusehen.
\end{proof}


Für $k=2$, also lineare Polynome, und $m=1$ (die Norm in $H^1$) gilt dann
\begin{align}\label{eq:2.16}
	\norm{u-I_hu}_1 \le c h \abs{u}_2
\end{align}
für $u \in H^2(\Omega)$.


\begin{kor}\label{kor:2.31}
Für lineare $C^0$-Elemente gilt bzgl. der Galerkin-Approximation $u_h$ die a priori Fehlerschätzung für unser Modellproblem \eqref{eq:DP}
\[
	\norm{u-u_h}_1 \le \tilde c h \abs{u}_2 \, .
\]
\end{kor}

\begin{proof}
Mit Theorem \ref{theorem:2.28} und Satz \ref{satz:2.27} folgt
\begin{align*}
	\norm{u-u_h}_1& \le \(\frac {c_1}\alpha\)^{\frac 1 2} \inf_{v\in V_h} \norm{u-v}_1
	 \le \(\frac {c_1}\alpha\)^{\frac 1 2}  \norm{u-I_h u}_1 \\
	& \!\!\!\! \:\!\: \!\stackrel{\scriptsize \eqref{eq:2.16}}\le \(\frac {c_1}\alpha\)^{\frac 1 2} c_2 h \abs{u}_2 \, .
\end{align*}
Mit $u\in H^2(\Omega)$ und $\tilde c := \(\frac {c_1}\alpha\)^{\frac 1 2} c_2 $ folgt dann die Behauptung.
\end{proof}


Mit Korollar \ref{kor:2.31} gilt also, dass für $h \ra 0$ die Galerkin-Approximation $u_h$ gegen die exakte Lösung $u$ konvergiert. Es ist also sinnvoll das Netz zu verfeinern, allerdings bringt jede Verfeinerung auch mehr Knoten und damit ein größeres Gleichungssystem mit sich. Daher stellt sich die Frage, ob es sinnvoll ist, nur einzelne Teile des Gitters zu verfeinern, womit wir uns im Kapitel \ref{kap:2.4} beschäftigen wollen.




\section{Adaptive Verfeinerungsstrategien}
\label{kap:2.4}

Wir wollen nun betrachten, wie sich das Gitter geschickt verfeinern lässt, so dass die Anzahl der Knoten im Vergleich zur Verringerung des Fehlers hinreichend groß ist. Diese Verfeinerung im $n$-ten Schritt geschieht adaptiv in Abhängigkeit des aktuellen Gitters $\mcal T_{h_n}$ bzw. der aktuellen Lösung $u_{h_n}$. Hierbei wird \textit{a posteriori} der Fehler im nächsten Schritt $\norm{u-u_{h_{n+1}}}$ mit dem Fehler im aktuellen $\norm{u-u_{h_n}}$ verglichen und daraus die notwendige Größe der Verfeinerung abgeschätzt. Da die Lösung $u$ für ein Problem, wie  schon beschrieben, nicht bekannt oder berechenbar sein muss, gilt es den oben beschriebenen Fehler zu schätzen. Für solche a posteriori Fehlerschätzer gibt es mehrere Ansätze.


\subsection{A posteriori Fehlerschätzer}
\label{kap:2.4.1}

Wie auch in \cite{BraeFEM} Kapitel III, \S 8, Seite 176 genauer beschrieben, gibt es verschiedene Arten von \idx{a posteriori Fehlerschätzer}n:

\begin{enumerate}[(a)]
\item Residuale Schätzer
\item Schätzung über ein lokales Neumann-Problem
\item Schätzung über ein lokales Dirichlet-Problem
\item Schätzung durch Mitteilung
\item Hierarchische Schätzer
\end{enumerate}

Als Fehlerschätzer für Hindernis- oder auch Kontaktprobleme sind häufig residuale Schätzer zu finden. Wir wollen uns in dieser Arbeit mit hierarchischen Fehlerschätzern beschäftigen.

Die Idee dabei ist, dass wir den Fehler durch eine genauere Lösung aus einem "`besseren"' Ansatzraum abschätzen, in dem Sinne, dass für den Ansatzraum $V_h$, in dem die berechnete Lösung $u_h$ liegt, gilt:
\begin{align}
	V_h \subset W_h \text{ mit } W_h = V_h \oplus Z_h \, ,
\end{align}
wobei bzgl. der Obermenge $W_h$ die Lösung $u_h^W$ genauer ist als $u_h$, d.h. die \textit{\idx{Saturationseigenschaft}}
\begin{align}\label{eq:2.18}
	a(u-u_h^W,u-u_h^W) \le \beta^2 a(u-u_h,u-u_h)
\end{align}
mit einer Konstanten $0\le \beta < 1$ erfüllt.

\begin{bem}
Da wir zur Berechnung unser \idx{Galerkin-Approximation} den Raum $V_h = \mcal S_h$ der linearen Ansatzfunktionen verwenden werden, wählen wir später als Hierarchie
\[
	W_h = \mcal Q_h \coloneqq \{v \in C^0(\Omega) \mid v|_T \in \mcal P_2 \text{ für } T \in \mcal T_h, v|_{\partial\Omega} = 0\} \, ,
\]
den Raum der quadratischen Ansatzfunktionen. Daraus lässt sich dann der Raum $Z_h$ ableiten.
\end{bem}


Wir sollen nun aber nicht die exakte (bessere) Approximation $u_h^W\in W_h$ berechnen, da dies ein zu großer Aufwand wäre, sondern schränken das sogenannte \textit{\idx{Defektproblem}} lokal auf die Erweiterung $Z_h$ ein, d.h.
\begin{align}\label{eq:2.19}
	a(u_h^W,z_h) = F(z_h) \quad \forall \, z_h \in Z_h \, ,
\end{align}
wobei in \eqref{eq:2.19} $u_h^W$ aufgeteilt werden kann, da $W_h$ aus direkter Summe von $V_h, Z_h$ entsteht, d.h. $u_h^W = u_h+e_h$ mit $u_h \in V_h, e_h\in Z_h$. Da die Lösung $u_h$ in jedem Schritt bekannt ist, lässt sich das lokale Defektproblem \eqref{eq:2.19} schreiben als
\begin{align}\label{eq:2.20}
	e_h \in Z_h : \quad a(e_h,z_h) = \underbrace{F(z_h) - a(u_h,z_h)}_{= \widetilde F(z_h)} \quad \forall \, z_h \in Z_h \, .
\end{align}
Man kann zeigen, dass für die Lösung $e_h$ aus \eqref{eq:2.20} unter der Bedingung \eqref{eq:2.18} der Term $a(e_h,e_h)$ beschränkt ist und daher gibt  der auf ein Dreieck $T$ bezogene lokale Anteil $\eta_T \coloneqq a_T(e_h,e_h)^{\frac 12}$ den hierarchischen Fehlerschätzer an.

Damit lässt sich $a(e_h,e_h)$ in die lokalen Anteile $\eta_T$ aufteilen, so dass
\[
	\sum_{T\in \mcal T_h} a_T(e_h,e_h) = a(e_h,e_h) \, .
\]
Daher verwenden wir als adaptive Strategie: Verfeinere alle Dreiecke $T\in \mcal T_h$, deren lokaler Fehleranteil größer gleich dem skalierten Gesamtfehler sind, also
\[
	\eta_T \ge \sigma \(\sum_{T\in \mcal T_h} \eta_T^2\)^{\frac 12} ,
\]
wobei $\sigma \in (0,1)$ ist. Wählen wir $\sigma$ sehr klein, so werden viele Dreiecke verfeinert, da auch kleinere lokale Fehleranteile die Ungleichung erfüllen. Umgekehrt gilt für ein großes $\sigma$, dass man viele Adaptionsschritte benötigt, um einen hinreichend kleinen Fehler zu erhalten, da nur wenige Dreiecke pro Verfeinerungsschritt ausgewählt werden.

Die Idee der Hierarchie bzgl. der Räume wollen wir in Kapitel \ref{Kap:4} auch auf Variationsprobleme unter Nebenbedingung anwenden, um einen a posteriori Schätzer herzuleiten.


\subsection{Verfeinerung des Netzes}
\label{kap:2.4.2}

Auf der Grundlage des a posteriori Schätzers müssen die ausgewählten Dreiecke verfeinert werden. Dabei ist es essentiell, dass eine konforme Triangulierung\index{Triangulierung!konform} auch konform bleibt, was durch Verwendung von verschiedenen Verfeinerungsmethoden möglich ist (s. Abbildung \ref{abb:2.6}).



\begin{enumerate}[(i)]
\item \textit{Rote} (\textit{reguläre})\index{Verfeinerung!regulär}\index{Verfeinerung!rot} Verfeinerung,
\item \textit{Grüne} Verfeinerung,\index{Verfeinerung!grün}
\item \textit{Blaue} Verfeinerung.\index{Verfeinerung!blau}
\end{enumerate}


\begin{figure}[h]
\begin{center}
\begin{pspicture}(0,0)(5,3)
	\psset{linewidth=0.5pt}
	% die Dreiecke:
	\pspolygon(-2,0)(-1,3)(0.5,1.7)
	\pspolygon(1.5,0)(2.5,3)(4,1.7)
	\pspolygon(5,0)(6,3)(7.5,1.7)
	\psdots(-2,0)(-1,3)(0.5,1.7)(1.5,0)(2.5,3)(4,1.7)(5,0)(6,3)(7.5,1.7)
	
	% Verfeinerungen:
	\pspolygon(-1.5,1.5)(-0.25,2.35)(-0.75,0.85)
	\psdots[linecolor=red](-1.5,1.5)(-0.25,2.35)(-0.75,0.85)
	\psline(4,1.7)(2,1.5)
	\psdot[linecolor=green](2,1.5)
	\psline(7.5,1.7)(5.5,1.5)
	\psline(6.25,0.85)(5.5,1.5)
	\psdots[linecolor=blue](5.5,1.5)(6.25,0.85)
\end{pspicture}
\caption{Verfeinerungen von Dreiecken\label{abb:2.6}}
\end{center}
\end{figure}

Bei der regulären Verfeinerung werden die drei Mittelpunkte der Kanten eines  Dreiecks miteinander verbunden. Der Vorteil dabei ist, dass die Winkel der entstandenen Dreiecke identisch zu den vorherigen Winkeln sind. Damit bleibt das Verhältnis zwischen dem Radius des Um- zum Innenkreises $\frac{h_T}{\rho_T}$, was den Parameter für die Quasi-Uniformität angibt, gleich bleibt.

Beim regulären Verfeinern können allerdings in anliegenden Dreiecken, die bzgl. des a posteriori Fehlerschätzers nicht verfeinert werden müssen, hängende Knoten (vgl. Abbildung \ref{abb:2.2}) entstehen, wodurch eine nichtzulässige Triangulierung entstehen würde. Um die hängenden Knoten zu eliminieren, kann Verfeinerungsmethode (ii) und (iii) verwendet werden.

Bei der grünen Verfeinerung\index{Verfeinerung!grün} wird der Mittelpunkt genau einer Kante mit dem gegenüberliegendem Eckpunkt verbunden.

Bei der blauen Verfeinerung\index{Verfeinerung!blau} verbindet man den Mittelpunkt der längsten Kante im Dreieck mit dem gegenüberliegendem Eckpunkt und einem weiteren Mittelpunkt einer anderen Kante.

Matlab verwendet diese Verfeinerungsstrategien in der Methode \textit{refinemesh}\index{Matlab!refinemesh}. Diese Methode werden wir bei der Implementierung der numerischen Beispiele auch verwenden und daher wollen wir das Vorgehen von Matlab kurz vorstellen. Ist $\widetilde{\mcal T_h}$ die Menge der zu verfeinernden Dreiecke, dann werden folgende Schritte durchgeführt:
\begin{enumerate}[(i)]
\item Halbiere alle Kanten der ausgewählten Dreiecke $T \in \widetilde{\mcal T_h}$.
\item Halbiere jeweils die längste Kante aller Dreiecke, die schon eine geteilte Kante haben. 
\item Bilde die neuen Dreiecke nach folgender Strategie:
\begin{enumerate}[(a)]
\item Wenn alle drei Kanten eines Dreiecks geteilt sind, verwende die reguläre (rote) Verfeinerung.
\item Sind genau zwei Seiten halbiert, so verwende die blaue Verfeinerung.
\item Ist genau eine Kante eines Dreiecks halbiert, dann benutze die grüne Verfeinerung.
\end{enumerate}
\end{enumerate}





\section{Einführung in die Strukturmechanik}
\label{kap:2.5}


Um in Kapitel \ref{kap:3.2} Kontaktprobleme bzgl. der Mechanik beschreiben zu können, wollen wir in diesem Kapitel eine kurze Einführung in die Kontinuumsmechanik geben.

\begin{defi}[\idx{Kontinuum}]
Ein \textit{Kontinuum} ist eine Teilmenge $\mathscr{B}\subset \R^3$, dessen  Punkte stetig verteilt sind. Den Punkten werden gewisse Materialeigenschaften zugewiesen.
\end{defi}


\subsection{Kinematik}
\label{kap:2.5.1}

Um die Kinematik für ein \idx{Kontinuum} zu beschreiben, betrachten wir dieses in zeitlich abhängigen Zuständen. Dabei unterscheiden wir zwischen den \textit{materiellen Punkten}\index{materielle Punkte} des Kontinuums, die mit $\bs X = (X_1,X_2,X_3)$ beschrieben werden, und den \textit{räumlichen Punkten}\index{räumliche Punkte} $\bs x = (x_1,x_2,x_3)$.

\begin{defi}[\idx{Konfiguration}]
Eine zu jedem Zeitpunkt $t$ differenzierbare und stetige Zuordnung $\bs x = \varphi(\bs X, t)$ nennen wir \textit{Konfiguration}. Die Konfiguration zum Zeitpunkt $t = t_0$ nennen wir \textit{Ausgang-} oder \textit{\idx{Referenzkonfiguration}}\index{Ausgangskonfiguration}\index{Konfiguration!Ausgangs-}\index{Konfiguration!Referenz-}, die Konfiguration zum aktuellen Zeitpunkt $t$ \textit{\idx{aktuelle Konfiguration}} oder \textit{\idx{Momentankonfiguration}}\index{Konfiguration!aktuelle}\index{Konfiguration!Momentan-}.
\end{defi}


\begin{figure}[h!]
\begin{center}
\begin{pspicture}(0,-1)(4,3)
	% materielle Punkte:
	\psccurve(-2,0)(0,0.3)(-0.6,2.3)(-2.2,1.2)
	\psdot(-1,0.6)
	\rput(-1.3,0.5){$\bs X$}
	\rput(-0.6,1.8){$\mscr B$}
	
	% Momentankonfiguration:
	\psccurve(3,1)(5,0.5)(6.4,2)(5.4,3)(4,2)
	\rput(6,2){$\mscr B$}
	\psdot(4,1)
	\rput(4.2,0.9){$\bs x$}
	
	% Koordinatensystem:
	\psline{->}(2,0)(3,0)
	\psline{->}(2,0)(2,1)
	\psline{->}(2,0)(1.3,-0.42)
	\rput(1.6,-0.5){$\bs e_1$}
	\rput(2.25,1){$\bs e_3$}
	\rput(3,-0.2){$\bs e_2$}
	
	% Zuordnung:
	\pscurve{->}(0.5,1.5)(1.8,2.5)(3.5,2.1)
	\rput(2,2.8){$\varphi$}
\end{pspicture}
\caption{Konfiguration eines Kontinuums $\mscr B$\label{abb:2.7}}
\end{center}
\end{figure}


\begin{bem}
Wir können der Einfachheit halber zu einem Startzeitpunkt $t = t_0$ die materiellen Punkte durch die räumlichen Koordinaten der Ausgangskonfiguration $\bs X = \bs x(t_0)$ beschreiben. So liegt der Körper $\mscr B$ aus Abbildung \ref{abb:2.7} im selben Koordinatensystem.
\end{bem}


Damit ist die Bewegung (oder auch Deformation) eines Kontinuums als zeitlich stetige Folge von Konfigurationen $\bs x = \bs x(\bs X,t)$ zu verstehen. Hierfür gibt es zwei grundlegende Betrachtungsweisen, die \textit{{Lagrange'sche}}\index{Betrachtungsweise!Lagrange'sche} und die \textit{Euler'sche Betrachtungsweise}\index{Betrachtungsweise!Euler'sche}. Bei der Lagrange'schen Betrachtung ist der Beobachter mit dem materiellen Punkt $\bs X$ verbunden und misst alle Änderungen des Kontinuums an diesem Punkt; jene wird auch \textit{materielle Betrachtungsweise}\index{Betrachtungsweise!materielle} genannt. Bei der Euler'schen Betrachtungsweise befindet sich der Beobachter am räumlichen Punkt $\bs x$ und misst zu jedem Zeitpunkt $t$ die Änderungen am Punkt $\bs x$, die sich durch das Passieren von Teilchen $\bs X$ ergeben; jene wird auch \textit{räumliche Betrachtungsweise}\index{Betrachtungsweise!räumliche} genannt.


Um die Deformation eines Körpers $\mscr B$ nun beschreiben zu können, betrachten wir die Veränderung von Linienelementen $d \bs x$ bzgl. der Ausgangs- und Momentankonfiguration. Wir beschreiben den sogenannten \textit{\idx{Deformationsgradient}} $\bs F$ durch den Faktor, der zwischen der Deformation dieser Linienelemente liegt, d.h.
\begin{align}\label{eq:2.21}
	d\bs x = \bs F \, d\bs X \, .
\end{align}
Damit ergibt sich der Deformationsgradient als
\begin{align}\label{eq:2.22}
	\bs F = \frac{\partial \bs x}{\partial \bs X} = \Grad \bs x \, , 
\end{align}
dies ist der Gradient bzgl. der materiellen Betrachtung\index{Betrachtungsweise!materielle} und wird daher auch \textit{materieller Deformationsgradient} genannt. Analog können wir den \textit{räumlichen Deformationsgradienten} erhalten als
\begin{align}\label{eq:2.23}
	\bs F^{-1} = \frac{\partial \bs X}{\partial \bs x} = \grad \bs X \, .
\end{align}
Die Deformationsgradienten sind zweistufige Tensoren, die sich jeweils auf die Referenz- und Momentankonfiguration bzgl. der Basen beziehen.

Damit können wir nun die Verzerrung eines Kontinuums mittels des Deformationsgradienten ausdrücken. Um jene zu beschreiben, betrachten wir die Längenänderung zwischen den Linienelementen $d\bs X$ und $d\bs x$.
\begin{align*}
	\norm{d\bs x}^2-\norm{d\bs X}^2 &= (d\bs x)^T d\bs x - (d\bs X)^Td\bs X \\
	& \! \! \!\!\: \!\!\: \stackrel{\scriptsize \eqref{eq:2.21}}= (d\bs X)^T \bs F^T \bs F d\bs X - (d\bs X)^Td\bs X \\
	& = (d\bs X)^T (\bs F^T \bs F- \bs 1) d\bs X \, ,
\end{align*}
wobei $\bs 1$ den zweistufigen Einheitstensor beschreibt. Also lässt sich die Verzerrung bzgl. der Ausgangskonfiguration durch den \textit{Green-Lagrange'schen Verzerrungssensor}\index{Verzerrungssensor!Green-Lagrange}
\begin{align}\label{eq:2.24}
	\bs E = \frac 1 2 (\bs C- \bs 1)
\end{align}
mit dem \textit{rechten Cauchy-Green-Tensor}\index{Cauchy-Green-Tensor!rechter} $\bs C = \bs F^T\bs F$ beschreiben.


\begin{bem}
Dies ist natürlich nur eine Möglichkeit. Wichtig für die Wahl eines Verzerrungsmaßes ist jedoch, dass für reine Translation oder Rotation dieses Null wird. 

Daher ist beispielsweise $\bs F - \bs 1$ als Verzerrungsmaß unbrauchbar. Der Deformationsgradient lässt sich in Rotation $\bs R$ und Streckung $\bs U$ durch $$\bs F = \bs R \cdot \bs U$$ aufteilen. Sollten wir nun eine reine Rotation betrachten, so ist $\bs U = \bs 1$ und damit $\bs F = \bs R$, was nicht zwangsläufig der Einheitstensor sein muss.
\end{bem}


Wenn wir einen Punkt bzgl. seiner Ausgangs- und Momentankonfiguration vergleichen, so erhalten wir die \textit{\idx{Verschiebung}}
\begin{align}\label{eq:2.25}
	\bs u (\bs X,t) = \bs x(\bs X,t)-\bs X \, .
\end{align}
Damit ergeben sich analog zum materiellen und räumlichen Deformationsgradienten\index{Deformationsgradient!materiell}\index{Deformationsgradient!räumlich} \eqref{eq:2.22} und \eqref{eq:2.23} mit \eqref{eq:2.25} der materielle und räumliche \textit{Verschiebungsgradient}\index{Verschiebungsgradient!materiell}\index{Verschiebungsgradient!räumlich}:
\begin{align*}
	\bs H& = \Grad \bs u = \frac{\partial(\bs x - \bs X)}{\partial \bs X} = \frac{\partial\bs x}{\partial \bs X} - \frac{\partial \bs X}{\partial \bs X} = \bs F - \bs 1 \, ,  \\
	\bs h& = \grad \bs u = \frac{\partial(\bs x - \bs X)}{\partial \bs x} = \frac{\partial\bs x}{\partial \bs x} - \frac{\partial \bs X}{\partial \bs x} = \bs 1 - \bs F^{-1} \, .
\end{align*}
Damit ergibt sich beispielsweise für den Green-Lagrange'schen Verzerrungstensor \index{Verzerrungstensor!Green-Lagrange} \eqref{eq:2.24} bzgl. des materiellen Verschiebungsgradienten die Beziehung
\begin{align}\label{eq:2.26}
\begin{aligned}
	\bs E& = \frac 12 (\bs C - \bs 1) = \frac 1 2 (\bs F^T\bs F - \bs 1) = \frac 12((\bs 1+\bs H)^T(\bs 1+\bs H)-\bs 1) \\
	& = \frac 12 (\bs 1 + \bs H^T + \bs H +\bs H^T \bs H - \bs 1) = \frac 12(\bs H^T+\bs H+\bs H^T\bs H) \, .
\end{aligned}
\end{align}
Wenn wir von kleinen Deformationen ausgehen, so ist es sinnvoll für die spätere numerische Berechnung, die nichtlinearen Verzerrungsgrößen (wie in \eqref{eq:2.26})  zu linearisieren. Dies können wir mithilfe von Taylor und der \idx{Gâteaux-Ableitung} \eqref{eq:A.1} bewerkstelligen. Für eine vektor- oder tensorwertige Funktion $\bs A$ gilt dann
\begin{align}\label{eq:2.27}
	\operatorname{lin}(\bs A)_{\bs x,\bs u} = \bs A(\bs x) + \frac d{ d\eps} (\bs A(\bs x + \eps \bs u))\bigg|_{\eps = 0} \, .
\end{align}
Wenden wir also \eqref{eq:2.27} auf den Green-Lagrange'schen Verzerrungstensor \eqref{eq:2.26} an, so erhalten wir die Linearisierung:
\begin{align*}
	\operatorname{lin}(\bs E)_{\bs X,\bs u} & = \underbrace{\bs E(\bs X)}_{=0} + \frac d{d\eps} (\bs E(\bs X+\eps \bs u))\bigg|_{\eps = 0} \\
	& = \frac 12 \frac d{d\eps} \(\(\frac{\partial(\bs X+\eps\bs u)}{\partial \bs X}\)^T \frac{\partial(\bs X+\eps\bs u)}{\partial \bs X}-\bs  1\)\Bigg|_{\eps=0} \\
	& = \frac 12 \frac d{d\eps} \((\bs 1+\eps \bs H)^T (\bs 1+\eps \bs H) - \bs 1\) \bigg|_{\eps=0} \\
	& = \frac 12 \frac d{d\eps} (\bs 1+\eps \bs H^T+\eps\bs H+ \eps^2 \bs H^T\bs H-\bs 1) \bigg|_{\eps=0} \\
	& = \frac 12 (\bs H^T + \bs H + 2\eps \bs H^T\bs H)\bigg|_{\eps=0} = \frac 12(\bs H^T+ \bs H) \eqqcolon \bs \eps \, .
\end{align*} 

Wir wollen der Einfachheit halber in dieser Arbeit von kleinen Deformationen ausgehen, dann gilt $\bs H \approx \bs h = \nabla \bs u$. Damit werden wir also immer die linearisierten Verzerrungstensor
\begin{align}\label{eq:2.28}
	\bs \eps = \frac 1 2 (\nabla^T \bs u + \nabla \bs u)
\end{align}
verwenden.


\subsection{Kinetik}
\label{kap:2.5.2}

Da wir von kleinen Deformationen ausgehen, betrachten wir den \textit{\idx{Cauchy-Spannungstensor}} $\bs\sigma$, der die aktuelle Kraft bzgl. der Querschnittsfläche in der Momentankonfiguration setzt. Wie in \cite{WriggersContact} Kapitel 3.2.2 beschrieben, gilt das \textit{Cauchy Theorem}, das besagt, dass die Spannung $\bs t$ auf einer Schnittfläche eines beliebigen Schnittes im Körper $\mscr B$ gleich der Spannung $\bs\sigma$ in Normaleinrichtung $\bs n$ ist, d.h.
\begin{align}\label{eq:2.29}
	\bs t = \bs \sigma \cdot \bs n \, .
\end{align}
Mit den Bilanzgleichungen für das Momentengleichgewicht kann man herleiten, dass der Cauchy-Spannungstensor symmetrisch ist, also $\bs \sigma^T = \bs \sigma$ gilt.

Betrachten wir nun eine Volumenkraft $\bar{\bs b}$ und eine Oberflächenlast $\bar{\bs t}$, die auf den Körper $\mscr B$ wirken, so erhalten wir das (\textit{globale}) Kräftegleichgewicht
\begin{align}\label{eq:2.30}
	\int_{\Omega} \bar{\bs b} \, dv + \int_{\partial \Omega} \bar{\bs t} \, da = \bs 0 \, ,
\end{align}
wobei $\Omega \subset \mscr B \subset \R^3$ eine Teilmenge des Kontinuums $\mscr B$ beschreibt. Mit dem Cauchy Theorem \eqref{eq:2.29} und dem \idx{Satz von Gauß} lässt sich \eqref{eq:2.30} als Integral über $\Omega$ schreiben durch
\begin{align}\label{eq:2.31}
	\int_{\Omega} \bar{\bs b} + \div \bs\sigma \, dv = \bs 0 \, .
\end{align}
Die Gleichung \eqref{eq:2.31} muss nach dem Schnittprinzip auf jeder Teilmenge $\Omega$ gelten, was nur erfüllt werden kann, wenn der Integrand Null ist. Damit erhalten wir die sogenannte \textit{starke} oder auch \textit{lokale} Form des Gleichgewichts
\begin{align}\label{eq:2.32}
	\div \bs \sigma + \bar{\bs b} = \bs 0 \text{ auf } \Omega \, .
\end{align}
Wir wollen in der Arbeit von einem konstanten Wärmefeld ausgehen, sodass wir thermodynamische Prozesse vernachlässigen können.

\subsection{Konstitutive Gleichungen und Prinzipien}
\label{kap:2.5.3}


Auch für die konstitutiven Gleichungen (Materialannahmen etc.) machen wir uns zu Nutze, dass wir von kleinen Deformationen ausgehen. Wir gehen daher von einem linear elastischen Material aus, d.h. dass Spannung und Verzerrung in einem linearen Zusammenhang stehen. Hierfür werden wir in der Arbeit das \textit{Hooke'sche Materialmodell}\index{Hooke'sches Materialmodell} verwenden:
\begin{align}\label{eq:2.33}
	\bs \sigma = \mcal C :\bs \eps = 2 \mu \bs \eps + \lambda (\tr \bs \eps) \bs I \, ,
\end{align}
wobei $\lambda, \mu$ die \textit{\idx{Lamé-Konstanten}} darstellen. Dabei handelt es sich um materialabhängige Parameter, die im Zusammenhang mit dem \idx{Elastizitätsmodul} $E$ und der Querkontraktionszahl $\nu$ stehen.
\[
	\lambda = \frac{\nu E}{(1+\nu)(1-2\nu)} \, ,\quad \nu = \frac E{2(1+\nu)} \, .
\]
Der Materialtensor $\mcal C$ ist 4-stufig und mit "`:"' ist das doppelt verjüngende Skalarprodukt, das in Anhang \ref{anhang:C} beschrieben wird, gemeint. Es sei bemerkt, dass $\nu$ gleich dem \idx{Schubmodul} $G$ ist.


Da wir in dieser Arbeit immer $\Omega \subset \R^2$ annehmen wollen, wollen wir an dieser Stelle noch zwei Prinzipien zur Behandlung von dreidimensionalen strukturmechanischen Problemen im $\R^2$ vorstellen (vgl. \cite{WriggersFEMSkript} Kapitel 5.4.1 und 5.4.2).

Ist beispielsweise die dritte Richtung dünn gegenüber den anderen zwei, so können wir das Prinzip des \textit{ebenen Spannungszustand} verwenden. Hierbei ist die Spannung in der dritten Richtung $\sigma_{33} = 0$. Dies impliziert jedoch nicht, dass es in dieser Richtung keine Verzerrung geben muss. Aus dem Hooke'schen Materialgesetz \eqref{eq:2.33} errechnen wir nämlich, dass die Verzerrung $\eps_{33}$ abhängig von $\eps_{11}$ und $\eps_{22}$ ist.
\[
	\eps_{33} = -\frac \nu{1-\nu} (\eps_{11}+\eps_{22}) 
\]
Damit lässt sich der Materialtensor $\mcal C$ als Matrix schreiben, wenn wir die \textit{\idx{Voigt-Notation}} bzgl. des Spannungs- und Verzerrungstensors verwenden.


Analog erhalten wir den \textit{ebenen Verzerrungszustand}\index{ebener Verzerrungszustand}, indem wir die Verzerrung in die dritte Richtung $\eps_{33} = 0$ setzen. Anschaulich heißt das, dass die dritte Richtung unendlich ausgedehnt ist. Auch hier ergeben sich Abhängig-keiten zwischen den Spannungen:
\[
	\sigma_{33} = \nu(\sigma_{11}+\sigma_{22}) \, ,
\]
was einerseits bedeutet, dass die Spannung in der dritten Richtung nicht zwangsläufig Null sein muss und andererseits uns wieder Einträge aus dem Materialtensor $\mcal C$ eliminieren lässt.



\newpage

%%% Local Variables: 
%%% mode: latex
%%% TeX-master: "Skript"
%%% End: 
