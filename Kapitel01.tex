\newchapter{Einleitung}
\label{kap:1}

\begin{itemize}
\item Thema (worum geht es?) $\ra$ Fehlerabschätzung $\ra$ analytische Lösung oftmals nicht bekannt und damit Fehlerschätzer interessant
\item[$\ra$] in FEM soll Lösung genauer mit weniger Rechenzeit sein, daraus folgt Anwendung adaptiver Verfahren mit verschiedenen Fehlerschätzern
\item Lücke zum neuen (Kontaktproblematik) füllen in dieser Arbeit
\item[$\ra$] Übertragung unseres Fehlerschätzers auf Kontaktprobleme, wie und warum?! $\ra$ möglicher Grund: Hindernisprobleme beinhalten Kontaktbereiche (später für Kapitel 4 interessant)
\item[wichtig:] Vorgehen einer adaptiven Verfeinerungsstrategie mit "`solve $\ra$ estimate $\ra$ ...."' umschreiben
\item Struktur der Arbeit
\end{itemize}


\newpage

%%% Local Variables: 
%%% mode: latex
%%% TeX-master: "Skript"
%%% End: 
