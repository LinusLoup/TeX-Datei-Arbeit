\newchapter{Ein hierarchischer Fehlerschätzer für Hindernisprobleme}
\label{kap:4}

\begin{itemize}
\item Vergleich Hindernisprobleme zu Kontaktproblemen $\ra$ warum gerade dieser Fehlerschätzer bei Hindernis- bzw. Kontaktproblemen
\end{itemize}

Dieses Kapitel basiert größtenteils auf \cite{ZouVee}.


\section{Herleitung eines a posteriori hierarchischen Fehlerschätzers}
\label{kap:4.1}

\begin{itemize}
\item der Einfachheit halber gehen wir von folgendem Sachverhalt aus
\begin{vor}
Das Hindernis  wird durch eine stückweise lineare stetige Funktion $\psi$ beschrieben.
\end{vor}

\item nicht nichtstetige oder auch glatte Hindernisse sind analoge Aussagen, aber schwerer, beweisbar
\end{itemize}






\subsection{Diskretisierung}
\label{kap:4.1.1}

\begin{itemize}
\item $\mcal B_h$ sei eine nodale Basis bzgl. einer quasi-uniformen Triangulierung $\mcal T_h$  für $\mcal S_h$ (s. auch Kapitel \ref{kap:2}), $K_h$ wie in Kapitel \ref{kap:3.1.3}
\[
	K_h  = \{v_h \in \mcal S_h \with v_h (p) \ge \psi(p) \, \forall \, p \in \mcal N \cap \Omega\} \, , 
\]
wobei $\mcal N$ wieder die Menge der Knoten von $\mcal T_h$ darstellt.

\item betrachte wieder die diskrete Variationsungleichung \eqref{eq:3.12}: Finde $u_h \in K_h$ mit
\[
	a(u_h,v_h-u_h) \ge (f,v_h-u_h) \quad \forall \, v_h \in K_h\, .
\]

\item oder äquivalent die Minimierung des Funktionals $J(v) = \frac 1 2 a(v,v)-(f,v)$ über $K_h$, d.h.
\[
	u_h \in K_h : J(u_h)\le J(v_h) \quad \forall \, v_h \in K_h
\]
\end{itemize}





\subsection{Lokaler Anteil des Fehlerschätzers}
\label{kap:4.1.2}



\subsection{Oszillationsterme}
\label{kap:4.1.3}



\subsection{Zuverlässigkeit des Fehlerschätzers}
\label{kap:4.1.4}



\subsection{Effektivität des Fehlerschätzers}
\label{kap:4.1.5}



\section{Ein adaptiver Algorithmus}
\label{kap:4.2}



\section{Erfüllung einer Saturationseigenschaft}
\label{kap:4.3}



\section{Übertragung des Fehlerschätzers auf Kontaktprobleme}
\label{kap:4.4}


\newpage

%%% Local Variables: 
%%% mode: latex
%%% TeX-master: "Skript"
%%% End: 
