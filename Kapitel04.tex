\newchapter{Ein hierarchischer Fehlerschätzer für Hindernisprobleme}
\label{kap:4}

\begin{itemize}
\item Vergleich Hindernisprobleme zu Kontaktproblemen $\ra$ warum gerade dieser Fehlerschätzer bei Hindernis- bzw. Kontaktproblemen
\end{itemize}

Dieses Kapitel basiert größtenteils auf \cite{ZouVee}.


\section{Herleitung eines a posteriori hierarchischen Fehlerschätzers}
\label{kap:4.1}

\begin{itemize}
\item der Einfachheit halber gehen wir von folgendem Sachverhalt aus
\begin{vor}\label{vor:4.1}
Das Hindernis  wird durch eine stückweise lineare stetige Funktion $\psi$ beschrieben.
\end{vor}

\item nicht nichtstetige oder auch glatte Hindernisse sind analoge Aussagen, aber schwerer, beweisbar
\end{itemize}






\subsection{Diskretisierung}
\label{kap:4.1.1}

\begin{itemize}
\item $\mcal B_h$ sei eine nodale Basis bzgl. einer quasi-uniformen Triangulierung $\mcal T_h$  für $\mcal S_h$ (s. auch Kapitel \ref{kap:2}), $K_h$ wie in Kapitel \ref{kap:3.1.3}
\[
	K_h  = \{v_h \in \mcal S_h \with v_h (p) \ge \psi(p) \, \forall \, p \in \mcal N \cap \Omega\} \, , 
\]
wobei $\mcal N$ wieder die Menge der Knoten von $\mcal T_h$ darstellt.

\item betrachte wieder die diskrete Variationsungleichung \eqref{eq:3.12}: Finde $u_h \in K_h$ mit
\[
	a(u_h,v_h-u_h) \ge (f,v_h-u_h) \quad \forall \, v_h \in K_h\, .
\]

\item oder äquivalent die Minimierung des Funktionals $J(v) = \frac 1 2 a(v,v)-(f,v)$ über $K_h$, d.h.
\begin{align}\label{eq:4.1}
	u_h \in K_h :\quad J(u_h)\le J(v_h) \quad \forall \, v_h \in K_h
\end{align}

\item wegen der Voraussetzung, dass $\psi$ stückweise linear ist, gilt $K_h \subset K$, da die linearen Ansatzfunktionen nicht nur punktuell, sondern auch kontinuierlich die Nebenbedingung erfüllen

\item damit ist \eqref{eq:3.12} eine konforme FEM $\ra$ nichtkonforme wollen wir hier nicht betrachten (s. bel. stetige Hindernisse)

\item wir wollen einen a posteriori Fehlerschätzer für den Fehler bzgl. der Funktionswerte der Funktionale $J(u),J(u_h)$ herleiten. Hierbei gilt $J(u_h) - J(u) \ge 0$, denn aus den beiden Minimierungsproblemen über $K$ und $K_h$ folgt
\[
	J(u) \le J(v) \, \forall \, v \in K \, , \quad J(u_h) \le J(v_h) \, \forall \, v_h \in K_h \, .
\]
Da $K_h \subset K$ gilt, folgt auch $J(u)\le J(v_h)$ für alle $v_h \in K_h$. Setze $v_h = u_h$, so gilt
\[
	J(u) \le J(u_h) \Llra J(u_h)-J(u) \ge 0
\]
\item 
\begin{bem}
Gilt $\psi = -\infty$, d.h. ist kein Hindernis vorhanden, so folgt
\begin{align*}
	J(u_h)-J(u)  =& \frac 1 2 a(u_h,u_h)-(f,u_h) - \(\frac 1 2 a(u,u)-(f,u)\) \\
	 =& \frac 1 2 a(u_h,u_h)-(f,u_h) - \frac 1 2 a(u,u)+(f,u) \\
	 & +\overbrace{(a(u,u-u_h)-\underbrace{(f,u-u_h)}_{=(f,u)-(f,u_h)})}^{=0} \\
	 = & \frac 1 2a(u_h,u_h) - \frac 1 2 a(u,u) + a(u,u-u_h) \\
	 = & \frac 1 2 a(u_h,u_h) - \frac 1 2a(u,u)+a(u,u) - a(u,u_h) \\
	 = & \frac1 2 (a(u_h,u_h)+ a(u,u) - 2 a(u,u_h)) \\
	 = & \frac 1 2 a(u_h-u,u_h - u) = \frac 1 2 \norm{u_h-u}^2_E \, .
\end{align*}
Ist nun ein $\psi > -\infty$ gegeben, dann addieren wir im zweiten Schritt nicht mehr Null, sondern es gilt für den Term
\[
	a(u,u-u_h)-(f,u-u_h) \le 0
\]
und damit gilt $J(u_h)-J(u) \ge \frac 1 2\norm{u_h-u}_E^2$, d.h. eine obere Schranke des Fehlers im Funktional schätzt auch den Fehler zwischen exakter und approximierter Lösung in der Energienorm ab.
\end{bem}

\item Herleitung eines hierarchischen a posteriori Fehlerschätzers:

\item 
\begin{notation}
Um im Folgenden den hierarchischen Split leichter beschreiben zu können, schreiben wir für die Galerkin-Lösung $u_h$ die Notation $u_{\mcal S}$, um auszudrücken, dass diese im linearen Ansatzraum $\mcal S_h$ liegt. Analog sind die im Weiteren übrigen verwendeten Indizes zu verstehen.
\end{notation}

\item wir führen Fehlerfunktion $e= u-u_{\mcal S}$ ein

\item weiter sei $\mcal I(v) = \frac 12 a(v,v)-\rho_{\mcal S}(v)$ mit $\rho_{\mcal S} (v) = (f,v)-a(u_{\mcal S},v), v \in H^1_0(\Omega)$.

\item
\begin{bem}
\begin{enumerate}[(a)]
\item Die Linearform $\rho_{\mcal S}$ stellt das Residuum der Variationsgleichung (d.h. ohne Hindernis) dar.
\item Nach dem Darstellungssatz von Riesz existiert ein $v^* \in H^1_0(\Omega)$, so dass
\[
	(v^*,v)_1 = \rho_{\mcal S} (v) \quad\forall \, v \in H^1_0(\Omega)
\]
ist. Wir können also $v^*$ als Lagrange-Multiplikator bzgl. der Nebenbedingung $v \ge \psi$ interpretieren.
\end{enumerate}
\end{bem}

\item neues Minimierungsproblem, jetzt für den Fehler $e$.
\begin{satz}[Lösung des Defektproblems]
Mit den obigen Bezeichnungen löst die Fehlerfunktion $e$ folgendes Defektproblem:
\begin{align}\label{eq:4.2}
	e \in \mcal A:\quad  \mcal I(e) \le \mcal I(v) \quad \forall \, v \in \mcal A \, ,
\end{align}
wobei $\mcal A \coloneqq \{v \in H^1_0(\Omega) \with v \ge \psi-u_{\mcal S}\} = -u_{\mcal S} + K$.
\end{satz}

\begin{proof}
Es sei $u$ die Lösung von \eqref{eq:3.2} und $u_{\mcal S}$ die Lösung von \eqref{eq:4.1}. Dann gilt
\begin{align*}\tag{$\ast$}
	 u \in K : \quad\qquad\quad \,\, \, \, \,  J(u)& \le J(\tilde v )\, \, \, \, \,  \qquad\qquad \forall \, \tilde v \in K \\
	\Llra u \in K : \quad J(u)-J(u_{\mcal S})& \le J(\tilde v) - J(u_{\mcal S}) \quad \forall \, \tilde v \in K \, .
\end{align*}
Wir rechnen für die linke Seite nach, dass gilt
\begin{align*}
	J(u)-J(u_{\mcal S}) & = \frac 1 2 a(u,u) - (f,u) -\(\frac 1 2 a(u_{\mcal S},u_{\mcal S}) - (f,u_{\mcal S})\) \\
	& = \frac 1 2 a(u,u)  + \frac 1 2 a(u_{\mcal S},u_{\mcal S}) - a(u_{\mcal S},u_{\mcal S}) - (f,u-u_{\mcal S}) \\
	& = \frac 1 2 a(u,u)  + \frac 1 2 a(u_{\mcal S},u_{\mcal S}) - a(u_{\mcal S},u) - ((f,u-u_{\mcal S})- a(u_{\mcal S},u-u_{\mcal S})) \\
	& = \frac 1 2a(u-u_{\mcal S},u-u_{\mcal S})- \rho_{\mcal S} (u-u_{\mcal S}) \\
	& = \frac 1 2 a(e,e)-\rho_{\mcal S} (e) = \mcal I(e) \, .
\end{align*}
Analog gilt für die rechte Seite $J(\tilde v ) - J(u_{\mcal S}) = \mcal I(\tilde v - u_{\mcal S})$. Mit $v \coloneqq \tilde v - u_{\mcal S}$ gilt $v \in \mcal A$ und damit ist ($*$) äquivalent zu: Finde $e \in \mcal A$, so dass
\[
	\mcal I(e) \le \mcal I(v) \quad \forall \, v \in\mcal A \, . \qedhere
\]
\end{proof}

\item
\begin{kor}
Das Problem \eqref{eq:4.2} ist äquivalent zur Variationsungleichung: Finde $e \in \mcal A$ mit
\begin{align}\label{eq:4.3}
	a(e,v-e) \ge \rho_{\mcal S} (v-e) \quad \forall \, v \in \mcal A \, .
\end{align}
\end{kor}

\begin{proof}
Analog zu Lemma \ref{lem:3.1} lässt sich zeigen, dass $\mcal A$ abgeschlossen und konvex ist. Mit Satz \ref{satz:A.10} folgt dann die Behauptung.
\end{proof}

\item da $\psi$ stückweise linear ist, liegt $0 \in \mcal A$, d.h. das "`gewünschte"' Ergebnis für $e$ liegt im betrachteten Raum

\item
\begin{bem*}
Wir werden noch zeigen, dass $\rho_{\mcal S}$ eine Schlüsselgröße für die a posteriori Abschätzung darstellt.
\end{bem*}

\item a posteriori Schätzer in 2 Schritten
\begin{enumerate}[(i)]
\item	diskreditiere \eqref{eq:4.3} bzgl. einer Erweiterung von $\mcal S_h$ (hier quadratische Funktionen), so dass $e$ hinreichend genau approximiert wird.
\item Aufteilung des neuen Raumes, sodass \eqref{eq:4.3} lokal in der Erweiterung exakt gelöst werden kann
\end{enumerate}

\item als Erweiterung von $\mcal S_h$ betrachten wir einen Raum $\mcal Q_h$ mit $\mcal S_h \subset \mcal Q_h$.

\item hier bietet sich an: $\mcal Q_h \coloneqq \{v \in C^0(\Omega) \with v|_T \in \mcal P_2 \text{ für } T\in \mcal T_h, v|_{\partial\Omega} = 0\}$, also der Raum der quadratischen Spline über einer quasi-uniformen Zerlegung $\mcal T_h$.

\item damit definiere $\mcal N_{\mcal Q} \coloneqq \mcal N \cup \{x_E \with E \in \mcal E\}$, wobei $x_E$ den Mittelpunkt der Kante $E$ darstellt und $\mcal E$ somit die Menge aller Kanten ist.

\item damit ergibt sich $\mcal A$ über $\mcal Q$ diskret als
\begin{align}\label{eq:4.4}
	\mcal A_{\mcal Q} \coloneqq \{v \in \mcal Q_h \with v(p) \ge \psi(p)-u_{\mcal S}(p) \, \forall \, p \in \mcal N_{\mcal Q} \cap \Omega\}
\end{align}

\item im Bezug zu \eqref{eq:4.4} ergibt sich dann das diskrete Defektproblem
\begin{align}\label{eq:4.5}
	e_{\mcal Q} \in \mcal A_{\mcal Q} : \quad a(e_{\mcal Q}, v-e_{\mcal Q}) \ge \rho_{\mcal S}(v-e_{\mcal Q}) \quad \forall \, v \in \mcal A_{\mcal Q}
\end{align}

\item 
\begin{bem}
Im Allgemeinen gilt hierbei nicht $\mcal A_{\mcal Q} \subset \mcal A$. So kann man sich anschaulich eine quadratische Funktion $v_{\mcal Q} \in \mcal A_{\mcal Q}$ vorstellen, die allerdings zwischen den übereinstimmenden Werten aufgrund ihrer Krümmung das lineare Hindernis aus $\mcal A$ durchdringt.

\begin{figure}[h]
\begin{center}
	\begin{pspicture}(0,0)(5,3)
		% Koordinatensystem:
		\psaxes[linewidth=0.65pt]{->}(0,0)(-0.3,-0.3)(6.7,2.7)
		\rput(6.7,-0.2){$x$}
		\rput(-0.2,2.7){$y$}
		
		% Das Hindernis:
		\psline[linewidth=0.6pt](1,0)(2,1)
		\psline[linewidth=0.6pt](2,1)(4,1.5)
		\psline[linewidth=0.6pt](4,1.5)(5,0)
		\rput(4.5,0.3){$\psi$}
		
		\psdots(2,1)(3,1.25)(4,1.5)
		
		% Die Funktion v:
		\psline[linestyle=dashed,linewidth=0.6pt](0,0)(2,1.5)
		\pscurve[linestyle=dashed,linewidth=0.6pt](2,1.5)(3,1.25)(4,1.5)
		\psline[linestyle=dashed,linewidth=0.6pt](4,1.5)(6,0)
		\rput(0.5,0.7){$v$}
	\end{pspicture}
\end{center}
\caption{Beispiel eines affinen Hindernisses $\psi$ mit $v \in \mcal A_{\mcal Q}$ in $\R$}
\end{figure}
\end{bem}

\item hierarchische Aufteilung von $\mcal Q_h$ durch $\mcal Q_h = \mcal S_h \oplus \mcal V_h$, wobei $\mcal V_h \coloneqq \{\phi_E \with E \in \mcal E\}$ ist und $\phi_E$ die \textit{\idx{Bubble-Funktion}} mit
\[
	\phi_E (p) = \delta_{x_E,p} = \begin{cases}
								1, & p = x_E \\
								0, & \text{sonst}
							\end{cases}
\]
ist

\item
\begin{bsp}
allgemeine Skizze und die drei bubble Funktionen auf einem Referenzdreieck
\end{bsp}

\item
\begin{satz}
Mit den oben verwendeten Notationen gilt $\mcal Q_h = \mcal S_h \oplus \mcal V_h$.
\end{satz}

\begin{proof}
Wir zeigen, dass $\mcal Q_h = \mcal S_h \oplus \mcal V_h$ auf dem Referenzdreieck gilt und damit gilt es auch für beliebige Dreiecke $T \in \mcal T_h$, da ein allgemeines Dreieck $T$ aus dem Referenzelement $\tilde T$ durch affine Transformation hervorgeht.

Auf dem Referenzelement $\tilde T$ ist $\{\phi_1,\phi_2,\phi_3\}$ eine Basis von $\mcal S_h$ mit
\[
	\phi_1(\xi,\eta) = 1-\xi-\eta \, , \quad \phi_2(\xi,\eta) = \xi \, , \quad \phi_3(\xi,\eta) = \eta
\]
und $\{\phi_4,\phi_5,\phi_6\}$ eine Basis von $\mcal V_h$ mit
\[
	\phi_1(\xi,\eta) = 4\xi (1-\xi-\eta) \, , \quad \phi_2(\xi,\eta) = 4\xi\eta \, , \quad \phi_3(\xi,\eta) = 4\eta(1-\xi-\eta) \, .
\]
Damit ist $\{\phi_1,\ldots,\phi_6\}$ ein Erzeugendensystem von $\mcal Q_h$, da jedes Element
\[
	a_0+a_1\xi+a_2\eta + a_3 \xi^2+ a_4\xi\eta +a_5\eta^2 \in \mcal Q_h
\]
als Linearkombination aus den Funktionen beschrieben werden kann ($\phi_1$ bis $\phi_6$ enthalten alle vorkommenden Summanden eines Polynom 2. Grades). Außerdem ist leicht nachzurechnen, dass die Funktionen $\phi_i,i = 1,\ldots,6,$ linear unabhängig sind und damit gilt
\[
	\mcal Q_h = \operatorname{span} \{\phi_1,\ldots,\phi_6\} \, .
\]
Aus der linearen Unabhängigkeit folgt damit auch $\mcal S_h \cap \mcal V_h = \{0\}$ gilt und damit die Behauptung.
\end{proof}

\item daher kann jedes Element $v_{\mcal Q} \in \mcal Q_h$  als $v_{\mcal Q} = v_{\mcal S} + v_{\mcal V}$ mit $v_{\mcal S} \in \mcal S_h, v_{\mcal V}\in \mcal V_h$ geschrieben werden

\item aus diesem Grund führen wir folgende Bilinearform ein:
\begin{align*}
	a_{\mcal Q} (v,w) \coloneqq a(v_{\mcal S},w_{\mcal S}) + \sum_{E \in \mcal E} u_{\mcal V}(x_E) w_{\mcal V}(x_E) a(\phi_E,\phi_E) \quad \forall \, v,w \in \mcal Q_h \, ,
\end{align*}
welche aufgrund der Eigenschaften der direkten Summe von $\mcal S_h$ und $\mcal V_h$ wohldefiniert ist.

\item dabei ergibt sich $a_{\mcal Q}$ durch Entkopplung von $\mcal S_h$ und $\mcal V_h$ und anschließender "`Diagonalisierung"' auf $\mcal V$

\item sinnvoll $a_{\mcal Q}$ so einzuführen, denn:
\begin{satz}
Die zu $a_{\mcal Q}$ assoziierte Energienorm
\begin{align*}
	\norm v_{\mcal Q}\coloneqq a_{\mcal Q}(v,v)^{\frac 1 2} \, , \quad v \in \mcal Q_h
\end{align*}
ist äquivalent zur Energienorm $\norm\cdot_E$, d.h. es gibt Konstanten $c_1,c_2$ $($die insbesondere nur von der Quasi-Uniformität von $\mcal T_h$ abhängen$)$, so dass
\[
	c_1 \norm v_E \le \norm v_{\mcal Q} \le c_2 \norm v_E \, , \quad \forall \, v \in \mcal Q_h \, .
\]
\end{satz}

\begin{proof}
Die Aussage folgt aus Theorem 4.1 bzw. Bemerkung 4.3 in \cite{HoppeKorn} zusammen mit dem Lemma auf Seite 14 in \cite{Deufl}.
\end{proof}

\item daher führen wir die approximierte Energie
\begin{align}\label{eq:4.6}
	\mcal I_{\mcal Q} (v) \coloneqq \frac 1 2 a_{\mcal Q}(v,v)-\rho_{\mcal S}(v) \, , \quad v \in \mcal Q_h 
\end{align}
ein.

\item das damit verbundene Defektproblem ist allerdings noch durch die Nebenbedingung aus $\mcal A_{\mcal Q}$ mit $\mcal S_h$ gekoppelt und daher noch nicht alleine auf die Raumerweiterung $\mcal V_h$ bezogen.

\item Als Abhilfe ignorieren wir einfach die linearen Beiträge in $\mcal A_{\mcal Q}$ und führen eine echte Teilmenge 
\begin{align}\label{eq:4.7}
	\mcal A_{\mcal V} \coloneqq \{v \in \mcal V \with v(x_E) \ge \psi(x_E)-u_{\mcal S}(x_E) \, \forall \, E \in \mcal E\}
\end{align}
von $\mcal A_{\mcal Q}$ ein.

\item zusammen mit \eqref{eq:4.6} und \eqref{eq:4.7} erhalten wir das lokale diskrete Defektproblem
\begin{align}\label{eq:4.8}
	\eps_{\mcal V} \in \mcal A_{\mcal V} : \quad \mcal I_{\mcal Q}(\eps_{\mcal V}) \le \mcal I_{\mcal Q} (v) \quad \forall \, v \in \mcal A_{\mcal V}
\end{align}
bzw. die dazu äquivalente Variationsungleichung
\begin{align}\label{eq:4.9}
	\eps_{\mcal V} \in \mcal A_{\mcal V} : \quad a_{\mcal Q} (\eps_{\mcal V},v-\eps_{\mcal V})\ge \rho_{\mcal S} (v-\eps_{\mcal V}) \quad \forall \, v \in \mcal A_{\mcal V} \, .
\end{align}

\item
\begin{bem}
\begin{enumerate}[(a)]
\item Da $\psi$ stetig stückweise linear ist und somit $u_{\mcal S} \ge \psi$ gilt, folgt $0 \in \mcal A_{\mcal V}$. Damit ist auch hier die gewünschte Lösung für $\eps_{\mcal V}$ in $\mcal A_{\mcal V}$ enthalten
\item	Auch für $\mcal A_{\mcal V}$ lässt sich mit analogem Vorgehen zu Lemma \ref{lem:3.1} die Konvexität zeigen.
\end{enumerate}
\end{bem}

\item
\begin{lemma}
Das Energiefunktional $\mcal I_{\mcal Q}$ ist konvex.
\end{lemma}

\begin{proof}
Da $a$ eine stetige koerzive Bilinearform, werden aufgrund der Konstruktion von $a_{\mcal Q}$ diese Eigenschaften auch auf $a_{\mcal Q}$ übertragen. Weiterhin ist leicht zu überprüfen, dass $\rho_{\mcal S}$ eine stetige Linearform ist. Dann folgt aus Lemma \ref{lem:2.3} direkt die Behauptung.
\end{proof}

\item Lösung des lokalen Defektproblems
\begin{satz}
Die Lösung von \eqref{eq:4.8} bzw. \eqref{eq:4.9} ist explizit gegeben durch
\begin{align}\label{eq:4.10}
	\eps_{\mcal V} (x_E) = \frac{\max \{-d_E,\rho_E\}}{\norm{\phi_E}} \, 
\end{align}
wobei
\begin{align}\label{eq:4.11}
	d_E = (u_{\mcal S}(x_E) - \psi (x_E))\norm{\phi_E} \ge 0 \, , \quad \rho_E = \frac{\rho_{\mcal S}(\phi_E)}{\norm{\phi_E}} \, .
\end{align}
\end{satz}

\begin{proof}
Es sei $M = \abs{\mcal E}$ die Anzahl der Kanten. Zunächst berechnen wir zur besseren Übersicht $\eps_{\mcal V}(x_E)$ konkret, d.h.
\begin{align}\notag
	\eps_{\mcal V} (x_E) & =  \frac{\max \{-d_E,\rho_E\}}{\norm{\phi_E}} \\
	\notag
	& = \frac{\max \left\{(\psi (x_E)-u_{\mcal S}(x_E) )\norm{\phi_E} ,\frac{\rho_{\mcal S}(\phi_E)}{\norm{\phi_E}}\right\}}{\norm{\phi_E}} \\
	\notag
	& = \max \left\{\psi (x_E)-u_{\mcal S}(x_E)  ,\frac{\rho_{\mcal S}(\phi_E)}{\norm{\phi_E}^2}\right\} \\
	\label{eq:4.12}
	& = \max \left\{\psi (x_E)-u_{\mcal S}(x_E)  ,\frac 1{\norm{\phi_E}^2} ((f,\phi_E)-a(u_{\mcal S},\phi_E))	\right\} \, .
\end{align}
Da  $\eps_{\mcal V} = \sum_{E \in \mcal E} \eps_{\mcal V}(x_E) \phi_E$ ist, können wir \eqref{eq:4.8} bzgl. der Basis $\{\phi_E \with E \in \mcal E\}$ von $\mcal V_h$ diskret schreiben als
\begin{align*}
	\min \frac 1 2 \bs v^T D \bs v - \bs g^T \bs v \quad \text{s.t.} \quad\bs v \ge \bs \psi - \bs u_{\mcal S}\, , 
\end{align*}
wobei $\bs v = [\eps_{\mcal V}(x_{E_i})]_{1\le i \le M}, D = \operatorname{diag}(a(\phi_{E_1},\phi_{E_1}),\ldots,a(\phi_{E_M},\phi_{E_M})), \bs g = [(f,\phi_{E_i})-a(u_{\mcal S},\phi_{E_i})]_{1\le i\le M}, \bs \psi = [\psi(x_{E_i})]_{1\le i \le M}$ und $ \bs u_{\mcal S} = [u_{\mcal S}(x_{E_i})]_{1\le i \le M}$. Da $\mcal A_{\mcal V}$ und $\mcal I_{\mcal Q}$ konvex sind, existiert ein Minimum $\bs v^* \in \mcal A_{\mcal V}$ von $\mcal I_{\mcal Q}$, das die KKT-Bedingungen erfüllt. Damit gilt
\begin{subequations}\label{eq:4.13}
\begin{align}\label{eq:4.13a}
	D \bs v-\bs g - \bs \lambda & = \bs 0 \, , \\
	\label{eq:4.13b}
	\bs \lambda &\ge 0\, ,\\
	\label{eq:4.13c}
	\bs v & \ge \bs \psi - \bs u_{\mcal S}\,  , \\
	\label{eq:4.13d}
	\lambda_i\, (\bs v - \bs \psi+\bs u_{\mcal S})_i &= 0 \quad \forall \, i=1,\ldots,M \, .
\end{align}
\end{subequations}
Es sei $k \in \{1,\ldots,M\}$ beliebig.

\underline{Fall 1:} Gilt $\lambda_k = 0$, so folgt aus \eqref{eq:4.13a}
\[
	\eps_{\mcal V} (x_{E_k}) = v_k = \frac {g_k}{a(\phi_{E_k},\phi_{E_k})} = \frac 1{\norm{\phi_{E_k}}^2} ((f,\phi_{E_k})-a(u_{\mcal S},\phi_{E_k})) \, .
\]
\underline{Fall 2:} Gilt $\lambda_k \not= 0$, dann folgt wegen \eqref{eq:4.13d}
\[
	\eps_{\mcal V} (x_{E_k}) =v_k = (\bs \psi - \bs u_{\mcal S})_k = \psi(x_{E_k}) - u_{\mcal S}(x_{E_k})\, .
\]
Insgesamt folgt mit \eqref{eq:4.13c} und \eqref{eq:4.12} die Behauptung.
\end{proof}

\item wir wollen im weiteren den a posteriori Fehlerschätzer
\[
	-\mcal I_{\mcal Q} (\eps_{\mcal V})= -\frac 1 2 a_{\mcal Q}(\eps_{\mcal V},\eps_{\mcal V}) + \rho_{\mcal S}(\eps_{\mcal V})
\]
betrachten und werden zeigen, dass er äquivalent zu $J(u_{\mcal S}) - J(u)$ ist (vgl. Kapitel \ref{kap:4.1.4} und \ref{kap:4.1.5})

\item zunächst aber Einführung des lokalen Anteils des Fehlerschätzers $-\mcal I_{\mcal Q} (\eps_{\mcal V})$
\end{itemize}





\subsection{Lokaler Anteil des Fehlerschätzers}
\label{kap:4.1.2}

\begin{itemize}
\item 
\begin{notation}
\begin{enumerate}[(a)]
\item Wir schreiben im Folgenden "`$\lesssim$"' statt "`$\le C$"', wenn die Konstante $C$ nur von der Quasi-Uniformität von $\mcal T_h$ abhängt.
\item Weiter schreiben wir "`$A \approx B$"' für "`$A\lesssim B$"' und "`$B \lesssim A$"'.
\end{enumerate}
\end{notation}

\item zunächst zeigen wir ein paar Eigenschaften von der Fehlerfunktion $e = u-u_{\mcal S}$

\item
\begin{lemma}\label{lem:4.12}
Die Fehlerfunktion $e = u-u_{\mcal S}$ erfüllt die Ungleichungen
\begin{align}\label{eq:4.14}
	\frac 12 \norm e^2 \le \frac 12 \rho_{\mcal S}(e) \le -\mcal I (e) \le \rho_{\mcal S}(e) \, .
\end{align}
\end{lemma}

\begin{proof}
Wir erinnern uns, dass
\[
	-\mcal I (e) \coloneqq - \frac 12 \underbrace{a(e,e)}_{\ge 0} + \rho_{\mcal S} (e) \le \rho_{\mcal S}(e) \, , 
\]
da $a$ koerziv ist. Dann gilt weiter
\begin{align*}
	-\mcal I(e) & = - \frac 12 a(e,e) + \rho_{\mcal S} (e) \\
	& = -\frac 1 2 a(u-u_{\mcal S},e) + \rho_{\mcal S} (e) \\
	& = -\frac 1 2 a(u,e) \underbrace{\frac 1 2 a(u_{\mcal S},e)-\frac 1 2(f,e)}_{=-\frac 12 \rho_{\mcal S}(e)} + \frac 1 2 (f,e)+\rho_{\mcal S}(e) \\
	& = -\frac 12 (\underbrace{a(u,u-u_{\mcal S})-(f,u-u_{\mcal S})}_{\le 0}) + \frac 1 2\rho_{\mcal S}(e) \ge \frac 1 2\rho_{\mcal S}(e) \, .
\end{align*}
Es bleibt also die erste Ungleichung von \eqref{eq:4.14} zu zeigen. Wir rechnen nach, dass
\begin{align*}
	\frac 1 2 \rho_{\mcal S} (e) &= \frac 1 2 (f,e)-\frac 1 2a(u_{\mcal S},e) \\
	& = \frac 1 2 (\underbrace{(f,u-u_{\mcal S}) - a(u,u-u_{\mcal S})}_{\ge 0} + a(u-u_{\mcal S},e)) \\
	& \ge \frac 1 2 a(u-u_{\mcal S},e) =\frac 1 2 a(e,e) =  \frac 1 2 \norm e^2 
\end{align*}
gilt, womit \eqref{eq:4.14} insgesamt bewiesen ist.
\end{proof}

\item 
\begin{kor}
Für die Lösungen $e_{\mcal Q}, \eps_{\mcal V}$ von \eqref{eq:4.5} und \eqref{eq:4.9} gilt
\begin{align}\label{eq:4.15}
	\frac 12 \norm{e_{\mcal Q}}^2 \le \frac 12 \rho_{\mcal S}(e_{\mcal Q})& \le -\mcal I (e_{\mcal Q}) \le \rho_{\mcal S}(e_{\mcal Q}) \, ,\\
	\label{eq:4.16}
	\frac 12 \norm {\eps_{\mcal V}}_{\mcal Q}^2 \le \frac 12 \rho_{\mcal S}(\eps_{\mcal V})& \le -\mcal I_{\mcal Q} (\eps_{\mcal V}) \le \rho_{\mcal S}(\eps_{\mcal V})\, .
\end{align}
\end{kor}

\begin{proof}
Da $e_{\mcal Q}$ und $\eps_{\mcal V}$ Lösungen der Variationsungleichungen \eqref{eq:4.5} und \eqref{eq:4.9} sind, folgt die Behauptung analog zum Beweis von Lemma \ref{lem:4.12}.
\end{proof}

\item wegen \eqref{eq:4.16} ist $\rho_{\mcal S} (\eps_{\mcal V})$ äquivalent zum Fehlerschätzer $-\mcal I_{\mcal Q}(\eps_{\mcal V})$ und kann daher als Indikator für $-\mcal I_{\mcal Q} (\eps_{\mcal V})$ verwendet werden (verkleinern wir $\rho_{\mcal S}$, so wird auch $-\mcal I_{\mcal Q}$ kleiner)

\item in Kapitel \ref{kap:4.1.4} und \ref{kap:4.1.5} werden wir die Äquivalenz von $-\mcal I_{\mcal Q}(\eps_{\mcal V})$ zum exakten Fehler in den Funktionalen $J(u_{\mcal S}) - J(u)=-\mcal I(e)$ zeigen

\item damit folgt auch aus Lemma \ref{lem:4.12}, dass der Fehler $J(u_{\mcal S})-J(u)$ äquivalent zu $\rho_{\mcal S}(e)$ ist $\Ra$ daher betrachten wir ein paar weitere Eigenschaften von $\rho_{\mcal S}$.

\item nun zu den lokalen Anteilen von $\rho_{\mcal S}(\eps_{\mcal V})$:

\item es sei $u_{\mcal S}$ die Lösung von \eqref{eq:3.12}, dann auf jedem $T\in \mcal T_h$ die Gleichung $\Delta u_{\mcal S} = 0$, da $u_{\mcal S}$ auf jedem $T$ linear ist.

\item dann gilt mit $\Omega = \bigcup_{T \in \mcal T_h} T$ für alle $v \in H^1(\Omega)$
\begin{align}\notag 
	\rho_{\mcal S} (v) & = (f,v)-a(u_{\mcal S},v) = \int_{\Omega} fv \, d\Omega - \int_{\Omega} \nabla u_{\mcal S} \nabla v \, d\Omega \\
	\notag
	& = \int_{\Omega} fv \, d\Omega - \sum_{T\in \mcal T_h} \int_T \nabla u_{\mcal S}\nabla v \, dT \\
	\notag
	& = \int_{\Omega} fv \, d\Omega - \sum_{T\in \mcal T_h} \(\int_{\partial T} v \partial_{\bs n} u_{\mcal S} \, d\Gamma -  \int_T \underbrace{\Delta u_S}_{=0} v \, dT \) \\
	\label{eq:4.17}
	& = \int_{\Omega} fv \, d\Omega - \sum_{T\in \mcal T_h} \int_{\partial T} v \partial_{\bs n} u_{\mcal S} \, d\Gamma \, ,
\end{align}
wobei im vorletzten Schritt die 1. Green'sche Formel angewendet wurde und d$\bs n$ die äußere Einheitsnormale von $T$ ist.

\item Betrachten wir zwei beliebige Dreiecke $T_1,T_2$ wie in Abbildung \ref{abb:4.2}, wobei $\bs n$ hierbei die Einheitsnormale, die von $T_1$ nach $T_2$ zeigt, bezeichnet, so können wir die Summe aus \eqref{eq:4.17} bzgl. der Menge der Kanten $\mcal E$ darstellen, da der Rand $\partial T = E_1 \cup E_2 \cup E_3$ für jedes $T$ disjunkt in seine Kantenstücke aufgeteilt werden kann.

Dabei sei $E$ nun die Kante, die $T_1$ und $T_2$ zugleich enthalten, d.h. $\bs n$ steht rechtwinklig auf $E$. Dann gilt, dass die Richtungsableitung $\partial_{\bs n} u_{\mcal S}|_{T_2}$ negativ ist bzgl. \eqref{eq:4.17} wegen der negativen Orientierung von $\bs n$ bzgl. $T_2$.

\begin{figure}[h]\label{abb:4.2}
  \begin{center}
    \begin{pspicture}(-1.5,0)(2,2)
    	% Die zwei Dreiecke:
	\psline(-1.5,1.7)(1,2)
	\psline(1,2)(-0.2,-0.2)
	\psline(-0.2,-0.2)(-1.5,1.7)
	\psline(1,2)(2.5,0.4)
	\psline(2.5,0.4)(-0.2,-0.2)
	\rput(-0.7,1.4){$T_1$}
	\rput(1.9,0.55){$T_2$}
	
	% Normalenvektor:
	\psline{->}(0.35,0.8)(1,0.45)
	\rput(0.8,0.9){$\bs n$}
	\psellipticarc[linewidth=0.5pt](0.35,0.8)(0.25,0.25){242}{335}
	\psdot[dotsize=1.3pt](0.38,0.68)
	
	% Beschriftung der Kante E:
	\rput(0.55,1.65){$E$}
    \end{pspicture}
  \end{center}
\caption{Dreiecke $T_1$ und $T_2$ mit Einheitsnormalen $\bs n$}
\end{figure}

Hiermit ergibt sich aus \eqref{eq:4.17}
\begin{align}\notag
	\rho_{\mcal S} (v) &= \int_{\Omega} fv \, d\Omega - \sum_{T\in \mcal T_h} \int_{\partial T} v \partial_{\bs n} u_{\mcal S} \, d\Gamma \\
	\notag
	& = \int_{\Omega} fv \, d\Omega - \sum_{E\in \mcal E} \int_{E} v\, (\underbrace{\partial_{\bs n} u_{\mcal S}|_{T_1}-\partial_{\bs n} u_{\mcal S}|_{T_2}}_{\eqqcolon -j_E}) \, d\Gamma \\
	\label{eq:4.18}
	& =  \int_{\Omega} fv \, d\Omega + \sum_{E\in \mcal E} \int_{E} j_E v \, d\Gamma \, .
\end{align}

\item da für die nodalen Basisfunktionen $\{\phi_p \with p \in \mcal N\cap \Omega\}$ gilt
\[
	\sum_{p \in \mcal N} \phi_p = 1 \text{ auf ganz } \Omega \, , 
\]
sodass wir $\rho_{\mcal S}$ wie folgt in lokale Anteile aufteilen können:
\begin{align}\label{eq:4.19}
	\rho_p(v) \coloneqq \rho_{\mcal S} (v \phi_p) \, , \quad v \in H^1(\Omega) \, .
\end{align}

\item
\begin{lemma}\label{lem:4.14}
Für $\rho_p$ gilt
\begin{align*}
	\rho_p (v) = \int_{\omega_p} f v \phi_p \, d\Omega + \sum_{E\in \mcal E_p} \int_E j_E v \phi_p \, d\Gamma  \, , \quad v \in H^1(\Omega)
\end{align*}
mit $\omega_p \coloneqq \supp \phi_p$ und $\mcal E_p \coloneqq \{E \in \mcal E \with E \ni p\}$, d.h. die Menge der Kanten, in denen $p$ enthalten ist.
\end{lemma}

\begin{proof}
Wir rechnen einfach mit der Definition \eqref{eq:4.19} und \eqref{eq:4.18} nach, dass für ein beliebiges $v \in H^1(\Omega)$ gilt
\begin{align*}
	\rho_p(v) &= \rho_{\mcal S} (v \phi_p) =  \int_{\Omega} fv\phi_p \, d\Omega + \sum_{E\in \mcal E} \int_{E} j_E v \phi_p\, d\Gamma \\
	& = \int_{\omega_p} f v \phi_p \, d\Omega + \sum_{E\in \mcal E_p} \int_E j_E v \phi_p \, d\Gamma  \, ,
\end{align*}
da $\phi_p \equiv 0$ auf $\mcal O \coloneqq \overline{\Omega \setminus \omega_p}$ und damit auch auf $\mcal F \coloneqq\mcal E \setminus \mcal E_p$, da $\mcal F \subset\mcal O$.

\begin{figure}[h]
\begin{center}
	\begin{pspicture}(0,0)(7,5)
		%\psset{xunit=1.5cm,yunit=1.2cm}
		
		% \omega_p:
		\pspolygon[fillstyle=solid,fillcolor=lightgray](2.5,1.25)(3,2.6)(1.5,2.5)
		\pspolygon[fillstyle=solid,fillcolor=lightgray](2.5,1.25)(3,2.6)(4.7,2)
		\pspolygon[fillstyle=solid,fillcolor=lightgray](4.3,3.5)(3,2.6)(4.7,2)
		\pspolygon[fillstyle=solid,fillcolor=lightgray](4.3,3.5)(3,2.6)(2.5,3.5)
		\pspolygon[fillstyle=solid,fillcolor=lightgray](1.5,2.5)(3,2.6)(2.5,3.5)
		
		% Rest:
		\pspolygon(1.5,2.5)(0.75,3)(2.5,3.5)
		\pspolygon(1.5,2.5)(0.75,3)(1.2,1.5)
		\pspolygon(1.5,2.5)(2.5,1.25)(1.2,1.5)
		\pspolygon(1.2,1.5)(2.5,1.25)(1.9,0.5)
		\pspolygon(4.7,2)(2.5,1.25)(4.3,0.9)
		\pspolygon(4,-0.1)(2.5,1.25)(4.3,0.9)
		\pspolygon(4.7,2)(6.5,1.5)(4.3,0.9)
		\pspolygon(4.7,2)(6.5,1.5)(6.5,2.75)
		\pspolygon(4.7,2)(4.3,3.5)(6.5,2.75)
		
		\psline(1.2,1.5)(0.3,0.6)
		\psline(0.75,3)(0.25,3.25)
		\psline(2.5,3.5)(2.2,3.8)
		\psline(2.5,3.5)(2.9,3.9)
		\psline(4.3,3.5)(5.1,3.8)
		\psline(6.5,2.75)(6.3,3.3)
		\psline(6.5,1.5)(7.1,0.8)
		\psline(6.5,1.5)(7,1.8)
		\psline(6.5,2.75)(7,2.3)
		\psline(4.3,0.9)(5.9,0.3)
		
		% Hutfunktion:
		\psline[linestyle=dotted](3,2.6)(3,4.8)
		\psline[linewidth=0.6pt](3,4.8)(2.5,1.25)
		\psline[linewidth=0.6pt](3,4.8)(4.3,3.5)
		\psline[linewidth=0.6pt](3,4.8)(1.5,2.5)
		\psline[linewidth=0.6pt](3,4.8)(4.7,2)
		\psline[linewidth=0.6pt](3,4.8)(2.5,3.5)
		
		% Beschriftungen:
		\rput(3.2,5){$\phi_p$}
		\rput(3.1,2.3){$p$}
		\rput(2.2,2.1){$\omega_{p}$}
		
	\end{pspicture}
\end{center}
\caption{Darstellung von $\omega_p$ (grau) und $\mcal E_p$ (abgehende Kanten von $p$) für ein beliebiges $\phi_p$}
\end{figure}
\end{proof}

\item 
\begin{kor}
Der Indikator $\rho_{\mcal S}$ lässt sich schreiben als
\begin{align}\label{eq:4.20}
	\rho_{\mcal S} = \sum_{p \in \mcal N} \rho_p \, .
\end{align}
\end{kor}

\begin{proof}
Die Behauptung folgt direkt aus Lemma \ref{lem:4.14} zusammen mit
\[
	\Omega = \bigcup_{p\in \mcal N} \omega_p \, , \quad \mcal E = \bigcup_{p \in \mcal N} \mcal E_p \quad \text{und}\quad \sum_{p \in \mcal N} \phi_p = 1 \, 
\]
durch einfaches Nachrechnen.
\end{proof}

\item im unbeschränkten Fall gilt $\rho_{\mcal S} = 0 \Lra e = 0$, denn zu $\rho_{\mcal S} = 0$ ist äquivalent
\[
	a(e,v) = \rho_{\mcal S}(v) = 0 \quad \forall \, v \in \mcal V\, .
\]
Da $e\in \mcal V$ ist, folgt wegen der Galerkin-Orthogonalität, dass $e=0$ sein muss. Die Umkehrung gilt analog.

\item bei Variationsungleichungen gilt dies im allgemeinen nicht.

\item aber: aus Lemma \ref{lem:4.12} folgt allgemeiner, falls $\rho_{\mcal S} (v) \le 0$ für alle $v \in \mcal A$ gilt
\[
	\frac 12 \norm{e}^2 \le \rho_{\mcal S}(e) \le 0\,  \lra\,  \norm e = 0 \, \lra\, e = 0 \, ,
\]
wodurch $\rho_{\mcal S} = 0$ folgt, dass $e = 0$ ist.

\item es gilt, ist $u_{\mcal S}$ die Lösung von \eqref{eq:3.12}, so gilt für alle $p \in \mcal N \cap \Omega$, dass $v = u_{\mcal S} + \phi_p  \ge \psi$, d.h. $v \in K_h$.

Damit folgt mit Einsetzen von $v$ in  \eqref{eq:3.12}
\begin{align}\notag
	&a(u_{\mcal S}, u_{\mcal S} + \phi_p - u_{\mcal S}) \ge (f,u_{\mcal S}+\phi_p-u_{\mcal S}) \\
	\label{eq:my4.20}
	\Llra \,  & 0 \ge (f,\phi_p)-a(u_{\mcal S},\phi_p) = \rho_{\mcal S}(\phi_p)
\end{align}
dies bedeutet, dass die lineare Approximation des Fehlers $e$ gleich Null ist.

\item falls an einem Punkt $p$ kein Kontakt zwischen $u_{\mcal S}$ und $\psi$ vorliegt, also $u_{\mcal S} (p) > \psi(p)$ ist, dann können wir ein $\alpha > 0$ hinreichend klein wählen, sodass $v = u_{\mcal S} - \alpha \phi_p \in K_h$ liegt. Dann folgt analog durch Einsetzen von $v$ in \eqref{eq:3.12}
\begin{align*}
	0 & \ge (f,-\alpha \phi_p) - a(u_{\mcal S},-\alpha\phi_p) \\
	\Llra \, 0 & \le (f,\phi_p) - a(u_{\mcal S},\phi_p) = \rho_{\mcal S} (\phi_p) \stackrel[\scriptsize\eqref{eq:my4.20}]{}\le 0
\end{align*}
und damit gilt $\rho_{\mcal S} (\phi_p) = 0$

\item zusammen ergeben sich die Bedingungen
\begin{align}\label{eq:4.21}
	\rho_{\mcal S} (\phi_p) \le 0 \, , \quad \psi(p)-u_{\mcal S} (p) \le 0 \, , \quad \rho_{\mcal S} (\phi_p) (\psi(p)-u_{\mcal S} (p)) = 0
\end{align}

\item dies berechtigt zur Definition von Kontakt- und Nichtkontaktpunkten
\begin{defi}
Wir definieren die Mengen von \textit{Kontaktpunkten}\index{Kontaktpunkte} $\mcal N^0$ und \textit{Nichtkontaktpunkten}\index{Nichtkontaktpunkte} $\mcal N^+$ durch
\begin{align*}
	\mcal N^0 \coloneqq  \{p \in \mcal N \cap \Omega \with u_{\mcal S}(p) = \psi (p) \} \, , \quad 
	\mcal N^+ \coloneqq  \{p \in \mcal N \cap \Omega \with u_{\mcal S}(p) > \psi (p) \}\, .
\end{align*}
\end{defi}

\item 
\begin{bem}
Die Bedingungen \eqref{eq:4.21} können wir auch auf den lokalen Anteil $\rho_p$ übertragen, damit ergibt sich für alle $p \in \mcal N \cap \Omega$
\begin{subequations}\label{eq:4.22}
\begin{align}\label{eq:4.22a}
	\rho_p(1) &  \le 0\, ,  \\
	\label{eq:4.22b}
	u_{\mcal S} (p) > \psi (p) \, & \lra \, \rho_p(1) =0 \, ,
\end{align}
\end{subequations}
denn $\rho_p (1) = \rho_{\mcal S} (\phi_p)$.
\end{bem}

\item damit ist also die Approximation von $e$ über $\mcal S_h$ gleich Null, wenn die lokalen Anteile (im Vektor später) kleiner gleich Null sind

\end{itemize}






\subsection{Oszillationsterme}
\label{kap:4.1.3}

\begin{itemize}
\item in Kapitel \ref{kap:4.1.4} werden wir zeigen, dass $-\mcal I_{\mcal Q} (\eps_{\mcal V})$ eine obere Schranke von $-\mcal I (e)$ bis auf Terme höherer Ordnung bereitstellen, d.h. Terme, die nicht in $\mcal V$ enthalten sind $\ra$ Oszillationsterme (hier eingeführt)

\item man kann in den numerischen Beispielen später sehen, dass (wie auch in der Theorie) eine Verkleinerung der Oszillation auch eine Verringerung des Fehlers mit sich bringt $\Ra$ wir führen die Oszillationsterme ein (auch ohne präzise Beweise)

\item die Oszillation ist in zwei Teile kaufteilbar
\begin{align}\label{eq:4.23}
	\osc (u_{\mcal S}, \psi, f) \coloneqq \osc_1(u_{\mcal S},\psi)+\osc_2(u_{\mcal S},\psi,f)
\end{align}
(später vielleicht mit Wurzeln anders)

\item im unbeschränkten Fall ist die Oszillation nur von $f$ abhängig (s. \cite{MorNoc}) und dort wird daher von "`Daten-Oszillation"' gesprochen

\item $\osc_1$ ist ein Maß für die Oszillation zwischen Hindernis $\psi$ und der Galerkin-Lösung $u_{\mcal S}$, d.h.
\begin{align}
	\osc_1(u_{\mcal S},\psi) \coloneqq \(\sum_{p \in \mcal N^{0+}} \norm{\nabla (\psi - u_{\mcal S})}_{0,\omega_p}^2\)^{\frac 12} \, ,
\end{align}
wobei $\mcal N^{0+} \coloneqq \{p \in \mcal N^0 \with u_{\mcal S} > \psi \text{ in }\omega_p \setminus \{p\}\}$, also die Megne der \textit{isolierten Kontaktknoten}, d.h. $u_{\mcal S}$ ist in $\omega_p$ nur mit $p$ in Kontakt

\item anschaulich: da $\psi, u_{\mcal S}$ linear, gilt: je größer die Differenz zwischen $\psi$ und $u_{\mcal S}$, umso größer die Differenz $\nabla(\psi-u_{\mcal S})$, d.h. auch $\osc_1$

\item das kontinuierliche Gegenstück zu $\mcal N^{0+}$ ist die Menge der \textit{isolierten Kontaktpunkte} $x_c$, die aufgrund von $u-\psi >0$ für alle $x \in \mcal U(x_c, \eps) \subset \Omega$ mit $u(x_c) = \psi(x_c)$ alle strikten Minima $x_c \in \Omega$ enthält $\Ra (\nabla u -\nabla \psi) = 0$ für alle isolierten Kontaktpunkte, wenn $u, \psi$ hinreichend glatt sind

\item da laut Theorem \ref{theorem:3.14} $u_h \ra u$ für $h\ra 0$ geht, folgt: wenn ein isolierter Kontaktknoten $p \in \mcal N^{0+}$ bei Verfeinerung bestehen bleibt, so hat die exakte Lösung $u$ einen korrespondierenden Kontaktpunkt $\tilde p$, dann gilt
\begin{align*}
	\bigcup_{p \in \mcal N^{0+}} \omega_p \xrightarrow[h\ra 0]{} \tilde p
\end{align*}

\item damit gilt $\osc_1$ hat wenigstens den Grad vom Fehler $e$ (warum?)

\item wegen oben (mit dem hinreichend glatten $u, \psi$) verschwindet $\osc_1$ für $h \ra 0$

\item $\osc_2$ ist über zwei Mengen definiert:
\begin{align}
	\mcal N^{++} \coloneqq \{p \in \mcal N^{+}\with \rho_E \ge - d_E \, \forall \, E \in \mcal E_p\}
\end{align}
d.h. alle Punkte ohne Kontakt, in denen der Fehler $\eps_{\mcal V}$ nicht in Kontakt mit $\mcal A_{\mcal V}$ steht (wie in Beweis von Satz \ref{satz:4.11} ersichtlich)
\begin{align}
	\mcal N^{0-}\coloneqq \{ p \in \mcal N^0 \with u_{\mcal S} = \psi , f \le 0 \text{ auf } \omega_p, j_E \le 0 \, \forall \, E \in \mcal E_p \}
\end{align}
d.h. voller Kontakt (s. auch \cite{SiebVee} Gleichung (2.11)) mit Last $f$ auf Druck und negativem Normalenfluss $j_E$

\item aus der Nebenbedingung von $\mcal N^{0-}$ folgt
\begin{align*}
	0 \ge f + \sum_{E \in \mcal E_p} j_E
\end{align*}
durch Multiplikation mit geeigneten Testfunktionen $v$ und multiplizieren über $\omega_p$ ergibt
\begin{align*}
	0 &  \ge \int_{\omega_p} f v \, d\Omega + \sum_{E\in \mcal E_p} \int_E j_E v \, d \Gamma \\
	& = \int_{\omega_p} f v \, d \Omega - \int_{\omega_p} \underbrace{\nabla u_{\mcal S}}_{=\nabla \psi} \nabla v \, d\Omega
\end{align*}
und damit gilt
\begin{align}
	 \int_{\omega_p} {\nabla \psi} \nabla v \, d\Omega \ge \int_{\omega_p} f v \, d \Omega
\end{align}
es gilt also laut Satz \ref{satz:3.4}, dass $-\Delta \psi - f \ge 0$ auf $\omega_p$ im distributionellem Sinne (vgl. auch \cite{Walker} Kapitel 3)

dies ist laut Satz \ref{satz:3.4} auch notwendig, damit $u = \psi$ auf $\omega_p$ ist

\item damit ergibt sich $\osc_2$ als
\begin{align}
	\osc_2(u_{\mcal S},\psi, f) \coloneqq \(\sum_{p \in \mcal N^{++}} h_p^2 \norm{f-\bar f_p}_{0,\omega_p}^2 + \sum_{p \in \mcal N\setminus (\mcal N^{0-}\cup \mcal N^{++})} h_p^2 \norm f_{0,\omega_p}^2 \)^{\frac 12}
\end{align}
wobei $h_p \coloneqq \max_{E\in \mcal E_p} \abs E$ für jedes $p \in \mcal N$ ($h_p$ ist ein Maß für den Durchmesser von $\omega_p$) und $\bar f_p$ den Mittelwert von $f$ über $\omega_p$ bezeichne, d.h.
\begin{align}
	\bar f _p = \frac 1{\abs{\omega_p}} \int_{\omega_p} f \, d\Omega
\end{align}

\item anschaulich: damit kann man die Summanden der ersten Summe als Varianz der Last $f$ auf $\omega_p$ interpretieren 

$\mcal N\setminus (\mcal N^{0-}\cup \mcal N^{++})$ ist die Menge von Punkten, die keinen vollen Kontakt und in der $\eps_{\mcal V}$ keinen Kontakt mit $\mcal A_{\mcal V}$ hat

\item Beachte: Im Term $\osc_2$ fehlen nur die Punkte, die vollen Kontakt haben, d.h. wir betrachten also wirklich nur die Punkte außerhalb des Hindernisses!! (genauer noch: $\mcal N \setminus (\mcal N^{0-}\cup \mcal N^{++})$ sind nur die Randpunkte!)

\item damit enthält $\osc_2$ nur Anteile aus Knoten, ohne vollen Kontakt

\item Bem.: die Oszillationsterme können leicht berechnet werden (siehe hierfür auch Kapitel 5)

\item im unbeschränkten Fall, also $\psi = -\infty$, ist $\eps_{\mcal V}$ nicht im Kontakt mit dem Hindernis für alle Punkte aus $\mcal N$, also gilt $\mcal N^{++} = \mcal N$.

\item damit wird \eqref{eq:4.28} ($\osc_2$) zu
\begin{align}
	\osc_2(u_{\mcal S},\psi, f) = \(\sum_{p \in \mcal N\cap \Omega} h_p^2 \norm{f-\bar f_p}_{0,\omega_p}^2 + \sum_{p \in \mcal N \cap \partial \Omega} h_p^2 \norm f_{0,\omega_p}^2 \)^{\frac 12}
\end{align}

\item damit ist (4.28) eine Verallgemeinerung von (4.30): wenn der Teil ohne Kontakt also bekannt wäre, dann wäre der beschränkte Fall auf dieser Menge äquivalent zu einem unbeschränkten Dirichtlet-Problem

\item WICHTIG: Noch einmal in \cite{Zhang} schauen, ob dies in Verbindung des letzten Absatzes im Mainpaper verwendet werden kann!!!


\end{itemize}






\subsection{Zuverlässigkeit des Fehlerschätzers}
\label{kap:4.1.4}

\begin{itemize}
\item wir wollen in diesem Kapitel eine obere Schranke des Fehlers im Energiefunktional, die vom hierarchischen Fehlerschätzer abhängt, herleiten.

\item die Reduktion des Fehlers $e = u-u_{\mcal S} \in H^1_0(\Omega)$ auf den approximierten Fehler $\eps_{\mcal V}\in \mcal V$ erhalten wir durch lokale Projektionen für jedes $p \in \mcal N$ mit
\begin{align}
	\pi_p: H^1(\Omega) \ra \mcal Q_p = \Span \{ \phi_p\} \cup \mcal V_p \, , \quad \mcal V_p = \Span\{\phi_E \with E \in \mcal E_p\} \, .
\end{align}

\item $\pi_p$ ist für jedes $v \in H^1(\Omega)$ aus Dimensionsgründen ($\dim (\mcal Q_p) = p+1$) eindeutig bestimmt durch
\begin{align}\label{eq:4.32}
	\int_E \pi_p v \, d \Gamma = \int_E v \, d\Gamma \quad \forall \, E \in \mcal E_p \text{ und } \begin{cases}
														\int_{\omega_p} \pi_p v \, d \Omega = \int_{\omega_p}  v \, d\Omega  &,  p \in \mcal N^{++} \\
														0  &, \text{sonst}
													\end{cases} \, .
\end{align}

\item
\begin{lemma}
Es sei $\pi_p$ die oben beschriebene Projektion. Dann gelten für die Koordinaten bzgl. der Basis $\{\phi_p\}\cup \{\phi_E \with E \in \mcal E_p\}$ von 
\[
	\pi_p v = \alpha_p(v) \phi_p + \sum_{E \in \mcal E_p} \alpha_E(v) \phi_E
\]
die Beziehungen
\begin{subequations}\label{eq:4.33}
\begin{align}
	\alpha_p(v)& = \begin{cases}
					\frac{c_p(v)}{c_p(\phi_p)} &, p \in \mcal N^{++} \\
					0 &, \text{sonst}
				\end{cases} , \\
	 \alpha_E(v) &= \frac{\int_E v \, d\Gamma - \alpha_p(v)\int_E \phi_p \, d\Gamma}{\int_E \phi_E \, d\Gamma} \, ,
\end{align}
\end{subequations}
wobei
\begin{align*}
	c_p(v) = \int_{\omega_p} v \, d\Omega - \sum_{E\in \mcal E_p} \(\int_E v \, d\Gamma\) \(\int_{\omega_p} \phi_E \, d\Omega\)\(\int_E \phi_E \, d\Gamma\)^{-1}
\end{align*}
Insbesondere gilt $c_p(\phi_p) = -\frac 16\, \abs{\omega_p}$.
\end{lemma}

\begin{proof}
Für eine bessere Übersicht im Beweis werden wir die Differentialformen $d\Omega$ und $d\Gamma$ weg. Es sei $v \in H^1(\Omega)$ beliebig. Dann gilt für jede Kante $E \in \mcal E_p$ mit
\[
	\pi_p v = \alpha_p(v) \phi_p + \alpha_E (v) \phi_E \in \mcal Q_p 
\]
nach \eqref{eq:4.32}, dass
\begin{align}\notag
	 & \int_Ev  = \int_E \pi_p v  = \int_E \alpha_p(v) \phi_p + \alpha_E(v) \phi_E \\
	\label{eq:4.34}
	\lra \,  &  \alpha_E(v)= \(\int_E v - \alpha_p(v) \int_E \phi_p \)\( \int_E \phi_E \)^{-1} \, .
\end{align}
Wenn $p \not\in \mcal N^{++}$ ist, so gilt $\pi_p v \in \mcal V_p = \Span\{\phi_E\mid E \in \mcal E_p\}$, d.h. $\alpha_p(v) = 0$.

Es sei nun also $p \in \mcal N^{++}$. Dann folgt aus der zweiten Eigenschaft von \eqref{eq:4.32} und \eqref{eq:4.34} für $\pi_p v = \alpha_p(v) \phi_p + \sum_{E\in \mcal E_p} \alpha_E(v) \phi_E\in \mcal Q_p$
\begin{align*}
	\int_{\omega_p} v    =& \int_{\omega_p} \pi_p v  = \int_{\omega_p} \alpha_p(v) \phi_p + \sum_{E\in \mcal E_p} \alpha_E(v) \phi_E \\
	 =& \alpha_p(v) \int_{\omega_p} \phi_p  + \sum_{E\in \mcal E_p} \alpha_E(v) \int_{\omega_p}\phi_E  \\
	= & \alpha_p(v) \int_{\omega_p} \phi_p  + \sum_{E\in \mcal E_p} \(\int_E v  - \alpha_p(v) \int_E \phi_p \)\( \int_E \phi_E\)^{-1} \( \int_{\omega_p}\phi_E\) \\
	 =& \alpha_p(v) \Bigg(\underbrace{\int_{\omega_p} \phi_p-\sum_{E\in \mcal E_p}\( \int_E \phi_p\)\( \int_{\omega_p} \phi_E \)\(\int_E \phi_E\)^{-1}}_{= c_p(\phi_p)}\Bigg) \\
	 & + \sum_{E\in\mcal E_p} \(\int_E v\) \(\int_{\omega_p} \phi_E\) \(\int_E \phi_E\)^{-1} \, .
\end{align*}
Nach dem Umformen nach $\alpha_p(v)$ ergibt sich dann
\[
	\alpha_p(v) = \frac{c_p(v)}{c_p(\phi_p)}
\]
mit dem oben definierten $c_p(\cdot)$.

Es bleibt also zu zeigen, dass $c_p(\phi_p) = -\frac 1 6 \, \abs{\omega_p}$. Hierfür betrachten wir die einzelnen Summanden von $c_p(\phi_p)$. Zunächst berechnet das Integral von $\phi_p$ über $\omega_p$ das Volumen der von $\phi_p$ erzeugten Pyramide mit Grundfläche $\abs{\omega_p}$, d.h.
\begin{align}\label{eq:4.35}
	\int_{\omega_p} \phi_p = \frac 13 \, \abs{\omega_p} \, .
\end{align}
Weiter ist $\phi_p$ auf jeder Kante $E \in \mcal E_p$ eine von 1 zu 0 abfallende Gerade und damit ist das Integral über $E$ gerade der Flächeninhalt vom darüber liegenden Dreieck, also
\begin{align}\label{eq:4.36}
	\int_E \phi_p = \frac 1 2 \, \abs E \, .
\end{align}
Die letzten beiden Integrale berechnen wir über die Referenzelemente in $\R$ oder $\R^2$ für das Kurven- bzw. Flächenintegral. Die Funktion $\phi_E$ über eine Kante $E$ ist eine nach unten geöffnete Parabel. Auf dem Referenzelement $[-1,1]\subset\R$ ist dies die Funktion
\[
	\hat \phi_E = 1-\xi^2 
\]
und mit einer affinen Transformation $s: [-1,1]\ra [a,b] = E, s(\xi) = \frac{b-a}2 \xi + \frac{b+a}2$ lässt sich das Referenzelement auf das Element $E$ abbilden. Damit ergibt sich mit dem Transformationssatz der Integration
\begin{align}\label{eq:4.37}
	\int_E \phi_E = \frac{b-a}2 \int_{-1}^1 \hat\phi_E = \frac 12 \, \abs{E} \cdot \frac 43 = \frac 23 \, \abs E\, .
\end{align}
Der letzte Fall ist komplizierter zu beschreiben. Zunächst sei erwähnt, dass $\supp (\phi_E) = T_i \cup T_j, T_i,T_j \subset \omega_p, i\not=j$ gilt, $\phi_E$ also nur auf zwei Dreiecken, die in $\omega_p$ enthalten sind, ungleich Null ist. Damit wird für jede Kante $E \in \mcal E_p$ über jedes Dreieck $T \subset \omega_p$ genau zweimal integriert.

Auf dem Referenzelement 
\[	
	\hat T \coloneqq \{(\xi,\eta)\in \R^2 \mid 0\le \xi \le 1, 0 \le \eta \le 1-\xi\}
\]
haben wir die drei Bubble-Funktionen
\begin{align*}
	\hat \phi_{E_1} = 4 \xi (1-\xi-\eta) \, , \quad \hat \phi_{E_2} = 4 \xi \eta \, , \quad \hat \phi_{E_3} = 4 \eta (1-\xi-\eta) \, ,
\end{align*}
für die man leicht nachrechnen kann, dass
\begin{align*}
	\int_{\hat T} \hat\phi_{E_1} = \int_{\hat T} \hat\phi_{E_2} = \int_{\hat T} \hat\phi_{E_3} = \frac 16
\end{align*}
gilt. Es sei nun $J_T$ die Jacobi-Determinante bzgl. einer affinen Transformation $r:\hat T \ra T$, dann gilt nach Transformationssatz mit einem $T \subset \supp(\phi_E)$
\[
	\int_T \phi_E = \abs{J_T} \int_{\hat T} \hat \phi_E = \frac 16 \, \abs{J_T} \, .
\]
Weiter rechnen wir nach, dass
\[
	\abs T = \int_T d\Omega = \abs{J_T} \int_{\hat T} d\Omega = \frac 12 \, \abs{J_T} \, \lra \, \abs{J_T} = 2 \, \abs{T} 
\]
gilt und damit folgt insgesamt zusammen mit \eqref{eq:4.35} bis \eqref{eq:4.37}
\begin{align*}
	c_p(\phi_p) & = \int_{\omega_p} \phi_p - \sum_{E\in\mcal E_p} \(\int_E \phi_p\) \(\int_{\omega_p} \phi_E \)\(\int_E \phi_E\)^{-1} \\
	& = \frac 1 3 \, \abs{\omega_p} - 2 \sum_{T \subset \omega_p} \frac 1 2 \, \abs E \cdot \frac 1 6 \, \abs{J_T} \cdot \frac 3 2 \, \abs E^{-1} \\
	& = \frac 1 3 \, \abs{\omega_p} - \sum_{T\subset \omega_p} \frac 12 \, \abs T = \( \frac 1 3  -  \frac 1 2 \)\abs{\omega_p} \\
	& = - \frac 16 \, \abs{\omega_p} \, . \qedhere
\end{align*}
\end{proof}

\item für den wichtigsten Satz dieser Arbeit benötigen wir noch eine Eigenschaft der lokalen Projektionen
\begin{lemma}
Die Koeffizienten in \eqref{eq:4.33} erfüllen die Eigenschaft
\begin{align}\label{eq:my4.39}
	\max_{\mcal Q \in \{p\} \cup \mcal E_p} \abs{\alpha_{\mcal Q} (v)} \lesssim h^{-1}_p (\norm v_{0,\omega_p} + h_p \norm{\nabla v}_{0,\omega_p}) 
\end{align}
und $\pi_p$ ist stabil im Sinne von
\begin{align}\label{eq:my4.40}
	\norm{\pi_p v}_{0,\omega_p} \lesssim \norm v_{0,\omega_p} + h_p \norm{\nabla v}_{0,\omega_p} \, .
\end{align}
Insbesondere gilt, wenn $p \not\in \mcal N^{++}$ ist, dass für $\alpha_E(v) = \(\int_{E} v \, d\Gamma \) \(\int_E \phi_E\)^{-1}$ die Eigenschaft gilt:
\begin{align}\label{eq:my4.41}
	\int_E v \, d\Gamma \ge \int_E (\psi-u_{\mcal S}) \, d\Gamma \lra \alpha_E(v) \gtrsim \psi(x_E)-u_{\mcal S} (x_E) \quad \forall \, E \in \mcal E_p \, .
\end{align}
\end{lemma}

\begin{proof}
Beweis machen???
\end{proof}


\item folgendes Lemma ist zentral, um die obere Schranke von $J(u_{\mcal S})-J(u)$ bzgl. $-\mcal I_{\mcal Q}(\eps_{\mcal V})$ zu zeigen

\begin{lemma}\label{lem:4.19}
Es sei Voraussetzung \ref{vor:4.1} erfüllt. Dann gilt
\begin{align}
	\rho_{\mcal S}(e) \lesssim \sum_{E\in \mcal E} \eta_E \, \abs{\rho_E} + \osc(u_{\mcal S},\psi,f)^2 
\end{align}
mit $\rho_E$ wie in \eqref{eq:4.11}, $\osc (u_{\mcal S},\psi,f)$ wie in \eqref{eq:4.23} und $\eta_E = \abs{\eps_{\mcal V}(x_E)} \, \norm{\phi_E}$.
\end{lemma}

\begin{proof}
Die Idee des Beweises beruht auf Gleichung \eqref{eq:4.20}; damit können wir den Indikator in die lokalen Anteile bzgl. der Punkte $p \in \mcal N$ aufteilen, d.h.
\begin{align}\label{eq:4.40}
	\rho_{\mcal S} (e) = \sum_{p \in \mcal N} \rho_p (e) \, .
\end{align}
Hierbei ist die Abschätzung der lokalen Anteile $\rho_p(e)$ abhängig von einer Anwendung der \textit{\idx{Poincaré-Ungleichung}}\index{Ungleichungen!Poincaré} (vgl. auch \cite{Rudin})
\begin{align}\label{eq:4.41}
	\norm{v-c}_{0,\omega_p} \lesssim h_p \norm{\nabla v}_{0,\omega_p}
\end{align}
mit einer Konstanten $c$ und $v\in H^1(\Omega)$, so dass $v=0$ auf einer Menge $\Gamma \subset \partial \omega_p$ mit einem Maß $\mu (\Gamma)\not= 0$. Da \eqref{eq:4.41} von $p\in \mcal N$ abhängt, werden wir sehen, dass die Anwendung von der Poincaré-Ungleichung vom Typ des Knotens $p$ abhängt. Deshalb betrachten wir die disjunkte Vereinigung
\begin{align}\label{eq:4.42}
\begin{aligned}
	& \overbrace{\mcal N^{++} \cup (\mcal N^+ \setminus \mcal N^{++})}^{=\mcal N^+} \cup (\mcal N \cap \partial \Omega ) \cup\overbrace{ (\mcal N^0 \setminus (\mcal N^{0+} \cup \mcal N^{0-}))  \cup \mcal N^{0+} \cup \mcal N^{0-}}^{= \mcal N^0} \\
	= \, & \mcal N^+ \cup \mcal N^0 \cup (\mcal N \cap \partial \Omega)  = (\mcal N \cap \Omega) \cup (\mcal N \cap \partial \Omega) = \mcal N \, .
\end{aligned}
\end{align}
Wir wollen im Folgenden die in \eqref{eq:4.42} aufgeführten Fälle chronologisch abarbeiten.

\textit{Fall 1}: Es sei $p \in \mcal N^{++}$. Wir behaupten, dass
\begin{align}\label{eq:4.46}
	\rho_p(e) \lesssim \(\sum_{E\in \mcal E^+_p} \abs{\rho_E} + h_p \norm{f-\bar f_p}_{0,\omega_p} \) \norm {\nabla e}_{0,\omega_p}
\end{align}
gilt, wobei $\mcal E^+_p = \{E \in \mcal E_p \mid \rho_E \ge -d_E\}$. Da $p \in \mcal N^{++}$ ist, gilt für alle $E\in \mcal E_p$ die Ungleichung $\rho_E \ge -d_E$ ist, d.h. $\mcal E^+_P = \mcal E_p$. Wir setzen
\[
	w = (e-c)\phi_p \, , \quad c = \frac 1{\abs{\omega_p}} \int_{\omega_p} e\, d\Omega \, .
\]
Da $\mcal N^{++} \subset \mcal N^+ \cap \Omega$ ist, d.h. $u_{\mcal S}(p) > \psi (p)$ ist,  gilt  
\begin{align*}
	\rho_p(c) = c \, \rho_p(1) \stackrel{\scriptsize\eqref{eq:4.22b}}= c\cdot 0 = 0 \, .
\end{align*}
Damit erhalten wir
\begin{align}\notag
	\rho_p (e) & = \rho_p(e) - \rho_p(c) = \rho_p(e-c) \stackrel{\scriptsize \text{Lem. \ref{lem:4.14}}}= \int_{\omega_p} fw \, d \Omega + \sum_{E\in \mcal E_p} \int_E j_E w \, d\Gamma \\
	\notag
	&\!\! \!\stackrel{\scriptsize \text{"`}+0\text{"'}}= \int_{\omega_p} f \pi_p w \, d\Omega + \sum_{E\in \mcal E_p}\underbrace{ \int_E j_E \pi_pw \, d \Gamma}_{\stackrel[\scriptsize\eqref{eq:4.32}]{}= \int_E j_Ew \, d\Gamma} + \int_{\omega_p} f(w-\pi_p w) \, d\Omega \\
	\notag
	& = \rho_{\mcal S} (\pi_pw ) +  \int_{\omega_p} f(w-\pi_p w) \, d\Omega - \bar f_p \underbrace{ \int_{\omega_p} (w-\pi_p w) \, d\Omega }_{= 0 \text{ wegen } \eqref{eq:4.32}} \\
	\notag
	& = \rho_{\mcal S} (\pi_pw ) +  \int_{\omega_p} (f-\bar f_p)(w-\pi_p w) \, d\Omega \\
	\label{eq:4.47}
	& \!\!\! \; \; \!\! \stackrel{\scriptsize \text{C.S.}}\le \sum_{E \in \mcal E_p} \alpha_E(w) \rho_E \norm{\phi_E} + \norm{f-\bar f_p}_{0,\omega_p} \norm{w-\pi_p w}_{0,\omega_p} \, , 
\end{align}
wobei im letzten Schritt zusätzlich zur Cauchy-Schwarz-Ungleichung im zweiten Summanden noch angewendet wurde, dass
\begin{align*}
	\rho_{\mcal S} (\pi_p w) &= \rho_{\mcal S} \(\alpha_p(w) \phi_p + \sum_{E\in \mcal E_p} \alpha_E(w) \phi_E\) \\
	& = \alpha_p(w) \underbrace{\rho_{\mcal S} (\phi_p)}_{=\rho_p(1) = 0} \sum_{E\in \mcal E_p} \alpha_E(w) \!\!\!\! \underbrace{\rho_{\mcal S}(\phi_E)}_{\stackrel[\scriptsize \eqref{eq:4.11}]{}=\rho_E \norm{\phi_E}} \\
	& = \sum_{E \in \mcal E_p} \alpha_E(w) \rho_E\norm{\phi_E}
\end{align*}
ist. Da $\norm{\phi_p}_{\infty,\omega_p} \le 1$, gilt mit der Poincaré-Ungleichung \eqref{eq:4.41}
\begin{align}\label{eq:4.48}
	\norm w_{0,\omega_p} = \norm{(e-c)\phi_p}_{0,\omega_p} \le \norm{e-c}_{0,\omega_p} \lesssim h_p \norm{\nabla e}_{0,\omega_p} \, ,
\end{align}
wobei die erste Ungleichung aus dem Mittelwertsatz der Integralrechnung folgt.
Es sei weiterhin darauf hingewiesen, dass $\norm{\nabla \phi_p}_{\infty,\omega_p} \lesssim h^{-1}_p$, da die Steigung der Hutfunktion nur von der Form von $\omega_p$ abhängt. Damit erhalten wir dann durch Anwenden von \eqref{eq:my4.39} für alle $E \in \mcal E_p$
\begin{align}
\begin{aligned}\label{eq:4.49}
	\abs{\alpha_E(w)} & \lesssim h_p^{-1} (\norm{w}_{0,\omega_p} + h_p \norm{\nabla w}_{0,\omega_p}) \\
	& = h_p^{-1} (\norm{(e-c)\phi_p}_{0,\omega_p} + h_p \norm{\nabla ((e-c)\phi_p)}_{0,\omega_p}) \\
	& \le h_p^{-1} (h_p \norm{\nabla e}_{0,\omega_p} + h_p \norm{\underbrace{\nabla(e-c)}_{=\nabla e} \phi_p + (e-c) \nabla \phi_p}_{0,\omega_p}) \\
	& \! \;\!\!\stackrel{\triangle \not=}\le \norm{\nabla e}_{0,\omega_p} + \norm{\nabla e \, \phi_p}_{0,\omega_p}+ \norm{(e-c) \nabla \phi_p}_{0,\omega_p} \\
	& \lesssim 2 \, \norm{\nabla e}_{0,\omega_p} + h_p^{-1} \norm{e-c}_{0,\omega_p} \stackrel{\scriptsize \eqref{eq:4.48}} \lesssim  \norm{\nabla e}_{0,\omega_p} \, .
\end{aligned}
\end{align}
Über das Referenzelement $\hat T$ kann man zeigen, dass
\[
	\norm{\phi_E}  = a(\phi_E,\phi_E)^{\frac 1 2} =\(\int_{\Omega} \nabla \phi_E \nabla \phi_E \, d\Omega \)^{\frac 12} = \(\frac 83 (\abs{J_{T_1}}+\abs{J_{T_2}})\)^{\frac 12} \lesssim \tilde c \, ,
\]
da die Jacobi-Determinanten $J_{T_1},J_{T_2}$ (wobei $E \subset T_i,i=1,2$ gilt) endlich sind, jedoch von der Form von $\omega_p$ abhängen. Damit gilt
\begin{align}\label{eq:4.50}
	\norm{\nabla e}_{0,\omega_p} \approx \norm{\phi_E}^{-1} \norm{\nabla e}_{0,\omega_p} \, .
\end{align}
Analog folgt mit Anwendung von \eqref{eq:my4.40} und \eqref{eq:4.48}, dass gilt:
\begin{align}\label{eq:4.51}
\begin{aligned}
	\norm{w-\pi_p w}_{0,\omega_p} &\! \;\!\!\stackrel{\scriptsize \triangle \not=} \le \norm w_{0,\omega_p} + \norm{\pi_p w}_{0,\omega_p} \\
	& \lesssim h_p\norm{\nabla e}_{0,\omega_p}+\norm w_{0,\omega_p} + h_p \norm{\nabla e}_{0,\omega_p} \lesssim h_p \norm{\nabla e}_{0,\omega_p}
\end{aligned}
\end{align}
Setzen wir \eqref{eq:4.49} bis \eqref{eq:4.51} in \eqref{eq:4.47}, so folgt nach Ausklammern von $\norm{\nabla e}_{0,\omega_p}$ die Behauptung \eqref{eq:4.46} mit $\mcal E_p = \mcal E_p^+$.

\textit{Fall 2}: Es sei $p \in \mcal N^+\setminus \mcal N^{++}$. Wir behaupten, dass gilt:
\begin{align}\label{eq:4.52}
	\rho_p (e) \lesssim \(\sum_{E \in \mcal E^+_p} \abs{\rho_E} + h_p \norm f_{0,\omega_p}\) \norm{\nabla e}_{0,\omega_p} \, .
\end{align}
Auch hier können wir analog zu \eqref{eq:4.47} eine Ungleichung herleiten, wobei die zweite Addition der Null nicht gilt, da $p \not \in \mcal N^{++}$. Damit erhalten wir die Aussage
\begin{align}\label{eq:4.53}
	\rho_p (e) \le \sum_{E\in \mcal E_p} \alpha_E (w) \rho_E \norm{\phi_E} + \norm f_{0,\omega_p} \norm{w-\pi_p w}_{0,\omega_p} \, ,
\end{align} 
wobei wir hier
\begin{align}\label{eq:4.54}
	w = (e-c)\phi_p\, , \quad c = \min\left\{ \(\int_E e\phi_p \, d\Gamma \)\(\int_E \phi_p\, d\Gamma\)^{-1} \mid E \in \mcal E_p \right\}
\end{align}
setzen. Damit gilt
\begin{align*}
	\alpha_E(w) & = \(\int_E w \, d\Gamma \) \(\int_E \phi_E\, d\Gamma\)^{-1} = \(\int_E (e-c)\phi_p \, d\Gamma \) \(\int_E \phi_E\, d\Gamma\)^{-1} \\
	& = \(\int_E e\phi_p \, d\Gamma-c\int_E \phi_p \, d\Gamma \) \(\int_E \phi_E\, d\Gamma\)^{-1} \\
	&\!\!\!\!\!\;\stackrel{\scriptsize \eqref{eq:4.54}} \ge \!\!\;  \(\int_E e\phi_p \, d\Gamma-\(\int_E e\phi_p \, d\Gamma \)\(\int_E \phi_p\, d\Gamma\)^{-1} \(\int_E \phi_p \, d\Gamma \)\) \(\int_E \phi_E\, d\Gamma\)^{-1} \\
	& = 0 \, .
\end{align*}
Daraus können wir folgern, dass für die Kanten $E\in \mcal E_p$ mit $\rho_E < -d_E \le 0$
\[
	\alpha_E(w)\rho_E \le 0
\]
gilt. Ersetzen wir dies in \eqref{eq:4.53}, so folgt \eqref{eq:4.52} insgesamt mit denselben Abschätzungen aus \eqref{eq:4.49}, \eqref{eq:4.50} und \eqref{eq:4.51}.

\textit{Fall 3}: Sei $p \in \mcal N \cap \partial \Omega$. Wir behaupten für diesen Fall, dass
\begin{align}\label{eq:4.55}
	\rho_p (e) \lesssim \sum_{E\in \mcal E^0_p} d_E \, \abs{\rho_E} + \(\sum_{E\in \mcal E_p^+} \abs{\rho_E} + h_p \norm{f}_{0,\omega_p}\)\norm{\nabla e}_{0,\omega_p}
\end{align}
mit $\mcal E_p^0 = \mcal E_p \setminus \mcal E_p^+ = \{E \in \mcal E_p \mid \rho_E < -d_E\}$. Auch hier betrachten wir die Ungleichung \eqref{eq:4.53}, wobei wir dieses Mal $w = e\phi_p$ setzen, d.h. mit der obigen Wahl $c = 0$ setzen. In diesem Fall kann kein anderes $c$ gewählt werden, da wir wegen $p \not\in \Omega$ nicht direkt $\rho_p (c) = 0$ aus \eqref{eq:4.22b} folgern können. Da $p \in \partial \Omega$ ist, gilt mindestens auf einer Kante $E \in \mcal E_p$ von $\partial \omega_p$
\[
	e = u- u_{\mcal S} = 0 \, .
\]
Damit ist $e$ auf einer Teilmenge, vom Maße ungleich Null, des Randes gleich Null und wir können daher die allgemeine Poincaré-Ungleichung \eqref{eq:4.41} anwenden.



\vspace{3cm}



Wir machen uns, bevor wir die sechs Fälle zusammenführen, klar, dass gilt
\begin{align*}
	 & (\mcal N^+\setminus \mcal N^{++})  \cup (\mcal N\cap \partial \Omega)  \cup \overbrace{(\mcal N^0 \setminus (\mcal N^{0-} \cup \mcal N^{0+}))  \cup\mcal N^{0+}}^{=\mcal N^0\setminus \mcal N^{0-}} \\
	 = \, & (\mcal N\cap \Omega) \setminus (\mcal N^{0-}\cup \mcal N^{++})\cup (\mcal N \cap \partial \Omega) = \mcal N \setminus (\mcal N^{0-} \cup \mcal N^{++}) \, .
\end{align*}
Verwenden wir nun die sechs gezeigten Fälle, so ergibt sich:
\begin{align*}
	\rho_{\mcal S}(e) = & \sum_{p \in \mcal N} \rho_p(e) \\
	= & \sum_{E\in \mcal E^0} d_E\, \abs{\rho_E} 
		+ \sum_{p \in \mcal N \setminus \mcal N^{0-}} \Bigg( \sum_{E \in \mcal E_p^+} \abs{\rho_E} \Bigg) \norm{\nabla e}_{0,\omega_p} \\
	 & + \! \! \! \sum_{p \in \mcal N\setminus(\mcal N^{0-}\cup\mcal N^{++})}\!\!\! h_p \norm{f}_{0,\omega_p} \norm{\nabla e}_{0,\omega_p} + \sum_{p \in \mcal N^{++}} h_p \norm{f-\bar f_p}_{0,\omega_p} \norm{\nabla e}_{0,\omega_p} \\
	 & + \sum_{p \in \mcal N^{0+}} \Bigg(\sum_{E\in \mcal E^+_p} \abs{\rho_E} + h_p \norm f_{0,\omega_p}\Bigg) \norm{\nabla(\psi-u_{\mcal S})}_{0,\omega_p}  \, .
\end{align*}
Damit folgt nach der Cauchy-Schwarz-Ungleichung, dass mit einer Konstante $C>0$, die nur von der Quasi-Uniformität von $\mcal T_h$ abhängt, gilt:
\begin{align*}
	C \rho_{\mcal S}(e) \le & \sum_{E\in \mcal E^0} d_E\, \abs{\rho_E} + \Bigg(\sum_{E\in \mcal E^+} \abs{\rho_E}^2 + \sum_{p \in \mcal N\setminus(\mcal N^{0-}\cup\mcal N^{++})} h_p^2 \norm{f}^2_{0,\omega_p}  \\
	& +  \sum_{p \in \mcal N^{++}} h_p^2 \norm{f-\bar f_p}_{0,\omega_p}^2\Bigg)^{\frac 12} \Bigg( \norm{\nabla e}_{0,\Omega} + \sum_{p \in \mcal N^{0+}} \norm{\nabla(\psi-u_{\mcal S})}_{0,\omega_p}^2 \Bigg)^{\frac 1 2} \\
	= &  \sum_{E\in \mcal E^0} d_E\, \abs{\rho_E} + \Bigg(\sum_{E\in \mcal E^+} \abs{\rho_E}^2 + \osc_2(u_{\mcal S},\psi,f)^2\Bigg)^{\frac 12} \Big( \norm{\nabla e}_{0,\Omega} + \osc_1(u_{\mcal S},\psi)^2 \Big)^{\frac 1 2}
\end{align*}
\end{proof}




\item Theorem für obere Schranke des Fehlerschätzers:
\begin{theorem}
Es sei Voraussetzung \ref{vor:4.1} für $\psi$ erfüllt. Dann ist der hierarchische Fehlerschätzer $-\mcal I_{\mcal Q}(\eps_{\mcal V})$ eine obere Schranke für den Fehler im Energiefunktional bis auf Addition von Oszillationstermen und einer Konstante $C$, die nur von der Quasi-Uniformität von $\mcal T_h$ abhängt, d.h.
\begin{align}
	J(u_{\mcal S}) - J(u) \lesssim -\mcal I_{\mcal Q}(\eps_{\mcal V}) + \osc (u_{\mcal S},\psi,f)^2 \, .
\end{align}
\end{theorem}

\begin{proof}
Die Aussage folgt direkt durch Lemma \ref{lem:4.12} und \ref{lem:4.19}, denn
\begin{align*}
	J(u_{\mcal S})-J(u) & = - \mcal I (e) \le \rho_{\mcal S}(e) \\
	& \lesssim \underbrace{\sum_{E\in \mcal E}\eta_E \, \abs{\rho_E}}_{=\rho_{\mcal S}(\eps_{\mcal V})} + \osc (u_{\mcal S},\psi,f)^2 \\
	& \le 2\cdot (-\mcal I_{\mcal Q}(\eps_{\mcal V}))+ \osc (u_{\mcal S},\psi,f)^2 \\
	& \le 2 \cdot (-\mcal I_{\mcal Q}(\eps_{\mcal V})+ \osc (u_{\mcal S},\psi,f)^2)
\end{align*}
und damit folgt die Behauptung.
\end{proof}

\item an dieser Abschätzung können wir sehen, dass es sinnvoll ist, nicht nur den hierarchischen Fehlerschätzer $-\mcal I_{\mcal Q}(\eps_{\mcal V})$ zum Abschätzen des Fehlers zu verwenden, sondern auch die Oszillationsterme (diese sollten von Verfeinerungsschritt zum Verfeinerungsschritt kleiner werden, sonst ist die Verringerung im exakten Fehler nicht gesichert)

\item damit nachher eine Abschätzung analog zu \cite{MorNoc} Lemma 3.8 (wäre schön, wenn diese noch gezeigt werden würde....)
\begin{lemma}
Es sei $0 < \gamma < 1$ ein Parameter, der die Reduktion der Größe des Dreiecks bei Verfeinerung wiedergibt. Weiter sei $0 < \hat \theta < 1$ gegeben und eine Menge an Punkten $\hat{\mcal N}\subset \mcal N$, die die zu verfeinernden Dreiecke anzeigen, gegeben, so dass
\[
	\osc(u_{\mcal S},\psi,f,\hat{\mcal N}) \ge \hat\theta \osc (u_{\mcal S},\psi,f,\mcal N) \, .
\]
Dann existiert ein $\hat\alpha \in (0,1)$, so dass
\begin{align}
	\osc(u_{\mcal S},\psi,f,\tilde{\mcal N}) \le \hat \alpha \osc (u_{\mcal S},\psi,f,\mcal N) \, ,
\end{align}
wobei $\tilde{\mcal N}$ die Menge an Punkten nach Verfeinerung der Triangulierung $\mcal T_h$ bzgl. der Punkte $\hat{\mcal N}$ ist.
\end{lemma}

Hierfür vllt eine äquivalente Darstellung von $\osc_2$ bzgl. der Dreiecke und dann das Lemma nur auf $\osc_2$ beziehen.
\end{itemize}






\subsection{Effektivität des Fehlerschätzers}
\label{kap:4.1.5}

\begin{itemize}
\item wir zeigen, dass der hierarchische Fehlerschätzer $-\mcal I_{\mcal Q}(\eps_{\mcal V})$ ist auch eine untere Schranke für $-\mcal I(e) = J(u_{\mcal S}) - J(u)$

\item
\begin{theorem}
Das Hindernis $\psi$ sei stückweise linear und stetig. Dann ist der hierarchische a posteriori Fehlerschätzer $\mcal I_{\mcal Q}(\eps_{\mcal V})$ auch eine untere Schranke für den Fehler im Energiefunktional im Sinne von
\begin{align}
	-\mcal I_{\mcal Q}(\eps_{\mcal V}) \le 6 (J(u_{\mcal S})-J(u)) \, .
\end{align}
\end{theorem}

\begin{proof}
Zunächst folgt mit \eqref{eq:4.16}
\begin{align}\notag
	-\mcal I_{\mcal Q} (\eps_{\mcal V}) & \le \rho_{\mcal S} (\eps_{\mcal V}) = \rho_{\mcal S} \(\sum_{E\in \mcal E} \eps_{\mcal V}(x_E) \phi_E\) \\
	\notag
	& = \sum_{E\in \mcal E} \eps_{\mcal V} (x_E) \rho_{\mcal S}(\phi_E) \\
	& = \sum_{E \in \mcal E} \eta_E \, \abs {\rho_E}  \label{eq:my4.32}
\end{align}
mit $\eta_E = \abs{\eps_{\mcal V}(x_E)} \cdot \norm{\phi_E}$ und $\rho_E = \frac {\rho_{\mcal S}(\phi_E)}{\norm{\phi_E}}$, wobei man zeigen kann, dass $\sign (\eps_{\mcal V}(x_E)) = \sign (\rho_{\mcal S} (\phi_E))$ gilt. Weiter sollte man erwähnen, dass \eqref{eq:my4.32} äquivalent ist zu \cite{SiebVee} Gleichung (2.16).

Das weitere Vorgehen ist ähnlich zum Beweis von Theorem 3.2 aus \cite{SiebVee}. Es sei
\[
	\varphi = \frac 13 \sum_{E\in \mcal E} \beta_E \phi_E
\]
eine Linearkombination aus Bubble-Funktionen. Dann lässt sich $u_{\mcal S} + \varphi$ auf jedem $T \in \mcal T_h$ durch eine Konvexkombination aus $v_E \coloneqq u_{\mcal S} + \beta_E \phi_E, E \in \mcal E$ schreiben, d.h.
\[
	(u_{\mcal S} + \varphi)\Big|_{T} = \frac 13 \sum_{E\in\mcal E, E \subset T} v_E \Big|_T \, .
\]
Da $\R^2 \ni x \mapsto \frac 12 \abs x^2$ konvex ist, rechnen wir mit den obigen Bezeichnungen schnell nach, dass gilt
\begin{align*}
	J(u_{\mcal S} + \varphi) & = \int_\Omega \frac 12 \abs{\nabla(u_{\mcal S}+\varphi)}^2 - f (u_{\mcal S} + \varphi )\, d\Omega \\
	& = \sum_{T\in \mcal T_h} \int_T \frac 12 \Abs{\nabla(u_{\mcal S}+\varphi)\Big|_{T}}^2 - f (u_{\mcal S} + \varphi )\Big|_T\, d\Omega \\
	& =  \sum_{T\in \mcal T_h} \int_T \frac 12 \Abs{\(\frac 13 \sum_{E\in\mcal E, E \subset T}\nabla v_E \Big|_T\)}^2 - f\( \frac 13 \sum_{E\in\mcal E, E \subset T} v_E \Big|_T\)\, d\Omega \\
	& \le \frac 13 \sum_{E\in\mcal E, E \subset T} \, \sum_{T\in \mcal T_h} \int_T \frac 12 \Abs{\nabla v_E \Big|_T}^2 - f v_E \Big|_T\, d\Omega \, .
\end{align*}
Da wir drei Kanten pro Dreieck $T$ haben, gilt analog die Gleichung
\[
	J(u_{\mcal S}) = \frac 13 \sum_{E\in\mcal E , E \subset T}\,  \sum_{T\in \mcal T_h} \int_T \frac 12 \Abs{\nabla u_{\mcal S}}^2 - f u_{\mcal S}\, d\Omega  \, .
\]
Durch Subtraktion der letzten beiden Terme und einigen Umformungen ergibt sich dann
\begin{align}
	J(u_{\mcal S}) - J(u_{\mcal S}+\varphi) \ge \frac 13 \sum_{E\in \mcal E} (J(u_{\mcal S})-J(u_{\mcal S} + \beta_E \phi_E)) \, .
\end{align}
Wir rechnen nach, dass für alle $E\in \mcal E$
\begin{align*}
	J(u_{\mcal S} + \beta_E \phi_E) & = \frac 12 a(u_{\mcal S}+\beta_E \phi_E, u_{\mcal S} + \beta_E \phi_E) - (f, u_{\mcal S} + \beta_E \phi_E) \\
	& = J(u_{\mcal S}) + \frac 12 a(\beta_E\phi_E,\beta_E\phi_E)-((f,\beta_E\phi_E)-a(u_{\mcal S},\beta_E\phi_E)) \\
	& = J(u_{\mcal S}) + \mcal I (\beta_E\phi_E) 
\end{align*}
gilt. Damit ist das Minimieren von $J(u_{\mcal S}+\beta_E\phi_E)$, so dass $\beta_E \ge -d_E$, mit $d_E$ wie oben definiert, äquivalent ist zum Problem:
\[
	\min_{\beta_E \ge -d_E} \mcal I (\beta_E\phi_E) \, ,
\]
was den nächsten Schritt legitimiert. Wir setzen nun $\beta_E = \eps_{\mcal V}(x_E)$, dann gilt, dass $u_{\mcal S} + \beta_E \phi_E \in  K$ ist für alle $E \in \mcal E$ und damit aufgrund der Konvexität von $K$ auch $u_{\mcal S}+\varphi \in \mcal K$. Damit folgt insgesamt
\begin{align*}
	J(u_{\mcal S}) -J(u) & \ge J(u_{\mcal S})-J(u_{\mcal S}+\varphi) \\
	& \ge  \frac 13 \sum_{E\in \mcal E} (J(u_{\mcal S})-J(u_{\mcal S} + \beta_E \phi_E))=  \frac 13 \sum_{E\in \mcal E} -\mcal I (\beta_E \phi_E) \\
	&  =   \frac 13 \sum_{E\in \mcal E} \(\rho_{\mcal S}(\beta_E \phi_E) - \frac 12 a(\beta_E\phi_E,\beta_E\phi_E)\) \\
	& =   \frac 13 \sum_{E\in \mcal E} \(\beta_E \, \rho_{\mcal S}(\phi_E) - \frac 12 \beta_E^2 a(\phi_E,\phi_E)\) \\
	& \ge   \frac 13 \sum_{E\in \mcal E}\(\frac{\max\{-d_E,\rho_E\}}{\norm{\phi_E}} \, \rho_{\mcal S}(\phi_E) - \frac 1 2 \frac{\max\{-d_E,\rho_E\}^2}{\norm{\phi_E}^2} \norm{\phi_E}^2 \) \\
	& = \frac 13 \sum_{E\in \mcal E} \bigg( \underbrace{\max\{-d_E,\rho_E\} \rho_E}_{=\eta_E \, \abs{\rho_E}} - \frac 12 \underbrace{\max\{-d_E,\rho_E\}^2}_{\ge \eta_E \, \abs{\rho_E}} \bigg) \\
	& \ge \frac 13 \sum_{E\in \mcal E} \frac 12 \eta_E \, \abs{\rho_E} = \frac 16 \sum_{E\in \mcal E} \eta_E \, \abs{\rho_E} \, .
\end{align*}
Zusammen mit \eqref{eq:my4.32} folgt dann die Behauptung.
\end{proof}
\end{itemize}






\section{Ein adaptiver Algorithmus}
\label{kap:4.2}

\begin{itemize}
\item
\end{itemize}






\section{Erfüllung einer Saturationseigenschaft}
\label{kap:4.3}

\begin{itemize}
\item
\end{itemize}






\section{Übertragung des Fehlerschätzers auf Kontaktprobleme}
\label{kap:4.4}

\begin{itemize}
\item
\end{itemize}






\newpage

%%% Local Variables: 
%%% mode: latex
%%% TeX-master: "Skript"
%%% End: 
