\documentclass[a4paper,11pt,oneside]{book}

\usepackage[utf8]{inputenc}
%\usepackage[T1]{fontenc}
%\usepackage{pslatex}
\usepackage[intlimits]{amsmath}
\usepackage{amsfonts}
\usepackage{amssymb}
\usepackage{amsthm}
\usepackage{enumerate}
\usepackage[ngerman]{babel}
\usepackage[dvips]{graphicx}
%\usepackage{ps4pdf}
%\PSforPDF{
	\usepackage{pstricks}
	\usepackage{pst-all}
	\usepackage{multido}
	\usepackage{pst-plot}
	\usepackage{pst-3dplot}
%}
%\usepackage{pdftricks}
\usepackage{hyperref}
\usepackage[mathscr]{euscript}
\usepackage{makeidx}
%\usepackage{multind}
\usepackage{fancyhdr}
\usepackage{mathtools}
\usepackage[cmtip,arrow]{xy}
\usepackage{pb-diagram,pb-xy}
\usepackage{breqn}
%\usepackage{dsfont}
%\usepackage{bbold}
%\usepackage[active]{preview}
\usepackage{stackrel}
\usepackage[makeroom]{cancel}
\usepackage[section]{algorithm}
\usepackage{algorithmicx}
\usepackage{algpseudocode}


% Theoreme
%\theoremstyle{plain}
\newtheorem{satz}{Satz}[chapter]
\newtheorem{lemma}[satz]{Lemma}
\newtheorem{theorem}[satz]{Theorem}
\newtheorem{kor}[satz]{Korollar}
\theoremstyle{definition}
\newtheorem{defi}[satz]{Definition}
\newtheorem{bemdef}[satz]{Bemerkungen und Definitionen}
\newtheorem{bsp}[satz]{Beispiel}
\newtheorem*{bsp*}{Beispiel}
\newtheorem{bem}[satz]{Bemerkung}
\theoremstyle{remark}
\newtheorem*{bem*}{Bemerkung}
\newtheorem*{folge}{Folgerung}
\newtheorem*{einschr}{Einschränkung}
\newtheorem*{erinnerung}{Erinnerung}
\newtheorem*{notation}{Notation}
\newtheorem*{vor}{Voraussetzungen}

% kleine Helferlein
\newcommand{\R}{\mathbb{R}}
\newcommand{\N}{\mathbb{N}}
\newcommand{\K}{\mathbb{K}}
\newcommand{\C}{\mathbb{C}}
\newcommand{\B}{\mathbb{B}}
\newcommand{\D}{\mathscr{D}}
\newcommand{\E}{\mathscr{E}}
\newcommand{\F}{\mathscr{F}}
\renewcommand{\H}{\mathbb{H}}
\newcommand{\Ext}{\text{Ext}}
\newcommand{\id}{\text{id}}
\renewcommand{\L}{\mathcal{L}}
\newcommand{\Lloc}{L_{1,\scriptsize{\textnormal{loc}}}}
\newcommand{\diff}[1]{\frac{d}{d #1}}
\newcommand{\pdiff}[1]{\frac{\partial}{\partial #1}}
\renewcommand{\div}{\operatorname{div}}
\newcommand{\spur}{\operatorname{spur}}
\DeclareMathOperator{\graph}{graph}
\DeclareMathOperator{\supp}{supp}
\DeclareMathOperator{\vol}{vol}
\DeclareMathOperator{\dist}{dist}
\DeclareMathOperator{\im}{im}
\newcommand{\spn}{\operatorname{span}}
\renewcommand{\Re}{\operatorname{Re}}
\renewcommand{\Im}{\operatorname{Im}}
\renewcommand{\d}{\,\mathrm{d}}
\newcommand{\lra}{\Longrightarrow}
\newcommand{\Ra}{\Rightarrow}
\newcommand{\la}{\leftarrow}
\newcommand{\La}{\Leftarrow}
\newcommand{\ra}{\rightarrow}
\newcommand{\Lra}{\Leftrightarrow}
\newcommand{\Llra}{\Longleftrightarrow}
\newcommand{\hhookrightarrow}{\lhook\mkern-3mu\relbar\mkern-12mu\hookrightarrow}
\newcommand{\abs}[1]{\lvert #1\rvert}
\newcommand{\Abs}[1]{\left\lvert #1\right\rvert}
\newcommand{\norm}[1]{\lVert #1\rVert}
\newcommand{\Norm}[1]{\left\lVert #1\right\rVert}
\newcommand{\fa}{\forall}
\newcommand{\with}{\;\vert\,}
\newcommand{\<}{\langle}
\renewcommand{\>}{\rangle}
\renewcommand{\epsilon}{\varepsilon}
\newcommand{\dom}{\operatorname{dom}}
\newcommand{\mcal}{\mathcal}
\newcommand{\mscr}{\mathscr}
\newcommand{\eps}{\varepsilon}
\newcommand{\bs}{\boldsymbol}
\newcommand{\tr}{\textnormal{tr }}
\newcommand{\sign}{\textnormal{sign}}

\newcommand{\rz}[1]{%
	\ensuremath{% stellt Mathe-Modus sicher
		\mathord{% für richtige Abstände, siehe TeX Book
			\mathrm{% aufrechte Buchstaben
				#1%
			}%
		}%
	}%
}

\renewcommand{\(}{\left(}
\renewcommand{\)}{\right)}



% Gleichung nummerieren
\newcommand{\eq}[1]{\
  \addtocounter{equation}{1}
  \tag{\theequation}
  \label{#1}
}

% Index
\newcommand{\idx}[1]{\index{#1}#1}
\makeindex{}

\newcommand{\nameofchapter}{}
\newcommand{\newchapter}[1]{\chapter{#1}
	\renewcommand{\nameofchapter}{#1}
	\setcounter{satz}{0}
	\setcounter{equation}{0}}


% Dokument
\begin{document}
\frontmatter

%Titelseite 
\begin{titlepage} 

\begin{flushright}
 \includegraphics[width=6.75cm]{luh_logo3.pdf}\\
 {\large Fakultät für Mathematik und Physik \\
Institut für Angewandte Mathematik} \\
\end{flushright}

\centering 

\vspace{2cm}

\Large
 Diplomarbeit\\[2cm]\huge 
Ein hierarchischer Fehlerschätzer für Hindernisprobleme\\[2.5cm]\Large 
von Cornelius Rüther \\
Matr.-Nr.: 2517350 \\[2.5cm]
\today \\  
\vfill
Erstprüfer: Prof. Dr. Gerhard Starke\\ % hier auch gerne eine Tabelle 
Zweitprüfer: Prof. Dr. Peter Wriggers  
\end{titlepage} 

\setcounter{page}{2}


% Inhaltsverzeichnis

\tableofcontents
%\addcontentsline{toc}{chapter}{Inhaltsverzeichnis}
\thispagestyle{fancy}{
	\rhead{}
	\lhead{\sl Inhaltsverzeichnis}
	%\renewcommand{\headrulewidth}{0.4pt}
	\renewcommand{\headheight}{14pt}
	\renewcommand{\footrulewidth}{0.4pt}
	\cfoot{\thepage}
}

%Abbildungsverzeichnis
\listoffigures
\addcontentsline{toc}{chapter}{Abbildungsverzeichnis}

% Tabellenverzeichnis
\listoftables
\addcontentsline{toc}{chapter}{Tabellenverzeichnis}


\mainmatter

\setcounter{page}{6}

%\setcounter{page}{\thecountersave}

\pagestyle{fancy}{
	\rhead{}
	\lhead{\sl\thechapter. \nameofchapter}
	%\renewcommand{\headrulewidth}{0.4pt}
	\renewcommand{\headheight}{14pt}
	\renewcommand{\footrulewidth}{0.4pt}
	\cfoot{\thepage}
}


\newchapter{Einleitung}
\label{kap:1}

\begin{itemize}
\item Thema (worum geht es?) $\ra$ Fehlerabschätzung $\ra$ analytische Lösung oftmals nicht bekannt und damit Fehlerschätzer interessant
\item[$\ra$] in FEM soll Lösung genauer mit weniger Rechenzeit sein, daraus folgt Anwendung adaptiver Verfahren mit verschiedenen Fehlerschätzern
\item Lücke zum neuen (Kontaktproblematik) füllen in dieser Arbeit
\item[$\ra$] Übertragung unseres Fehlerschätzers auf Kontaktprobleme, wie und warum?! $\ra$ möglicher Grund: Hindernisprobleme beinhalten Kontaktbereiche (später für Kapitel 4 interessant)
\item[wichtig:] Vorgehen einer adaptiven Verfeinerungsstrategie mit "`solve $\ra$ estimate $\ra$ ...."' umschreiben
\item Struktur der Arbeit
\end{itemize}


\newpage

%%% Local Variables: 
%%% mode: latex
%%% TeX-master: "Skript"
%%% End: 

\newchapter{Grundlagen}
\label{kap:2}

In diesem Kapitel wollen wir uns mit grundlegender Theorie beschäftigen, die nicht im Anhang aufgeführt, zum Verständnis von den darauffolgenden Kapiteln jedoch notwendig ist.

Dieses Kapitel basiert auf \cite{BraeFEM}, \cite{StarkePDE}, \cite{EPS}, \cite{Walker}, \cite{AltKonti}.


\section{Hilberträume}
\label{kap:2.1}


Wir benötigen für die Variationsrechnung Hilberträume und wollen uns daher in diesem Kapitel mit wichtigen Eigenschaften solcher Räume im Allgemeinen beschäftigen. Zunächst führen wir ein, was wir unter einem Hilbertraum verstehen.


\begin{defi}\label{def:2.1}
Ein \textit{\idx{Hilbertraum}} ist ein reeller oder komplexer Vektorraum $H$ mit Skalarprodukt $(\cdot, \cdot)_H$, der vollständig bzgl. der durch das Skalarprodukt induzierten Norm, $\norm v_H^2 \coloneqq{(v,v)_H}$ für alle $v \in H$, ist, d.h. in dem jede Cauchy-Folge konvergiert.
\end{defi}


\begin{bsp*}
Es sei $H = \R^n$ und $(\cdot,\cdot)_H : \R^n\times \R^n \ra \R$ definiert durch das Standartskalarprodukt. Dann konvergiert jede \idx{Cauchy-Folge} in $H$ bzgl. der durch $(\cdot,\cdot)_H$ induzierten (euklidischen) Norm (vgl. \cite{Ana2}, metrische Räume) und damit ist $H$ ein Hilbertraum.
\end{bsp*}


Wir wollen die im Folgenden aufgeführten Eigenschaften später auf weniger triviale Räume anwenden, vor allem den Funktionenraum $H^1_0(\Omega)$ (s. Anhang \ref{anhang:A} Sobolev-Räume\index{Sobolev-Raum}). Um alle Aussagen auch allgemein verwenden zu können, sei in diesem Kapitel $H$ ein reeller Hlbertraum mit Skalarprodukt $(\cdot,\cdot)_H$ und der dazu induzierten Norm $\norm v_H^2 = (v,v)_H$ für alle $v \in H$.


\begin{bem*}
Für alle $v,w \in H$ gilt die \idx{Cauchy-Schwarz'sche Ungleichung}
\[
	(v,w)_H \le \norm v_H \, \norm w_H \, .
\]
\end{bem*}


Da wir uns in dieser Arbeit mit Variationsproblemen über konvexen abgeschlossenen Mengen beschäftigen werden, sammeln wir zunächst einige Aussagen bzgl. dieser Mengen.


\begin{satz}[\idx{Approximationssatz}]\label{satz:2.2}
Es sei $\emptyset\neq M\subset H$ konvex und abgeschlossen. Dann existiert für alle $v\in H$ ein $m_v\in M$ mit
\[ 
  	\norm{v-m_v}=\dist(v,M)\coloneqq \inf_{w\in M}\norm{v-w}\, .
\]
Wir nennen $P_M:H\ra M$ mit $v\mapsto m_v$ die \idx{Projektionen} auf $M$.
\end{satz}

\begin{proof}
Der Beweis ist in \cite{Walker} Kapitel 7.1 Satz 7.2 zu finden.
\end{proof}


\begin{satz}[Charakterisierung der Projektionen]\label{satz:2.3}
$\emptyset\neq M\subset H$ sei abgeschlossen und konvex und $v\in H$. Dann gilt:
\[ 
  	m_0=P_M(v)\quad\Longleftrightarrow\quad (m-m_0, v-m_0)_H\leq0 
\]
für alle $m\in M$.
\end{satz}

\begin{proof}
 Es sei o.B.d.A. $0\in M$ und $m_0=0$.
 
  "`$\Rightarrow$"' Wegen $0=P_M(x)$ muss $\norm{v-tm}_H\geq\norm v_H$ für $m\in M$ und $0\leq t\leq1$ sein. Dann ist
\begin{align*}
    	 \norm v^2_H\leq\norm v^2_H-2t(v, m)_H+t^2\norm m^2_H
	\ \lra \ 0 \le - 2t (v,m)_H + \underbrace{t^2 \norm m_H^2}_{\ge 0} \, .
 \end{align*}
Damit ist $2(v, m)_H\leq0$.
 
 "`$\La$"' Für alle $m\in M$ ist $(v, m)_H\leq0$. Es folgt
\[ 
	\norm v^2_H\leq\norm v^2_H+\norm m^2-2(v,m)_H=\norm{v-m}^2_H\, . 
\]
Wegen $0\in M$ ist $\dist(v,M)=\norm v^2_H$ und damit $0=P_M(v)$.
\end{proof}


\begin{satz}\label{satz:2.4}
Es sei $\emptyset \not = M \subset H$ konvex und abgeschlossen. Dann gilt:
\[
	\norm{P_M(v)-P_M(w)}_H \le \norm{v-w}_H \quad \forall \, v,w \in H \, .
\]
\end{satz}

\begin{proof}
Da $P_M(v), P_M(w) \in M$ für alle $v,w \in H$ ist, folgt aus Satz \ref{satz:2.3}
\begin{align}\label{eq:2.1}
	(P_M(w)-P_M(v),v-P_M(v))_H  \le 0 \, , \\
	(P_M(v)-P_M(w),w-P_M(w))_H \le 0 \, .\label{eq:2.2}
\end{align}
Addieren wir \eqref{eq:2.1} und \eqref{eq:2.2}, so erhalten wir
\begin{align*}
	0 &\ge (P_M(w)-P_M(v),v-P_M(v))_H + (P_M(v)-P_M(w),w-P_M(w))_H \\
	& = (P_M(w)-P_M(v),v-w+P_M(w)-P_M(v))_H \\
	& = \norm{P_M(w)-P_M(v)}_H^2-(P_M(w)-P_M(v),w-v)_H \\
	&\!\!\!\: \stackrel{\scriptsize \text{C.S.}}\ge \norm{P_M(w)-P_M(v)}_H^2 - \norm{P_M(w)-P_M(v)}_H \, \norm{w-v}_H \, .
\end{align*}
Nach Umstellen der Ungleichung folgt die Behauptung.
\end{proof}


\begin{defi}\label{def:2.5}
Es sei $\emptyset\neq M\subset H$ und wir definieren das \textit{{orthogonale Komplement}}\index{orthogonales~Komplement} von $M$ durch
\[
	M^\perp\coloneqq\{v\in H\mid v\perp M\}\coloneqq\{v\in H\mid (v, m)_H=0\;\fa\,  m\in M\}\, .
\]
\end{defi}


\begin{satz}\label{satz:2.6}
 Es sei $M$ ein abgeschlossener Untervektorraum von $H$. Dann ist
  \[
  	H=M\oplus M^\perp\, , 
  \]
  d.h. jedes $v\in M$ hat eine eindeutige Zerlegung $v=v_M+v_{M^\perp}$ mit $v_M\in M$ und $v_{M^\perp}\in M^\perp$.
\end{satz}

\begin{proof}
Der Beweis findet sich in \cite{Walker} Kapitel 7.1 Theorem 7.6.
\end{proof}


\begin{kor}\label{kor:2.6}
Es sei $\emptyset \not = M \subset H$   ein Untervektorraum. Dann ist $\bar M = H$ genau dann, wenn $M^\perp = \{0\}$ ist.
\end{kor}

\begin{proof}
Man kann zeigen, dass $\overline{\spn M} = (M^\perp)^\perp =: M^{\perp\perp}$ ist und dann unter Verwendung von Satz \ref{satz:2.6} die Behauptung folgern. Den kompletten Beweis können wir in \cite{Walker} Kapitel 7.1 Korollar 7.7 (iii) einsehen.
\end{proof}



\section{Variationsformulierung}
\label{kap:2.2}


Bevor wir uns mit Variationsproblemen auf konvexen Teilmengen eines Hilbertraumes beschäftigen, wollen wir die Variationsrechnung an einem einfachen Modellproblem ohne Nebenbedingung beschreiben.

      \begin{figure}[ht!]
        \centering
        \begin{pspicture}(-3,-2.7)(3,2.6)
          \psset{Beta=15}
          
          % Membran
          \pstThreeDCircle[fillstyle=shape,fillcolor=lightgray](0,0,0)(2,2,0)(-2,2,0)
          \pstThreeDEllipse[beginAngle=0,endAngle=180](0,0,0)(2,-2,0)(0,0,-2)
          \pstThreeDLine[arrows=|-|](2.5,-2.5,0)(2.5,-2.5,-2)
          \pstThreeDPut[origin=rt](2.9,-2.9,-1){\large$u(x)$}
          \pstThreeDPut(-1.8,1.2,0){\large$\Omega$}

          % Kraft
          \pstThreeDLine[arrows=->,arrowscale=3](0,0,2)(0,0,0)
          \pstThreeDPut[origin=lt](-0.2,0.1,1.3){\large$f$}
        \end{pspicture}
        \caption{Membran $\Omega$ mit Flächenlast $f$ und Auslenkung $u(x)$\label{abb:2.1}}
      \end{figure}




Wir betrachten als Modellproblem die Auslenkung $u: \Omega \ra \R$ einer in $\Omega \subset \R^d$ eingespannten Membran unter Krafteinwirkung $f$. Mathematisch beschrieben wird dies durch das  \textit{\idx{Dirichlet-Problem}}
\begin{subequations}\label{eq:2.1a}
\begin{align}\label{eq:2.1aa}
%\begin{aligned}
	-\Delta u &= f \text{ in } \Omega \, ,\\
	\label{eq:2.1ab}
	u & = g \text{ auf } \partial \Omega \, ,
%\end{aligned}
\end{align}
\end{subequations}
dabei ist $g: \partial \Omega \ra \R$ eine für die Randwerte von $u$ gegebene Funktion.


\begin{notation} 
In der Praxis übliche Dimensionen sind $d = 2,3$. Der Einfachheit halber sei im Folgenden $d = 2$ und $\Omega \subset \R^2$ ein durch ein Polygonzug berandetes Gebiet, den Rand $\partial \Omega$ bezeichnen wir auch mit $\Gamma$.
\end{notation}


\begin{bem*}
Sollte $\Omega$ ein allgemeiner berandetes Gebiet sein, so können wir dieses beliebig genau durch ein polygonales Gebiet approximieren; hierbei entsteht schon bei der Gebietszerlegung ein Fehler.

Diesen Fehler kann man durch Verwendung von \textit{isoparametrischen Elementen}\index{isoparametrisches Element} (vgl. \cite{BraeFEM} Kapitel III, \S2, Isoparametrische Elemente) verringern. Dies soll in dieser Arbeit aber nicht weiter vertieft werden.
\end{bem*}


Es sei $u_0: \Omega \ra \R$ eine für das \idx{Dirichlet-Problem} zulässige Funktion, d.h. die für   \eqref{eq:2.1a} hinreichend regulär ist und für die $u_0 = g$ auf $\Gamma$ gilt. Dann gilt für 
$\tilde u = u-u_0$
\begin{subequations}\label{eq:2.2a}
\begin{align}\label{eq:2.2aa}
	-\Delta \tilde u &= \tilde f \text{ in } \Omega \, ,\\
	\label{eq:2.2ab}
	\tilde u & = 0 \text{ auf } \Gamma 
\end{align}
\end{subequations}
mit $\tilde f = f-\Delta u_0$. Also reicht es aus, sich auf das \textit{homogene Dirichlet-Problem}\index{Dirichlet-Problem!homogenes} \eqref{eq:2.2a} zu beschränken. Im Folgenden betrachten wir somit \eqref{eq:2.1a} mit $g \equiv 0$.

Mit $H^1_0 (\Omega)$ bezeichnen wir, wie in Bemerkung \ref{bem:A.8} beschrieben, den Raum der in $\Omega$ einmal schwach differenzierbaren Funktionen, die am Rand $\Gamma$ verschwinden im Sinne der Spur. Wählen wir nun ein beliebiges $v \in H^1_0(\Omega)$, dann folgt durch Multiplikation von \eqref{eq:2.1aa} mit $v$ und Integration über $\Omega$ die Beziehung
\begin{align*}
	\int_\Omega -\Delta u \cdot v \, dx = \int_\Omega f v \, dx \, .
\end{align*}
Wir betrachten nun \eqref{eq:2.1aa} also nicht mehr punktweise (lokal), sondern im gewichteten Mittel über ganz $\Omega$ (global). Durch Anwenden der 1. Green'schen Formel (bzw. dem Satz von Gauß) ergibt sich
\begin{align}\notag
	& \int_\Omega \nabla u \cdot \nabla v \, dx -\underbrace{\int_\Gamma v \partial_\nu u \, ds}_{=0, \text{ da } v|_\Gamma = 0} = \int_\Omega f v \, dx \, \\
	\label{eq:2.3}	
	\Llra \, & \quad \qquad \int_\Omega \nabla u \cdot \nabla v \, dx =\int_\Omega f v \, dx \, .
\end{align}
Die Gleichung \eqref{eq:2.3} wird als \textit{\idx{Variationsgleichung}} bezeichnet. Wenn wir die Notationen aus Satz \ref{satz:A.5} (b) verwenden, so können wir \eqref{eq:2.3} kurz schreiben als
\[
	(\nabla u, \nabla v)_0 = (f,v)_0 \, ,
\]
daher definieren wir die Bilinearform $a: (H^1_0(\Omega))^2 \ra \R, a(u,v) := (\nabla u, \nabla v)_0$ und $(f,v):=(f,v)_0$.


\begin{bem*}
Wir werden in dieser Arbeit oftmals auch $a(\cdot,\cdot)$ für eine beliebige Bilinearform $a: H\times H\ra \R$ verwenden.
\end{bem*}


\begin{defi}\label{def:2.8}
Eine Funktion $u \in H^1_0(\Omega)$ heißt \textit{\idx{schwache Lösung}} vom homogenen Dirichlet-Problem\index{homogenes Dirichlet-Problem}\index{Dirichlet-Problem!homogenes}
\begin{align}\label{eq:DP}\tag{DP}
\begin{aligned}
	-\Delta  u &=  f \text{ in } \Omega \, ,\\
	 u & = 0 \text{ auf } \Gamma \, ,
\end{aligned}
\end{align}
wenn die Gleichung
\begin{align}\label{eq:2.4}
	a(u,v) = (f,v)\quad \forall \, v \in H^1_0(\Omega) 
\end{align}
gilt.
\end{defi}


Anschaulich ist eine solche Lösung deshalb schwach, da sie das Problem \eqref{eq:DP} nur im gewichteten Mittel löst. Eine schwache Lösung muss das \textit{starke Problem}\index{starkes Problem} nicht lösen, da sie beispielsweise die Regularitätsanforderungen an das Problem nicht erfüllen muss.

Wir wollen uns nun die Frage nach der Eindeutigkeit und Existenz einer Lösung für die Variationsgleichung \eqref{eq:2.4} stellen. Diese Frage wollen wir zunächst allgemein für einen beliebigen reellen Hilbertraum $H$ beantworten. Wie wir nachher im Beweis des zentralen Satzes von Lax-Milgram sehen werden, ist hierfür explizit ein Hilbertraum notwendig.

Zuvor benötigen wir allerdings noch ein paar Definitionen und Eigenschaften für Bilinearformen.


\begin{defi}\label{def:2.9}
Sei $H$ ein Hilbertraum. Die Bilinearform  $a : H\times H \ra \R$ heißt \textit{stetig}\index{Bilinearform!stetig}, falls mit einem $c>0$
\[
	\abs{a(u,v)} \le c \, \norm{u}_H   \norm{v}_H \quad \forall \, u,v \in H
\]
gilt. Sie heißt $H$-\textit{elliptisch} (oder kurz \textit{elliptisch} oder \textit{koerziv})\index{Bilinearform!koerziv}\index{Bilinearform!elliptisch}, falls es ein $\alpha > 0$ gibt, so dass
\[
	a(v,v) \ge \alpha \, \norm{v}_H^2 \quad \forall \, v \in H 
\]
gilt.
\end{defi}


Da man die Variationsgleichung \eqref{eq:2.4} auch aus der Minimierung eines quadratischen Energiefunktionals $J: (H^1_0(\Omega))^2\ra \R, J(v) \coloneqq \frac 12 a(v,v)-(f,v)$ herleiten kann, wollen wir für ein solches Funktional zuvor einige Eigenschaften sammeln.


\begin{lemma}\label{lem:2.10}
Es sei $H$ ein Hilbertraum. Das Funktional
\[
	J: H \ra \R \, , \quad J(v) := \frac 1 2 a(v,v) - F(v) \, ,
\]
wobei $a: H\times H \ra \R$ eine stetige bilineare koerzive und $F: H\ra \R$ eine lineare Abbildung ist, ist konvex.
\end{lemma}

\begin{proof}
Es seien $u,v \in H$, dann gilt $u + t(v-u) = (1-t)u + tv \in H$ (dies gilt auch, wenn wir den Satz auf eine konvexe Teilmenge $M \subset H$ beschränken). Damit folgt mit $t \in [0,1]$
\begin{align*}
	J((1-t)u+tv)  = & \frac 1 2 a((1-t)u+tv,(1-t)u+tv) - F((1-t)u+tv) \\
	= &(1-t) \, J(u) + t \, J(v) +  \frac 1 2 a((1-t)u+tv,(1-t)u+tv) \\
	 & - \frac 1 2(1-t) \, a(u,u)-\frac 1 2 t \, a(v,v) \\
	= & (1-t) \, J(u) + t \, J(v)  + \frac 1 2 a(u,u) + t \, a(u,v-u)  \\
	&+ \frac {t^2} 2 a(v-u,v-u) - \frac 12 (1-t)\, a(u,u) -\frac 1 2 t \, a(v,v) \\
	= & (1-t) \, J(u) + t \, J(v) + \frac {t^2} 2 a(v-u,v-u)  \\
	 &\underbrace{+ t \, a(u,v)  - \frac 12 t\, a(u,u) -\frac 1 2 t \, a(v,v) }_{=  -\frac 1 2 t\, a(v-u,v-u)}\\
	= &  (1-t) \, J(u) + t \, J(v) - \frac {1} 2 \underbrace{t \, (1-t)}_{\ge 0} \,\underbrace{ a(v-u,v-u) }_{\ge \alpha  \norm{v-u}_H^2 \ge 0} \\
	\le &   (1-t) \, J(u) + t \, J(v) \, % \qedhere
\end{align*}
Daraus folgt die Behauptung.
\end{proof}


\begin{lemma} \label{lem:2.11}
Sei $H$ ein Hilbertraum. Das Funktional $J: H \ra \R, J(v) =\frac 1 2 a(v,v)-F(v)$ aus Lemma \ref{lem:2.10} ist Gâteaux-differenzierbar $($s. Definition \ref{def:Gateaux-Ableitung}$)$.
\end{lemma}

\begin{proof}
Wir rechnen einfach nach, dass der Grenzwert des Differenzenquotienten existiert und verwenden dabei die Bilinearität von $a$ und Linearität von $F$. Seien $u,v \in H$, dann gilt
\begin{align*}
	\mscr D_v J(u) & = \lim_{t\ra 0} \frac{J(u+tv)-J(u)}t \\ 
	&= \lim_{t\ra 0} \frac{J(u) + t \, (a(u,v)-F(v)) + \frac {t^2}2 a(v,v)-J(u)}t \\
	& =  \lim_{t\ra 0}  (a(u,v)-F(v)) + \frac {t}2 a(v,v) \\
	& = a(u,v)-F(v) < \infty\, ,
\end{align*}
da $a$ und $F$ jeweils stetig sind und daher durch $\norm u_H,\norm v_H$ beschränkt sind. Damit folgt die Behauptung.
\end{proof}


Nun können wir die Existenz und Eindeutigkeit einer Lösung durch das folgende Theorem zeigen.

\begin{theorem}[\idx{Lax-Milgram}]\label{theorem:2.12}
Es sei $H$ ein Hilbertraum und  $a : H \times H \ra \R$ eine symmetrische, in $H$ stetige, koerzive Bilinearform. Weiter sei $F:H\ra \R$ ein stetiges lineares Funktional, d.h.
\[
	\abs{F(v)} \le c \, \norm{v}_H \quad \forall \, v \in H
\]
mit einer Konstante $c >0$. Dann gibt es eine eindeutige Lösung $u \in H$, für die
\[
	a(u,v) = F(v) \quad \forall \, v \in H \, .
\]
gilt. Diese minimiert den Ausdruck
\[
	J(v) = \frac 1 2 a(v,v) - F(v)
\]
unter allen $v \in H$.
\end{theorem}

\begin{proof}
(i) Zunächst zeigen wir die Äquivalenz der beiden oberen Probleme.

"`$\Ra$"' Es sei $u\in H$, so dass $a(u,v) = F(v) \, \forall \, v \in H$. Für $t>0$ und $v\in H$ gilt dann
\begin{align*}
	J(u+tv) & = \frac 1 2 a(u+tv,u+tv) -F(u+tv) \\
	& = \frac 1 2 a(u,u) + t \, a(u,v) + \frac {t^2} 2 a(v,v)-F(u)-t \, F(v) \\
	& = \frac 1 2 a(u,u)-F(u) + t\, (\underbrace{a(u,v)-F(v)}_{=0}) + \frac{t^2}2 \underbrace{a(v,v)}_{\parbox{1.2cm}{\scriptsize$\ge 0$, da $a$ koerziv}} \\
	& > \frac 1 2 a(u,u) - F(u) = J(u) \, ,
\end{align*}
also ist $u = \arg\min\limits_{v\in H} J(v)$.

"`$\La$"' Es sei $u \in H$ das Minimum von dem Problem
\[
	\min_{v\in H} J(v) = \frac 1 2 a(v,v) -F(v) \, .
\]
Da $J:H\ra \R$ nach Lemma \ref{lem:2.10} ein konvexes Funktional ist und $J$ nach Lemma \ref{lem:2.11} Gâteaux-differenzierbar, gilt mit Satz \ref{satz:A.10} für alle $v \in H$
\begin{align*}
	0& = \mscr D_vJ(u) = \frac d{dt} J(u+tv)\Big|_{t=0} \\
	& = \frac d{dt}(J(u) + t \, (a(u,v)-F(v))+\frac{t^2}2 a(v,v))\Big|_{t=0} \\
	& = a(u,v)-F(v) + t \, a(v,v) \Big|_{t=0} = a(u,v)-F(v)
\end{align*}
(ii) Eindeutigkeit: Es seien $u,\tilde u \in H$ Lösungen der Variationsungleichung, d.h.
\begin{align*}
	a(u,v) = F(v) \, \wedge \, a(\tilde u,v) = F(v) \quad \forall \, v \in H \, .
\end{align*}
Damit folgt durch Subtraktion der beiden Gleichungen für alle $v \in H$
\begin{align}\label{eq:2.5}
	a(u,v) = a(\tilde u,v) \Llra a(u-\tilde u,v) = 0 \, .
\end{align}
Da $H$ ein Vektorraum ist, gilt auch $u-\tilde u \in H$. Ersetzen wir also in \eqref{eq:2.5} $v = u-\tilde u$, dann ergibt sich
\begin{align*}
	&0 = a(u-\tilde u,u-\tilde u) \stackrel{\scriptsize a\text{ koerziv}}\ge \underbrace{\alpha}_{>0} \norm{u-\tilde u}_H^2 \ge 0 
	\lra\norm{u-\tilde u}_H^2 = 0 \, ,
\end{align*}
also folgt $u = \tilde u$.

(iii) Existenz: Die Existenz einer Lösung weisen wir über das Funktional nach.
\begin{align*}
	J(v) & = \frac 1 2 a(v,v)-F(v) \stackrel[\scriptsize F \text{ linear}]{\scriptsize a \text{ koerziv}}\ge \frac 1 2 \alpha \norm v_H^2 - c \norm v_H \\
	& = \frac 1 2 \alpha \(\norm v_H^2 - \frac{2c}\alpha \norm v_H\) = \frac 1 2 \alpha\(\norm v_H - \frac c\alpha\)^2 - \frac {c^2}{2\alpha} \\
	& \ge - \frac{c^2}{2\alpha}
\end{align*}
Folglich ist $J$ nach unten beschränkt. Sei $\eta := \inf \{J(v)\mid v \in H\}$ und $(v_n)_{n\in\N}$ eine Folge mit $J(v_n) \ra\eta$ für $n\ra \infty$. Dann folgt mit der Koerzivität von $a$
\begin{align*}
	\alpha \norm{v_n-v_m}^2_H  \le & a(v_n-v_m,v_n-v_m) \\
	 = &a(v_n,v_n)+a(v_m,v_m)-a(v_n,v_m)-a(v_m,v_n) \\
	=& 2a(v_n,v_n)+2a(v_m,v_m) \underbrace{-a(v_n,v_n+v_m)-a(v_m,v_n+v_m)}_{=-a(v_n+v_m,v_n+v_m)} \\
	=& 2a(v_n,v_n)-4F(v_n)+2a(v_m,v_m)-4F(v_m) \\
	& -a(v_n+v_m,v_n+v_m)+4F(v_n+v_m) \\
	= & 4 J(v_n) + 4J(v_m) - 4 a\(\frac{v_n+v_m}2,\frac{v_n+v_m}2\)+8F\(\frac{v_n+v_m}2\) \\
	= & 4 J(v_n) + 4J(v_m) - 8 J\(\frac{v_n+v_m}2\) \\
	\le &4 J(v_n) + 4J(v_m) - 8\eta  \xrightarrow[n,m\ra\infty]{} 4\eta+4\eta-8\eta = 0 \, ,
\end{align*}
d.h. $(v_n)_{n\in \N}$ ist eine Cauchy-Folge. Da $H$ ein Hilbertraum ist, gilt somit: $\exists \, u \in H : v_n \xrightarrow[n\ra \infty]{} u$ mit $J(u) = \eta$.
\end{proof}


Um die allgemeine Aussage aus dem Theorem von \idx{Lax-Milgram} auf unser Modellproblem \eqref{eq:2.4} übertragen zu können, benötigen wir die \textit{\idx{Poincaré-Friedrich-Ungleichung}}, die auch später noch eine zentrale Rolle für den hierarchischen Fehlerschätzer spielen wird.

\begin{satz}[\idx{Poincaré-Friedrich-Ungleichung}]\label{satz:2.13}
Es sei $\Omega$ in einem $d$-dimensionalen Würfel der Kantenlänge $s>0$ enthalten. Dann gilt
\[
	\norm v_0 \le s \norm{\nabla v}_0 \quad \forall \, v \in H^1_0(\Omega) \, ,
\]
wobei $\norm\cdot_0$ die durch das Skalarprodukt $(\cdot,\cdot)_0$ induzierte Norm ist.
\end{satz}

\begin{proof}
Der Beweis ist in \cite{BraeFEM} Kapitel II, \S1 Sobolev-Räume, Satz 1.5 oder \cite{StarkePDE} Satz 1.5 zu finden.
\end{proof}


\begin{bem}\label{bem:2.14}
Für die Gültigkeit der Poincaré-Friedrich-Ungleichung, muss $v$ nicht auf ganz $\Gamma$ gleich Null sein, sondern es reicht aus, dass
\[
	v \in H^1_{\Gamma_u} (\Omega)\coloneqq \{v \in H^1(\Omega) \mid v = 0 \text{ auf } \Gamma_u\}
\]
ist mit $\Gamma_u \subset \Gamma$, wobei mit einem Maß $\mu$ gilt: $\mu (\Gamma_u) \not = 0$, d.h. $\Gamma_u$ ist keine Nullmenge (vgl. \cite{BraeFEM} Kapitel II, \S1, Bemerkung 1.6).
\end{bem}


Jetzt können wir mittels Theorem \ref{theorem:2.12} überprüfen, ob  Problem \eqref{eq:2.4} mit $a:(H^1_0(\Omega))^2\ra \R, a(u,v) = (\nabla u,\nabla v)_0$ und $F:H^1_0(\Omega) \ra \R, F(v) := (f,v)$ eine eindeutige Lösung hat. Es seien $u,v \in H^1_0(\Omega)$, dann gilt
\begin{align*}
	 a(v,v) = & \int_\Omega \nabla v \nabla v \, dx = \norm{\nabla v}_0^2  \\
	\ge& \frac{s^2+1}{(1+s)^2}\norm{\nabla v}_0^2  \stackrel{\scriptsize \text{Satz \ref{satz:2.13}}}\ge \frac 1{(1+s)^2} (\norm v_0^2 + \norm{\nabla v}_0^2) \\
	= & \frac 1{(1+s)^2} \norm v_1^2 \, .
\end{align*}
Damit ist $a$ mit $\alpha :=  \frac 1{(1+s)^2}$ koerziv. Weiter rechnen wir unter Verwendung der Cauchy-Schwarz-Ungleichung\index{Cauchy-Schwarz'sche Ungleichung} nach:
\begin{align*}
	\abs{a(u,v)} &= \Abs{\int_\Omega \nabla u \nabla v \, dx} \le \sum_{i = 1}^d \int_\Omega\abs{\partial_i u}\abs{\partial_i v} \, dx \\
	&\!\!\:\! \stackrel{\scriptsize\text{C.S.}}\le  \sum_{i = 1}^d \(\int_\Omega \abs{\partial_i u}^2 \, dx\)^{\frac 12} \(\int_\Omega \abs{\partial_i v}^2 \, dx\)^{\frac 12} \\
	&\le \(\sum_{i = 1}^d \int_\Omega \abs{\partial_i u}^2 \, dx\)^{\frac 12} \(\sum_{i=1}^d\int_\Omega \abs{\partial_i v}^2 \, dx\)^{\frac 12} \\
	&\le \( \int_\Omega \abs{\nabla u}^2 \, dx + \int_\Omega u^2 \, dx\)^{\frac 12} \(\int_\Omega \abs{\nabla v}^2 \, dx+\int_\Omega v^2\, dx\)^{\frac 12} \\
	&= \norm u_1 \, \norm v_1 \, ,
\end{align*}
d.h.  $a$ ist stetig mit $c := 1$. Die Symmetrie von $a$ ist trivial, also bleibt nur noch die Stetigkeit von $F$ zu zeigen. Es sei $v \in H^1_0(\Omega)$, dann gilt
\begin{align*}
	\abs{F(v)} &= \abs{(f,v)} =  \Abs{\int_\Omega fv \, dx} \stackrel{\scriptsize\text{C.S.}}\le\( \int_\Omega \abs{f}^2 \,dx\)^{\frac 12} \( \int_\Omega \abs v^2 \, dx\)^{\frac 12} \\
	&\le  c \, \bigg( \int_\Omega \underbrace{\abs{\nabla v}^2}_{\ge 0} +  \abs v^2 \, dx\bigg)^{\frac 12} = c \, \norm v_1
\end{align*}
mit $0<c := \(\int_\Omega \abs f^2 \, dx\)^{\frac 12} < \infty$, wenn $f \in L_2(\Omega)$ ist. Damit ist $F$ ein stetiges lineares Funktional und somit existiert nach Theorem \ref{theorem:2.12} eine eindeutige Lösung $u \in H^1_0(\Omega)$ für die schwache Formulierung des homogenen Dirichlet-Problems. 

Weiter minimiert die Lösung $u \in H^1_0(\Omega)$ auch das Funktional
\begin{align*}
	J(v) = \frac 12 \int_\Omega \nabla v\nabla v \, dx - \int_\Omega fv \, dx \, ,
\end{align*}
welches die gespeicherte Energie der durch die Kraft $f$ belasteten in $\Omega$ eingespannten Membran beschreibt.


\begin{bem*}
Die Stetigkeit vom Funktional $F$ zeigt, dass  die Kraft $f$ aus dem Dirichlet-Problem wenigstens quadratisch integrierbar, also in $L^2(\Omega)$, sein muss, damit es eine schwache Lösung geben kann.
\end{bem*}


\begin{notation}
\begin{enumerate}[(a)]
\item Mit $H'$ bezeichnen wir den Dualraum zum Hilbertraum $H$.
\item Den Dualraum zu $H^1(\Omega)$ bezeichnen wir mit $H^{-1}(\Omega)$.
\end{enumerate}
\end{notation}


Als Folgerung aus dem Theorem von \idx{Lax-Milgram} betrachten wir den nächsten Satz.


\begin{satz}[\idx{Riesz'scher Darstellungssatz}]\label{satz:2.15}
Es sei $H$ ein Hilbertraum mit einem Skalarprodukt $(\cdot,\cdot)_H$. Es sei $F \in H'$, dann existiert genau ein $u \in H$, so dass
\[
	(u,v)_H = F(v) \quad \forall \, v \in H \, .
\]
\end{satz}

\begin{proof}
Dies ist eine direkte Folgerung aus dem Theorem \ref{theorem:2.12}. Die Abbildung $(\cdot,\cdot)_H:H\times H \ra \R$ ist als Skalarprodukt bilinear, symmetrisch und positiv definit, damit auch bzgl. der auf $H$ durch das Skalarprodukt induzierten Norm $\norm v_H := \sqrt{(v,v)_H}$, koerziv. $F$ ist als Element des Dualraumes $H'$ eine lineare stetige Abbildung $F:H\ra\R$ und damit folgt mit $a(\cdot,\cdot) :=(\cdot,\cdot)_H$ aus dem Theorem von Lax-Milgram die Behauptung.
\end{proof}


\begin{lemma}\label{lem:2.16}
Es sei $H$ ein Hilbertraum mit Skalarprodukt $(\cdot,\cdot)_H$ und $a:H\times H\ra \R$ eine stetige koerzive Bilinearform. Dann existiert genau ein linearer Operator $A : H \ra H$, so dass gilt:
\[
	a(u,v)  = (Au,v)_H \quad \forall \, u , v \in H \, .
\]
\end{lemma}

\begin{proof}
Es sei $u\in H$ fest, dann ist $L: H\ra \R, L(v) := a(u,v)$ eine lineare Abbildung, die stetig ist, da
\[
	\abs{L(v)} = \abs{a(u,v)} \stackrel{\scriptsize\text{stetig}}\le c \,  \norm{u}_H \norm v_H  = \tilde c\, \norm v_H
\]
mit $0<\tilde c := c \, \norm u_H$ gilt. Damit folgt nach dem Darstellungssatz von Riesz, dass es ein eindeutiges $l \in H$ gibt, so dass
\[
	a(u,v) = L(v) = (l,v)_H \quad \forall \, v \in H
\]
gilt. Da $u \in H$ jedoch beliebig ist, bleibt zu zeigen, dass es ein eindeutiges $A:H\ra H$ gibt, so dass $Au = l$ ist.

Wir zeigen zunächst mithilfe der Bilinearform $a$, dass $A$ linear ist. Es gilt für $\lambda,\mu \in \R$ und $u,v \in H$
\begin{align*}
	(A(\lambda u + \mu v),w)_H &= a(\lambda u + \mu v, w) = \lambda a(u,w) + \mu a(v,w) \\
	& = \lambda (Au,w)_H + \mu(Av,w)_H \\
	& = (\lambda \, Au+\mu \, Av,w)_H
\end{align*}
für alle $w \in H$. Weiter gilt
\[
	\norm{Au}_H^2 = (Au,Au)_H = a(u,Au) \stackrel{\scriptsize \text{stetig}}\le c \, \norm u_H \norm{Au}_H \, ,
\]
d.h. $\norm{Au}_H \le c \, \norm u_H$ und damit ist nach \cite{Werner} Satz II.1.2 der Operator $A$ stetig.

Betrachten wir den Kern von $A$, so ergibt sich
\begin{align}\label{eq:2.6}
	\ker A := \{ v \in H \mid Av = 0\} = \{0\} \, ,
\end{align}
denn
\[
	\alpha \norm v_H^2 \stackrel{\scriptsize \text{koerziv}}\le a(v,v)  = (Av,v)_H \stackrel{\scriptsize \text{CS}}\le \norm{Av}_H \norm v_H 
\]
und damit gilt $\norm{Av}_H \ge \alpha \norm v_H$, d.h. $Av = 0 \Lra v = 0$. Dies impliziert, dass $A$ injektiv ist, denn mit $v_1,v_2\in H, Av_1 = Av_2$ folgt
\[
	0 = Av_1 - Av_2 = A(v_1-v_2) \ \stackrel{\scriptsize\eqref{eq:2.6}}\lra \ v_1 = v_2 \, .
\]
Weiter betrachten wir das Bild von $A$, d.h.
\[
	\im A := \{ v \in H \mid \exists \, u \in H : Au = v \} \subset H\, .
\]
Sei $(v_n)_{n \in \N}$ eine Folge mit $v_k \in \im A $ für alle $k \in \N$. Dann folgt, dass für jedes $v_k$ ein $u_k \in H$ existiert mit $A u_k = v_k$. Es gelte, dass $Au_k = v_k \ra v \in H$ geht, dann folgt
\begin{align*}
	\alpha \, \norm{u_n -u_m}_H & \le \norm{A(u_n-u_m)}_H = \norm{Au_n-Au_m}_H \\
	& = \norm{v_n-v_m}_H \xrightarrow[n,m\ra \infty]{} 0 \, ,
\end{align*}
d.h. $(u_n)_{n\in \N} \subset H$ ist eine Cauchy-Folge und konvergiert daher in $H$. Also existiert ein $u \in H$ mit $u_n \ra u$. Mit der Stetigkeit von $A$ folgt dann
\[
	v_n = A u_n \xrightarrow[n\ra \infty]{} Au = v\, ,
\]
d.h. $v \in \im A$ und damit ist $\im A$ abgeschlossen. Wir betrachten nun ein $v \in H$ mit $v \perp \im A \subset H$, dann gilt
\[
	(Au,v)_H = 0 \quad \forall \, u \in H \, .
\]
Damit folgt mit $u = v \in H$ oben eingesetzt
\[
	 0 = (Av,v)_H = a(v,v) \ge \alpha \, \norm v_H^2 \, \lra \, v = 0 \, .
\]
Also besteht der zu $\im A$ orthogonale Raum nur aus dem Nullelement und mit Korollar \ref{kor:2.6} gilt dann $\im A = \overline{\im A} = H$. Damit ist $A$ bijektiv.

Es seien nun $0 \not = l\in H$ sowie $A_1,A_2 \in \mcal L(H,H)$ zwei lineare Operatoren mit $A_1u = l$ und $A_2 u = l$, die nach der obigen Weise konstruiert sind. Dann gilt
\[
	0 = A_1 u - A_2 u = (A_1 - A_2)u \, \lra \, A_1 = A_2 \, , 
\]
da $u \not = 0$ und die Summe zweier bijektiver linearer Operatoren wieder bijektiv ist, also ist ein so konstruierter Operator eindeutig.
\end{proof}






\section{Finite Elemente Methode}
\label{kap:2.3}


Für unser Modellproblem kann man zeigen, dass es für bestimmte Gebiete $\Omega$ eine exakte, d.h. analytische, Lösung gibt (vgl. \cite{Walker} Kapitel 5). Diese muss im Allgemeinen nicht für jedes Problem bekannt oder gar berechenbar sein. Daher wollen wir nicht mehr die exakte Lösung von unserer Variationsgleichung \eqref{eq:2.4} berechnen, sondern eine Approximation davon, die sogenannte \textit{\idx{Galerkin-Approximation}}.

Unter dem \textit{\idx{Galerkin-Verfahren}} verstehen wir, dass wir die Variationsgleichung
\begin{align}\label{eq:2.8}
	a(u,v) = F(v) \quad \forall \, v \in H
\end{align}
nur noch auf einem endlich dimensionalen Unterraum $V_h \subset H$ lösen wollen, d.h. finde $u_h \in V_h$, so dass
\begin{align}\label{eq:2.9}
	a(u_h,v_h) = F(v_h) \quad \forall \, v_h \in V_h \, .
\end{align}


\begin{satz}\label{satz:2.17}
Das "`Galerkin-Problem"' \eqref{eq:2.9} hat eine eindeutige Lösung.
\end{satz}

\begin{proof}
Da $V_h$ als Unterraum von $H$ auch ein Hilbertraum ist und die Eigenschaften von $a, F$ weiterhin erfüllt sind, gilt auch hier der Satz von Lax-Milgram, was die Eindeutigkeit und Existenz einer Lösung sichert.
\end{proof}


Da $V_h$ ein endlich dimensionaler Unterraum von $H$ ist, wird jener von einer endlichen Basis $\mcal B_h \coloneqq \{\phi_1,\ldots,\phi_N\}$ aufgespannt, d.h. für $u_h \in V_h$ gilt:
\begin{align}\label{eq:2.10}
	\exists! \, \bs\mu \in \R^N :  u_h(x) = \sum_{i = 1}^N \mu_i \,  \phi_i(x) \, .
\end{align}
Da $F(\cdot),a(u,\cdot)$ linear sind und alle $v_h \in V_h$ analog zu oben darstellbar sind, ist \eqref{eq:2.9} äquivalent zum Problem
\[
	a(u_h,\phi_i) = F(\phi_i) \quad \forall \, i = 1, \ldots,N \, ,
\]
mit $u_h$, wie in \eqref{eq:2.10} dargestellt, eingesetzt ergibt sich
\[
	a(u_h,\phi_i) = a \Big( \sum_{j = 1}^N \mu_j \,  \phi_j,\phi_i \Big) = \sum_{j = 1}^N \mu_j \, a(\phi_j,\phi_i) \, ,
\]
also
\[
	 \sum_{j = 1}^N \mu_j \, a(\phi_j,\phi_i) = F(\phi_i)\quad \forall \, i = 1, \ldots,N \, .
\]
Damit erhalten wir ein lineares Gleichungssystem
\[
	A \bs \mu = \bs f 
\]
mit $A = [a(\phi_j,\phi_i)]_{i,j=1}^N, \bs \mu = [\mu_i]_{i=1}^N$ und $\bs f = [F(\phi_i)]_{i=1}^N$. Dieses Gleichungssystem gilt es zu lösen, um die gesuchten Koordinaten $\mu_i, i = 1,\ldots,N,$ bzgl. der Basis $\mcal B_h$ für die approximierte Lösung $u_h$ zu finden.

In den Ingenieurswissenschaften, insbesondere bei kontinuumsmechanischen Problemen, wird $A$ als \textit{\idx{Steifigkeitsmatrix}} bezeichnet.

\begin{bem}\label{bem:2.18}
Ist die Bilinearform $a$ symmetrisch, so ist es auch die Matrix $A$, denn
\[
	a_{ij} = a(\phi_i,\phi_j) = a(\phi_j,\phi_i)= a_{ji} \, .
\]
Außerdem folgt aus der Koerzivität von $a$, dass  mit $\bs 0 \not = \bs \nu \in \R^N$ gilt
\begin{align*}
	\bs \nu^T A \bs \nu & = \sum_{i,j=1}^N \nu_i a_{ij} \nu_j  =  \sum_{i=1}^N \nu_i\sum_{j=1}^N \, a(\phi_i,\phi_j) \, \nu_j   \\
	& = \sum_{i=1}^N \nu_i \, a\Big(\phi_i,\sum_{j=1}^N\nu_j\phi_j\Big) =  a\Big(\sum_{i=1}^N \nu_i\phi_i,\sum_{j=1}^Nv_j\phi_j\Big) \\
	& = a(v_h,v_h) \ge \alpha \norm{v_h}^2_H > 0 \, ,
\end{align*}
da $\sum \nu_i \phi_i = v_h \not = 0$ wegen $\bs \nu \not = \bs 0$. Damit ist $A$ also positiv definit und es folgt nochmals, dass $A \bs \mu = \bs f$ eine eindeutige Lösung hat.
\end{bem}


Um eine Basis $\mcal B_h$ bzgl. $V_h$ beschreiben zu können, muss das Gebiet $\Omega\subset \R^2$ in endliche Elemente zerlegt werden. Die Basis $\mcal B_h$ und damit der Raum $V_h$ wird dann bzgl. einer Zerlegung $\mcal T_h$ beschrieben. Eine gebräuchliche Zerlegung $\mcal T_h$ kann durch Dreiecke oder auch Vierecke geschehen. Wir wollen in dieser Arbeit nur Zerlegungen durch Dreiecke betrachten, hierfür führen wir den folgenden Begriff ein (vgl. \cite{BraeFEM} Seite 58 oder \cite{StarkePDE} Seite 19).


\begin{defi}[\idx{Triangulierung}]\label{def:2.19}
Es sei $\Omega \subset \R^2$ ein durch einen Polygonzug berandetes Gebiet. Dann heißt eine Zerlegung aus Dreiecken
\[
	\mcal T = \{T_1,T_2,\ldots,T_M\}
\]
\textit{\idx{Triangulierung}}, wenn gilt:
\begin{enumerate}[(a)]
\item Für alle Dreiecke $T \in \mcal T$ gilt: $T$ ist abgeschlossen.
\item	Ganz $\Omega$ wird durch alle Dreiecke aus $\mcal T$ überdeckt, d.h. $\bar\Omega = \bigcup_{T\in \mcal T} T$.
\item Der Schnitt zweier Dreiecke $T_i\cap T_j$ mit $i \not = j$ überlappt sich nicht, d.h. $\operatorname{int}(T_i)\cap \operatorname{int}(T_j) = \emptyset$.
\end{enumerate}
Wir nennen eine Triangulierung \textit{konform}\index{Triangulierung!konform} oder \textit{zulässig}\index{Triangulierung!zulässig}, wenn zusätzlich gilt:
\begin{enumerate}[(d)]
\item Für jede Kante $k$ eines Dreiecks $T \in \mcal T$ gilt entweder $k \subset \partial \Omega$ oder $k = \tilde k$ für eine weitere Kante $\tilde k$ eines weiteren Dreiecks $\widetilde T \in \mcal T$. 
\end{enumerate}
Der Radius des Umkreis eines Dreieckes $T$ wird mit $h$ bezeichnet und beschreibt die Größe eines Dreiecks. Wenn jedes Dreieck $T \in \mcal T$ höchstens einen Radius von $h$ hat, so schreiben wir $\mcal T_h$ statt $\mcal T$.
\end{defi}


\begin{bem}\label{bem:2.20}
\begin{enumerate}[(a)]
\item Ein Dreieck $T\in \mcal T_h$ bezeichnen wir auch als (\textit{finites}) \textit{Element}.
\item Sollte $\Omega \subset \R^3$ sein, so können wir  analog zu Definition \ref{def:2.19} eine Zerlegung mit Tetraedern definieren.
\end{enumerate}
\end{bem}


Für die Netzverfeinerung führen wir zwei unterschiedliche Familien von Zerlegungen ein (vgl. \cite{BraeFEM} Seite 58).


\begin{defi}[\idx{(quasi-) uniforme Zerlegung}]\label{def:2.21}
Eine Familie von Zerlegungen $\{\mcal T_h\}$ heißt \textit{quasi-uniform}\index{Triangulierung!quasi-uniform}, wenn es eine Zahl $\kappa > 0$ gibt, so dass jedes $T \in \mcal T_h$ einen Kreis vom Radius
\[
	\rho_T \ge \frac{h_T}\kappa
\]
enthält, wobei $h_T$ der Radius des Dreiecks $T$ ist.

Eine Familie von Zerlegungen $\{\mcal T_h\}$ heißt \textit{uniform}\index{Triangulierung!uniform}, wenn es eine Zahl $\kappa > 0$ gibt, so dass jedes $T \in \mcal T_h$ einen Kreis vom Radius
\[
	\rho_T \ge \frac{h}\kappa
\]
enthält, wobei $h := \max_{T \in \mcal T_h} h_T$ ist.
\end{defi}



\begin{figure}[h]
\begin{center}
\begin{pspicture}(-3,0)(3,2)
	% zulässige Triangulierung:
	\psline(-3,0)(-3,2)
	\psline(-3,0)(-1,0)
	\psline(-1,0)(-3,2)
	\psline(-3,2)(-1,2)
	\psline(-1,2)(-1,0)
	\psline(-3,0)(-1,2)
	\psline(-3,1)(-2,1)
	\psline(-2,2)(-2,1)
	\psline(-2,2)(-3,1)
	
%	\psline(-2,2)(-1.5,1.5)
%	\psline(-1.5,1.5)(-1,0)

%	\psline(-2.5,1.5)(-3,1)
%	\psline(-2.5,1.5)(-1,2)
	
	% nichtzulässige Triangulierung
	\psline(1,2)(2,1)
	\psline(2,1)(1,0)
	\psline(1,0)(1,2)
	\psline(1,0)(3,0)
	\psline(3,0)(2,1)
	\psline(1,2)(3,2)
	\psline(3,2)(3,0)
	\psdot[dotstyle=o, dotsize=4pt](2,1)
\end{pspicture}
\end{center}
\caption{Zulässige und unzulässige Triangulierung (mit hängendem Knoten)\label{abb:2.2}}
\end{figure}


In der Abbildung \ref{abb:2.2} ist eine quasi-uniforme zulässige Zerlegung und eine unzulässige Zerlegung zu sehen; zweitere ist daher unzulässig, da es Kanten gibt, die weder am Rand liegen, noch mit einer Kante eines anderen Dreiecks übereinstimmt. Die Knoten, die zu diesem Phänomen führen, nennen wir \textit{hängende Knoten}\index{hängender Knoten}. Dies sind Knoten, die nicht Eckpunkt jedes angrenzenden Dreiecks sind.


\begin{bem}\label{bem:2.22}
Wie man leicht sehen kann, ist jede uniforme Zerlegung auch quasi-uniform. Umgekehrt gilt dies nicht (s. Abbildung oben).

Allerdings lassen uniforme Zerlegungen keine lokalen Verfeinerungen zu. Da dies für adaptive Verfeinerungsstrategien allerdings ausschlaggebend ist, gehen wir im Folgenden immer von einer quasi-uniformen Zerlegung $\mcal T_h$ aus.
\end{bem}


Nun wollen wir uns Gedanken über unseren Ansatzraum $V_h$ machen. Hierfür gibt es, abhängig von der Konstruktion des verwendeten Elements, viele Möglichkeiten -- vgl. hierzu auch \cite{BraeFEM} Kapitel II, \S5, Tabelle 2. Wir wollen uns weitestgehend aber nur auf ein Element konzentrieren. Zuvor betrachten wir hierfür ein wichtiges Resultat, wobei noch bemerkt sei, dass eine Funktion $u$ auf $\Omega$ bei gegebener Zerlegung $\mcal T_h$ eine Eigenschaft stückweise hat, wenn sie auf jedem Element diese Eigenschaft besitzt.


\begin{satz}\label{satz:2.23}
Sei $k \ge 1$ und $\Omega\subset \R^2$ ein polygonales Gebiet. Eine stückweise beliebig oft differenzierbare Funktion $v : \bar \Omega \ra \R$ liegt in $H^k(\Omega)$ genau dann, wenn $v \in C^{k-1}(\bar\Omega)$ ist.
\end{satz}

\begin{proof}
Der Beweis ist in \cite{BraeFEM} Kapitel II, \S5, Satz 5.2 zu finden.
\end{proof}


Der Satz \ref{satz:2.23} rechtfertigt, dass wir für das Modellproblem \eqref{eq:2.4} auf einer Triangulierung $\mcal T_h$ einen Ansatzraum $V_h$ mit stetigen Funktionen $v \in C^0(\Omega)$ verwendet, da dann auch $v \in H^1(\Omega)$ gilt. Daher wählen wir
\[
	V_h \coloneqq \{v \in C^0(\Omega) \mid v|_T \in \mcal P_m \text{ für } T\in \mcal T_h, v|_{\partial \Omega} = 0\} \, ,
\]
wobei $\mcal P_m$ der Raum der Polynome vom Grad $m$ ist. Es stellt sich nun die Frage, wie wir geschickt eine Basis wählen können, um $V_h$ aufzuspannen. Die einfachste Möglichkeit stellen \textit{\idx{nodale Basisfunktion}en} dar.

\begin{defi}[\idx{nodale Basisfunktion}]\label{def:2.24}
Zu einem Finiten Element Raum $V_h$ und einer gegebenen Zerlegung $\mcal T_h$ sei eine Menge von Punkten $P$ bekannt mit $\abs P = N$. Die Menge $\mcal B_h = \{\phi_1,\ldots,\phi_N\}$ mit $\phi_i \in \mcal P_m, i = 1,\ldots, N$, heißt \textit{\idx{nodale Basis}} (oder \textit{\idx{Lagrange-Basis}}), wenn
\[
	\phi_i (x_j) = \delta_{ij} = \begin{cases}
							1, &  i = j \\
							0 ,& i \not = j
						\end{cases} \, , %\qquad \forall \,\phi_i \in \mcal B_h, x_j \in P
\]
für alle $\phi_i \in \mcal B_h$ und $x_j \in P$ gilt.
\end{defi}


Mit der Menge der vorgegebenen Punkte $P$ kann durch die Anzahl $N$ der Grad des zur Interpolation verwendeten Polynoms gesteuert werden.


\begin{bem}
Sei $m \ge 0$. In einem Dreieck $T$ seien auf $m+1$ Linien $l = 1+2+\ldots+(m+1)$ Punkte $z_1,\ldots,z_l$ angeordnet (s. Abb. \ref{abb:2.3}). Dann gibt es zu jedem $C^0(T)$ genau ein Polynom $p$ vom Grad $m$ mit der Eigenschaft
\[
	p(z_i) = f(z_i) \quad \forall \, i = 1,\ldots,m \, .
\]
\end{bem}

\begin{proof}
Der Beweis steht in \cite{BraeFEM} Kapitel II, \S 5, Bemerkung 5.4.
\end{proof}


\begin{figure}[h]
\begin{center}
\begin{pspicture}(-3,0)(5,2)
	% lineares Element:
	\psline(-3,0)(-3,2)
	\psline(-3,0)(-1,0)
	\psline(-1,0)(-3,2)
	\psdots(-1,0)(-3,0)(-3,2)
	
	% quadratisches Element:
	\psline(0,0)(0,2)
	\psline(0,0)(2,0)
	\psline(2,0)(0,2)
	\psdots(0,0)(0,2)(2,0)(0,1)(1,1)(1,0)
	
	% kubisches Element:
	\psline(3,0)(3,2)
	\psline(3,0)(5,0)
	\psline(5,0)(3,2)
	\psdots(3,0)(5,0)(3,2)(3,1.3333)(3,0.6667)(3.6667,0)(4.3333,0)(3.6667,0.6667)(3.6667,1.3333)(4.3333,0.6667)
	
\end{pspicture}
\end{center}
\caption{Dreiecke für nodale Basen (linear, quadratisch, kubisch)\label{abb:2.3}}
\end{figure}


Damit lässt sich für $V_h$ mit einem beliebigen Polynomgrad $m$ eine eindeutige \idx{nodale Basis} finden, die den Raum aufspannt. Im weiteren wollen wir lineare Ansatzfunktionen verwenden. Wir bezeichnen, sofern nicht anders beschrieben, also im Folgenden $\mcal S_h$ mit
\[
	\mcal S_h \coloneqq \{v \in C^0(\Omega) \mid v|_T \in \mcal P_1 \text{ für } T\in \mcal T_h, v|_{\partial \Omega} = 0\} \, ,
\]
also sind in diesem Raum die Eckpunkte der Dreiecke vorgegeben bzw. später im LGS gesucht. Das \idx{Galerkin-Verfahren} mit dem Ansatzraum $\mcal S_h$ wird auch \idx{Finite-Elemente-Methode} (kurz: FEM) genannt. 


Wir wollen folgendes Beispiel zur Berechnung von den Matrixeinträgen der Matrix $A$ betrachten.

\begin{bsp}\label{bsp:2.26}
Wir betrachten das Variationsproblem \eqref{eq:2.9} auf $\Omega = [-1,1]^2$ mit $\mcal S_h$ wie oben eingeführt als den Raum der linearen Ansatzfunktionen auf einer Zerlegung $\mcal T_h$ aus 8 \textit{\idx{Courant-Element}en}, wie in Abbildung \ref{abb:2.4} zu sehen ist, wobei wir auf die rechte Seite $F(v_h)$ zunächst noch nicht genauer eingehen möchten.


\begin{figure}[h]
\begin{center}
\begin{pspicture}(-2,-2)(2,2.5)
	% Skalierung:
	\psset{xunit=2cm,yunit=2cm}

	% Die 8 Courant-Elemente:
	\psline(-1,-1)(-1,1)
	\psline(-1,1)(1,1)
	\psline(1,1)(1,-1)
	\psline(1,-1)(-1,-1)
	\psline(-1,0)(1,0)
	\psline(0,-1)(0,1)
	\psline(0,-1)(-1,0)
	\psline(0,1)(1,0)
	\psline(-1,1)(1,-1)
	
	% Beschriftung der Elemente:
	\rput(0.3,0.3){I}
	\rput(0.7,0.7){II}
	\rput(-0.3,-0.3){VI}
	\rput(-0.7,-0.7){V}
	\rput(-0.65,0.3){III}
	\rput(-0.3,0.7){IV}
	\rput(0.3,-0.7){VII}
	\rput(0.7,-0.3){VIII}
	
	% Beschriftung und Markierung der Punkte:
	\psdots(-1,-1)(-1,0)(-1,1)(0,-1)(0,0)(0,1)(1,-1)(1,0)(1,1)
	% links:
	\rput(-1.13,1.13){1}
	\pscircle[linewidth=0.5pt](-1.13,1.13){0.21}
	\rput(-1.15,0){4}
	\pscircle[linewidth=0.5pt](-1.15,0){0.21}
	\rput(-1.13,-1.13){7}
	\pscircle[linewidth=0.5pt](-1.13,-1.13){0.21}
	% rechts:
	\rput(1.13,1.13){3}
	\pscircle[linewidth=0.5pt](1.13,1.13){0.21}
	\rput(1.15,0){6}
	\pscircle[linewidth=0.5pt](1.15,0){0.21}
	\rput(1.13,-1.13){9}
	\pscircle[linewidth=0.5pt](1.13,-1.13){0.21}
	% mitte:
	\rput(0,1.15){2}
	\pscircle[linewidth=0.5pt](0,1.15){0.21}
	\rput(0.15,0.15){5}
	\pscircle[linewidth=0.5pt](0.15,0.15){0.21}
	\rput(0,-1.15){8}
	\pscircle[linewidth=0.5pt](0,-1.15){0.21}
\end{pspicture}
\end{center}
\caption{Triangulierung von $\Omega = [-1,1]^2$ in 8 Courant-Elemente\label{abb:2.4}}
\end{figure}

Wir stellen für die nodale Basisfunktion $\phi_5$ die Einträge in der Steifigkeitsmatrix $A$ auf. Man rechnet leicht nach, dass
\begin{align*}
	\phi_5(x,y) = \begin{cases}
					1-x-y , & \text{auf } \rz I \\
					1+x, & \text{auf } \rz{III} \\
					1-y, & \text{auf } \rz{IV} \\
					1+x+y, & \text{auf } \rz{VI} \\
					1+y, & \text{auf }\rz{VII}\\
					1-x, & \text{auf }\rz{VIII} \\
					0, & \text{sonst}
				\end{cases}
\end{align*}

\begin{table}[htpb]
\centering
\begin{tabular}[c]{|c|c|c|c|c|c|c|c|c|}
	\hline
      & I & II & III & IV & V & VI & VII & VIII\\
	\hline
     $\partial_x \phi_5$ &-1 &0 &1& 0&0 &1 &0 &-1  \\
     $\partial_y \phi_5$ & -1&0 & 0& -1& 0& 1& 1& 0\\
	\hline
\end{tabular}
\caption{\label{tab:2.1}Ableitungen der nodalen Basisfunktion $\phi_5$.}
\end{table}

ist und es ergeben sich die Ableitungen aus Tabelle \ref{tab:2.1}. Dann gilt
\begin{align*}
	a(\phi_5,\phi_5) & = \int_\Omega \nabla \phi_5 \nabla \phi_5 \, dx dy  = \int_{\rz I  \cup \ldots \cup \rz{VIII}} \underbrace{(\partial_x \phi_5)^2}_{\ge0}+ \underbrace{(\partial_y \phi_5)^2}_{\ge 0} \, dx dy \\
	& = 2 \int_{\rz I \cup \rz{III} \cup \rz{IV}} {(\partial_x \phi_5)^2}+ {(\partial_y \phi_5)^2} \, dx dy \\
	& = 2 \(\int_{\rz I \cup \rz{III}}\underbrace{(\partial_x \phi_5)^2}_{=1} \, dx dy + \int_{\rz I \cup \rz{IV}} \underbrace{(\partial_y \phi_5)^2}_{=1} \, dxdy\) \\
	& = 2 (\mscr A(\rz I) + \mscr A(\rz{III}) +\mscr A(\rz{I}) +\mscr A(\rz{IV})) \\
	& = 8 \cdot \mscr A(\rz I) = 8 \cdot \frac 1 2= 4 \, ,
\end{align*}
wobei verwendet wurde, dass die Dreiecke kongruent zueinander sind und $\mscr A(\cdot)$ den Flächeninhalt eines Dreiecks berechnet. Analog können wir auch die übrigen acht nodalen Basisfunktionen aufstellen und damit die Einträge der Steifigkeitsmatrix
\begin{align*}
	a(\phi_5,\phi_2) = a(\phi_5,\phi_4) = a(\phi_5,\phi_6) = a(\phi_5,\phi_8) &= -1 \, , \\
	a(\phi_5,\phi_1) = a(\phi_5,\phi_3) = a(\phi_5,\phi_7) = a(\phi_5,\phi_9)& = 0
\end{align*}
berechnen. Damit ist der Einteil der Basisfunktion $\phi_5$ an der Steifigkeitsmatrix $A$ von der Form
\[
	\widetilde A = \begin{pmatrix}
		0 & -1 & 0 \\
		-1 & 4 & -1 \\
		0 & -1 & 0
	\end{pmatrix} \, .
\]
Hierbei müssen die Einträge aus $\widetilde A$ in die Matrix $A \in \R^{9 \times 9}$ an die richtige Stelle zugeordnet werden, wie durch die Formel $a_{ij} = a(\phi_i,\phi_j)$ beschrieben wird. Daher nennen wir $\widetilde A$ \textit{\idx{lokale Steifigkeitsmatrix}} bzgl. des Knoten 5.

Dieses Vorgehen müssten wir noch für die übrigen Basisfunktion analog durchführen, um die vollständige Steifigkeitsmatrix $A$ zu erhalten. Dies soll hier aber nicht weiter ausgeführt werden.
\end{bsp}

Wie man sieht, ist das Vorgehen aus Beispiel \ref{bsp:2.26} sehr aufwendig. Außerdem ist es schwer jenes auf diese Weise zu verallgemeinern, damit man es gut implementieren kann, da die Ansatzfunktionen auf das Gitter bezogen von individueller Form sind.


\begin{figure}[h!]
\begin{center}
\begin{pspicture}(-0.5,0)(7,3)
	% Skalierung
	\psset{xunit=2cm,yunit=2cm}
	
	% lokales Koordinatensystem:
	\psline{->}(0,0)(1.3,0)
	\psline{->}(0,0)(0,1.3)
	\rput(1.3,-0.12){$\xi$}
	\rput(-0.1,1.3){$\eta$}
	
	% globales Koordinatensystem:
	\psline{->}(2,0)(2.5,0)
	\psline{->}(2,0)(2,0.5)
	\rput(2.5,-0.13){$x$}
	\rput(1.9,0.5){$y$}
	
	% Referenz-Dreieck:
	\psline(0,0)(1,0)
	\psline(0,0)(0,1)
	\psline(1,0)(0,1)
	\rput(0.3,0.3){$\widetilde T$}
	
	% Punktbeschriftung:
	\rput(0,-0.13){\small (0,0)}
	\rput(0.95,-0.13){\small (1,0)}
	\rput(-0.22,1){\small (0,1)}
	
	% allgemeines Dreieck:
	\psline(2.2,0.7)(3.2,1.2)
	\psline(3.2,1.2)(3.7,0.2)
	\psline(3.7,0.2)(2.2,0.7)
	\rput(3,0.8){$T$}
	
	% Punktbeschriftung:
	\rput(1.9,0.8){\small$(x_1,y_1)$}
	\rput(3.2,1.35){\small $(x_3,y_3)$}
	\rput(3.7,0.1){\small $(x_2,y_2)$}
	
	% affine Trafo:
	\pscurve{->}(0.7,0.7)(1.4,1.2)(2.2,1)
\end{pspicture}
\end{center}
\caption{Referenzelement $\widetilde T$ für ein allgemeines Dreieck $T \in \mcal T_h$\label{abb:2.5}}
\end{figure}


Um die Berechnung der Einträge der Steifigkeitsmatrix $A$ zu verallgemeinern, betrachten wir das Referenzelement
\[
	\widetilde T \coloneqq \{ (\xi,\eta) \in \R^2 \mid 0\le \xi \le 1, 0 \le \eta \le 1-\xi\} \, .
\]
Auf diesem können wir dann die Integrale für die Ansatzfunktionen lokal berechnen, um dann die berechneten Werte affin auf ein beliebiges Dreieck $T$ zu transformieren (s. Abbildung \ref{abb:2.5}). Diese auf das Element $T$ bezogene lokale Steifigkeitsmatrix müssen wir dann durch \textit{\idx{global-local node ordering}} in die globale Steifigkeitsmatrix $A$ assemblieren. Die genaue Berechnungsvorschrift für das oben beschriebene Vorgehen wird in Kapitel \ref{kap:5} noch mal genauer hergeleitet.







\subsection{A priori Fehlerabschätzung}
\label{kap:2.3.1}

Wir wollen nun zeigen, dass durch Netzverfeinerung, d.h. Verkleinern von $h$, der Fehler zwischen der exakten Lösung $u$ und der Galerkin-Approximation $u_h$ kleiner wird.

\begin{lemma}\label{lem:2.27}
Durch $\norm \cdot _E: H^1_0(\Omega) \ra \R, \norm v_E \coloneqq (a(v,v))^{\frac 1 2}$ mit einer stetigen koerziven Bilinearform $a$ wird eine Norm auf $H_0^1(\Omega)$ definiert.
\end{lemma}

\begin{proof}
Aus der Stetigkeit und Koerzivität von $a$ folgt direkt
\begin{align}\label{eq:2.12}
	\alpha \, \norm{v}_1^2 \le \underbrace{a(v,v)}_{= \norm v_E^2} \le c \, \norm v_1^2 \, .
\end{align}
Damit ist $\norm\cdot_E$ nach oben und unten durch die Norm auf $H^1_0(\Omega)$ beschränkt und somit eine zu dieser äquivalente Norm.
\end{proof}


\begin{bem*}
\begin{enumerate}[(a)]
\item Die Norm $\norm \cdot_E$ bezeichnen wir als \textit{\idx{Energie-Norm}}. Sie gibt für die von uns später in der Strukturmechanik betrachtete Bilinearform  die Verzerrungsenergie eines Kontinuums an.
\item Für die Bilinearform
\[
	a(u,v) = \int_\Omega \nabla u \nabla v \, dx
\]
mit $u,v \in H^1_0(\Omega)$ gilt dann $\norm\cdot_E = \abs\cdot_1$ (s. Bemerkung \ref{bem:A.6}).
\item Auf $H^1(\Omega)$ wäre $\norm{\cdot}_E$ also nur eine Halbnorm, da konstante Funktionen $v = c$ auch die Norm $\norm{v}_E = 0$ hätten.
\end{enumerate}
\end{bem*}


\begin{satz}\label{satz:2.28}
Die \idx{Galerkin-Approximation} $u_h$ ist die beste Approximation von $u$ bzgl. der \idx{Energie-Norm}, also
\[
	\norm{u-u_h}_E = \inf_{v \in V_h} \norm{u-v}_E \, .
\]
\end{satz}

\begin{proof}
Zunächst betrachten wir die exakte und approximierte Variationsgleichung \eqref{eq:2.8} und \eqref{eq:2.9}, d.h.
\begin{align}\label{eq:2.13}
	a(u,v) &= F(v) \quad \ \, \forall \, v \in H \, , \\
	\label{eq:2.14}
	a(u_h,v_h) &= F(v_h) \quad \forall \, v_h \in V_h \, .
\end{align}
Da $V_h \subset H$ ist, gilt \eqref{eq:2.13} auch für alle $v_h\in V_h$. Ersetzen wir dies  in \eqref{eq:2.13} und subtrahieren \eqref{eq:2.13} und \eqref{eq:2.14}, so erhalten wir
\begin{align}\label{eq:2.15}
	a(u-u_h,v_h ) = 0 \quad \forall \, v_h \in V_h \, .
\end{align}
Damit rechnen wir für ein beliebiges $v \in V_h$ einfach nach:
\begin{align*}
	\norm{u-u_h}_E^2 & = a(u-u_h,u-u_h) \\
	& = a(u-u_h, u-v+v-u_h) \\
	& = a(u-u_h,u-v)+\underbrace{a(u-u_h,\underbrace{v-u_h}_{\in V_h})}_{=0\text{ wegen \eqref{eq:2.14}}} \\
	& = a(u-u_h,u-v) \\
	&\!\!\!\:\!\:\!\:\!\:\! \stackrel{\scriptsize\text{C.S.}}\le \norm{u-u_h}_E \norm{u-v}_E 
\end{align*}
und damit folgt nach Division $\norm{u-u_h}_E \le \norm{u-v}_E$, was zu zeigen war.
\end{proof}


\begin{bem*}
Die Gleichung \eqref{eq:2.15} drückt aus, dass die Verbindung $u-u_h$ orthogonal zum Raum $V_h$ steht und wird daher auch \textit{\idx{Galerkin-Orthogonalität}} genannt. Diese wird bei Hindernisproblemen im Allgemeinen nicht mehr erfüllt, da diese, wie wir später sehen werden, nicht mehr auf Variationsgleichungen führen.
\end{bem*}


\begin{satz}[Céa]\label{satz:2.27}
Der Fehler der \idx{Galerkin-Approximation} $u_h$ hat in der $H^1$-Norm die Eigenschaft
\[
	\norm{u-u_h}_1 \le \tilde c \inf_{v\in V_h} \norm{u-v}_1 \, .
\]
\end{satz}

\begin{proof}
Aus \eqref{eq:2.12} und Satz \ref{satz:2.28} folgt
\[
	\norm{u-u_h}_1 \le \(\frac 1\alpha\)^{\frac 1 2} \norm{u-u_h}_E \le \(\frac 1\alpha\)^{\frac 1 2} \norm{u-v}_E \le \(\frac c\alpha\)^{\frac 1 2} \norm{u-v}_1  \, .
\]
Damit folgt die Behauptung mit $\tilde c :=  \sqrt{\frac c\alpha}$.
\end{proof}


Nun kommen wir zum zentralen Satz dieses Unterkapitels, mit dem man direkt die gewünschte Aussage folgern kann. Vergleiche hierzu auch \cite{BraeFEM} Kapitel II, \S6, Satz 6.4.

\begin{theorem}[\idx{Approximationssatz für Interpolationen}]\label{theorem:2.28}
Es sei $k \ge 2$ und $\mcal T_h$ eine quasi-uniforme Triangulierung\index{Triangulierung!quasi-uniform} von $\Omega$. Dann gilt für die Interpolation $I_h$ auf die stetigen, stückweise durch Polynome vom Grad $k-1$ gegebenen Funktionen mit einer von $\Omega, \kappa$ und $k$ abhängigen Kontanten $c$ die a priori Fehlerabschätzung
\[
	\norm{u-I_hu}_m \le c h^{k-m} \abs{u}_k
\]
für $u \in H^k(\Omega)$ und $0\le m\le k$.
\end{theorem}

\begin{proof}
Für den Beweis würden wir noch weitere Ausführungen über affine Transformationen benötigen, die wir hier nicht weiter aufführen wollen. Der komplette Beweis ist in \cite{BraeFEM} auf Seite 75ff einzusehen.
\end{proof}


Für $k=2$, also lineare Polynome, und $m=1$ (die Norm in $H^1$) gilt dann
\begin{align}\label{eq:2.16}
	\norm{u-I_hu}_1 \le c h \abs{u}_2
\end{align}
für $u \in H^2(\Omega)$.


\begin{kor}\label{kor:2.31}
Für lineare $C^0$-Elemente gilt bzgl. der Galerkin-Approximation $u_h$ die a priori Fehlerschätzung für unser Modellproblem \eqref{eq:DP}
\[
	\norm{u-u_h}_1 \le \tilde c h \abs{u}_2 \, .
\]
\end{kor}

\begin{proof}
Mit Theorem \ref{theorem:2.28} und Satz \ref{satz:2.27} folgt
\begin{align*}
	\norm{u-u_h}_1& \le \(\frac {c_1}\alpha\)^{\frac 1 2} \inf_{v\in V_h} \norm{u-v}_1
	 \le \(\frac {c_1}\alpha\)^{\frac 1 2}  \norm{u-I_h u}_1 \\
	& \!\!\!\! \:\!\: \!\stackrel{\scriptsize \eqref{eq:2.16}}\le \(\frac {c_1}\alpha\)^{\frac 1 2} c_2 h \abs{u}_2 \, .
\end{align*}
Mit $u\in H^2(\Omega)$ und $\tilde c := \(\frac {c_1}\alpha\)^{\frac 1 2} c_2 $ folgt dann die Behauptung.
\end{proof}


Mit Korollar \ref{kor:2.31} gilt also, dass für $h \ra 0$ die Galerkin-Approximation $u_h$ gegen die exakte Lösung $u$ konvergiert. Es ist also sinnvoll das Netz zu verfeinern, allerdings bringt jede Verfeinerung auch mehr Knoten und damit ein größeres Gleichungssystem mit sich. Daher stellt sich die Frage, ob es sinnvoll ist, nur einzelne Teile des Gitters zu verfeinern, womit wir uns im Kapitel \ref{kap:2.4} beschäftigen wollen.




\section{Adaptive Verfeinerungsstrategien}
\label{kap:2.4}

Wir wollen nun betrachten, wie sich das Gitter geschickt verfeinern lässt, so dass die Anzahl der Knoten im Vergleich zur Verringerung des Fehlers hinreichend groß ist. Diese Verfeinerung im $n$-ten Schritt geschieht adaptiv in Abhängigkeit des aktuellen Gitters $\mcal T_{h_n}$ bzw. der aktuellen Lösung $u_{h_n}$. Hierbei wird \textit{a posteriori} der Fehler im nächsten Schritt $\norm{u-u_{h_{n+1}}}$ mit dem Fehler im aktuellen $\norm{u-u_{h_n}}$ verglichen und daraus die notwendige Größe der Verfeinerung abgeschätzt. Da die Lösung $u$ für ein Problem, wie  schon beschrieben, nicht bekannt oder berechenbar sein muss, gilt es den oben beschriebenen Fehler zu schätzen. Für solche a posteriori Fehlerschätzer gibt es mehrere Ansätze.


\subsection{A posteriori Fehlerschätzer}
\label{kap:2.4.1}

Wie auch in \cite{BraeFEM} Kapitel III, \S 8, Seite 176 genauer beschrieben, gibt es verschiedene Arten von \idx{a posteriori Fehlerschätzer}n:

\begin{enumerate}[(a)]
\item Residuale Schätzer
\item Schätzung über ein lokales Neumann-Problem
\item Schätzung über ein lokales Dirichlet-Problem
\item Schätzung durch Mitteilung
\item Hierarchische Schätzer
\end{enumerate}

Als Fehlerschätzer für Hindernis- oder auch Kontaktprobleme sind häufig residuale Schätzer zu finden. Wir wollen uns in dieser Arbeit mit hierarchischen Fehlerschätzern beschäftigen.

Die Idee dabei ist, dass wir den Fehler durch eine genauere Lösung aus einem "`besseren"' Ansatzraum abschätzen, in dem Sinne, dass für den Ansatzraum $V_h$, in dem die berechnete Lösung $u_h$ liegt, gilt:
\begin{align}
	V_h \subset W_h \text{ mit } W_h = V_h \oplus Z_h \, ,
\end{align}
wobei bzgl. der Obermenge $W_h$ die Lösung $u_h^W$ genauer ist als $u_h$, d.h. die \textit{\idx{Saturationseigenschaft}}
\begin{align}\label{eq:2.18}
	a(u-u_h^W,u-u_h^W) \le \beta^2 a(u-u_h,u-u_h)
\end{align}
mit einer Konstanten $0\le \beta < 1$ erfüllt.

\begin{bem}
Da wir zur Berechnung unser \idx{Galerkin-Approximation} den Raum $V_h = \mcal S_h$ der linearen Ansatzfunktionen verwenden werden, wählen wir später als Hierarchie
\[
	W_h = \mcal Q_h \coloneqq \{v \in C^0(\Omega) \mid v|_T \in \mcal P_2 \text{ für } T \in \mcal T_h, v|_{\partial\Omega} = 0\} \, ,
\]
den Raum der quadratischen Ansatzfunktionen. Daraus lässt sich dann der Raum $Z_h$ ableiten.
\end{bem}


Wir sollen nun aber nicht die exakte (bessere) Approximation $u_h^W\in W_h$ berechnen, da dies ein zu großer Aufwand wäre, sondern schränken das sogenannte \textit{\idx{Defektproblem}} lokal auf die Erweiterung $Z_h$ ein, d.h.
\begin{align}\label{eq:2.19}
	a(u_h^W,z_h) = F(z_h) \quad \forall \, z_h \in Z_h \, ,
\end{align}
wobei in \eqref{eq:2.19} $u_h^W$ aufgeteilt werden kann, da $W_h$ aus direkter Summe von $V_h, Z_h$ entsteht, d.h. $u_h^W = u_h+e_h$ mit $u_h \in V_h, e_h\in Z_h$. Da die Lösung $u_h$ in jedem Schritt bekannt ist, lässt sich das lokale Defektproblem \eqref{eq:2.19} schreiben als
\begin{align}\label{eq:2.20}
	e_h \in Z_h : \quad a(e_h,z_h) = \underbrace{F(z_h) - a(u_h,z_h)}_{= \widetilde F(z_h)} \quad \forall \, z_h \in Z_h \, .
\end{align}
Man kann zeigen, dass für die Lösung $e_h$ aus \eqref{eq:2.20} unter der Bedingung \eqref{eq:2.18} der Term $a(e_h,e_h)$ beschränkt ist und daher gibt  der auf ein Dreieck $T$ bezogene lokale Anteil $\eta_T \coloneqq a_T(e_h,e_h)^{\frac 12}$ den hierarchischen Fehlerschätzer an.

Damit lässt sich $a(e_h,e_h)$ in die lokalen Anteile $\eta_T$ aufteilen, so dass
\[
	\sum_{T\in \mcal T_h} a_T(e_h,e_h) = a(e_h,e_h) \, .
\]
Daher verwenden wir als adaptive Strategie: Verfeinere alle Dreiecke $T\in \mcal T_h$, deren lokaler Fehleranteil größer gleich dem skalierten Gesamtfehler sind, also
\[
	\eta_T \ge \sigma \(\sum_{T\in \mcal T_h} \eta_T^2\)^{\frac 12} ,
\]
wobei $\sigma \in (0,1)$ ist. Wählen wir $\sigma$ sehr klein, so werden viele Dreiecke verfeinert, da auch kleinere lokale Fehleranteile die Ungleichung erfüllen. Umgekehrt gilt für ein großes $\sigma$, dass man viele Adaptionsschritte benötigt, um einen hinreichend kleinen Fehler zu erhalten, da nur wenige Dreiecke pro Verfeinerungsschritt ausgewählt werden.

Die Idee der Hierarchie bzgl. der Räume wollen wir in Kapitel \ref{Kap:4} auch auf Variationsprobleme unter Nebenbedingung anwenden, um einen a posteriori Schätzer herzuleiten.


\subsection{Verfeinerung des Netzes}
\label{kap:2.4.2}

Auf der Grundlage des a posteriori Schätzers müssen die ausgewählten Dreiecke verfeinert werden. Dabei ist es essentiell, dass eine konforme Triangulierung\index{Triangulierung!konform} auch konform bleibt, was durch Verwendung von verschiedenen Verfeinerungsmethoden möglich ist (s. Abbildung \ref{abb:2.6}).



\begin{enumerate}[(i)]
\item \textit{Rote} (\textit{reguläre})\index{Verfeinerung!regulär}\index{Verfeinerung!rot} Verfeinerung,
\item \textit{Grüne} Verfeinerung,\index{Verfeinerung!grün}
\item \textit{Blaue} Verfeinerung.\index{Verfeinerung!blau}
\end{enumerate}


\begin{figure}[h]
\begin{center}
\begin{pspicture}(0,0)(5,3)
	\psset{linewidth=0.5pt}
	% die Dreiecke:
	\pspolygon(-2,0)(-1,3)(0.5,1.7)
	\pspolygon(1.5,0)(2.5,3)(4,1.7)
	\pspolygon(5,0)(6,3)(7.5,1.7)
	\psdots(-2,0)(-1,3)(0.5,1.7)(1.5,0)(2.5,3)(4,1.7)(5,0)(6,3)(7.5,1.7)
	
	% Verfeinerungen:
	\pspolygon(-1.5,1.5)(-0.25,2.35)(-0.75,0.85)
	\psdots[linecolor=red](-1.5,1.5)(-0.25,2.35)(-0.75,0.85)
	\psline(4,1.7)(2,1.5)
	\psdot[linecolor=green](2,1.5)
	\psline(7.5,1.7)(5.5,1.5)
	\psline(6.25,0.85)(5.5,1.5)
	\psdots[linecolor=blue](5.5,1.5)(6.25,0.85)
\end{pspicture}
\caption{Verfeinerungen von Dreiecken\label{abb:2.6}}
\end{center}
\end{figure}

Bei der regulären Verfeinerung werden die drei Mittelpunkte der Kanten eines  Dreiecks miteinander verbunden. Der Vorteil dabei ist, dass die Winkel der entstandenen Dreiecke identisch zu den vorherigen Winkeln sind. Damit bleibt das Verhältnis zwischen dem Radius des Um- zum Innenkreises $\frac{h_T}{\rho_T}$, was den Parameter für die Quasi-Uniformität angibt, gleich bleibt.

Beim regulären Verfeinern können allerdings in anliegenden Dreiecken, die bzgl. des a posteriori Fehlerschätzers nicht verfeinert werden müssen, hängende Knoten (vgl. Abbildung \ref{abb:2.2}) entstehen, wodurch eine nichtzulässige Triangulierung entstehen würde. Um die hängenden Knoten zu eliminieren, kann Verfeinerungsmethode (ii) und (iii) verwendet werden.

Bei der grünen Verfeinerung\index{Verfeinerung!grün} wird der Mittelpunkt genau einer Kante mit dem gegenüberliegendem Eckpunkt verbunden.

Bei der blauen Verfeinerung\index{Verfeinerung!blau} verbindet man den Mittelpunkt der längsten Kante im Dreieck mit dem gegenüberliegendem Eckpunkt und einem weiteren Mittelpunkt einer anderen Kante.

Matlab verwendet diese Verfeinerungsstrategien in der Methode \textit{refinemesh}\index{Matlab!refinemesh}. Diese Methode werden wir bei der Implementierung der numerischen Beispiele auch verwenden und daher wollen wir das Vorgehen von Matlab kurz vorstellen. Ist $\widetilde{\mcal T_h}$ die Menge der zu verfeinernden Dreiecke, dann werden folgende Schritte durchgeführt:
\begin{enumerate}[(i)]
\item Halbiere alle Kanten der ausgewählten Dreiecke $T \in \widetilde{\mcal T_h}$.
\item Halbiere jeweils die längste Kante aller Dreiecke, die schon eine geteilte Kante haben. 
\item Bilde die neuen Dreiecke nach folgender Strategie:
\begin{enumerate}[(a)]
\item Wenn alle drei Kanten eines Dreiecks geteilt sind, verwende die reguläre (rote) Verfeinerung.
\item Sind genau zwei Seiten halbiert, so verwende die blaue Verfeinerung.
\item Ist genau eine Kante eines Dreiecks halbiert, dann benutze die grüne Verfeinerung.
\end{enumerate}
\end{enumerate}





\section{Einführung in die Strukturmechanik}
\label{kap:2.5}


Um in Kapitel \ref{kap:3.2} Kontaktprobleme bzgl. der Mechanik beschreiben zu können, wollen wir in diesem Kapitel eine kurze Einführung in die Kontinuumsmechanik geben.

\begin{defi}[\idx{Kontinuum}]
Ein \textit{Kontinuum} ist eine Teilmenge $\mathscr{B}\subset \R^3$, dessen  Punkte stetig verteilt sind. Den Punkten werden gewisse Materialeigenschaften zugewiesen.
\end{defi}


\subsection{Kinematik}
\label{kap:2.5.1}

Um die Kinematik für ein \idx{Kontinuum} zu beschreiben, betrachten wir dieses in zeitlich abhängigen Zuständen. Dabei unterscheiden wir zwischen den \textit{materiellen Punkten}\index{materielle Punkte} des Kontinuums, die mit $\bs X = (X_1,X_2,X_3)$ beschrieben werden, und den \textit{räumlichen Punkten}\index{räumliche Punkte} $\bs x = (x_1,x_2,x_3)$.

\begin{defi}[\idx{Konfiguration}]
Eine zu jedem Zeitpunkt $t$ differenzierbare und stetige Zuordnung $\bs x = \varphi(\bs X, t)$ nennen wir \textit{Konfiguration}. Die Konfiguration zum Zeitpunkt $t = t_0$ nennen wir \textit{Ausgang-} oder \textit{\idx{Referenzkonfiguration}}\index{Ausgangskonfiguration}\index{Konfiguration!Ausgangs-}\index{Konfiguration!Referenz-}, die Konfiguration zum aktuellen Zeitpunkt $t$ \textit{\idx{aktuelle Konfiguration}} oder \textit{\idx{Momentankonfiguration}}\index{Konfiguration!aktuelle}\index{Konfiguration!Momentan-}.
\end{defi}


\begin{figure}[h!]
\begin{center}
\begin{pspicture}(0,-1)(4,3)
	% materielle Punkte:
	\psccurve(-2,0)(0,0.3)(-0.6,2.3)(-2.2,1.2)
	\psdot(-1,0.6)
	\rput(-1.3,0.5){$\bs X$}
	\rput(-0.6,1.8){$\mscr B$}
	
	% Momentankonfiguration:
	\psccurve(3,1)(5,0.5)(6.4,2)(5.4,3)(4,2)
	\rput(6,2){$\mscr B$}
	\psdot(4,1)
	\rput(4.2,0.9){$\bs x$}
	
	% Koordinatensystem:
	\psline{->}(2,0)(3,0)
	\psline{->}(2,0)(2,1)
	\psline{->}(2,0)(1.3,-0.42)
	\rput(1.6,-0.5){$\bs e_1$}
	\rput(2.25,1){$\bs e_3$}
	\rput(3,-0.2){$\bs e_2$}
	
	% Zuordnung:
	\pscurve{->}(0.5,1.5)(1.8,2.5)(3.5,2.1)
	\rput(2,2.8){$\varphi$}
\end{pspicture}
\caption{Konfiguration eines Kontinuums $\mscr B$\label{abb:2.7}}
\end{center}
\end{figure}


\begin{bem}
Wir können der Einfachheit halber zu einem Startzeitpunkt $t = t_0$ die materiellen Punkte durch die räumlichen Koordinaten der Ausgangskonfiguration $\bs X = \bs x(t_0)$ beschreiben. So liegt der Körper $\mscr B$ aus Abbildung \ref{abb:2.7} im selben Koordinatensystem.
\end{bem}


Damit ist die Bewegung (oder auch Deformation) eines Kontinuums als zeitlich stetige Folge von Konfigurationen $\bs x = \bs x(\bs X,t)$ zu verstehen. Hierfür gibt es zwei grundlegende Betrachtungsweisen, die \textit{{Lagrange'sche}}\index{Betrachtungsweise!Lagrange'sche} und die \textit{Euler'sche Betrachtungsweise}\index{Betrachtungsweise!Euler'sche}. Bei der Lagrange'schen Betrachtung ist der Beobachter mit dem materiellen Punkt $\bs X$ verbunden und misst alle Änderungen des Kontinuums an diesem Punkt; jene wird auch \textit{materielle Betrachtungsweise}\index{Betrachtungsweise!materielle} genannt. Bei der Euler'schen Betrachtungsweise befindet sich der Beobachter am räumlichen Punkt $\bs x$ und misst zu jedem Zeitpunkt $t$ die Änderungen am Punkt $\bs x$, die sich durch das Passieren von Teilchen $\bs X$ ergeben; jene wird auch \textit{räumliche Betrachtungsweise}\index{Betrachtungsweise!räumliche} genannt.


Um die Deformation eines Körpers $\mscr B$ nun beschreiben zu können, betrachten wir die Veränderung von Linienelementen $d \bs x$ bzgl. der Ausgangs- und Momentankonfiguration. Wir beschreiben den sogenannten \textit{\idx{Deformationsgradient}} $\bs F$ durch den Faktor, der zwischen der Deformation dieser Linienelemente liegt, d.h.
\begin{align}\label{eq:2.21}
	d\bs x = \bs F \, d\bs X \, .
\end{align}
Damit ergibt sich der Deformationsgradient als
\begin{align}\label{eq:2.22}
	\bs F = \frac{\partial \bs x}{\partial \bs X} = \Grad \bs x \, , 
\end{align}
dies ist der Gradient bzgl. der materiellen Betrachtung\index{Betrachtungsweise!materielle} und wird daher auch \textit{materieller Deformationsgradient} genannt. Analog können wir den \textit{räumlichen Deformationsgradienten} erhalten als
\begin{align}\label{eq:2.23}
	\bs F^{-1} = \frac{\partial \bs X}{\partial \bs x} = \grad \bs X \, .
\end{align}
Die Deformationsgradienten sind zweistufige Tensoren, die sich jeweils auf die Referenz- und Momentankonfiguration bzgl. der Basen beziehen.

Damit können wir nun die Verzerrung eines Kontinuums mittels des Deformationsgradienten ausdrücken. Um jene zu beschreiben, betrachten wir die Längenänderung zwischen den Linienelementen $d\bs X$ und $d\bs x$.
\begin{align*}
	\norm{d\bs x}^2-\norm{d\bs X}^2 &= (d\bs x)^T d\bs x - (d\bs X)^Td\bs X \\
	& \! \! \!\!\: \!\!\: \stackrel{\scriptsize \eqref{eq:2.21}}= (d\bs X)^T \bs F^T \bs F d\bs X - (d\bs X)^Td\bs X \\
	& = (d\bs X)^T (\bs F^T \bs F- \bs 1) d\bs X \, ,
\end{align*}
wobei $\bs 1$ den zweistufigen Einheitstensor beschreibt. Also lässt sich die Verzerrung bzgl. der Ausgangskonfiguration durch den \textit{Green-Lagrange'schen Verzerrungssensor}\index{Verzerrungssensor!Green-Lagrange}
\begin{align}\label{eq:2.24}
	\bs E = \frac 1 2 (\bs C- \bs 1)
\end{align}
mit dem \textit{rechten Cauchy-Green-Tensor}\index{Cauchy-Green-Tensor!rechter} $\bs C = \bs F^T\bs F$ beschreiben.


\begin{bem}
Dies ist natürlich nur eine Möglichkeit. Wichtig für die Wahl eines Verzerrungsmaßes ist jedoch, dass für reine Translation oder Rotation dieses Null wird. 

Daher ist beispielsweise $\bs F - \bs 1$ als Verzerrungsmaß unbrauchbar. Der Deformationsgradient lässt sich in Rotation $\bs R$ und Streckung $\bs U$ durch $$\bs F = \bs R \cdot \bs U$$ aufteilen. Sollten wir nun eine reine Rotation betrachten, so ist $\bs U = \bs 1$ und damit $\bs F = \bs R$, was nicht zwangsläufig der Einheitstensor sein muss.
\end{bem}


Wenn wir einen Punkt bzgl. seiner Ausgangs- und Momentankonfiguration vergleichen, so erhalten wir die \textit{\idx{Verschiebung}}
\begin{align}\label{eq:2.25}
	\bs u (\bs X,t) = \bs x(\bs X,t)-\bs X \, .
\end{align}
Damit ergeben sich analog zum materiellen und räumlichen Deformationsgradienten\index{Deformationsgradient!materiell}\index{Deformationsgradient!räumlich} \eqref{eq:2.22} und \eqref{eq:2.23} mit \eqref{eq:2.25} der materielle und räumliche \textit{Verschiebungsgradient}\index{Verschiebungsgradient!materiell}\index{Verschiebungsgradient!räumlich}:
\begin{align*}
	\bs H& = \Grad \bs u = \frac{\partial(\bs x - \bs X)}{\partial \bs X} = \frac{\partial\bs x}{\partial \bs X} - \frac{\partial \bs X}{\partial \bs X} = \bs F - \bs 1 \, ,  \\
	\bs h& = \grad \bs u = \frac{\partial(\bs x - \bs X)}{\partial \bs x} = \frac{\partial\bs x}{\partial \bs x} - \frac{\partial \bs X}{\partial \bs x} = \bs 1 - \bs F^{-1} \, .
\end{align*}
Damit ergibt sich beispielsweise für den Green-Lagrange'schen Verzerrungstensor \index{Verzerrungstensor!Green-Lagrange} \eqref{eq:2.24} bzgl. des materiellen Verschiebungsgradienten die Beziehung
\begin{align}\label{eq:2.26}
\begin{aligned}
	\bs E& = \frac 12 (\bs C - \bs 1) = \frac 1 2 (\bs F^T\bs F - \bs 1) = \frac 12((\bs 1+\bs H)^T(\bs 1+\bs H)-\bs 1) \\
	& = \frac 12 (\bs 1 + \bs H^T + \bs H +\bs H^T \bs H - \bs 1) = \frac 12(\bs H^T+\bs H+\bs H^T\bs H) \, .
\end{aligned}
\end{align}
Wenn wir von kleinen Deformationen ausgehen, so ist es sinnvoll für die spätere numerische Berechnung, die nichtlinearen Verzerrungsgrößen (wie in \eqref{eq:2.26})  zu linearisieren. Dies können wir mithilfe von Taylor und der \idx{Gâteaux-Ableitung} \eqref{eq:A.1} bewerkstelligen. Für eine vektor- oder tensorwertige Funktion $\bs A$ gilt dann
\begin{align}\label{eq:2.27}
	\operatorname{lin}(\bs A)_{\bs x,\bs u} = \bs A(\bs x) + \frac d{ d\eps} (\bs A(\bs x + \eps \bs u))\bigg|_{\eps = 0} \, .
\end{align}
Wenden wir also \eqref{eq:2.27} auf den Green-Lagrange'schen Verzerrungstensor \eqref{eq:2.26} an, so erhalten wir die Linearisierung:
\begin{align*}
	\operatorname{lin}(\bs E)_{\bs X,\bs u} & = \underbrace{\bs E(\bs X)}_{=0} + \frac d{d\eps} (\bs E(\bs X+\eps \bs u))\bigg|_{\eps = 0} \\
	& = \frac 12 \frac d{d\eps} \(\(\frac{\partial(\bs X+\eps\bs u)}{\partial \bs X}\)^T \frac{\partial(\bs X+\eps\bs u)}{\partial \bs X}-\bs  1\)\Bigg|_{\eps=0} \\
	& = \frac 12 \frac d{d\eps} \((\bs 1+\eps \bs H)^T (\bs 1+\eps \bs H) - \bs 1\) \bigg|_{\eps=0} \\
	& = \frac 12 \frac d{d\eps} (\bs 1+\eps \bs H^T+\eps\bs H+ \eps^2 \bs H^T\bs H-\bs 1) \bigg|_{\eps=0} \\
	& = \frac 12 (\bs H^T + \bs H + 2\eps \bs H^T\bs H)\bigg|_{\eps=0} = \frac 12(\bs H^T+ \bs H) \eqqcolon \bs \eps \, .
\end{align*} 

Wir wollen der Einfachheit halber in dieser Arbeit von kleinen Deformationen ausgehen, dann gilt $\bs H \approx \bs h = \nabla \bs u$. Damit werden wir also immer die linearisierten Verzerrungstensor
\begin{align}\label{eq:2.28}
	\bs \eps = \frac 1 2 (\nabla^T \bs u + \nabla \bs u)
\end{align}
verwenden.


\subsection{Kinetik}
\label{kap:2.5.2}

Da wir von kleinen Deformationen ausgehen, betrachten wir den \textit{\idx{Cauchy-Spannungstensor}} $\bs\sigma$, der die aktuelle Kraft bzgl. der Querschnittsfläche in der Momentankonfiguration setzt. Wie in \cite{WriggersContact} Kapitel 3.2.2 beschrieben, gilt das \textit{Cauchy Theorem}, das besagt, dass die Spannung $\bs t$ auf einer Schnittfläche eines beliebigen Schnittes im Körper $\mscr B$ gleich der Spannung $\bs\sigma$ in Normaleinrichtung $\bs n$ ist, d.h.
\begin{align}\label{eq:2.29}
	\bs t = \bs \sigma \cdot \bs n \, .
\end{align}
Mit den Bilanzgleichungen für das Momentengleichgewicht kann man herleiten, dass der Cauchy-Spannungstensor symmetrisch ist, also $\bs \sigma^T = \bs \sigma$ gilt.

Betrachten wir nun eine Volumenkraft $\bar{\bs b}$ und eine Oberflächenlast $\bar{\bs t}$, die auf den Körper $\mscr B$ wirken, so erhalten wir das (\textit{globale}) Kräftegleichgewicht
\begin{align}\label{eq:2.30}
	\int_{\Omega} \bar{\bs b} \, dv + \int_{\partial \Omega} \bar{\bs t} \, da = \bs 0 \, ,
\end{align}
wobei $\Omega \subset \mscr B \subset \R^3$ eine Teilmenge des Kontinuums $\mscr B$ beschreibt. Mit dem Cauchy Theorem \eqref{eq:2.29} und dem \idx{Satz von Gauß} lässt sich \eqref{eq:2.30} als Integral über $\Omega$ schreiben durch
\begin{align}\label{eq:2.31}
	\int_{\Omega} \bar{\bs b} + \div \bs\sigma \, dv = \bs 0 \, .
\end{align}
Die Gleichung \eqref{eq:2.31} muss nach dem Schnittprinzip auf jeder Teilmenge $\Omega$ gelten, was nur erfüllt werden kann, wenn der Integrand Null ist. Damit erhalten wir die sogenannte \textit{starke} oder auch \textit{lokale} Form des Gleichgewichts
\begin{align}\label{eq:2.32}
	\div \bs \sigma + \bar{\bs b} = \bs 0 \text{ auf } \Omega \, .
\end{align}
Wir wollen in der Arbeit von einem konstanten Wärmefeld ausgehen, sodass wir thermodynamische Prozesse vernachlässigen können.

\subsection{Konstitutive Gleichungen und Prinzipien}
\label{kap:2.5.3}


Auch für die konstitutiven Gleichungen (Materialannahmen etc.) machen wir uns zu Nutze, dass wir von kleinen Deformationen ausgehen. Wir gehen daher von einem linear elastischen Material aus, d.h. dass Spannung und Verzerrung in einem linearen Zusammenhang stehen. Hierfür werden wir in der Arbeit das \textit{Hooke'sche Materialmodell}\index{Hooke'sches Materialmodell} verwenden:
\begin{align}\label{eq:2.33}
	\bs \sigma = \mcal C :\bs \eps = 2 \mu \bs \eps + \lambda (\tr \bs \eps) \bs I \, ,
\end{align}
wobei $\lambda, \mu$ die \textit{\idx{Lamé-Konstanten}} darstellen. Dabei handelt es sich um materialabhängige Parameter, die im Zusammenhang mit dem \idx{Elastizitätsmodul} $E$ und der Querkontraktionszahl $\nu$ stehen.
\[
	\lambda = \frac{\nu E}{(1+\nu)(1-2\nu)} \, ,\quad \nu = \frac E{2(1+\nu)} \, .
\]
Der Materialtensor $\mcal C$ ist 4-stufig und mit "`:"' ist das doppelt verjüngende Skalarprodukt, das in Anhang \ref{anhang:C} beschrieben wird, gemeint. Es sei bemerkt, dass $\nu$ gleich dem \idx{Schubmodul} $G$ ist.


Da wir in dieser Arbeit immer $\Omega \subset \R^2$ annehmen wollen, wollen wir an dieser Stelle noch zwei Prinzipien zur Behandlung von dreidimensionalen strukturmechanischen Problemen im $\R^2$ vorstellen (vgl. \cite{WriggersFEMSkript} Kapitel 5.4.1 und 5.4.2).

Ist beispielsweise die dritte Richtung dünn gegenüber den anderen zwei, so können wir das Prinzip des \textit{ebenen Spannungszustand} verwenden. Hierbei ist die Spannung in der dritten Richtung $\sigma_{33} = 0$. Dies impliziert jedoch nicht, dass es in dieser Richtung keine Verzerrung geben muss. Aus dem Hooke'schen Materialgesetz \eqref{eq:2.33} errechnen wir nämlich, dass die Verzerrung $\eps_{33}$ abhängig von $\eps_{11}$ und $\eps_{22}$ ist.
\[
	\eps_{33} = -\frac \nu{1-\nu} (\eps_{11}+\eps_{22}) 
\]
Damit lässt sich der Materialtensor $\mcal C$ als Matrix schreiben, wenn wir die \textit{\idx{Voigt-Notation}} bzgl. des Spannungs- und Verzerrungstensors verwenden.


Analog erhalten wir den \textit{ebenen Verzerrungszustand}\index{ebener Verzerrungszustand}, indem wir die Verzerrung in die dritte Richtung $\eps_{33} = 0$ setzen. Anschaulich heißt das, dass die dritte Richtung unendlich ausgedehnt ist. Auch hier ergeben sich  durch Einsetzen von $\eps_{33}=0$ in \eqref{eq:2.33} Abhängigkeiten zwischen den Spannungen:
\[
	\sigma_{33} = \nu(\sigma_{11}+\sigma_{22}) \, ,
\]
was einerseits bedeutet, dass die Spannung in der dritten Richtung nicht zwangsläufig Null sein muss und andererseits uns wieder Einträge aus dem Materialtensor $\mcal C$ eliminieren lässt.



\newpage

%%% Local Variables: 
%%% mode: latex
%%% TeX-master: "Skript"
%%% End: 

\newchapter{Variationsungleichungen}

\section{Ein Hindernisproblem}

\begin{itemize}
\item Hindernisproblem: Auslenkung $u$ einer Membran $\Omega$ unter Krafteinwirkung $f$, wobei die Membran durch ein Hindernis $\psi$ behindert wird. Mathematische modelliert bedeutet dies:
\begin{align}\label{eq:Hindernis}
\min_{v\in K} J(v) = \frac 1 2 a(v,v)-(f,v)
\end{align}
mit $K := \{v \in H^1_0(\Omega) \with v\ge \psi$ fast überall in $\Omega\}$. 

\item $J$ gibt wieder die in der Membran gespeicherte Energie an.
\item wobei jetzt die Lösung nicht auf ganz $H^1_0(\Omega)$ gesucht ist, sondern in einer Teilmenge $K \subset H^1_0(\Omega)$.

\item wir können auch hier eine Variationsformulierung, die äquivalent zu \eqref{eq:Hindernis} ist, herleiten

\item zu Beginn noch eine Skizze von einem Hindernisproblem

%\item hierfür zunächst wieder etwas allgemeineres Energiefunktional $J: H \ra \R, J(v) = \frac 1 2 a(v,v)-F(v)$, wobei $H$ Hilbertraum
\end{itemize}


\subsection{Variationsformulierung für das Hindernisproblem}

\begin{itemize}
\item zeigen zunächst, dass $K$ konvex und abgeschlossen ist.
\begin{lemma}
Die Menge $K =  \{v \in H^1_0(\Omega) \with v\ge \psi\text{ fast überall in }\Omega\}$ ist eine konvexe abgeschlossene Teilmenge von $H^1_0(\Omega)$.
\end{lemma}

\begin{proof}
(i) Es seien $u,v \in K$, d.h. $u \ge \psi$ und $v \ge \psi$ fast überall in $\Omega$. Dann gilt für $t \in [0,1]$
\begin{align*}
	(1-t) u + tv \ge (1-t)\psi + t \psi = \psi  \, ,
\end{align*}
somit ist $(1-t)u + tv \in K$, also $K$ konvex.

(ii) abgeschlossen fehlt noch
\end{proof}

\item \begin{satz}\label{satz:3.2}
Es sei $K =   \{v \in H^1_0(\Omega) \with v\ge \psi\text{ fast überall in }\Omega\}$. Das Minimierungsproblem
\begin{align}\label{eq:3.2}
	\min_{v\in K} J(v) = \frac 1 2 a(v,v)-(f,v)
\end{align}
ist äquivalent zur Variationsungleichung: Finde $u \in K$, so dass
\begin{align}\label{eq:3.3}
	a(u,v-u) \ge (f,v-u) \quad \forall \, v \in K \, .
\end{align}
\end{satz}

\begin{proof}
Aus Lemma \ref{lem:2.3} folgt, dass $J$ konvex ist und damit gilt mit Satz \ref{satz:A.10}, dass $u \in K$ genau dann eine Lösung von \eqref{eq:3.2} ist, wenn
\begin{align}\label{eq:3.4}
	\mscr D_{v-u} J(u) \ge 0 \quad \forall \, v \in K 
\end{align}
gilt. Analog zu der berechneten Gâteaux-Ableitung von $J$ in Lemma \ref{lem:2.4}, gilt
\[
	\mscr D_{v-u} J(u) = \frac d{dt} J(u+t(v-u))\Big|_{t=0} = a(u,v-u) -(f,v-u)
\]
und damit folgt mit \eqref{eq:3.4} die Behauptung.
\end{proof}

\item \begin{bem}
Wie man mit Satz \ref{satz:A.10} sehen kann, gilt analog zu Satz \ref{satz:3.2} auch allgemeiner: Es sei $K\subset H$ eine konvexe Teilmenge eines Hilbertraumes $H$. Dann ist
\begin{align*}
	\min_{v\in K} J(v) = \frac 1 2 a(v,v)-F(v)
\end{align*}
äquivalent zur Variationsungleichung: Finde $u \in K$, so dass
\begin{align*}
	a(u,v-u) \ge F(v-u) \quad \forall \, v \in K \, ,
\end{align*}
wobei $F :H\ra \R$ eine lineare stetige Abbildung ist.
\end{bem}

\item auch für das Hindernisproblem gibt es analog zum homogenen Dirichlet-Problem \eqref{eq:2.2} eine äquivalente starke Formulierung

\item \begin{satz}\textnormal{(Starke Formulierung des Hindernisproblems)} Jede Lösung $u \in H^2(\Omega) \cap H^1_0(\Omega)$ des Problems
\begin{align}\label{eq:3.5}
\begin{aligned}
	-\Delta u -f&\ge 0 \\
	u-\psi &\ge 0 \\
	(u-\psi) (-\Delta u &- f) = 0
\end{aligned}
\end{align}
mit $\psi \in H^1(\Omega)$ erfüllt die Variationsungleichung \eqref{eq:3.3}. Umgekehrt ist jede Lösung $u\in H^2(\Omega) \cap K$ von \eqref{eq:3.3} auch eine Lösung von \eqref{eq:3.5}.
\end{satz}

\begin{proof}
"`$\Ra$"' Sei $u \in H^2(\Omega) \cap H^1_0(\Omega)$ eine Lösung von \eqref{eq:3.5}, dann gilt für ein beliebiges $v \in K$
\begin{align*}
	\int_\Omega (-\Delta u - f) (v-u) \, dx & = \underbrace{- \int_\Omega \Delta u  (v-u) \, dx}_{\parbox{3.5cm}{\scriptsize$\stackrel{\text{Green}}= \int_\Omega \nabla u \nabla (v-u) \, dx$ \\ \text{ }\text{ } \text{ }\text{ } $ -\int_\Gamma \underbrace{(v-u)}_{=0} \partial_\nu u \, ds$}} - \int_\Omega  f (v-u) \, dx  \\
	& = \int_\Omega \nabla u \nabla(v-u) \, dx - \int_\Omega f(v-u) \\
	& = a(u,v-u) - (f,v-u) \, .
\end{align*}
Mit $\Omega_0 := \{ x \in \Omega \with u = \psi\}$ folgt, dass $-\Delta u = f$ auf $\Omega_1 := \Omega \setminus \bar\Omega_0$ gelten muss.
\begin{align*}
	 \lra\,   \int_{\Omega = \Omega_0 \cup \Omega_1} \underbrace{(-\Delta u - f)}_{=0 \text{ auf } \Omega_1} (v-u) \, dx  = \int_{\Omega_0}\underbrace{ (-\Delta u - f)}_{\ge0} \underbrace{(v-\psi)}_{\ge 0} \, dx \ge 0
\end{align*}
Damit ist $u$ eine Lösung von \eqref{eq:3.3}
\[
	a(u,v-u) \ge (f,v-u) \quad \forall \, v \in K\, .
\]
"`$\La$"' Es sei $u \in H^2(\Omega) \cap K$ Lösung von \eqref{eq:3.3}. Weiter sei $v \in K$  beliebig, dann gilt
\begin{align}\label{eq:3.6}
\begin{aligned}
	0 & \le a(u,v-u) - (f,v-u) \\
	&= \int_\Omega \nabla u \nabla(v-u) \, dx - \int_\Omega f(v-u) \, dx \\
	& \stackrel{\scriptsize\text{Green}}= \int_\Omega -\Delta u (v-u) \, dx - \int_\Omega f(v-u) \, dx \\
	& = \int_\Omega (-\Delta u - f) (v-u) \, dx \, .
\end{aligned}
\end{align}
Wir nehmen an, dass $-\Delta u -f < 0$ in einem Ball $B_{r_0} := B_{r_0} (x_0)\subset \Omega$ mit Radius $r_0$ um $x_0 \in \Omega$ gilt. Sei weiter $\chi \in C^\infty(\Omega)$ mit $\chi = 0$ auf $\Omega \setminus \bar B_{r_0}, \rho(r) := \(1-\frac r{r_0}\)^2 \chi >0$ und $v := u + \rho (r) \in K$, da $u\in K$ und $\rho (r) >0$. Dann gilt
\[
	\int_\Omega (-\Delta u - f) (v-u) \, dx = \int_{B_{r_0}} \underbrace{(-\Delta u - f)}_{< 0} \underbrace{\rho(r)}_{>0} \, dx < 0 \, ,
\]
was im Widerspruch zu \eqref{eq:3.6} steht. Also muss $-\Delta u - f \ge 0$ gelten.

Nun nehmen wir an, dass $-\Delta u -f > 0$ und $u > \psi$ fast überall in einem Ball $B_{r_0}$ gilt. Wir betrachten $v:= u  + \eps \rho(r) (\psi - u) \in K$ mit $0< \eps\le 1$, dann folgt
\[
	\int_\Omega (-\Delta u - f) (v-u) \, dx = \eps \int_{B_{r_0}} \underbrace{(-\Delta u -f)}_{>0} \underbrace{\rho(r)}_{>0} \underbrace{(\psi-u)}_{<0} \, dx < 0 \, ,
\]
was wiederum im Widerspruch zu \eqref{eq:3.6} steht. Damit muss $u = \psi$ gelten, wenn $-\Delta u = f$ ist. Es folgt, dass $u \in H^2(\Omega) \cap K$ eine Lösung von \eqref{eq:3.5} ist.
\end{proof}
\end{itemize}

\subsection{Existenz und Eindeutigkeit der Lösung}

\begin{itemize}
\item Kapitel 3 in \cite{KikOden} mit Theorem 3.1-3.4 (\textbf{Beweis vgl. NPDE I von Stephan Seite 39}, auch in Solution of Variational Inequalities in Mechanics (Theorem 1.1 Seite 4))

\item für die Existenz und Eindeutigkeit der Lösung des Problems betrachten wir zunächst wieder das allgemeine reelle quadratische Funktional $J: H \ra \R, J(v) = \frac 1 2 a(v,v) - F(v)$.

\item \begin{vor}
Sei $H$ ein reeller Hilbertraum mit Skalarprodukt $(\cdot,\cdot)_H$ und der damit induzierten Norm $\norm\cdot_H$. Mit $H'$ bezeichnen wir den Dualraum zu $H$. Weiter sei vorausgesetzt:
\begin{enumerate}[(a)]
\item $a: H\times H \ra \R$ ist eine stetige koerzive Bilinearform,
\item $F:H\ra\R$ ist ein stetiges lineares Funktional,
\item $K\not = \emptyset$ ist eine abgeschlossene konvexe Teilmenge von $H$.
\end{enumerate}
\end{vor}

\item \begin{theorem}\textnormal{(Existenz und Eindeutigkeit)}
Unter den obigen Voraussetzungen hat die Variationsungleichung, finde $u\in K$, so dass
\begin{align}\label{eq:3.7}
	a(u,v-u) \ge F(v-u) \quad \forall \, v \in K
\end{align}
ist, genau eine Lösung.
\end{theorem}

\begin{proof}
(i) Eindeutigkeit: Es seien $u_1,u_2 \in K$ zwei Lösungen der Variationsungleichung \eqref{eq:3.7}, d.h.
\begin{align}\label{eq:3.8}
	a(u_1,v-u_1) \ge F(v-u_1) \quad \forall \, v \in K\, , \\
	a(u_2,v-u_2) \ge F(v-u_2) \quad \forall \, v \in K\, . \label{eq:3.9}
\end{align}
Addieren wir \eqref{eq:3.8} und \eqref{eq:3.9} miteinander und setzen zuvor $v = u_2$ in \eqref{eq:3.8} und $v = u_1$ in \eqref{eq:3.9}, so erhalten wir
\begin{align*}
	0 & \le a(u_1,u_2-u_1) - F(u_2-u_1) + a(u_2,u_1-u_2) \underbrace{- F(u_1-u_2)}_{=F(u_2-u_1)}  \\
	& = a(u_1,u_2-u_1)-a(u_2,u_2-u_1) = -a(u_2-u_1,u_2-u_1) \\
	& \le -\alpha \norm{u_2-u_1}_H^2 \, .
\end{align*}
Also gilt $\norm{u_2-u_1}_H^2 \le 0 \Ra \norm{u_2-u_1}_H^2 = 0$ und damit folgt $u_1 = u_2$.

(ii) Existenz: Aus dem Darstellungssatz von Riesz bzw. das Korollar \ref{kor:2.14} folgt, dass ein $A \in \mcal L(H,H), l \in H$ existiert, so dass
\begin{align*}
	a(u,v) &= (Au,v)_H \quad \forall \, u,v \in H\, , \\
	F(v) &= (l,v)_H \qquad \forall \, v \in H \, .
\end{align*}
Damit gilt
\begin{align*}
	  F(v-u) - a(u,v-u) &= (l,v-u)_H - (Au,v-u)_H \\
	 &=   (l-Au,v-u)_H \le 0 \, .
\end{align*}
Durch Multiplikation mit $\varrho > 0$ und Addition der Null erhalten wir das äquivalente Problem: Finde $u \in K$, so dass
\begin{align}
	(u-\varrho(Au-l)-u,v-u)_H \le 0 \quad \forall \, v \in K \, .
\end{align}
Nach Satz \ref{satz:2.3} ist $u$ damit das Bild der Projektion von $u-\varrho (Au-l)$ auf $K$, d.h.
\[
	u = P_K (u-\varrho (Au-l)) \,.
\]
Es bleibt zu zeigen, dass $W_\varrho : H \ra K, W_\varrho (v) \coloneqq P_K(v-\varrho (Av-l))$ einen Fixpunkt besitzt. Mit Anwendung von Satz \ref{satz:2.4} und der Koerzivität von $a$ rechnen wir nach, dass
\begin{align*}
	\norm{W_\varrho (v_1) - W_\varrho (v_2)}_H^2  & = \norm{ P_K (v_1-\varrho (Av_1-l))- P_K (v_2-\varrho (Av_2-l))  }_H^2 \\
	& \le \norm{v_1-\varrho (Av_1-l)- (v_2-\varrho (Av_2-l))  }_H^2  \\
	& = \norm{(v_1-v_2)-\varrho \, A(v_1-v_2)  }_H^2  \\
	& = \norm{v_1-v_2}_H^2 + \varrho^2 \norm{A(v_1-v_2)}^2_H \\
	& \ \ \, - \underbrace{ \varrho\, (A(v_1-v_2),v_1-v_2)_H - \varrho\, (v_1-v_2,A(v_1-v_2))_H}_{=2\varrho\, (A(v_1-v_2),v_1-v_2)_H  =2\varrho\, a(v_1-v_2,v_1-v_2)} \\
	& \le \norm{v_1-v_2}_H^2 + \varrho^2\, \norm A^2 \norm{v_1-v_2}_H^2 - 2\varrho\alpha \, \norm{v_1-v_2}_H^2 \\
	& = (1-2\varrho \alpha+\varrho^2 \, \norm A^2) \,\norm{v_1-v_2}_H^2
\end{align*}
mit $\norm A := \sup_{v \in H} \frac{\norm{Av}_H}{\norm v_H}$. Also ist die Abbildung $W_\varrho$  eine Kontraktion, wenn gilt
\begin{align*}
	1-2\varrho \alpha+\varrho^2 \, \norm A^2 < 1 \, \lra\, 0 < \varrho < \frac {2\alpha}{\norm A^2} \, .
\end{align*}
Nach dem Banach'scher Fixpunktsatz (vgl. \cite{Stoer} Satz 5.2.3) existiert für solch ein $\varrho$ ein $u \in H$ mit $u = W_\varrho (u) = P_K(u-\varrho (Au-l))$.

Insgesamt gibt es also für das Problem \eqref{eq:3.7} genau eine  Lösung.
\end{proof}

\item \begin{kor}
Das Problem \eqref{eq:Hindernis} hat eine eindeutige Lösung.
\end{kor}

\begin{proof}

\end{proof}
\end{itemize}







\subsection{Lösung des Hindernisproblems mittels FEM}

\begin{itemize}
\item Analog zum vorherigen Kapitel kann man auch im $\R^n$ Existenz und Eindeutigkeit der Lösung unter bestimmten Voraussetzungen zeigen. (vgl. Vug Skript Kapitel 2) $\Ra$ Beachte hierfür auch den Fixpunktsatz von Brouwer.
\end{itemize}


\section{Kontaktprobleme}

\subsection{Mathematische Modellierung von Kontaktproblemen}

\begin{itemize}
\item Starke Formulierung (s. Wriggers Paper) für Kontaktproblem mit Signorini-Kontakt (ohne Reibung).
\begin{align}
\div \bs \sigma + \bs b &= \bs 0 \text{ in } \Omega\\
\bs \sigma  - \mcal C \bs \eps & = \bs 0 \text{ in } \Omega\\
\bs \sigma \cdot \bs n &= \bs t  \text{ auf } \Gamma_N \\
\bs u &= \bs 0 \text{ auf } \Gamma_D \\
(\bs u \circ \chi - \bs u) \cdot \bs n_c + g& \ge 0 \text{ auf } \Gamma_C
\end{align}
sowie auf $\Gamma_C$ muss $\sigma_n \le 0$ (Normalenkraft $\sigma_n = \bs n\cdot ( \bs \sigma \cdot \bs n)$), $\bs \sigma_t = \bs 0$ (keine Tangentialkraft, da keine Reibung – $\bs \sigma_t = \bs \sigma \cdot \bs n - \sigma_n \bs n$) und $((\bs u \circ \chi - \bs u) \cdot \bs n_c + g)\sigma_n = 0$, d.h. wenn kein Kontakt ist, ist die Normalkraft in den Punkten Null, also herrscht Kräftegleichgewicht.
\item Anreißen von Kontaktproblem mit Tresca-Reibung (vgl. Numerik für Kontaktmechanik von Stephan und Vug von Starke) $\Ra$ Herleitung der Variationsungleichung durch Ableitung nicht mehr möglich, da Reibungspotential nicht mehr differenzierbar.
\end{itemize}

\subsection{Variationsformulierung für Kontaktprobleme}

\begin{itemize}
\item Minimierung von Energiefunktional (vgl. \cite{KikOden} Seite 112 unten) mit $\boldsymbol{u}: \Omega\ra \R^3$:
\begin{align*}
	E(u)& = \frac 1 2 a(u,u)-f(u) \text{ mit } \\
	 a(u,u) &= \int_\Omega {\mcal C} \bs\eps (\bs u): \bs\eps(\bs u) \, d\Omega , \, f(u) = \int_\Omega \bs b \cdot \bs u \, d\Omega + \int_{\Gamma_N} \bs t \cdot \bs u \, d\Gamma
\end{align*}
unter der Nebenbedingung $\bs n \cdot \bs u - g \le 0$ auf $\Gamma_C$  (siehe Vug Skript), bzw. $(\bs u\circ \chi - \bs u)\cdot \bs n_c + g \ge 0$ auf $\Gamma_C$ (etwas allgemeiner, vgl. Wriggers Paper).
\item Herleitung auch über starke Formulierung möglich, vgl. Stephan – Kontaktprobleme.
\item Herleitung der Variationsformulierung: Finde $\bs u \in K$: $a(\bs u,\bs v-\bs u) \ge f(\bs v-\bs u) \, \forall \, \bs v \in K$ (s. auch Wriggers Paper) analog zum Hindernisproblem (nicht mehr ausführlich, wenn oben schon ausführlich).
\item \cite{KikOden} Seite 113 für Bedingung für die Eindeutigkeit und Existenz der Lösung des Problems (hierfür wird Korn's Ungleichung benötigt $\Ra$ vielleicht Anhang?).
\end{itemize}

\subsection{Lösung des Kontaktproblems mittels FEM}

\begin{itemize}
\item Beschreibe das diskrete Problem, was man bekommt mit: Finde $\bs x^* \in \R^N$ mit $B\bs x^* \ge c$, so dass
\begin{align*}
	(A\bs x^* - \bs b)^T (\bs x - \bs x^*) \ge 0 \, \forall \bs x\in \R^N \text{ mit } B\bs x \ge \bs c \, ,
\end{align*}
wobei
\begin{align*}
A &= \left[\int_\Omega \mcal C \bs \eps (\bs \Psi_j):\bs \eps(\bs \Psi_i) \, d\Omega\right]_{1 \le i,j\le N} , \,  \bs b = \left[ \int_\Omega \bs b \cdot \bs\Psi_i \, d\Omega + \int_{\Gamma_N} \bs t \cdot \bs\Psi_i \, ds\right]_{1\le i \le N} \\
B & = [(\bs \Psi_j(\chi(\bs x_i))-\bs \Psi_j(\bs x_i))\cdot \bs n_c(\bs x_i)]_{\bs x_i \in \Gamma_c, 1\le j \le N} , \, c = [-g(\bs x_i)]_{\bs x_i \in \Gamma_c}
\end{align*}
Dieses Problem ist (wie vorher schon gezeigt) äquivalent zu einem quadratischen Problem
\begin{align*}
\min_{\bs x\in \R^N} \frac 1 2 \bs x^T A \bs x - \bs b^T \bs x \text{ s.t. } B\bs x \ge \bs c \, ,
\end{align*}
d.h. Lösbarkeit des quadratischen Programms sollte auch gezeigt sein (vgl. Vug Skript oder auch nichtlineare Optimierung).
\end{itemize}


\newpage

%%% Local Variables: 
%%% mode: latex
%%% TeX-master: "Skript"
%%% End: 

\newchapter{Ein hierarchischer Fehlerschätzer für Hindernisprobleme}
\label{kap:4}

\begin{itemize}
\item Vergleich Hindernisprobleme zu Kontaktproblemen $\ra$ warum gerade dieser Fehlerschätzer bei Hindernis- bzw. Kontaktproblemen
\end{itemize}

Dieses Kapitel basiert größtenteils auf \cite{ZouVee}.


\section{Herleitung eines a posteriori hierarchischen Fehlerschätzers}
\label{kap:4.1}

\begin{itemize}
\item der Einfachheit halber gehen wir von folgendem Sachverhalt aus
\begin{vor}\label{vor:4.1}
Das Hindernis  wird durch eine stückweise lineare stetige Funktion $\psi$ beschrieben.
\end{vor}

\item nicht nichtstetige oder auch glatte Hindernisse sind analoge Aussagen, aber schwerer, beweisbar
\end{itemize}






\subsection{Diskretisierung}
\label{kap:4.1.1}

\begin{itemize}
\item $\mcal B_h$ sei eine nodale Basis bzgl. einer quasi-uniformen Triangulierung $\mcal T_h$  für $\mcal S_h$ (s. auch Kapitel \ref{kap:2}), $K_h$ wie in Kapitel \ref{kap:3.1.3}
\[
	K_h  = \{v_h \in \mcal S_h \with v_h (p) \ge \psi(p) \, \forall \, p \in \mcal N \cap \Omega\} \, , 
\]
wobei $\mcal N$ wieder die Menge der Knoten von $\mcal T_h$ darstellt.

\item betrachte wieder die diskrete Variationsungleichung \eqref{eq:3.12}: Finde $u_h \in K_h$ mit
\[
	a(u_h,v_h-u_h) \ge (f,v_h-u_h) \quad \forall \, v_h \in K_h\, .
\]

\item oder äquivalent die Minimierung des Funktionals $J(v) = \frac 1 2 a(v,v)-(f,v)$ über $K_h$, d.h.
\begin{align}\label{eq:4.1}
	u_h \in K_h :\quad J(u_h)\le J(v_h) \quad \forall \, v_h \in K_h
\end{align}

\item wegen der Voraussetzung, dass $\psi$ stückweise linear ist, gilt $K_h \subset K$, da die linearen Ansatzfunktionen nicht nur punktuell, sondern auch kontinuierlich die Nebenbedingung erfüllen

\item damit ist \eqref{eq:3.12} eine konforme FEM $\ra$ nichtkonforme wollen wir hier nicht betrachten (s. bel. stetige Hindernisse)

\item wir wollen einen a posteriori Fehlerschätzer für den Fehler bzgl. der Funktionswerte der Funktionale $J(u),J(u_h)$ herleiten. Hierbei gilt $J(u_h) - J(u) \ge 0$, denn aus den beiden Minimierungsproblemen über $K$ und $K_h$ folgt
\[
	J(u) \le J(v) \, \forall \, v \in K \, , \quad J(u_h) \le J(v_h) \, \forall \, v_h \in K_h \, .
\]
Da $K_h \subset K$ gilt, folgt auch $J(u)\le J(v_h)$ für alle $v_h \in K_h$. Setze $v_h = u_h$, so gilt
\[
	J(u) \le J(u_h) \Llra J(u_h)-J(u) \ge 0
\]
\item 
\begin{bem}
Gilt $\psi = -\infty$, d.h. ist kein Hindernis vorhanden, so folgt
\begin{align*}
	J(u_h)-J(u)  =& \frac 1 2 a(u_h,u_h)-(f,u_h) - \(\frac 1 2 a(u,u)-(f,u)\) \\
	 =& \frac 1 2 a(u_h,u_h)-(f,u_h) - \frac 1 2 a(u,u)+(f,u) \\
	 & +\overbrace{(a(u,u-u_h)-\underbrace{(f,u-u_h)}_{=(f,u)-(f,u_h)})}^{=0} \\
	 = & \frac 1 2a(u_h,u_h) - \frac 1 2 a(u,u) + a(u,u-u_h) \\
	 = & \frac 1 2 a(u_h,u_h) - \frac 1 2a(u,u)+a(u,u) - a(u,u_h) \\
	 = & \frac1 2 (a(u_h,u_h)+ a(u,u) - 2 a(u,u_h)) \\
	 = & \frac 1 2 a(u_h-u,u_h - u) = \frac 1 2 \norm{u_h-u}^2_E \, .
\end{align*}
Ist nun ein $\psi > -\infty$ gegeben, dann addieren wir im zweiten Schritt nicht mehr Null, sondern es gilt für den Term
\[
	a(u,u-u_h)-(f,u-u_h) \le 0
\]
und damit gilt $J(u_h)-J(u) \ge \frac 1 2\norm{u_h-u}_E^2$, d.h. eine obere Schranke des Fehlers im Funktional schätzt auch den Fehler zwischen exakter und approximierter Lösung in der Energienorm ab.
\end{bem}

\item Herleitung eines hierarchischen a posteriori Fehlerschätzers:

\item 
\begin{notation}
Um im Folgenden den hierarchischen Split leichter beschreiben zu können, schreiben wir für die Galerkin-Lösung $u_h$ die Notation $u_{\mcal S}$, um auszudrücken, dass diese im linearen Ansatzraum $\mcal S_h$ liegt. Analog sind die im Weiteren übrigen verwendeten Indizes zu verstehen.
\end{notation}

\item wir führen Fehlerfunktion $e= u-u_{\mcal S}$ ein

\item weiter sei $\mcal I(v) = \frac 12 a(v,v)-\rho_{\mcal S}(v)$ mit $\rho_{\mcal S} (v) = (f,v)-a(u_{\mcal S},v), v \in H^1_0(\Omega)$.

\item
\begin{bem}
\begin{enumerate}[(a)]
\item Die Linearform $\rho_{\mcal S}$ stellt das Residuum der Variationsgleichung (d.h. ohne Hindernis) dar.
\item Nach dem Darstellungssatz von Riesz existiert ein $v^* \in H^1_0(\Omega)$, so dass
\[
	(v^*,v)_1 = \rho_{\mcal S} (v) \quad\forall \, v \in H^1_0(\Omega)
\]
ist. Wir können also $v^*$ als Lagrange-Multiplikator bzgl. der Nebenbedingung $v \ge \psi$ interpretieren.
\end{enumerate}
\end{bem}

\item neues Minimierungsproblem, jetzt für den Fehler $e$.
\begin{satz}[Lösung des Defektproblems]
Mit den obigen Bezeichnungen löst die Fehlerfunktion $e$ folgendes Defektproblem:
\begin{align}\label{eq:4.2}
	e \in \mcal A:\quad  \mcal I(e) \le \mcal I(v) \quad \forall \, v \in \mcal A \, ,
\end{align}
wobei $\mcal A \coloneqq \{v \in H^1_0(\Omega) \with v \ge \psi-u_{\mcal S}\} = -u_{\mcal S} + K$.
\end{satz}

\begin{proof}
Es sei $u$ die Lösung von \eqref{eq:3.2} und $u_{\mcal S}$ die Lösung von \eqref{eq:4.1}. Dann gilt
\begin{align*}\tag{$\ast$}
	 u \in K : \quad\qquad\quad \,\, \, \, \,  J(u)& \le J(\tilde v )\, \, \, \, \,  \qquad\qquad \forall \, \tilde v \in K \\
	\Llra u \in K : \quad J(u)-J(u_{\mcal S})& \le J(\tilde v) - J(u_{\mcal S}) \quad \forall \, \tilde v \in K \, .
\end{align*}
Wir rechnen für die linke Seite nach, dass gilt
\begin{align*}
	J(u)-J(u_{\mcal S}) & = \frac 1 2 a(u,u) - (f,u) -\(\frac 1 2 a(u_{\mcal S},u_{\mcal S}) - (f,u_{\mcal S})\) \\
	& = \frac 1 2 a(u,u)  + \frac 1 2 a(u_{\mcal S},u_{\mcal S}) - a(u_{\mcal S},u_{\mcal S}) - (f,u-u_{\mcal S}) \\
	& = \frac 1 2 a(u,u)  + \frac 1 2 a(u_{\mcal S},u_{\mcal S}) - a(u_{\mcal S},u) - ((f,u-u_{\mcal S})- a(u_{\mcal S},u-u_{\mcal S})) \\
	& = \frac 1 2a(u-u_{\mcal S},u-u_{\mcal S})- \rho_{\mcal S} (u-u_{\mcal S}) \\
	& = \frac 1 2 a(e,e)-\rho_{\mcal S} (e) = \mcal I(e) \, .
\end{align*}
Analog gilt für die rechte Seite $J(\tilde v ) - J(u_{\mcal S}) = \mcal I(\tilde v - u_{\mcal S})$. Mit $v \coloneqq \tilde v - u_{\mcal S}$ gilt $v \in \mcal A$ und damit ist ($*$) äquivalent zu: Finde $e \in \mcal A$, so dass
\[
	\mcal I(e) \le \mcal I(v) \quad \forall \, v \in\mcal A \, . \qedhere
\]
\end{proof}

\item
\begin{kor}
Das Problem \eqref{eq:4.2} ist äquivalent zur Variationsungleichung: Finde $e \in \mcal A$ mit
\begin{align}\label{eq:4.3}
	a(e,v-e) \ge \rho_{\mcal S} (v-e) \quad \forall \, v \in \mcal A \, .
\end{align}
\end{kor}

\begin{proof}
Analog zu Lemma \ref{lem:3.1} lässt sich zeigen, dass $\mcal A$ abgeschlossen und konvex ist. Mit Satz \ref{satz:A.10} folgt dann die Behauptung.
\end{proof}

\item da $\psi$ stückweise linear ist, liegt $0 \in \mcal A$, d.h. das "`gewünschte"' Ergebnis für $e$ liegt im betrachteten Raum

\item
\begin{bem*}
Wir werden noch zeigen, dass $\rho_{\mcal S}$ eine Schlüsselgröße für die a posteriori Abschätzung darstellt.
\end{bem*}

\item a posteriori Schätzer in 2 Schritten
\begin{enumerate}[(i)]
\item	diskreditiere \eqref{eq:4.3} bzgl. einer Erweiterung von $\mcal S_h$ (hier quadratische Funktionen), so dass $e$ hinreichend genau approximiert wird.
\item Aufteilung des neuen Raumes, sodass \eqref{eq:4.3} lokal in der Erweiterung exakt gelöst werden kann
\end{enumerate}

\item als Erweiterung von $\mcal S_h$ betrachten wir einen Raum $\mcal Q_h$ mit $\mcal S_h \subset \mcal Q_h$.

\item hier bietet sich an: $\mcal Q_h \coloneqq \{v \in C^0(\Omega) \with v|_T \in \mcal P_2 \text{ für } T\in \mcal T_h, v|_{\partial\Omega} = 0\}$, also der Raum der quadratischen Spline über einer quasi-uniformen Zerlegung $\mcal T_h$.

\item damit definiere $\mcal N_{\mcal Q} \coloneqq \mcal N \cup \{x_E \with E \in \mcal E\}$, wobei $x_E$ den Mittelpunkt der Kante $E$ darstellt und $\mcal E$ somit die Menge aller Kanten ist.

\item damit ergibt sich $\mcal A$ über $\mcal Q$ diskret als
\begin{align}\label{eq:4.4}
	\mcal A_{\mcal Q} \coloneqq \{v \in \mcal Q_h \with v(p) \ge \psi(p)-u_{\mcal S}(p) \, \forall \, p \in \mcal N_{\mcal Q} \cap \Omega\}
\end{align}

\item im Bezug zu \eqref{eq:4.4} ergibt sich dann das diskrete Defektproblem
\begin{align}\label{eq:4.5}
	e_{\mcal Q} \in \mcal A_{\mcal Q} : \quad a(e_{\mcal Q}, v-e_{\mcal Q}) \ge \rho_{\mcal S}(v-e_{\mcal Q}) \quad \forall \, v \in \mcal A_{\mcal Q}
\end{align}

\item 
\begin{bem}
Im Allgemeinen gilt hierbei nicht $\mcal A_{\mcal Q} \subset \mcal A$. So kann man sich anschaulich eine quadratische Funktion $v_{\mcal Q} \in \mcal A_{\mcal Q}$ vorstellen, die allerdings zwischen den übereinstimmenden Werten aufgrund ihrer Krümmung das lineare Hindernis aus $\mcal A$ durchdringt.

\begin{figure}[h]
\begin{center}
	\begin{pspicture}(0,0)(5,3)
		% Koordinatensystem:
		\psaxes[linewidth=0.65pt]{->}(0,0)(-0.3,-0.3)(6.7,2.7)
		\rput(6.7,-0.2){$x$}
		\rput(-0.2,2.7){$y$}
		
		% Das Hindernis:
		\psline[linewidth=0.6pt](1,0)(2,1)
		\psline[linewidth=0.6pt](2,1)(4,1.5)
		\psline[linewidth=0.6pt](4,1.5)(5,0)
		\rput(4.5,0.3){$\psi$}
		
		\psdots(2,1)(3,1.25)(4,1.5)
		
		% Die Funktion v:
		\psline[linestyle=dashed,linewidth=0.6pt](0,0)(2,1.5)
		\pscurve[linestyle=dashed,linewidth=0.6pt](2,1.5)(3,1.25)(4,1.5)
		\psline[linestyle=dashed,linewidth=0.6pt](4,1.5)(6,0)
		\rput(0.5,0.7){$v$}
	\end{pspicture}
\end{center}
\caption{Beispiel eines affinen Hindernisses $\psi$ mit $v \in \mcal A_{\mcal Q}$ in $\R$}
\end{figure}
\end{bem}

\item hierarchische Aufteilung von $\mcal Q_h$ durch $\mcal Q_h = \mcal S_h \oplus \mcal V_h$, wobei $\mcal V_h \coloneqq \{\phi_E \with E \in \mcal E\}$ ist und $\phi_E$ die \textit{\idx{Bubble-Funktion}} mit
\[
	\phi_E (p) = \delta_{x_E,p} = \begin{cases}
								1, & p = x_E \\
								0, & \text{sonst}
							\end{cases}
\]
ist

\item
\begin{bsp}
allgemeine Skizze und die drei bubble Funktionen auf einem Referenzdreieck
\end{bsp}

\item
\begin{satz}
Mit den oben verwendeten Notationen gilt $\mcal Q_h = \mcal S_h \oplus \mcal V_h$.
\end{satz}

\begin{proof}
Wir zeigen, dass $\mcal Q_h = \mcal S_h \oplus \mcal V_h$ auf dem Referenzdreieck gilt und damit gilt es auch für beliebige Dreiecke $T \in \mcal T_h$, da ein allgemeines Dreieck $T$ aus dem Referenzelement $\tilde T$ durch affine Transformation hervorgeht.

Auf dem Referenzelement $\tilde T$ ist $\{\phi_1,\phi_2,\phi_3\}$ eine Basis von $\mcal S_h$ mit
\[
	\phi_1(\xi,\eta) = 1-\xi-\eta \, , \quad \phi_2(\xi,\eta) = \xi \, , \quad \phi_3(\xi,\eta) = \eta
\]
und $\{\phi_4,\phi_5,\phi_6\}$ eine Basis von $\mcal V_h$ mit
\[
	\phi_1(\xi,\eta) = 4\xi (1-\xi-\eta) \, , \quad \phi_2(\xi,\eta) = 4\xi\eta \, , \quad \phi_3(\xi,\eta) = 4\eta(1-\xi-\eta) \, .
\]
Damit ist $\{\phi_1,\ldots,\phi_6\}$ ein Erzeugendensystem von $\mcal Q_h$, da jedes Element
\[
	a_0+a_1\xi+a_2\eta + a_3 \xi^2+ a_4\xi\eta +a_5\eta^2 \in \mcal Q_h
\]
als Linearkombination aus den Funktionen beschrieben werden kann ($\phi_1$ bis $\phi_6$ enthalten alle vorkommenden Summanden eines Polynom 2. Grades). Außerdem ist leicht nachzurechnen, dass die Funktionen $\phi_i,i = 1,\ldots,6,$ linear unabhängig sind und damit gilt
\[
	\mcal Q_h = \operatorname{span} \{\phi_1,\ldots,\phi_6\} \, .
\]
Aus der linearen Unabhängigkeit folgt damit auch $\mcal S_h \cap \mcal V_h = \{0\}$ gilt und damit die Behauptung.
\end{proof}

\item daher kann jedes Element $v_{\mcal Q} \in \mcal Q_h$  als $v_{\mcal Q} = v_{\mcal S} + v_{\mcal V}$ mit $v_{\mcal S} \in \mcal S_h, v_{\mcal V}\in \mcal V_h$ geschrieben werden

\item aus diesem Grund führen wir folgende Bilinearform ein:
\begin{align*}
	a_{\mcal Q} (v,w) \coloneqq a(v_{\mcal S},w_{\mcal S}) + \sum_{E \in \mcal E} u_{\mcal V}(x_E) w_{\mcal V}(x_E) a(\phi_E,\phi_E) \quad \forall \, v,w \in \mcal Q_h \, ,
\end{align*}
welche aufgrund der Eigenschaften der direkten Summe von $\mcal S_h$ und $\mcal V_h$ wohldefiniert ist.

\item dabei ergibt sich $a_{\mcal Q}$ durch Entkopplung von $\mcal S_h$ und $\mcal V_h$ und anschließender "`Diagonalisierung"' auf $\mcal V$

\item sinnvoll $a_{\mcal Q}$ so einzuführen, denn:
\begin{satz}
Die zu $a_{\mcal Q}$ assoziierte Energienorm
\begin{align*}
	\norm v_{\mcal Q}\coloneqq a_{\mcal Q}(v,v)^{\frac 1 2} \, , \quad v \in \mcal Q_h
\end{align*}
ist äquivalent zur Energienorm $\norm\cdot_E$, d.h. es gibt Konstanten $c_1,c_2$ $($die insbesondere nur von der Quasi-Uniformität von $\mcal T_h$ abhängen$)$, so dass
\[
	c_1 \norm v_E \le \norm v_{\mcal Q} \le c_2 \norm v_E \, , \quad \forall \, v \in \mcal Q_h \, .
\]
\end{satz}

\begin{proof}
Die Aussage folgt aus Theorem 4.1 bzw. Bemerkung 4.3 in \cite{HoppeKorn} zusammen mit dem Lemma auf Seite 14 in \cite{Deufl}.
\end{proof}

\item daher führen wir die approximierte Energie
\begin{align}\label{eq:4.6}
	\mcal I_{\mcal Q} (v) \coloneqq \frac 1 2 a_{\mcal Q}(v,v)-\rho_{\mcal S}(v) \, , \quad v \in \mcal Q_h 
\end{align}
ein.

\item das damit verbundene Defektproblem ist allerdings noch durch die Nebenbedingung aus $\mcal A_{\mcal Q}$ mit $\mcal S_h$ gekoppelt und daher noch nicht alleine auf die Raumerweiterung $\mcal V_h$ bezogen.

\item Als Abhilfe ignorieren wir einfach die linearen Beiträge in $\mcal A_{\mcal Q}$ und führen eine echte Teilmenge 
\begin{align}\label{eq:4.7}
	\mcal A_{\mcal V} \coloneqq \{v \in \mcal V \with v(x_E) \ge \psi(x_E)-u_{\mcal S}(x_E) \, \forall \, E \in \mcal E\}
\end{align}
von $\mcal A_{\mcal Q}$ ein.

\item zusammen mit \eqref{eq:4.6} und \eqref{eq:4.7} erhalten wir das lokale diskrete Defektproblem
\begin{align}\label{eq:4.8}
	\eps_{\mcal V} \in \mcal A_{\mcal V} : \quad \mcal I_{\mcal Q}(\eps_{\mcal V}) \le \mcal I_{\mcal Q} (v) \quad \forall \, v \in \mcal A_{\mcal V}
\end{align}
bzw. die dazu äquivalente Variationsungleichung
\begin{align}\label{eq:4.9}
	\eps_{\mcal V} \in \mcal A_{\mcal V} : \quad a_{\mcal Q} (\eps_{\mcal V},v-\eps_{\mcal V})\ge \rho_{\mcal S} (v-\eps_{\mcal V}) \quad \forall \, v \in \mcal A_{\mcal V} \, .
\end{align}

\item
\begin{bem}
\begin{enumerate}[(a)]
\item Da $\psi$ stetig stückweise linear ist und somit $u_{\mcal S} \ge \psi$ gilt, folgt $0 \in \mcal A_{\mcal V}$. Damit ist auch hier die gewünschte Lösung für $\eps_{\mcal V}$ in $\mcal A_{\mcal V}$ enthalten
\item	Auch für $\mcal A_{\mcal V}$ lässt sich mit analogem Vorgehen zu Lemma \ref{lem:3.1} die Konvexität zeigen.
\end{enumerate}
\end{bem}

\item
\begin{lemma}
Das Energiefunktional $\mcal I_{\mcal Q}$ ist konvex.
\end{lemma}

\begin{proof}
Da $a$ eine stetige koerzive Bilinearform, werden aufgrund der Konstruktion von $a_{\mcal Q}$ diese Eigenschaften auch auf $a_{\mcal Q}$ übertragen. Weiterhin ist leicht zu überprüfen, dass $\rho_{\mcal S}$ eine stetige Linearform ist. Dann folgt aus Lemma \ref{lem:2.3} direkt die Behauptung.
\end{proof}

\item Lösung des lokalen Defektproblems
\begin{satz}
Die Lösung von \eqref{eq:4.8} bzw. \eqref{eq:4.9} ist explizit gegeben durch
\begin{align}\label{eq:4.10}
	\eps_{\mcal V} (x_E) = \frac{\max \{-d_E,\rho_E\}}{\norm{\phi_E}} \, 
\end{align}
wobei
\begin{align}\label{eq:4.11}
	d_E = (u_{\mcal S}(x_E) - \psi (x_E))\norm{\phi_E} \ge 0 \, , \quad \rho_E = \frac{\rho_{\mcal S}(\phi_E)}{\norm{\phi_E}} \, .
\end{align}
\end{satz}

\begin{proof}
Es sei $M = \abs{\mcal E}$ die Anzahl der Kanten. Zunächst berechnen wir zur besseren Übersicht $\eps_{\mcal V}(x_E)$ konkret, d.h.
\begin{align}\notag
	\eps_{\mcal V} (x_E) & =  \frac{\max \{-d_E,\rho_E\}}{\norm{\phi_E}} \\
	\notag
	& = \frac{\max \left\{(\psi (x_E)-u_{\mcal S}(x_E) )\norm{\phi_E} ,\frac{\rho_{\mcal S}(\phi_E)}{\norm{\phi_E}}\right\}}{\norm{\phi_E}} \\
	\notag
	& = \max \left\{\psi (x_E)-u_{\mcal S}(x_E)  ,\frac{\rho_{\mcal S}(\phi_E)}{\norm{\phi_E}^2}\right\} \\
	\label{eq:4.12}
	& = \max \left\{\psi (x_E)-u_{\mcal S}(x_E)  ,\frac 1{\norm{\phi_E}^2} ((f,\phi_E)-a(u_{\mcal S},\phi_E))	\right\} \, .
\end{align}
Da  $\eps_{\mcal V} = \sum_{E \in \mcal E} \eps_{\mcal V}(x_E) \phi_E$ ist, können wir \eqref{eq:4.8} bzgl. der Basis $\{\phi_E \with E \in \mcal E\}$ von $\mcal V_h$ diskret schreiben als
\begin{align*}
	\min \frac 1 2 \bs v^T D \bs v - \bs g^T \bs v \quad \text{s.t.} \quad\bs v \ge \bs \psi - \bs u_{\mcal S}\, , 
\end{align*}
wobei $\bs v = [\eps_{\mcal V}(x_{E_i})]_{1\le i \le M}, D = \operatorname{diag}(a(\phi_{E_1},\phi_{E_1}),\ldots,a(\phi_{E_M},\phi_{E_M})), \bs g = [(f,\phi_{E_i})-a(u_{\mcal S},\phi_{E_i})]_{1\le i\le M}, \bs \psi = [\psi(x_{E_i})]_{1\le i \le M}$ und $ \bs u_{\mcal S} = [u_{\mcal S}(x_{E_i})]_{1\le i \le M}$. Da $\mcal A_{\mcal V}$ und $\mcal I_{\mcal Q}$ konvex sind, existiert ein Minimum $\bs v^* \in \mcal A_{\mcal V}$ von $\mcal I_{\mcal Q}$, das die KKT-Bedingungen erfüllt. Damit gilt
\begin{subequations}\label{eq:4.13}
\begin{align}\label{eq:4.13a}
	D \bs v-\bs g - \bs \lambda & = \bs 0 \, , \\
	\label{eq:4.13b}
	\bs \lambda &\ge 0\, ,\\
	\label{eq:4.13c}
	\bs v & \ge \bs \psi - \bs u_{\mcal S}\,  , \\
	\label{eq:4.13d}
	\lambda_i\, (\bs v - \bs \psi+\bs u_{\mcal S})_i &= 0 \quad \forall \, i=1,\ldots,M \, .
\end{align}
\end{subequations}
Es sei $k \in \{1,\ldots,M\}$ beliebig.

\underline{Fall 1:} Gilt $\lambda_k = 0$, so folgt aus \eqref{eq:4.13a}
\[
	\eps_{\mcal V} (x_{E_k}) = v_k = \frac {g_k}{a(\phi_{E_k},\phi_{E_k})} = \frac 1{\norm{\phi_{E_k}}^2} ((f,\phi_{E_k})-a(u_{\mcal S},\phi_{E_k})) \, .
\]
\underline{Fall 2:} Gilt $\lambda_k \not= 0$, dann folgt wegen \eqref{eq:4.13d}
\[
	\eps_{\mcal V} (x_{E_k}) =v_k = (\bs \psi - \bs u_{\mcal S})_k = \psi(x_{E_k}) - u_{\mcal S}(x_{E_k})\, .
\]
Insgesamt folgt mit \eqref{eq:4.13c} und \eqref{eq:4.12} die Behauptung.
\end{proof}

\item wir wollen im weiteren den a posteriori Fehlerschätzer
\[
	-\mcal I_{\mcal Q} (\eps_{\mcal V})= -\frac 1 2 a_{\mcal Q}(\eps_{\mcal V},\eps_{\mcal V}) + \rho_{\mcal S}(\eps_{\mcal V})
\]
betrachten und werden zeigen, dass er äquivalent zu $J(u_{\mcal S}) - J(u)$ ist (vgl. Kapitel \ref{kap:4.1.4} und \ref{kap:4.1.5})

\item zunächst aber Einführung des lokalen Anteils des Fehlerschätzers $-\mcal I_{\mcal Q} (\eps_{\mcal V})$
\end{itemize}





\subsection{Lokaler Anteil des Fehlerschätzers}
\label{kap:4.1.2}

\begin{itemize}
\item 
\begin{notation}
\begin{enumerate}[(a)]
\item Wir schreiben im Folgenden "`$\lesssim$"' statt "`$\le C$"', wenn die Konstante $C$ nur von der Quasi-Uniformität von $\mcal T_h$ abhängt.
\item Weiter schreiben wir "`$A \approx B$"' für "`$A\lesssim B$"' und "`$B \lesssim A$"'.
\end{enumerate}
\end{notation}

\item zunächst zeigen wir ein paar Eigenschaften von der Fehlerfunktion $e = u-u_{\mcal S}$

\item
\begin{lemma}\label{lem:4.12}
Die Fehlerfunktion $e = u-u_{\mcal S}$ erfüllt die Ungleichungen
\begin{align}\label{eq:4.14}
	\frac 12 \norm e^2 \le \frac 12 \rho_{\mcal S}(e) \le -\mcal I (e) \le \rho_{\mcal S}(e) \, .
\end{align}
\end{lemma}

\begin{proof}
Wir erinnern uns, dass
\[
	-\mcal I (e) \coloneqq - \frac 12 \underbrace{a(e,e)}_{\ge 0} + \rho_{\mcal S} (e) \le \rho_{\mcal S}(e) \, , 
\]
da $a$ koerziv ist. Dann gilt weiter
\begin{align*}
	-\mcal I(e) & = - \frac 12 a(e,e) + \rho_{\mcal S} (e) \\
	& = -\frac 1 2 a(u-u_{\mcal S},e) + \rho_{\mcal S} (e) \\
	& = -\frac 1 2 a(u,e) \underbrace{\frac 1 2 a(u_{\mcal S},e)-\frac 1 2(f,e)}_{=-\frac 12 \rho_{\mcal S}(e)} + \frac 1 2 (f,e)+\rho_{\mcal S}(e) \\
	& = -\frac 12 (\underbrace{a(u,u-u_{\mcal S})-(f,u-u_{\mcal S})}_{\le 0}) + \frac 1 2\rho_{\mcal S}(e) \ge \frac 1 2\rho_{\mcal S}(e) \, .
\end{align*}
Es bleibt also die erste Ungleichung von \eqref{eq:4.14} zu zeigen. Wir rechnen nach, dass
\begin{align*}
	\frac 1 2 \rho_{\mcal S} (e) &= \frac 1 2 (f,e)-\frac 1 2a(u_{\mcal S},e) \\
	& = \frac 1 2 (\underbrace{(f,u-u_{\mcal S}) - a(u,u-u_{\mcal S})}_{\ge 0} + a(u-u_{\mcal S},e)) \\
	& \ge \frac 1 2 a(u-u_{\mcal S},e) =\frac 1 2 a(e,e) =  \frac 1 2 \norm e^2 
\end{align*}
gilt, womit \eqref{eq:4.14} insgesamt bewiesen ist.
\end{proof}

\item 
\begin{kor}
Für die Lösungen $e_{\mcal Q}, \eps_{\mcal V}$ von \eqref{eq:4.5} und \eqref{eq:4.9} gilt
\begin{align}\label{eq:4.15}
	\frac 12 \norm{e_{\mcal Q}}^2 \le \frac 12 \rho_{\mcal S}(e_{\mcal Q})& \le -\mcal I (e_{\mcal Q}) \le \rho_{\mcal S}(e_{\mcal Q}) \, ,\\
	\label{eq:4.16}
	\frac 12 \norm {\eps_{\mcal V}}_{\mcal Q}^2 \le \frac 12 \rho_{\mcal S}(\eps_{\mcal V})& \le -\mcal I_{\mcal Q} (\eps_{\mcal V}) \le \rho_{\mcal S}(\eps_{\mcal V})\, .
\end{align}
\end{kor}

\begin{proof}
Da $e_{\mcal Q}$ und $\eps_{\mcal V}$ Lösungen der Variationsungleichungen \eqref{eq:4.5} und \eqref{eq:4.9} sind, folgt die Behauptung analog zum Beweis von Lemma \ref{lem:4.12}.
\end{proof}

\item wegen \eqref{eq:4.16} ist $\rho_{\mcal S} (\eps_{\mcal V})$ äquivalent zum Fehlerschätzer $-\mcal I_{\mcal Q}(\eps_{\mcal V})$ und kann daher als Indikator für $-\mcal I_{\mcal Q} (\eps_{\mcal V})$ verwendet werden (verkleinern wir $\rho_{\mcal S}$, so wird auch $-\mcal I_{\mcal Q}$ kleiner)

\item in Kapitel \ref{kap:4.1.4} und \ref{kap:4.1.5} werden wir die Äquivalenz von $-\mcal I_{\mcal Q}(\eps_{\mcal V})$ zum exakten Fehler in den Funktionalen $J(u_{\mcal S}) - J(u)=-\mcal I(e)$ zeigen

\item damit folgt auch aus Lemma \ref{lem:4.12}, dass der Fehler $J(u_{\mcal S})-J(u)$ äquivalent zu $\rho_{\mcal S}(e)$ ist $\Ra$ daher betrachten wir ein paar weitere Eigenschaften von $\rho_{\mcal S}$.

\item nun zu den lokalen Anteilen von $\rho_{\mcal S}(\eps_{\mcal V})$:

\item es sei $u_{\mcal S}$ die Lösung von \eqref{eq:3.12}, dann auf jedem $T\in \mcal T_h$ die Gleichung $\Delta u_{\mcal S} = 0$, da $u_{\mcal S}$ auf jedem $T$ linear ist.

\item dann gilt mit $\Omega = \bigcup_{T \in \mcal T_h} T$ für alle $v \in H^1(\Omega)$
\begin{align}\notag 
	\rho_{\mcal S} (v) & = (f,v)-a(u_{\mcal S},v) = \int_{\Omega} fv \, d\Omega - \int_{\Omega} \nabla u_{\mcal S} \nabla v \, d\Omega \\
	\notag
	& = \int_{\Omega} fv \, d\Omega - \sum_{T\in \mcal T_h} \int_T \nabla u_{\mcal S}\nabla v \, dT \\
	\notag
	& = \int_{\Omega} fv \, d\Omega - \sum_{T\in \mcal T_h} \(\int_{\partial T} v \partial_{\bs n} u_{\mcal S} \, d\Gamma -  \int_T \underbrace{\Delta u_S}_{=0} v \, dT \) \\
	\label{eq:4.17}
	& = \int_{\Omega} fv \, d\Omega - \sum_{T\in \mcal T_h} \int_{\partial T} v \partial_{\bs n} u_{\mcal S} \, d\Gamma \, ,
\end{align}
wobei im vorletzten Schritt die 1. Green'sche Formel angewendet wurde und d$\bs n$ die äußere Einheitsnormale von $T$ ist.

\item Betrachten wir zwei beliebige Dreiecke $T_1,T_2$ wie in Abbildung \ref{abb:4.2}, wobei $\bs n$ hierbei die Einheitsnormale, die von $T_1$ nach $T_2$ zeigt, bezeichnet, so können wir die Summe aus \eqref{eq:4.17} bzgl. der Menge der Kanten $\mcal E$ darstellen, da der Rand $\partial T = E_1 \cup E_2 \cup E_3$ für jedes $T$ disjunkt in seine Kantenstücke aufgeteilt werden kann.

Dabei sei $E$ nun die Kante, die $T_1$ und $T_2$ zugleich enthalten, d.h. $\bs n$ steht rechtwinklig auf $E$. Dann gilt, dass die Richtungsableitung $\partial_{\bs n} u_{\mcal S}|_{T_2}$ negativ ist bzgl. \eqref{eq:4.17} wegen der negativen Orientierung von $\bs n$ bzgl. $T_2$.

\begin{figure}[h]\label{abb:4.2}
  \begin{center}
    \begin{pspicture}(-1.5,0)(2,2)
    	% Die zwei Dreiecke:
	\psline(-1.5,1.7)(1,2)
	\psline(1,2)(-0.2,-0.2)
	\psline(-0.2,-0.2)(-1.5,1.7)
	\psline(1,2)(2.5,0.4)
	\psline(2.5,0.4)(-0.2,-0.2)
	\rput(-0.7,1.4){$T_1$}
	\rput(1.9,0.55){$T_2$}
	
	% Normalenvektor:
	\psline{->}(0.35,0.8)(1,0.45)
	\rput(0.8,0.9){$\bs n$}
	\psellipticarc[linewidth=0.5pt](0.35,0.8)(0.25,0.25){242}{335}
	\psdot[dotsize=1.3pt](0.38,0.68)
	
	% Beschriftung der Kante E:
	\rput(0.55,1.65){$E$}
    \end{pspicture}
  \end{center}
\caption{Dreiecke $T_1$ und $T_2$ mit Einheitsnormalen $\bs n$}
\end{figure}

Hiermit ergibt sich aus \eqref{eq:4.17}
\begin{align}\notag
	\rho_{\mcal S} (v) &= \int_{\Omega} fv \, d\Omega - \sum_{T\in \mcal T_h} \int_{\partial T} v \partial_{\bs n} u_{\mcal S} \, d\Gamma \\
	\notag
	& = \int_{\Omega} fv \, d\Omega - \sum_{E\in \mcal E} \int_{E} v\, (\underbrace{\partial_{\bs n} u_{\mcal S}|_{T_1}-\partial_{\bs n} u_{\mcal S}|_{T_2}}_{\eqqcolon -j_E}) \, d\Gamma \\
	\label{eq:4.18}
	& =  \int_{\Omega} fv \, d\Omega + \sum_{E\in \mcal E} \int_{E} j_E v \, d\Gamma \, .
\end{align}

\item da für die nodalen Basisfunktionen $\{\phi_p \with p \in \mcal N\cap \Omega\}$ gilt
\[
	\sum_{p \in \mcal N} \phi_p = 1 \text{ auf ganz } \Omega \, , 
\]
sodass wir $\rho_{\mcal S}$ wie folgt in lokale Anteile aufteilen können:
\begin{align}\label{eq:4.19}
	\rho_p(v) \coloneqq \rho_{\mcal S} (v \phi_p) \, , \quad v \in H^1(\Omega) \, .
\end{align}

\item
\begin{lemma}\label{lem:4.14}
Für $\rho_p$ gilt
\begin{align*}
	\rho_p (v) = \int_{\omega_p} f v \phi_p \, d\Omega + \sum_{E\in \mcal E_p} \int_E j_E v \phi_p \, d\Gamma  \, , \quad v \in H^1(\Omega)
\end{align*}
mit $\omega_p \coloneqq \supp \phi_p$ und $\mcal E_p \coloneqq \{E \in \mcal E \with E \ni p\}$, d.h. die Menge der Kanten, in denen $p$ enthalten ist.
\end{lemma}

\begin{proof}
Wir rechnen einfach mit der Definition \eqref{eq:4.19} und \eqref{eq:4.18} nach, dass für ein beliebiges $v \in H^1(\Omega)$ gilt
\begin{align*}
	\rho_p(v) &= \rho_{\mcal S} (v \phi_p) =  \int_{\Omega} fv\phi_p \, d\Omega + \sum_{E\in \mcal E} \int_{E} j_E v \phi_p\, d\Gamma \\
	& = \int_{\omega_p} f v \phi_p \, d\Omega + \sum_{E\in \mcal E_p} \int_E j_E v \phi_p \, d\Gamma  \, ,
\end{align*}
da $\phi_p \equiv 0$ auf $\mcal O \coloneqq \overline{\Omega \setminus \omega_p}$ und damit auch auf $\mcal F \coloneqq\mcal E \setminus \mcal E_p$, da $\mcal F \subset\mcal O$.

\begin{figure}[h]
\begin{center}
	\begin{pspicture}(0,0)(7,5)
		%\psset{xunit=1.5cm,yunit=1.2cm}
		
		% \omega_p:
		\pspolygon[fillstyle=solid,fillcolor=lightgray](2.5,1.25)(3,2.6)(1.5,2.5)
		\pspolygon[fillstyle=solid,fillcolor=lightgray](2.5,1.25)(3,2.6)(4.7,2)
		\pspolygon[fillstyle=solid,fillcolor=lightgray](4.3,3.5)(3,2.6)(4.7,2)
		\pspolygon[fillstyle=solid,fillcolor=lightgray](4.3,3.5)(3,2.6)(2.5,3.5)
		\pspolygon[fillstyle=solid,fillcolor=lightgray](1.5,2.5)(3,2.6)(2.5,3.5)
		
		% Rest:
		\pspolygon(1.5,2.5)(0.75,3)(2.5,3.5)
		\pspolygon(1.5,2.5)(0.75,3)(1.2,1.5)
		\pspolygon(1.5,2.5)(2.5,1.25)(1.2,1.5)
		\pspolygon(1.2,1.5)(2.5,1.25)(1.9,0.5)
		\pspolygon(4.7,2)(2.5,1.25)(4.3,0.9)
		\pspolygon(4,-0.1)(2.5,1.25)(4.3,0.9)
		\pspolygon(4.7,2)(6.5,1.5)(4.3,0.9)
		\pspolygon(4.7,2)(6.5,1.5)(6.5,2.75)
		\pspolygon(4.7,2)(4.3,3.5)(6.5,2.75)
		
		\psline(1.2,1.5)(0.3,0.6)
		\psline(0.75,3)(0.25,3.25)
		\psline(2.5,3.5)(2.2,3.8)
		\psline(2.5,3.5)(2.9,3.9)
		\psline(4.3,3.5)(5.1,3.8)
		\psline(6.5,2.75)(6.3,3.3)
		\psline(6.5,1.5)(7.1,0.8)
		\psline(6.5,1.5)(7,1.8)
		\psline(6.5,2.75)(7,2.3)
		\psline(4.3,0.9)(5.9,0.3)
		
		% Hutfunktion:
		\psline[linestyle=dotted](3,2.6)(3,4.8)
		\psline[linewidth=0.6pt](3,4.8)(2.5,1.25)
		\psline[linewidth=0.6pt](3,4.8)(4.3,3.5)
		\psline[linewidth=0.6pt](3,4.8)(1.5,2.5)
		\psline[linewidth=0.6pt](3,4.8)(4.7,2)
		\psline[linewidth=0.6pt](3,4.8)(2.5,3.5)
		
		% Beschriftungen:
		\rput(3.2,5){$\phi_p$}
		\rput(3.1,2.3){$p$}
		\rput(2.2,2.1){$\omega_{p}$}
		
	\end{pspicture}
\end{center}
\caption{Darstellung von $\omega_p$ (grau) und $\mcal E_p$ (abgehende Kanten von $p$) für ein beliebiges $\phi_p$}
\end{figure}
\end{proof}

\item 
\begin{kor}
Der Indikator $\rho_{\mcal S}$ lässt sich schreiben als
\[
	\rho_{\mcal S} = \sum_{p \in \mcal N} \rho_p \, .
\]
\end{kor}

\begin{proof}
Die Behauptung folgt direkt aus Lemma \ref{lem:4.14} zusammen mit
\[
	\Omega = \bigcup_{p\in \mcal N} \omega_p \, , \quad \mcal E = \bigcup_{p \in \mcal N} \mcal E_p \quad \text{und}\quad \sum_{p \in \mcal N} \phi_p = 1 \, 
\]
durch einfaches Nachrechnen.
\end{proof}

\item im unbeschränkten Fall gilt $\rho_{\mcal S} = 0 \Lra e = 0$, denn zu $\rho_{\mcal S} = 0$ ist äquivalent
\[
	a(e,v) = \rho_{\mcal S}(v) = 0 \quad \forall \, v \in \mcal V\, .
\]
Da $e\in \mcal V$ ist, folgt wegen der Galerkin-Orthogonalität, dass $e=0$ sein muss. Die Umkehrung gilt analog.

\item bei Variationsungleichungen gilt dies im allgemeinen nicht.

\item aber: aus Lemma \ref{lem:4.12} folgt allgemeiner, falls $\rho_{\mcal S} (v) \le 0$ für alle $v \in \mcal A$ gilt
\[
	\frac 12 \norm{e}^2 \le \rho_{\mcal S}(e) \le 0\,  \lra\,  \norm e = 0 \, \lra\, e = 0 \, ,
\]
wodurch $\rho_{\mcal S} = 0$ folgt, dass $e = 0$ ist.

\item es gilt, ist $u_{\mcal S}$ die Lösung von \eqref{eq:3.12}, so gilt für alle $p \in \mcal N \cap \Omega$, dass $v = u_{\mcal S} + \phi_p  \ge \psi$, d.h. $v \in K_h$.

Damit folgt mit Einsetzen von $v$ in  \eqref{eq:3.12}
\begin{align}\notag
	&a(u_{\mcal S}, u_{\mcal S} + \phi_p - u_{\mcal S}) \ge (f,u_{\mcal S}+\phi_p-u_{\mcal S}) \\
	\label{eq:4.20}
	\Llra \,  & 0 \ge (f,\phi_p)-a(u_{\mcal S},\phi_p) = \rho_{\mcal S}(\phi_p)
\end{align}
dies bedeutet, dass die lineare Approximation des Fehlers $e$ gleich Null ist.

\item falls an einem Punkt $p$ kein Kontakt zwischen $u_{\mcal S}$ und $\psi$ vorliegt, also $u_{\mcal S} (p) > \psi(p)$ ist, dann können wir ein $\alpha > 0$ hinreichend klein wählen, sodass $v = u_{\mcal S} - \alpha \phi_p \in K_h$ liegt. Dann folgt analog durch Einsetzen von $v$ in \eqref{eq:3.12}
\begin{align*}
	0 & \ge (f,-\alpha \phi_p) - a(u_{\mcal S},-\alpha\phi_p) \\
	\Llra \, 0 & \le (f,\phi_p) - a(u_{\mcal S},\phi_p) = \rho_{\mcal S} (\phi_p) \stackrel[\scriptsize\eqref{eq:4.20}]{}\le 0
\end{align*}
und damit gilt $\rho_{\mcal S} (\phi_p) = 0$

\item zusammen ergeben sich die Bedingungen
\begin{align}\label{eq:4.21}
	\rho_{\mcal S} (\phi_p) \le 0 \, , \quad \psi(p)-u_{\mcal S} (p) \le 0 \, , \quad \rho_{\mcal S} (\phi_p) (\psi(p)-u_{\mcal S} (p)) = 0
\end{align}

\item dies berechtigt zur Definition von Kontakt- und Nichtkontaktpunkten
\begin{defi}
Wir definieren die Mengen von \textit{Kontaktpunkten}\index{Kontaktpunkte} $\mcal N^0$ und \textit{Nichtkontaktpunkten}\index{Nichtkontaktpunkte} $\mcal N^+$ durch
\begin{align*}
	\mcal N^0 \coloneqq  \{p \in \mcal N \cap \Omega \with u_{\mcal S}(p) = \psi (p) \} \, , \quad 
	\mcal N^+ \coloneqq  \{p \in \mcal N \cap \Omega \with u_{\mcal S}(p) > \psi (p) \}\, .
\end{align*}
\end{defi}

\item 
\begin{bem}
Die Bedingungen \eqref{eq:4.21} können wir auch auf den lokalen Anteil $\rho_p$ übertragen, damit ergibt sich für alle $p \in \mcal N \cap \Omega$
\begin{subequations}\label{eq:4.22}
\begin{align}\label{eq:4.22a}
	\rho_p(1) &  \le 0\, ,  \\
	\label{eq:4.22b}
	u_{\mcal S} (p) > \psi (p) \, & \lra \, \rho_p(1) =0 \, ,
\end{align}
\end{subequations}
denn $\rho_p (1) = \rho_{\mcal S} (\phi_p)$.
\end{bem}

\item damit ist also die Approximation von $e$ über $\mcal S_h$ gleich Null, wenn die lokalen Anteile (im Vektor später) kleiner gleich Null sind

\end{itemize}






\subsection{Oszillationsterme}
\label{kap:4.1.3}

\begin{itemize}
\item in Kapitel \ref{kap:4.1.4} werden wir zeigen, dass $-\mcal I_{\mcal Q} (\eps_{\mcal V})$ eine obere Schranke von $-\mcal I (e)$ bis auf Terme höherer Ordnung bereitstellen, d.h. Terme, die nicht in $\mcal V$ enthalten sind $\ra$ Oszillationsterme (hier eingeführt)

\item man kann in den numerischen Beispielen später sehen, dass (wie auch in der Theorie) eine Verkleinerung der Oszillation auch eine Verringerung des Fehlers mit sich bringt $\Ra$ wir führen die Oszillationsterme ein (auch ohne präzise Beweise)

\item die Oszillation ist in zwei Teile kaufteilbar
\begin{align}\label{eq:4.23}
	\osc (u_{\mcal S}, \psi, f) \coloneqq \osc_1(u_{\mcal S},\psi)+\osc_2(u_{\mcal S},\psi,f)
\end{align}
(später vielleicht mit Wurzeln anders)

\item im unbeschränkten Fall ist die Oszillation nur von $f$ abhängig (s. \cite{MorNoc}) und dort wird daher von "`Daten-Oszillation"' gesprochen

\item $\osc_1$ ist ein Maß für die Oszillation zwischen Hindernis $\psi$ und der Galerkin-Lösung $u_{\mcal S}$, d.h.
\begin{align}
	\osc_1(u_{\mcal S},\psi) \coloneqq \(\sum_{p \in \mcal N^{0+}} \norm{\nabla (\psi - u_{\mcal S})}_{0,\omega_p}^2\)^{\frac 12} \, ,
\end{align}
wobei $\mcal N^{0+} \coloneqq \{p \in \mcal N^0 \with u_{\mcal S} > \psi \text{ in }\omega_p \setminus \{p\}\}$, also die Megne der \textit{isolierten Kontaktknoten}, d.h. $u_{\mcal S}$ ist in $\omega_p$ nur mit $p$ in Kontakt

\item anschaulich: da $\psi, u_{\mcal S}$ linear, gilt: je größer die Differenz zwischen $\psi$ und $u_{\mcal S}$, umso größer die Differenz $\nabla(\psi-u_{\mcal S})$, d.h. auch $\osc_1$

\item das kontinuierliche Gegenstück zu $\mcal N^{0+}$ ist die Menge der \textit{isolierten Kontaktpunkte} $x_c$, die aufgrund von $u-\psi >0$ für alle $x \in \mcal U(x_c, \eps) \subset \Omega$ mit $u(x_c) = \psi(x_c)$ alle strikten Minima $x_c \in \Omega$ enthält $\Ra (\nabla u -\nabla \psi) = 0$ für alle isolierten Kontaktpunkte, wenn $u, \psi$ hinreichend glatt sind

\item da laut Theorem \ref{theorem:3.14} $u_h \ra u$ für $h\ra 0$ geht, folgt: wenn ein isolierter Kontaktknoten $p \in \mcal N^{0+}$ bei Verfeinerung bestehen bleibt, so hat die exakte Lösung $u$ einen korrespondierenden Kontaktpunkt $\tilde p$, dann gilt
\begin{align*}
	\bigcup_{p \in \mcal N^{0+}} \omega_p \xrightarrow[h\ra 0]{} \tilde p
\end{align*}

\item damit gilt $\osc_1$ hat wenigstens den Grad vom Fehler $e$ (warum?)

\item wegen oben (mit dem hinreichend glatten $u, \psi$) verschwindet $\osc_1$ für $h \ra 0$

\item $\osc_2$ ist über zwei Mengen definiert:
\begin{align}
	\mcal N^{++} \coloneqq \{p \in \mcal N^{+}\with \rho_E \ge - d_E \, \forall \, E \in \mcal E_p\}
\end{align}
d.h. alle Punkte ohne Kontakt, in denen der Fehler $\eps_{\mcal V}$ nicht in Kontakt mit $\mcal A_{\mcal V}$ steht (wie in Beweis von Satz \ref{satz:4.11} ersichtlich)
\begin{align}
	\mcal N^{0-}\coloneqq \{ p \in \mcal N^0 \with u_{\mcal S} = \psi , f \le 0 \text{ auf } \omega_p, j_E \le 0 \, \forall \, E \in \mcal E_p \}
\end{align}
d.h. voller Kontakt (s. auch \cite{SiebVee} Gleichung (2.11)) mit Last $f$ auf Druck und negativem Normalenfluss $j_E$

\item aus der Nebenbedingung von $\mcal N^{0-}$ folgt
\begin{align*}
	0 \ge f + \sum_{E \in \mcal E_p} j_E
\end{align*}
durch Multiplikation mit geeigneten Testfunktionen $v$ und multiplizieren über $\omega_p$ ergibt
\begin{align*}
	0 &  \ge \int_{\omega_p} f v \, d\Omega + \sum_{E\in \mcal E_p} \int_E j_E v \, d \Gamma \\
	& = \int_{\omega_p} f v \, d \Omega - \int_{\omega_p} \underbrace{\nabla u_{\mcal S}}_{=\nabla \psi} \nabla v \, d\Omega
\end{align*}
und damit gilt
\begin{align}
	 \int_{\omega_p} {\nabla \psi} \nabla v \, d\Omega \ge \int_{\omega_p} f v \, d \Omega
\end{align}
es gilt also laut Satz \ref{satz:3.4}, dass $-\Delta \psi - f \ge 0$ auf $\omega_p$ im distributionellem Sinne (vgl. auch \cite{Walker} Kapitel 3)

dies ist laut Satz \ref{satz:3.4} auch notwendig, damit $u = \psi$ auf $\omega_p$ ist

\item damit ergibt sich $\osc_2$ als
\begin{align}
	\osc_2(u_{\mcal S},\psi, f) \coloneqq \(\sum_{p \in \mcal N^{++}} h_p^2 \norm{f-\bar f_p}_{0,\omega_p}^2 + \sum_{p \in \mcal N\setminus (\mcal N^{0-}\cup \mcal N^{++})} h_p^2 \norm f_{0,\omega_p}^2 \)^{\frac 12}
\end{align}
wobei $h_p \coloneqq \max_{E\in \mcal E_p} \abs E$ für jedes $p \in \mcal N$ ($h_p$ ist ein Maß für den Durchmesser von $\omega_p$) und $\bar f_p$ den Mittelwert von $f$ über $\omega_p$ bezeichne, d.h.
\begin{align}
	\bar f _p = \frac 1{\abs{\omega_p}} \int_{\omega_p} f \, d\Omega
\end{align}

\item anschaulich: damit kann man die Summanden der ersten Summe als Varianz der Last $f$ auf $\omega_p$ interpretieren 

$\mcal N\setminus (\mcal N^{0-}\cup \mcal N^{++})$ ist die Menge von Punkten, die keinen vollen Kontakt und in der $\eps_{\mcal V}$ keinen Kontakt mit $\mcal A_{\mcal V}$ hat

\item Beachte: Im Term $\osc_2$ fehlen nur die Punkte, die vollen Kontakt haben, d.h. wir betrachten also wirklich nur die Punkte außerhalb des Hindernisses!! (genauer noch: $\mcal N \setminus (\mcal N^{0-}\cup \mcal N^{++})$ sind nur die Randpunkte!)

\item damit enthält $\osc_2$ nur Anteile aus Knoten, ohne vollen Kontakt

\item Bem.: die Oszillationsterme können leicht berechnet werden (siehe hierfür auch Kapitel 5)

\item im unbeschränkten Fall, also $\psi = -\infty$, ist $\eps_{\mcal V}$ nicht im Kontakt mit dem Hindernis für alle Punkte aus $\mcal N$, also gilt $\mcal N^{++} = \mcal N$.

\item damit wird \eqref{eq:4.28} ($\osc_2$) zu
\begin{align}
	\osc_2(u_{\mcal S},\psi, f) = \(\sum_{p \in \mcal N\cap \Omega} h_p^2 \norm{f-\bar f_p}_{0,\omega_p}^2 + \sum_{p \in \mcal N \cap \partial \Omega} h_p^2 \norm f_{0,\omega_p}^2 \)^{\frac 12}
\end{align}

\item damit ist (4.28) eine Verallgemeinerung von (4.30): wenn der Teil ohne Kontakt also bekannt wäre, dann wäre der beschränkte Fall auf dieser Menge äquivalent zu einem unbeschränkten Dirichtlet-Problem

\item WICHTIG: Noch einmal in \cite{Zhang} schauen, ob dies in Verbindung des letzten Absatzes im Mainpaper verwendet werden kann!!!


\end{itemize}






\subsection{Zuverlässigkeit des Fehlerschätzers}
\label{kap:4.1.4}

\begin{itemize}
\item wir wollen in diesem Kapitel eine obere Schranke des Fehlers im Energiefunktional, die vom hierarchischen Fehlerschätzer abhängt, herleiten.

\item die Reduktion des Fehlers $e = u-u_{\mcal S} \in H^1_0(\Omega)$ auf den approximierten Fehler $\eps_{\mcal V}\in \mcal V$ erhalten wir durch lokale Projektionen für jedes $p \in \mcal N$ mit
\begin{align}
	\pi_p: H^1(\Omega) \ra \mcal Q_p = \Span \{ \phi_p\} \cup \mcal V_p \, , \quad \mcal V_p = \Span\{\phi_E \with E \in \mcal E_p\} \, .
\end{align}

\item $\pi_p$ ist für jedes $v \in H^1(\Omega)$ aus Dimensionsgründen ($\dim (\mcal Q_p) = p+1$) eindeutig bestimmt durch
\begin{align}\label{eq:4.32}
	\int_E \pi_p v \, d \Gamma = \int_E v \, d\Gamma \quad \forall \, E \in \mcal E_p \text{ und } \begin{cases}
														\int_{\omega_p} \pi_p v \, d \Omega = \int_{\omega_p}  v \, d\Omega  &,  p \in \mcal N^{++} \\
														0  &, \text{sonst}
													\end{cases} \, .
\end{align}

\item
\begin{lemma}
Es sei $\pi_p$ die oben beschriebene Projektion. Dann gelten für die Koordinaten bzgl. der Basis $\{\phi_p\}\cup \{\phi_E \with E \in \mcal E_p\}$ von 
\[
	\pi_p v = \alpha_p(v) \phi_p + \sum_{E \in \mcal E_p} \alpha_E(v) \phi_E
\]
die Beziehungen
\begin{subequations}
\begin{align}
	\alpha_p(v)& = \begin{cases}
					\frac{c_p(v)}{c_p(\phi_p)} &, p \in \mcal N^{++} \\
					0 &, \text{sonst}
				\end{cases} , \\
	 \alpha_E(v) &= \frac{\int_E v \, d\Gamma - \alpha_p(v)\int_E \phi_p \, d\Gamma}{\int_E \phi_p \, d\Gamma} \, ,
\end{align}
\end{subequations}
wobei
\begin{align*}
	c_p(v) = \int_{\omega_p} v \, d\Omega - \sum_{E\in \mcal E_p} \(\int_E v \, d\Gamma\) \(\int_{\omega_p} \phi_E \, d\Omega\)\(\int_E \phi_E \, d\Gamma\)^{-1}
\end{align*}
Insbesondere gilt $c_p(\phi_p) = -\frac 16\, \abs{\omega_p}$.
\end{lemma}

\begin{proof}
Für eine bessere Übersicht im Beweis werden wir die Differentialformen $d\Omega$ und $d\Gamma$ weg. Es sei $v \in H^1(\Omega)$ beliebig. Dann gilt für jede Kante $E \in \mcal E_p$ mit
\[
	\pi_p v = \alpha_p(v) \phi_p + \alpha_E (v) \phi_E \in \mcal Q_p 
\]
nach \eqref{eq:4.32}, dass
\begin{align}\notag
	 & \int_Ev  = \int_E \pi_p v  = \int_E \alpha_p(v) \phi_p + \alpha_E(v) \phi_E \\
	\label{eq:4.34}
	\lra \,  &  \alpha_E(v)= \(\int_E v - \alpha_p(v) \int_E \phi_p \)\( \int_E \phi_E \)^{-1} \, .
\end{align}
Wenn $p \not\in \mcal N^{++}$ ist, so gilt $\pi_p v \in \mcal V_p = \Span\{\phi_E\mid E \in \mcal E_p\}$, d.h. $\alpha_p(v) = 0$.

Es sei nun also $p \in \mcal N^{++}$. Dann folgt aus der zweiten Eigenschaft von \eqref{eq:4.32} und \eqref{eq:4.34} für $\pi_p v = \alpha_p(v) \phi_p + \sum_{E\in \mcal E_p} \alpha_E(v) \phi_E\in \mcal Q_p$
\begin{align*}
	\int_{\omega_p} v    =& \int_{\omega_p} \pi_p v  = \int_{\omega_p} \alpha_p(v) \phi_p + \sum_{E\in \mcal E_p} \alpha_E(v) \phi_E \\
	 =& \alpha_p(v) \int_{\omega_p} \phi_p  + \sum_{E\in \mcal E_p} \alpha_E(v) \int_{\omega_p}\phi_E  \\
	= & \alpha_p(v) \int_{\omega_p} \phi_p  + \sum_{E\in \mcal E_p} \(\int_E v  - \alpha_p(v) \int_E \phi_p \)\( \int_E \phi_E\)^{-1} \( \int_{\omega_p}\phi_E\) \\
	 =& \alpha_p(v) \Bigg(\underbrace{\int_{\omega_p} \phi_p-\sum_{E\in \mcal E_p}\( \int_E \phi_p\)\( \int_{\omega_p} \phi_E \)\(\int_E \phi_E\)^{-1}}_{= c_p(\phi_p)}\Bigg) \\
	 & + \sum_{E\in\mcal E_p} \(\int_E v\) \(\int_{\omega_p} \phi_E\) \(\int_E \phi_E\)^{-1} \, .
\end{align*}
Nach dem Umformen nach $\alpha_p(v)$ ergibt sich dann
\[
	\alpha_p(v) = \frac{c_p(v)}{c_p(\phi_p)}
\]
mit dem oben definierten $c_p(\cdot)$.

Es bleibt also zu zeigen, dass $c_p(\phi_p) = -\frac 1 6 \, \abs{\omega_p}$. Hierfür betrachten wir die einzelnen Summanden von $c_p(\phi_p)$. Zunächst berechnet das Integral von $\phi_p$ über $\omega_p$ das Volumen der von $\phi_p$ erzeugten Pyramide mit Grundfläche $\abs{\omega_p}$, d.h.
\begin{align}\label{eq:4.35}
	\int_{\omega_p} \phi_p = \frac 13 \, \abs{\omega_p} \, .
\end{align}
Weiter ist $\phi_p$ auf jeder Kante $E \in \mcal E_p$ eine von 1 zu 0 abfallende Gerade und damit ist das Integral über $E$ gerade der Flächeninhalt vom darüber liegenden Dreieck, also
\begin{align}\label{eq:4.36}
	\int_E \phi_p = \frac 1 2 \, \abs E \, .
\end{align}
Die letzten beiden Integrale berechnen wir über die Referenzelemente in $\R$ oder $\R^2$ für das Kurven- bzw. Flächenintegral. Die Funktion $\phi_E$ über eine Kante $E$ ist eine nach unten geöffnete Parabel. Auf dem Referenzelement $[-1,1]\subset\R$ ist dies die Funktion
\[
	\hat \phi_E = 1-\xi^2 
\]
und mit einer affinen Transformation $s: [-1,1]\ra [a,b] = E, s(\xi) = \frac{b-a}2 \xi + \frac{b+a}2$ lässt sich das Referenzelement auf das Element $E$ abbilden. Damit ergibt sich mit dem Transformationssatz der Integration
\begin{align}\label{eq:4.37}
	\int_E \phi_E = \frac{b-a}2 \int_{-1}^1 \hat\phi_E = \frac 12 \, \abs{E} \cdot \frac 43 = \frac 23 \, \abs E\, .
\end{align}
Der letzte Fall ist komplizierter zu beschreiben. Zunächst sei erwähnt, dass $\supp (\phi_E) = T_i \cup T_j, T_i,T_j \subset \omega_p, i\not=j$ gilt, $\phi_E$ also nur auf zwei Dreiecken, die in $\omega_p$ enthalten sind, ungleich Null ist. Damit wird für jede Kante $E \in \mcal E_p$ über jedes Dreieck $T \subset \omega_p$ genau zweimal integriert.

Auf dem Referenzelement 
\[	
	\hat T \coloneqq \{(\xi,\eta)\in \R^2 \mid 0\le \xi \le 1, 0 \le \eta \le 1-\xi\}
\]
haben wir die drei Bubble-Funktionen
\begin{align*}
	\hat \phi_{E_1} = 4 \xi (1-\xi-\eta) \, , \quad \hat \phi_{E_2} = 4 \xi \eta \, , \quad \hat \phi_{E_3} = 4 \eta (1-\xi-\eta) \, ,
\end{align*}
für die man leicht nachrechnen kann, dass
\begin{align*}
	\int_{\hat T} \hat\phi_{E_1} = \int_{\hat T} \hat\phi_{E_2} = \int_{\hat T} \hat\phi_{E_3} = \frac 16
\end{align*}
gilt. Es sei nun $J_T$ die Jacobi-Determinante bzgl. einer affinen Transformation $r:\hat T \ra T$, dann gilt nach Transformationssatz mit einem $T \subset \supp(\phi_E)$
\[
	\int_T \phi_E = \abs{J_T} \int_{\hat T} \hat \phi_E = \frac 16 \, \abs{J_T} \, .
\]
Weiter rechnen wir nach, dass
\[
	\abs T = \int_T d\Omega = \abs{J_T} \int_{\hat T} d\Omega = \frac 12 \, \abs{J_T} \, \lra \, \abs{J_T} = 2 \, \abs{T} 
\]
gilt und damit folgt insgesamt zusammen mit \eqref{eq:4.35} bis \eqref{eq:4.37}
\begin{align*}
	c_p(\phi_p) & = \int_{\omega_p} \phi_p - \sum_{E\in\mcal E_p} \(\int_E \phi_p \, d\Gamma\) \(\int_{\omega_p} \phi_E \, d\Omega\)\(\int_E \phi_E \, d\Gamma\)^{-1} \\
	& = \frac 1 3 \, \abs{\omega_p} - 2 \sum_{T \subset \omega_p} \frac 1 2 \, \abs E \cdot \frac 1 6 \, \abs{J_T} \cdot \frac 3 2 \, \abs E^{-1} \\
	& = \frac 1 3 \, \abs{\omega_p} - \sum_{T\subset \omega_p} \frac 12 \, \abs T = \( \frac 1 3  -  \frac 1 2 \)\abs{\omega_p} \\
	& = - \frac 16 \, \abs{\omega_p} \, . \qedhere
\end{align*}
\end{proof}




\item folgendes Lemma ist zentral, um die obere Schranke von $J(u_{\mcal S})-J(u)$ bzgl. $-\mcal I_{\mcal Q}(\eps_{\mcal V})$ zu zeigen

\begin{lemma}\label{lem:4.19}
Es sei Voraussetzung \ref{vor:4.1} erfüllt. Dann gilt
\begin{align}
	\rho_{\mcal S}(e) \lesssim \sum_{E\in \mcal E} \eta_E \, \abs{\rho_E} + \osc(u_{\mcal S},\psi,f)^2 
\end{align}
mit $\rho_E$ wie in \eqref{eq:4.11}, $\osc (u_{\mcal S},\psi,f)$ wie in \eqref{eq:4.23} und $\eta_E = \abs{\eps_{\mcal V}(x_E)} \, \norm{\phi_E}$.
\end{lemma}

\begin{proof}


Verwenden wir nun die sechs gezeigten Fälle, so ergibt sich:
\begin{align*}
	\rho_{\mcal S}(e) = & \sum_{p \in \mcal N} \rho_p(e) \\
	= & \sum_{E\in \mcal E^0} d_E\, \abs{\rho_E} 
\end{align*}
\end{proof}




\item Theorem für obere Schranke des Fehlerschätzers:
\begin{theorem}
Es sei Voraussetzung \ref{vor:4.1} für $\psi$ erfüllt. Dann ist der hierarchische Fehlerschätzer $-\mcal I_{\mcal Q}(\eps_{\mcal V})$ eine obere Schranke für den Fehler im Energiefunktional bis auf Addition von Oszillationstermen und einer Konstante $C$, die nur von der Quasi-Uniformität von $\mcal T_h$ abhängt, d.h.
\begin{align}
	J(u_{\mcal S}) - J(u) \lesssim -\mcal I_{\mcal Q}(\eps_{\mcal V}) + \osc (u_{\mcal S},\psi,f)^2 \, .
\end{align}
\end{theorem}

\begin{proof}
Die Aussage folgt direkt durch Lemma \ref{lem:4.12} und \ref{lem:4.19}, denn
\begin{align*}
	J(u_{\mcal S})-J(u) & = - \mcal I (e) \le \rho_{\mcal S}(e) \\
	& \lesssim \underbrace{\sum_{E\in \mcal E}\eta_E \, \abs{\rho_E}}_{=\rho_{\mcal S}(\eps_{\mcal V})} + \osc (u_{\mcal S},\psi,f)^2 \\
	& \le 2\cdot (-\mcal I_{\mcal Q}(\eps_{\mcal V}))+ \osc (u_{\mcal S},\psi,f)^2 \\
	& \le 2 \cdot (-\mcal I_{\mcal Q}(\eps_{\mcal V})+ \osc (u_{\mcal S},\psi,f)^2)
\end{align*}
und damit folgt die Behauptung.
\end{proof}

\item an dieser Abschätzung können wir sehen, dass es sinnvoll ist, nicht nur den hierarchischen Fehlerschätzer $-\mcal I_{\mcal Q}(\eps_{\mcal V})$ zum Abschätzen des Fehlers zu verwenden, sondern auch die Oszillationsterme (diese sollten von Verfeinerungsschritt zum Verfeinerungsschritt kleiner werden, sonst ist die Verringerung im exakten Fehler nicht gesichert)

\item damit nachher eine Abschätzung analog zu \cite{MorNoc} Lemma 3.8 (wäre schön, wenn diese noch gezeigt werden würde....)
\begin{lemma}
Es sei $0 < \gamma < 1$ ein Parameter, der die Reduktion der Größe des Dreiecks bei Verfeinerung wiedergibt. Weiter sei $0 < \hat \theta < 1$ gegeben und eine Menge an Punkten $\hat{\mcal N}\subset \mcal N$, die die zu verfeinernden Dreiecke anzeigen, gegeben, so dass
\[
	\osc(u_{\mcal S},\psi,f,\hat{\mcal N}) \ge \hat\theta \osc (u_{\mcal S},\psi,f,\mcal N) \, .
\]
Dann existiert ein $\hat\alpha \in (0,1)$, so dass
\begin{align}
	\osc(u_{\mcal S},\psi,f,\tilde{\mcal N}) \le \hat \alpha \osc (u_{\mcal S},\psi,f,\mcal N) \, ,
\end{align}
wobei $\tilde{\mcal N}$ die Menge an Punkten nach Verfeinerung der Triangulierung $\mcal T_h$ bzgl. der Punkte $\hat{\mcal N}$ ist.
\end{lemma}

Hierfür vllt eine äquivalente Darstellung von $\osc_2$ bzgl. der Dreiecke und dann das Lemma nur auf $\osc_2$ beziehen.
\end{itemize}






\subsection{Effektivität des Fehlerschätzers}
\label{kap:4.1.5}

\begin{itemize}
\item wir zeigen, dass der hierarchische Fehlerschätzer $-\mcal I_{\mcal Q}(\eps_{\mcal V})$ ist auch eine untere Schranke für $-\mcal I(e) = J(u_{\mcal S}) - J(u)$

\item
\begin{theorem}
Das Hindernis $\psi$ sei stückweise linear und stetig. Dann ist der hierarchische a posteriori Fehlerschätzer $\mcal I_{\mcal Q}(\eps_{\mcal V})$ auch eine untere Schranke für den Fehler im Energiefunktional im Sinne von
\begin{align}
	-\mcal I_{\mcal Q}(\eps_{\mcal V}) \le 6 (J(u_{\mcal S})-J(u)) \, .
\end{align}
\end{theorem}

\begin{proof}
Zunächst folgt mit \eqref{eq:4.16}
\begin{align}\notag
	-\mcal I_{\mcal Q} (\eps_{\mcal V}) & \le \rho_{\mcal S} (\eps_{\mcal V}) = \rho_{\mcal S} \(\sum_{E\in \mcal E} \eps_{\mcal V}(x_E) \phi_E\) \\
	\notag
	& = \sum_{E\in \mcal E} \eps_{\mcal V} (x_E) \rho_{\mcal S}(\phi_E) \\
	& = \sum_{E \in \mcal E} \eta_E \, \abs {\rho_E}  \label{eq:my4.32}
\end{align}
mit $\eta_E = \abs{\eps_{\mcal V}(x_E)} \cdot \norm{\phi_E}$ und $\rho_E = \frac {\rho_{\mcal S}(\phi_E)}{\norm{\phi_E}}$, wobei man zeigen kann, dass $\sign (\eps_{\mcal V}(x_E)) = \sign (\rho_{\mcal S} (\phi_E))$ gilt. Weiter sollte man erwähnen, dass \eqref{eq:my4.32} äquivalent ist zu \cite{SiebVee} Gleichung (2.16).

Das weitere Vorgehen ist ähnlich zum Beweis von Theorem 3.2 aus \cite{SiebVee}. Es sei
\[
	\varphi = \frac 13 \sum_{E\in \mcal E} \beta_E \phi_E
\]
eine Linearkombination aus Bubble-Funktionen. Dann lässt sich $u_{\mcal S} + \varphi$ auf jedem $T \in \mcal T_h$ durch eine Konvexkombination aus $v_E \coloneqq u_{\mcal S} + \beta_E \phi_E, E \in \mcal E$ schreiben, d.h.
\[
	(u_{\mcal S} + \varphi)\Big|_{T} = \frac 13 \sum_{E\in\mcal E, E \subset T} v_E \Big|_T \, .
\]
Da $\R^2 \ni x \mapsto \frac 12 \abs x^2$ konvex ist, rechnen wir mit den obigen Bezeichnungen schnell nach, dass gilt
\begin{align*}
	J(u_{\mcal S} + \varphi) & = \int_\Omega \frac 12 \abs{\nabla(u_{\mcal S}+\varphi)}^2 - f (u_{\mcal S} + \varphi )\, d\Omega \\
	& = \sum_{T\in \mcal T_h} \int_T \frac 12 \Abs{\nabla(u_{\mcal S}+\varphi)\Big|_{T}}^2 - f (u_{\mcal S} + \varphi )\Big|_T\, d\Omega \\
	& =  \sum_{T\in \mcal T_h} \int_T \frac 12 \Abs{\(\frac 13 \sum_{E\in\mcal E, E \subset T}\nabla v_E \Big|_T\)}^2 - f\( \frac 13 \sum_{E\in\mcal E, E \subset T} v_E \Big|_T\)\, d\Omega \\
	& \le \frac 13 \sum_{E\in\mcal E, E \subset T} \, \sum_{T\in \mcal T_h} \int_T \frac 12 \Abs{\nabla v_E \Big|_T}^2 - f v_E \Big|_T\, d\Omega \, .
\end{align*}
Da wir drei Kanten pro Dreieck $T$ haben, gilt analog die Gleichung
\[
	J(u_{\mcal S}) = \frac 13 \sum_{E\in\mcal E , E \subset T}\,  \sum_{T\in \mcal T_h} \int_T \frac 12 \Abs{\nabla u_{\mcal S}}^2 - f u_{\mcal S}\, d\Omega  \, .
\]
Durch Subtraktion der letzten beiden Terme und einigen Umformungen ergibt sich dann
\begin{align}
	J(u_{\mcal S}) - J(u_{\mcal S}+\varphi) \ge \frac 13 \sum_{E\in \mcal E} (J(u_{\mcal S})-J(u_{\mcal S} + \beta_E \phi_E)) \, .
\end{align}
Wir rechnen nach, dass für alle $E\in \mcal E$
\begin{align*}
	J(u_{\mcal S} + \beta_E \phi_E) & = \frac 12 a(u_{\mcal S}+\beta_E \phi_E, u_{\mcal S} + \beta_E \phi_E) - (f, u_{\mcal S} + \beta_E \phi_E) \\
	& = J(u_{\mcal S}) + \frac 12 a(\beta_E\phi_E,\beta_E\phi_E)-((f,\beta_E\phi_E)-a(u_{\mcal S},\beta_E\phi_E)) \\
	& = J(u_{\mcal S}) + \mcal I (\beta_E\phi_E) 
\end{align*}
gilt. Damit ist das Minimieren von $J(u_{\mcal S}+\beta_E\phi_E)$, so dass $\beta_E \ge -d_E$, mit $d_E$ wie oben definiert, äquivalent ist zum Problem:
\[
	\min_{\beta_E \ge -d_E} \mcal I (\beta_E\phi_E) \, ,
\]
was den nächsten Schritt legitimiert. Wir setzen nun $\beta_E = \eps_{\mcal V}(x_E)$, dann gilt, dass $u_{\mcal S} + \beta_E \phi_E \in  K$ ist für alle $E \in \mcal E$ und damit aufgrund der Konvexität von $K$ auch $u_{\mcal S}+\varphi \in \mcal K$. Damit folgt insgesamt
\begin{align*}
	J(u_{\mcal S}) -J(u) & \ge J(u_{\mcal S})-J(u_{\mcal S}+\varphi) \\
	& \ge  \frac 13 \sum_{E\in \mcal E} (J(u_{\mcal S})-J(u_{\mcal S} + \beta_E \phi_E))=  \frac 13 \sum_{E\in \mcal E} -\mcal I (\beta_E \phi_E) \\
	&  =   \frac 13 \sum_{E\in \mcal E} \(\rho_{\mcal S}(\beta_E \phi_E) - \frac 12 a(\beta_E\phi_E,\beta_E\phi_E)\) \\
	& =   \frac 13 \sum_{E\in \mcal E} \(\beta_E \, \rho_{\mcal S}(\phi_E) - \frac 12 \beta_E^2 a(\phi_E,\phi_E)\) \\
	& \ge   \frac 13 \sum_{E\in \mcal E}\(\frac{\max\{-d_E,\rho_E\}}{\norm{\phi_E}} \, \rho_{\mcal S}(\phi_E) - \frac 1 2 \frac{\max\{-d_E,\rho_E\}^2}{\norm{\phi_E}^2} \norm{\phi_E}^2 \) \\
	& = \frac 13 \sum_{E\in \mcal E} \bigg( \underbrace{\max\{-d_E,\rho_E\} \rho_E}_{=\eta_E \, \abs{\rho_E}} - \frac 12 \underbrace{\max\{-d_E,\rho_E\}^2}_{\ge \eta_E \, \abs{\rho_E}} \bigg) \\
	& \ge \frac 13 \sum_{E\in \mcal E} \frac 12 \eta_E \, \abs{\rho_E} = \frac 16 \sum_{E\in \mcal E} \eta_E \, \abs{\rho_E} \, .
\end{align*}
Zusammen mit \eqref{eq:my4.32} folgt dann die Behauptung.
\end{proof}
\end{itemize}






\section{Ein adaptiver Algorithmus}
\label{kap:4.2}

\begin{itemize}
\item
\end{itemize}






\section{Erfüllung einer Saturationseigenschaft}
\label{kap:4.3}

\begin{itemize}
\item
\end{itemize}






\section{Übertragung des Fehlerschätzers auf Kontaktprobleme}
\label{kap:4.4}

\begin{itemize}
\item
\end{itemize}






\newpage

%%% Local Variables: 
%%% mode: latex
%%% TeX-master: "Skript"
%%% End: 

\newchapter{Implementierung des Fehlerschätzers in Matlab}
\label{kap:5}

In diesem Kapitel wollen wir uns einen kurzen Überblick über den Quellcode für das programmierte Hindernis- bzw. Kontaktproblem verschaffen. Während wir an dieser Stelle einige wichtige Funktionen detailiert betrachten werden, können wir im Anhang \ref{anhang:D} den Matlab-Quellcode komplett einsehen.

In die Implementierung sind neben \cite{ZouVee} unter anderem auch Resultate aus \cite{MorNoc}, \cite{BarCar}, \cite{BraeFEM} und \cite{EPS} eingeflossen.


\section{Implementierung eines Hindernisproblems}
\label{kap:5.1}


\subsubsection{Grundlegender Aufbau des Programms}

In der Datei {\ttfamily start.m} werden die grundlegenden Daten für das betrachtete Beispiel festgelegt, wie beispielsweise
\begin{itemize}
\item die Geometriedaten für das Gitter und die Dirichlet-Randbedingungen,
\item die exakte Lösung des Funktionasl $J(u)$,
\item die Lastfunktion $f$ und
\item die initiale Triangulierung $\mcal T_0$.
\end{itemize}
Weiter werden nach Ausführung des adaptiven Algorithmus die Galerkin-Lösung und ein Fehlerdiagramm geplottet. 

Die Funktion {\ttfamily adaptive_refinement_solution.m} ist das Herzstück des Programms. Diese Datei beinhaltet den Ablauf des Algorithmus \ref{alg:4.1}, wobei hier die Abbruchbedingung aus Zeile 5 erweitert wurde. Alle weiteren Programmteile werden in {\ttfamily adaptive_refinement_solution.m} aufgerufen.


\subsubsection{Lokale Steifigkeitsmatrix und Assemblierung}

Wie schon in Kapitel \ref{kap:2.3} Beispiel \ref{bsp:2.26} beschrieben ist es für die Implementierung notwendig eine Verallgemeinerung zur Berechnung der Steifigkeitsmatrix herzuleiten. Die Idee ist an selber Stelle kurz vorgestellt worden; wir wollen nun angelehnt an Abbildung \ref{abb:2.5} die Formeln für die lokale Steifigkeitsmatrix eines beliebigen Elementes $T$ über das Referenzelement
\[
	\widetilde T = \{(\xi ,\eta)\in \R^2\mid 0\le \xi \le 1, 0\le \eta \le 1-\xi\}
\]
herleiten. Für die affine Transformation vom Referenzelement auf ein beliebiges Dreieck $T$ gelten die Bedingungen
\begin{subequations}\label{eq:5.1}
\begin{align}\label{eq:5.1a}
	x & = x_1 + (x_2-x_1)\xi + (x_3-x_1) \eta \, ,\\
	\label{eq:5.1b}
	y & = y_1 + (y_2-y_1)\xi + (y_3-y_1)\eta \, ,
\end{align}
\end{subequations}
wobei $(x_i,y_i),i=1,2,3,$ die Eckpunkte des Dreiecks $T$ sind (vgl. auch Abbildung \ref{abb:2.5}). Mit den Bedingungen \eqref{eq:5.1} erhalten wir dann die \idx{Funktionaldeterminante}
\begin{align}\label{eq:5.2}
\begin{aligned}
	J & = \det \begin{pmatrix}
				x_2-x_1 & x_3-x_1 \\
				y_2-y_1 & y_3-y_1
			\end{pmatrix}  \\
			& = (x_2-x_1)(y_3-y_1) - (x_3-x_1)(y_2-y_1)\, .
\end{aligned}
\end{align}
Dabei gilt $J>0$, da die Orientierung der Eckpunkte von $\widetilde T$ zu $T$ erhalten bleibt. Wir bezeichnen mit $\tilde u : \widetilde T \ra \R$ eine Funktion über dem Referenzdreieck und mit $u:T\ra \R$ die dazugehörige Funktion über dem allgemeinen Element $T$. Dann gilt zwischen $\tilde u $ und $u$ der Zusammenhang
\begin{align*}
	u(x,y) & = u(x_1 + (x_2-x_1)\xi + (x_3-x_1) \eta, y_1 + (y_2-y_1)\xi + (y_3-y_1)\eta) \\
	& =\tilde u(\xi,\eta) \, .
\end{align*}
Wir rechnen also unter Verwendung der \idx{Kettenregel} nach, dass
\begin{align*}
	\begin{pmatrix} \tilde u_\xi \\ \tilde u_\eta \end{pmatrix} & = \begin{pmatrix}
				x_2-x_1 & y_2-y_1 \\
				x_3-x_1 & y_3-y_1
			\end{pmatrix}   \begin{pmatrix}  u_x \\  u_y \end{pmatrix} \\
	\Llra  \begin{pmatrix}  u_x \\  u_y \end{pmatrix} & =\frac 1J \begin{pmatrix}
				y_3-y_1 & y_1-y_2 \\
				x_1-x_3 & x_2-x_1
			\end{pmatrix} \begin{pmatrix} \tilde u_\xi \\ \tilde u_\eta \end{pmatrix}
\end{align*}
gilt. Damit können wir $\nabla u \nabla v$ auf dem Referenzelement $\widetilde T$ ausdrücken durch
\begin{align*}
	\begin{pmatrix} u_x \\ u_y \end{pmatrix}^T \begin{pmatrix} v_x \\ v_y \end{pmatrix} & = \frac 1{J^2} \begin{pmatrix} \tilde u_\xi \\ \tilde u_\eta \end{pmatrix}^T \begin{pmatrix}
				y_3-y_1 & x_1-x_3 \\
				y_1-y_2 & x_2-x_1
			\end{pmatrix}  \begin{pmatrix}
				y_3-y_1 & y_1-y_2 \\
				x_1-x_3 & x_2-x_1
			\end{pmatrix}  \begin{pmatrix} \tilde v_\xi \\ \tilde v_\eta \end{pmatrix} \\
	& = \frac 1{J^2} \begin{pmatrix} \tilde u_\xi \\ \tilde u_\eta \end{pmatrix}^T     \begin{pmatrix}
				a & b \\
				b & c
			\end{pmatrix}    \begin{pmatrix} \tilde v_\xi \\ \tilde v_\eta \end{pmatrix} \\
	\text{mit }\qquad  & \!\!\!\!\!\!\!\!\!\!  \left\{ \begin{aligned}
			a & = (y_3-y_1)^2+ (x_3-x_1)^2 \\
			b  & = -( (y_3-y_1)(y_2-y_1) + (x_3-x_1)(x_2-x_1)) \\
			c & = (y_2-y_1)^2 + (x_2-x_1)^2
		\end{aligned} \right. \, .
\end{align*}

Insgesamt können wir nun die lokale Bilinearform $a_T(u,v)$ von einem allgemeinen Element $T$ auf das Referenzelement $\widetilde T$ transformieren.
\begin{align}\notag
	a_T (u,v) &\coloneqq \int_T \nabla u \nabla v \, dx dy= \int_T u_xv_x + u_yv_y \, dxdxdy \\
	\notag
	& = \int_{\widetilde T} \frac 1{J^2} (a\, \tilde u_\xi \tilde v_\xi + b\, (\tilde u_\xi \tilde v_\eta + \tilde u_\eta \tilde v_\xi) + c\,  \tilde u_\eta \tilde v_\eta ) J \, d\xi d\eta \\
	\notag
	& = \frac 1J \int_{\widetilde T} a\, \tilde u_\xi \tilde v_\xi + b\, (\tilde u_\xi \tilde v_\eta + \tilde u_\eta \tilde v_\xi) + c\,  \tilde u_\eta \tilde v_\eta  \, d\xi d\eta \\
	\label{eq:5.3}
	& = \frac 1J (a \, S_1 + b \, S_2 + c \, S_3)
\end{align}
Die Matrizen $S_k,k=1,2,3$, beinhalten dann die Anteile der einzelnen Summanden des Integranden aus dem oberen Integral von den jeweiligen lokalen Ansatzfunktionen. Eine Basis der linearen Ansatzfunktionen ist auf dem Referenzelement $\widetilde T$ beispielsweise von der Form
\begin{align}\label{eq:5.4}
	& \varphi_1(\xi,\eta) = 1-\xi-\eta \, ,\quad \varphi_2(\xi,\eta) = \xi \, , \quad \varphi_3(\xi,\eta) = \eta \, .
\end{align}
Damit lassen sich die Einträge von $S_1 \eqqcolon S = (s_{ij})_{i,j=1,2,3}$ berechnen durch
\begin{align}\label{eq:5.5}
	s_{ij} = \int_{\widetilde T} \varphi_{i,\xi} \, \varphi_{j,\xi} \, d\xi d\eta \, ,
\end{align}
d.h. mit \eqref{eq:5.4} berechnen wir die Gradienten
\begin{align*}
	  \nabla \varphi_1(\xi,\eta) = (-1,-1) \, , \quad \nabla \varphi_2(\xi,\eta) = (1,0) \, , \quad \nabla \varphi_3(\xi,\eta) = (0,1)
\end{align*}
und damit ergibt sich beispielsweise
\[
	s_{11} = \int_0^1 \int_0^{1-\xi} \varphi_{1,\xi} \, \varphi_{1,\xi} \, d\eta d\xi = \int_0^1 \int_0^{1-\xi} d\eta d\xi = \frac 12 \, .
\]
Analog lassen sich mit \eqref{eq:5.5} die weiteren Matrixeinträge aus $S_1$ berechnen bzw. mit \eqref{eq:5.3} und den dazugehörigen Formeln auch $S_2$ und $S_3$:
\begin{align*}
	S_1 = \begin{pmatrix}
				\frac 12 & -\frac 12 & 0 \\
				-\frac 12 & \frac 12 & 0 \\
				0 & 0 & 0
			\end{pmatrix} , \quad 
			S_2 = \begin{pmatrix}
				 1 & -\frac 12 & -\frac 12 \\
				-\frac 12 & 0 & \frac 12 \\
				-\frac 12 & \frac 12 & 0
			\end{pmatrix} , \quad 
			S_3 = \begin{pmatrix}
				\frac 12 & 0 & -\frac 12 \\
				0 & 0 & 0 \\
				-\frac 12 & 0 & \frac 12
			\end{pmatrix} .
\end{align*}
Insgesamt ist dann mit \eqref{eq:5.3} die lokale Steifigkeitsmatrix für die linearen Ansatzfunktionen eines beliebigen Elementes gegeben durch
\begin{align*}
	S = \frac 1{2J}\begin{pmatrix}
				a+2b+c & -a-b & -b-c \\
				-a-b & a & b \\
				-b-c & b & c
			\end{pmatrix}.
\end{align*}

Betrachten wir nun eine Basis von quadratischen Ansatzfunktionen auf $\widetilde T$, d.h.
\[
	\varphi_4(\xi,\eta) = 4\xi \, (1-\xi-\eta) \, , \quad \varphi_5(\xi,\eta) = 4\xi\eta \, , \quad \varphi_6(\xi,\eta) = 4\eta\, (1-\xi-\eta) \, ,
\]
dann können wir auch für diese nach Berechnung der Gradienten mit \eqref{eq:5.3}, \eqref{eq:5.5} und den analog resultierenden Formeln eine lokale Steifigkeitsmatrix $\bar S$ aufstellen. Diese hat dann die Form
\[
	\bar S  = \frac 4{3J}\begin{pmatrix}
				a+b+c & -b-c & b \\
				-b-c & a+b+c & -a-b \\
				b & -a-b & a+b+c
			\end{pmatrix}.
\]

Um das lokale Defektproblem \eqref{eq:4.9} zu lösen, ist nicht nur das Aufstellen von Steifigkeitsmatrizen über quadratische Ansatzfunktionen notwendig, sondern auch das Berechnen der rechten Seite $\rho_{\mcal S}(v) = (f,v)-a(u_{\mcal S},v)$. Auch hier können wir für zumindest einen Teil der Summe einen lokalen Vektor bestimmen. Analog zu \eqref{eq:5.3} rechnen wir nach, dass
\begin{align}\notag
	a_T(u_{\mcal S},v) & = \int_T \nabla u_{\mcal S} \nabla v \, dx dy \\
	\notag
	& = \int_{\widetilde T} (a\, \tilde u_{\mcal S,\xi} + b \, \tilde u_{\mcal S,\eta})\tilde v_\xi + (b\,  \tilde u_{\mcal S,\xi}+ c \, \tilde u_{\mcal S,\eta}) \tilde v_\eta \, d\xi d\eta \\
	\label{eq:5.6}
	& = \lambda \, \bs w_1 + \mu \, \bs w_2 
\end{align}
gilt, wobei $\bs w_i,i = 1,2$ Vektoren sind, deren Einträge gerade die lokalen Anteile an dem $i$-ten Summanden des Integranden bzgl. der quadratischen Ansatzfunktionen enthalten, analog zu den Matrizen $S_k$. Wenn wir also eine Lösung $u(x,y) = \alpha x+\beta y + \gamma$ auf $T$ gegeben haben, der Gradient $\nabla u(x,y) = (\alpha,\beta)$ noch auf $\widetilde T$ zu transformieren. Mit der affine Transformation \eqref{eq:5.1} lässt sich leicht nachrechnen, dass dann
\[
	\nabla \tilde u_{\mcal S} (\xi,\eta) = (\alpha\, (x_2-x_1)+\beta \, (y_2-y_1), \alpha \, (x_3-x_1) + \beta \, (y_3-y_1)) \eqqcolon (\tilde \alpha, \tilde \beta)
\]
ist. Damit ist $\lambda = a\tilde \alpha + b \tilde\beta$ und $\mu = b \tilde \alpha + c \tilde \beta$. Die Werte $\alpha, \beta$ des Gradienten auf $T$ lassen sich durch die gegebenen Funktionswerte von $u_{\mcal S}$ an den Eckpunkten des Elementes $T$ berechnen, da $u_{\mcal S}$ auf $T$ linear ist. Dies wird in der Datei {\ttfamily grad_u.m} verwendet. Die Vektoren $\bs w_1,\bs w_2$ ergeben sich nun aus
\[
	\bs w_1 = \int_0^1\int_0^{1-\xi} \begin{pmatrix}
								\varphi_{4,\xi} \\
								\varphi_{5,\xi} \\
								\varphi_{6,\xi}
							\end{pmatrix} d\eta d\xi =  \begin{pmatrix}
								0 \\
								\frac 23 \\
								-\frac 23
							\end{pmatrix} \, , \quad 
							\bs w_2 =  \begin{pmatrix}
								-\frac 23 \\
								\frac 23 \\
								0
							\end{pmatrix} \, .
\]
Damit ist der lokale Vektor aus \eqref{eq:5.6} bzgl. eines beliebigen Dreiecks $T$ gegeben durch
\[
	\bar{\bs \rho} = \lambda \begin{pmatrix}
								0 \\
								\frac 23 \\
								-\frac 23
							\end{pmatrix}+\mu  \begin{pmatrix}
								-\frac 23 \\
								\frac 23 \\
								0
							\end{pmatrix} = \begin{pmatrix}
					-\frac 23 (b\, \tilde \alpha + c\, \tilde\beta) \\
								\frac 23 (a\, \tilde\alpha + b\, (\tilde \beta + \tilde \alpha) + c\, \tilde\beta) \\
								-\frac 23 (a\,  \tilde \alpha + b\,  \tilde \beta)
							\end{pmatrix}.
\]

Die hier hergeleiteten Formeln für die lokalen Steifigkeitsmatrizen bzw. Vektoren werden in {\ttfamily local_mat.m} verwendet und ermöglichen einerseits eine leichtere Implementierung, da wir nur über ein Element integrieren müssen, andererseits erzeugen sie aber auch einen schnelleren Programmcode, weil wir keine Integrationen mehr durchführen müssen, um die Steifigkeitsmatrizen aufzustellen.

Die globalen Steifigkeitsmatrizen erzeugen wir dann durch Assemblierung, d.h. wir berechnen für alle Elemente die lokalen Steifigkeitsmatrizen und ordnen mit dem global-local node ordering jedem globalen Knoten in einem Element den jeweils lokalen zu. Damit addieren wir in der globalen Steifigkeitsmatrix im zugehörigen Eintrag den jeweiligen lokalen Eintrag auf. Dieses Vorgehen wir im Programmcode {\ttfamily assemble.m} verwendet. Natürlich ist die Assemblierung für den globalen Vektor aus $\rho_{\mcal S}$ analog.


\subsubsection{Lösung der quadratischen Programme}

Wie wir in Anhang \ref{anhang:B} beschrieben können wir die konvexen quadratischen Programme, die sich durch Hindernis- oder Kontaktproblem erzeugen, mit dem \idx{Active-Set-Algorithmus} lösen. Dieser ist unter anderem in der Matlab-Funktion {\ttfamily quadprog} hinterlegt. Allerdings ist der Active-Set-Algorithmus mit steigender Dimension schnell sehr langsam. Das Funktionen-Paket {\ttfamily quadprog} beinhaltet jedoch auch die \textit{\idx{Innere-Punkte-Methode}} (kurz: IPM), welche für große Systeme wesentlich schneller ist als die Active-Set-Methode. Daher verwenden wir die IPM zur Lösung der quadratischen Programme. Außerdem verwenden wir dazu \textit{Sparse}-Matrizen und die Option \textit{large-scale}, um die Auswertung zu beschleunigen.


\subsubsection{Quadratur auf einem Dreieck}


\subsubsection{Berechnung der einzelnen Knotenmengen}


\subsubsection{Berechnung des Normalenflußes}





\subsubsection{meine Stichpunkte}

\begin{itemize}
\item Gründe warum wo was.
\item Warum Verwendung von Sparse, IPM und large scale?
\end{itemize}



\newpage

%%% Local Variables: 
%%% mode: latex
%%% TeX-master: "Skript"
%%% End: 

\newchapter{Validierung}

\begin{itemize}
\item numerisches Beispiel (Problemstellung) $\ra$ vielleicht mit Kontakt und nur Hindernis
\item Vergleich mit Analytischer Lösung?! (Tabelle mit Ergebnissen) $\ra$ Ergebnisse diskutieren
\end{itemize}

\section{Numerisches Beispiel zum Hindernisproblem}

\section{Numerisches Beispiel zum Kontaktproblem}

\newpage

%%% Local Variables: 
%%% mode: latex
%%% TeX-master: "Skript"
%%% End: 

\newchapter{Zusammenfassung und Ausblick}

\begin{itemize}
\item kurz einleiten, worum es ging (Einleitung in einem Absatz zusammenfassen) 
\item Was ist rausgekommen?!
\item Ausblick: Was ist noch offen geblieben, was kann man noch machen...
\end{itemize}

\newpage

%%% Local Variables: 
%%% mode: latex
%%% TeX-master: "Skript"
%%% End: 


% Literatur

\thispagestyle{fancy}{
	\lhead{\sl Literaturverzeichnis}
	\rhead{}
}
\addcontentsline{toc}{chapter}{Literaturverzeichnis}
\nocite{*}
\bibliographystyle{alphadin}
\bibliography{Arbeit}
\newpage

\pagestyle{fancy}{
	\rhead{}
	\lhead{\sl\thechapter. \nameofchapter}
	%\renewcommand{\headrulewidth}{0.4pt}
	\renewcommand{\headheight}{14pt}
	\renewcommand{\footrulewidth}{0.4pt}
	\cfoot{\thepage}
}

\appendix

\newchapter{Funktionalanalysis}
\label{anhang:A}

\section{Sobolev-Räume}
\label{anhang:A.1}

Sei im Weiteren $\emptyset \not= \Omega \subset \R^n$. Wir definieren den Sobolev-Raum allgemein wie folgt (vgl. \cite{BraeFEM} Kaptitel II, \S 2 und \cite{Walker} Kapitel 6).

\begin{defi}\label{def:A.1}
  Seien $1\leq p\leq\infty$ und $m\in\N$. Die Menge
  \[
  W_p^m(\Omega):=\left(
    \{u\in L_p(\Omega)\with\partial^\alpha u\in L_p(\Omega)\,\fa\,\abs a\leq m\}
    , \norm \cdot_{W_p^m}
  \right)
  \]
  heißt \textit{\idx{Sobolev-Raum}} der Ordnung $m$. Dabei ist 
  \[
  \norm u_{W_p^m}:=\norm u_{W_p^m(\Omega)}:=
  \left(
    \sum_{\abs\alpha\leq m}\norm{\partial^\alpha u}_{L_p}^p
  \right)^{\frac 1p},
  \]
  wenn $1\leq p<\infty$. Im Fall $p=\infty$ ist $\norm u_{W_p^m}:=\max_{\abs\alpha\leq m}\norm{\partial^\alpha u}_\infty$. 
  
  Weiterhin bezeichne $L_p(\Omega)$ den \textit{\idx{Lebesgue-Raum}}, d.h. den Raum der messbaren Funktionen, deren $p$-te Potenz Lebesgue-integrierbar über $\Omega$ ist, d.h.
  \[
  	L_p(\Omega) := \left( \{u : \Omega \ra \R \with f \text{ messbar}, \norm{\cdot}_{L_p} < \infty\},\norm{\cdot}_{L_p} \right) \, ,
  \]
  wobei $\norm{u}_{L_p} := \norm{u}_{L_p(\Omega)} =\norm{u}_{W^0_p}$.
\end{defi}

\begin{defi}\label{def:A.2}
Der Raum
\[
	\mcal D(\Omega) := C_c^\infty (\Omega)= \{ \varphi \in C^\infty(\Omega) \with \supp (\varphi) \subset\subset \Omega \}
\]
heißt der \textit{\idx{Raum der Testfunktionen}}, wobei $K\subset\subset \Omega :\Lra  \bar K \subset \Omega$ kompakt.
\end{defi}

\begin{bem}\label{bem:A.3}
 Seien $u\in W_p^m(\Omega)$, $\varphi\in\D(\Omega)$ und $\alpha\in\N^n$ mit $\abs\alpha\leq m$. Dann  bezeichnen wir $v = \partial^\alpha u$ als \textit{\idx{schwache Ableitung}} von $u$, wenn gilt
    \[
      \int_\Omega v\cdot\varphi \, d x=(-1)^{\abs\alpha}\int_\Omega u\cdot\partial^\alpha\varphi \,  d x\, .
    \]
\end{bem}

\begin{bsp}\label{bsp:A.4}
Es sei $\Omega = (-1,1) \subset \R$ und $u (x) = \abs x \in L_2(\Omega)$. Betrachten wir $v (x) = \sign (x)$, so ergibt sich für $\varphi \in \mcal D(\Omega)$
\begin{align*}
	\int_\Omega v \cdot \varphi \, dx & = \int_{-1}^0 -1 \cdot \varphi(x) \, dx + \int_0^1 1\cdot \varphi(x) \, dx  \\
	&= -x\varphi(x)\Big|_{-1}^0 - \int_{-1}^0 -x \varphi'(x) \, dx +  x \varphi(x)\Big|_0^1 -\int_0^1 x \varphi'(x) \, dx \\
	& = - \int_{-1}^1 \abs x \varphi'(x) \, dx = (-1)^1 \int_\Omega u \cdot \varphi' \, dx \, ,
\end{align*}
da $\varphi(-1) = \varphi(1) = 0$. Also ist $v = \partial u$ und somit $u \in W^1_2(\Omega)$. Analog kann man nachrechnen, dass 
\[
	\int_\Omega v \cdot \varphi' \, dx = -2\varphi(0)
\]
ist und somit $u$ nicht zweimal schwach ableitbar ist, d.h. $u \not \in W^2_2(\Omega)$.
\end{bsp}

Wir wollen in der Theorie der Finiten Elemente Methode vor allem Sobolev-Räume über dem Raum $L_2(\Omega)$ betrachten, daher ist folgender Satz essentiell.

\begin{satz}\label{satz:A.5}
Es seien $1\le p \le \infty$ und $m \in \N$. Dann gilt:
\begin{enumerate}[\rm (a)]
\item $W_p^m(\Omega)$ ist ein Banachraum.
\item $H^m(\Omega) := W_2^m(\Omega)$ ist ein Hilbertraum mit Skalarprodukt
\[
	(u,v)_m := (u,v)_{H^m(\Omega)} := \sum_{\abs \alpha \le m} (\partial^\alpha u,\partial^\alpha v)_0\quad \forall \, u,v \in H^m(\Omega) \, ,
\]
wobei
\[
	(u,v)_0 := (u,v)_{L_2(\Omega)} := \int_\Omega uv \, dx \, .
\]
\end{enumerate}
\end{satz}

\begin{bem}\label{bem:A.6}
\begin{enumerate}[(a)]
\item Die Norm auf $H^m(\Omega)$ ergibt sich analog zur Norm des allgemeinen Sobolev-Raumes durch das Skalarprodukt, d.h. $\norm{u}_m: = \norm u_{H^m(\Omega)} := \norm u_{W^m_2}$.
\item Analog dazu definieren wir die Halbnorm $\abs\cdot_m$ auf $H^m$ wie folgt:
\[
		\abs{u}_m: = \abs u_{H^m(\Omega)} :=  \left(
    \sum_{\abs\alpha= m}\norm{\partial^\alpha u}^2_{L_2}
  \right)^{\frac 12}.
\]
\end{enumerate}
\end{bem}

\begin{defi}\label{def:A.7}
Der Raum $H^m_0(\Omega)$ ist die Vervollständigung von $\mcal D(\Omega)$ bzgl. der Norm $\norm\cdot_m$.
\end{defi}

\begin{bem}\label{bem:A.8}
Die Funktionen $u \in H^m_0(\Omega)$ können als die Funktionen $u \in H^m(\Omega)$ mit $u = 0$ auf $\partial \Omega$ aufgefasst werden. Weiter ist $H^m_0(\Omega)$ ein abgeschlossener Unterraum von $H^m(\Omega)$ (vgl. auch \cite{Walker} Bemerkung 6.7).
\end{bem}



\section{Optimalitätskriterien}
\label{anhang:A.2}

Zunächst definieren wir einen verallgemeinerten Begriff der Richtungsableitung, der auch auf unendlich dimensionalen Vektorräumen existiert.

\begin{defi}\label{def:Gateaux-Ableitung}
Es seien $V$ ein Vektorraum, $M\subset V$ und $W$ ein normierter Raum, sowie $F: M \ra W$ eine Abbildung, $x_0 \in M$ und $v \in V$. Dann heißt $F$ \textit{\idx{Gâteaux-differenzierbar}} (bzw. in Richtung $v$ an der Stelle $x_0$ differenzierbar), falls es ein $\eps >0$ mit $[x_0-\eps v, x_0 + \eps v] \subset M$ gibt und der Grenzwert
\begin{align}\label{eq:A.1}
	\mscr D_v F (x_0) :=\frac d{d t} F(x_0+tv)\Big|_{t=0} := \lim_{t\ra 0} \frac{F(x_0+tv)-F(x_0)}t
\end{align}
in $W$ existiert. $\mscr D_vF(x_0)$ heißt dann \textit{\idx{Gâteaux-Ableitung}} von $F$ an der Stelle $x_0$ in Richtung $v$.

Falls wir nur $[x_0,x_0 + \eps v] \subset M$ voraussetzen, so können wir in \eqref{eq:A.1} $\lim_{t\ra 0}$ durch $\lim_{t \ra + 0}$ ersetzen. Dann nennen wir \eqref{eq:A.1} die \textit{rechtsseitige Gâteaux-Ableitung}\index{Gâteaux-Ableitung!rechtsseitig} und bezeichnen diese mit $\mscr D^+_v F(x_0)$.
\end{defi}

Für die Variationsrechnung sind folgende zwei Sätze für uns von besonderer Bedeutung.

\begin{satz}\label{satz:A.10}
\textnormal{(Charakterisierungssatz der konvexen Optimierung)} Es seien $M \subset V$ eine konvexe Menge, $V$ ein Vektorraum und $F : M \ra \R$ ein konvexes Funktional. Dann gilt für $x_0, x \in M$:

$x_0$ ist Lösung von $\min\limits_{x \in M} F(x)$ genau dann, wenn für alle $x \in M$ gilt
\[
	\mscr D^+_{x-x_0} F(x_0) \ge 0 \, .
\]
\end{satz}

\begin{proof}
Siehe \cite{GopRieTam}, Kapitel 3.3.3, Satz 3.34.
\end{proof}

\begin{satz}
Es sei $U\subset V$ ein (Unter-)Vektorraum, $V$ ein Vektorraum und $F: U \ra \R$ eine \idx{Gâteaux-differenzierbar}e konvexe Funktion. Dann ist $x_0 \in U$ genau dann Lösung von $\min\limits_{x \in U} F(x)$, wenn für alle $u \in U$ gilt
\[
	\mscr D_u F(x_0) = 0 \, .
\]
\end{satz}

\begin{proof}
Siehe \cite{GopRieTam}, Kapitel 3.3.3, Satz 3.35.
\end{proof}


\section{Konvergenzbegriffe}
\label{anhang:A.3}

\begin{defi}\label{defi:A.12}
Es sei $m\in \N, 1\leq p < \infty, 1 = \frac 1 p + \frac 1 {p'}$.
\begin{enumerate}[(a)]
\item Eine Folge $(u_j)$ in $L_p$ konvergiert schwach gegen $u \in L_p(\Omega)$
\begin{align*}
	&: \Longleftrightarrow u_j \rightharpoonup u \text{ in } L_p(\Omega) \\
	& : \Longleftrightarrow \, \fa \, v \in L_{p'}(\Omega) : \int_\Omega u_j v \d x \longrightarrow \int_\Omega uv \d x \text{ in } \K \, .
\end{align*}
\item Eine Folge $(u_j) \in W^m_p(\Omega)$ konvergiert schwach gegen $u \in W_p^m (\Omega)$
\begin{align*}
	& : \Longleftrightarrow u_j \rightharpoonup u \text{ in } W_p^m(\Omega) \\
	& : \Longleftrightarrow \partial^\alpha u_j \rightharpoonup \partial^\alpha u \text{ in } L_p(\Omega) \, \forall \, \abs\alpha \leq m \, .
\end{align*}
\end{enumerate}
\end{defi}


\begin{bem}
\label{bem:A.13}
Sei $1\leq p < \infty, m \in \N$, dann ist:
\begin{enumerate}[(a)]
\item Ist $u_j \ra u$ in $W^m_p(\Omega)$, dann folgt $u_j \rightharpoonup u$ in $W^m_p(\Omega)$, d.h. "`starke Konvergenz ist stärker als schwache Konvergenz"'.
\begin{proof}
$\fa \, v \in L_{p'}(\Omega), \abs \alpha \leq m$ gilt
\[
	\Abs{\int_\Omega (\partial^\alpha u_j - \partial^\alpha u) v \d x} \stackrel[\scriptsize\text{Hölder}]{}\leq \norm v_{L_{p'}(\Omega)} \norm{\partial^\alpha u_j -\partial^\alpha u }_{L_p(\Omega)} \longrightarrow 0 \, .\qedhere
\]
\end{proof}
\item Sei $1 < p < \infty, (u_j)\subset W^m_p(\Omega)$ beschränkt (bzgl. $\norm{\, \cdot\, }_{W^m_p}$), dann folgt, dass eine Teilfolge $(u_{j'})$ und ein $u \in W^m_p(\Omega)$ existiert, so dass $u_{j'} \rightharpoonup u$ in $W^m_p(\Omega)$, d.h. "`beschränkte Folgen sind relativ schwach kompakt"'.
\begin{proof}
Vgl. \cite{Rudin}.
\end{proof}
\item Es sei $M\subset W^m_p(\Omega)$ konvex und abgeschlossen (bzgl. $\norm{\, \cdot\, }_{W^m_p}$), sowie $(u_j) \subset M$ mit $u_j \rightharpoonup u $ in $W^m_p(\Omega)$, dann ist $u \in M$, d.h. "`abgeschlossene konvexe Mengen sind schwach abgeschlossen"' (Theorem von Mazun; ohne Beweis, vgl. \cite{Rudin}).
\item Es sei $u_j \rightharpoonup u $ in $W^p_m (\Omega)$, dann folgt $(u_j)$ ist beschränkt in $W^m_p(\Omega)$ (bzgl. $\norm{\, \cdot\, }_{W^m_p}$), d.h. "`schwach konvergente Folgen sind beschränkt"'.
\begin{proof}
Theorem von Mackey, vgl. \cite{Rudin}.
\end{proof}
\item $u_ j \rightharpoonup u $ in $W^m_p(\Omega), u_j \rightharpoonup v$ in $W^m_p(\Omega)$, dann gilt $u=v$, d.h. "`Grenzwerte von schwach konvergenten Folgen sind eindeutig"'.
\begin{proof}
Aus dem Hauptsatz der Variationsrechnung folgt die Behauptung.
\end{proof}
\item Sei $u_j \rightharpoonup u $ in $W^m_p(\Omega)$, dann folgt $\norm u_{W^m_p(\Omega)} \leq \lim \inf \norm{u_j}_{W^m_p(\Omega)}$.
\end{enumerate}
\end{bem}

\begin{theorem}\label{theorem:A.14}
In einem reflexiven Raum $V$\index{reflexiver Raum}, d.h. der Bidualraum $V''$ ist isomorph zu $V$, besitzt jede beschränkte Folge $(v_n)_{n\in \N}$ eine schwach konvergente Teilfolge $(v_{n_j})$.
\end{theorem}

\begin{proof}
Der Beweis befindet sich in \cite{Werner} Kapitel III, Theorem 3.7.
\end{proof}

\begin{bem}\label{bem:A.15}
Jeder Hilbertraum $H$ ist reflexiv.
\end{bem}

\begin{proof}
Dies folgt aus dem Darstellungssatz von Riesz (Satz \ref{satz:2.14}).
\end{proof}










\newchapter{Optimierung}
\label{anhang:B}

\section{Quadratische Programmierung}
\label{anhang:B.1}

Um im folgenden die Idee des Algorithmus zu verstehen, führen wir zunächst grundlegende Begriffe ein. Ein quadratisches Problem mit Gleichungs- und Ungleichungsnebenbedingungen ist von der Form
\begin{align}\label{eq:QP}
\begin{aligned}
	\min_{\bs x} & \quad q(\bs x) = \frac 1 2 \bs x^T G\bs x + \bs x^T \bs c \\
	\text{s.t.} & \quad \bs a_i^T \bs x = b_i \, , \quad i \in \mcal E, \\
	& \quad \bs a_i^T \bs x\ge b_i \, , \quad i \in \mcal I,
\end{aligned}
\end{align}
wobei $\mcal E$ und $ \mcal I$ die Indexmengen der Gleichungs- und Ungleichungsnebenbedingungen darstellen und $\bs c,\bs x,\bs a_i \in \R^n, b_i \in \R, i \in \mcal E \cup \mcal I$, sowie $G$ eine symmetrische $(n\times n)$-Matrix ist, welche die Hesse-Matrix des Problems darstellt. Damit ist die Hesse-Matrix konstant und daher das Problem konvex, wenn $G$ positiv semidefinit ist. (Ist $G$ positiv definit, so nennen wir das Problem strikt konvex. Wenn $G$ indefinit ist, ist \eqref{eq:QP} "`nicht konvex"'.)

Da sonst das quadratische Problem (und damit der Active-Set Algorithmus) zu kompliziert wird, betrachten wir hier nur den konvexen Fall. Für diesen Fall können wir leicht zeigen, dass eine Lösung $\bs x^*$, die die Bedingungen 1. Ordnung erfüllt, auch globale Lösung des Problems ist (s. Theorem \ref{A.1}). Anschaulich kann es im indefiniten Fall mehrere optimale Punkte geben, die voneinander getrennt liegen, d.h. die Menge der optimalen Punkte ist nicht zusammenhängend, wodurch das Auffinden des globalen Minimums erschwert wird.

Die notwendigen Bedingungen 1. Ordnung sind die KKT-Bedingungen und können hier angewendet werden, da die Restriktionen und die Zielfunktion stetig differenzierbar sind. Die Lagrangefunktion $\mcal L$ für das quadratische Problem ist
\begin{align}
	\mcal L(\bs x,\bs\lambda) = \frac 1 2 \bs x^T G \bs x + \bs x^T \bs c- \sum_{i \in \mcal I \cup \mcal E} \lambda_i (\bs a_i^T \bs x-b_i) \, .
\end{align}
Damit ergeben sich –  vgl. \cite{NocWri}, Theorem 12.1 – mit der Menge der aktiven Nebenbedingungen $\mcal A(\bs x^*) = \{i\in \mcal E \cup \mcal I : \bs a_i^T \bs x^* = b_i\}$ die KKT-Bedingungen
\begin{align}\label{eq:KKT}
\begin{aligned}
	\nabla_{\bs x} \mcal L(\bs x^*,\bs \lambda^*) & = G\bs x^*+\bs c-\sum_{i \in \mcal A(\bs x^*)} \lambda^*_i \bs a_i  = 0 \, , \\
	\bs a_i^T \bs x^* &  = b_i \, , \quad \forall \, i \in \mcal A(\bs x^*) , \\
	\bs a_i^T \bs x^* &  \ge b_i \, , \quad \forall \, i \in \mcal I \setminus \mcal A(\bs x^*) ,\\
	\lambda_i^* & \ge 0 \, , \quad \, \forall i \in \mcal I \cap \mcal A(\bs x^*) .
\end{aligned}
\end{align}
Hierbei ist $\bs x^*$ Lösung von \eqref{eq:QP} und erfüllt die LICQ-Bedingung; $\bs\lambda^*$ ist dazugehöriger optimaler Lagrange-Multiplikator. In \eqref{eq:KKT} wird die Komplementaritätsbedingung $\lambda^*_i c_i(\bs x^*) = 0$ impliziert durch $\lambda_i^* = 0 \, \forall \, i \not\in \mcal A(\bs x^*)$.

\begin{theorem}\label{theorem:B.1}
Wenn $\bs x^*$ die Bedingungen \textnormal{\eqref{eq:KKT}} erfüllt mit $\lambda_i^*,i \in \mcal A(\bs x^*)$ und $G$ ist positiv semidefinit, dann ist $\bs x^*$ eine globale Lösung von \textnormal{\eqref{eq:QP}}.
\end{theorem}

\begin{proof}
Wenn $\bs x$ ein beliebiger weiterer zulässiger Punkt für (1.1) ist, gelten die Restriktionen $\bs a_i^T\bs x = b_i,  i \in \mcal E$,  sowie $\bs a_i^T \bs x \ge b_i, i \in \mcal I \cap \mcal A(\bs x^*)$ für $\bs x$ und damit gilt zusammen mit der ersten Bedingung von \eqref{eq:KKT}, dass
\[
	(\bs x-\bs x^*)^T (G\bs x^*+\bs c) = \sum_{i \in \mcal E} \underbrace{\lambda^*_i \bs a_i^T (\bs x-\bs x^*)}_{\ge 0} + \sum_{i \in \mcal A(x^*)\cap \mcal I} \underbrace{\lambda^*_i \bs a_i^T (\bs x-\bs x^*)}_{\ge 0} \ge 0 \, .
\]
Dann drücken wir $q(\bs x)$ durch $q(\bs x^*)$ aus und wenden die obere Ungleichung sowie die positive Semidefinitheit für $G$ an, um zu zeigen, dass $q(\bs x) \ge q(\bs x^*)$ ist. Damit ist $\bs x^*$ globale Lösung des quadratischen Problems.
\end{proof}

Daher ist im positiv semidefiniten Fall gesichert, dass ein optimaler Punkt auch gleichzeitig globale Lösung ist.


\section{Active Set-Methode für konvexe QPs}
\label{anhang:B.2}

Wenn wir eine Lösung $\bs x^*$ für das Problem \eqref{eq:QP} kennen, so ist auch die Menge der aktiven Nebenbedingungen $\mcal A(x^*)$ bekannt und wir können \eqref{eq:QP} vereinfachen zum Optimierungsproblem
\begin{align}\label{eq:active}
	\min_{\bs x} & \quad q(\bs x) = \frac 1 2 \bs x^T G \bs x+\bs x^T \bs c \, , \quad	\text{s.t.} \quad \bs a_i^T\bs x = b_i \, , \quad i \in \mcal A(\bs x^*)\, .
\end{align}
Dieses könnten wir dann beispielsweise mit direkten Verfahren wie der Schur-Komplement-Methode oder der Nullraum-Methode lösen. Natürlich ist die optimale Lösung zu Beginn noch nicht bekannt und damit auch nicht die aktiven Restriktionen. Jedoch können wir diese Idee für die Active-Set-Methode verwenden.

Das Hauptziel der Active-Set-Methode ist, die Menge der aktiven Restriktionen bzgl. der optimalen Lösung zu finden, wobei wir hier die primale Variante betrachten wollen, in der die Approximierte $\bs x_k$ zulässig bzgl. des primalen Problems ist. 

Die Grundidee ist, ein quadratisches Teilproblem zu lösen, bei dem wir bestimmte Nebenbedingungen aus Problem \eqref{eq:QP} bzgl. $\mcal I$ als aktiv annehmen. Die dadurch beschriebene Indexmenge der aktiven Restriktionen für $\bs x_k$ im $k$-ten Schritt heißt \textit{working set} und kann wie folgt beschrieben werden
\[
	\mcal W_k = \{ i \, | \, \bs a_i^T \bs x_k = b_i,  i \in \mcal E \cup \mcal J, \mcal J \subset \mcal I\} \, .
\]
Hierbei muss vorausgesetzt werden, dass die Nebenbedingungen in $\mcal W_k$ die LICQ-Bedingung erfüllen, selbst wenn diese bezogen auf alle Nebenbedingungen an der Stelle $x_k$ nicht erfüllt wird.

Wir betrachten nun den $k$-ten Schritt mit der Approximierten $\bs x_k$ und dem working set $\mcal W_k$. Wir berechnen die neue Iterierte $\bs x_{k+1}$, indem wir eine Richtung $\bs p$ finden, in der wir unter den Nebenbedingungen $\mcal W_k$ die Funktion $q$ minimieren. Hierfür betrachten wir $\bs x_{k+1} = \bs x_k + \bs p$ und setzen $\bs x_{k+1}$ in $q$ ein:
\begin{align*}
	 q(\bs x_{k+1}) & = q(\bs x_k+\bs p) = \frac 1 2 (\bs x_k + \bs p)^T G (\bs x_k + \bs p) + (\bs x_k + \bs p)^T \bs c \\
	& = \frac 1 2 \bs x^T_k G \bs x_k + \underbrace{\bs x_k^T G \bs p}_{\text{da $G$ symm.}} + \frac 1 2 \bs p^T G\bs p +\bs x_k^T \bs c +\bs p^T \bs c \\
	& = \frac 1 2\bs p^T G \bs p +\bs  g_k^T \bs p + \rho_k \, ,
\end{align*}
wobei $\bs g_k = G\bs x_k+\bs c$ und $\rho_k = \frac 1 2 \bs x_k^T G \bs x_k + \bs x_k^T \bs c$. Da wir den Parameter $\bs p$ so wählen wollen, so dass $q(\bs x_{k+1})$ minimal wird, ist der Term $\rho_k$ bzgl. des Problems konstant und  kann somit für die Lösung jenes weggelassen werden. Da weiterhin auch $\bs x_{k+1}$ die aktiven Nebenbedingungen $\mcal W_k$ erfüllen soll, gilt
\[
	\bs a_i^T \bs p =\bs a_i^T (\bs x_{k+1} - \bs x_k) = \underbrace{\bs a_i^T\bs x_{k+1}}_{=b_i} - \underbrace{\bs a_i^T \bs x_k}_{=b_i} = 0 \quad \forall \, i \in \mcal W_k \, .
\]
Zusammengefasst müssen wir also im $k$-ten Schritt das Teilproblem
\begin{align}\label{eq:subprob}
\begin{aligned}
	\min_{\bs p} & \quad \frac 1 2\bs p^T G \bs p + \bs g_k^T \bs p \, , \\
	\text{s.t.} & \quad \bs a_i^T\bs p = 0 \, , \quad \forall i \in \mcal W_k 
\end{aligned}
\end{align}
lösen. Die Lösung im $k$-ten Schritt von \eqref{eq:subprob} bezeichnen wir mit $\bs p_k$. Umgekehrt gilt damit, analog zur obigen Rechnung, natürlich auch, dass für alle $i \in \mcal W_k$ die Restriktion aktiv bleibt für $\bs x_k + \alpha \bs p_k$ mit beliebigem $\alpha$. Da $G$ positiv definit ist, kann \eqref{eq:subprob} nun – wie schon bei \eqref{eq:active} erwähnt – mit Schur-Komplement-Methode oder Nullraum-Methode gelöst werden.

Wie wir schon wissen, ist die neue Iterierte $\bs x_{k+1} = \bs x_k + \bs p_k$ bzgl. $\mcal W_k$ immer noch zulässig. Nun müssen wir jedoch feststellen, ob diese Iterierte auch alle übrigen Restriktionen mit $i\not\in \mcal W_k$ erfüllt. Ist dies der Fall, so setzen wir $\bs x_{k+1} = \bs x_k +\bs p_k$, ansonsten suchen wir das größtmögliche $\alpha_k \in [0,1]$, so dass
\[
	\bs x_{k+1} =\bs x_k + \alpha_k \bs p_k
\] 
zulässig bleibt. Hierfür betrachten wir zwei Fälle.

\underline{Fall 1:} Gilt für ein $i \not \in \mcal W_k$, dass $\bs a_i^T \bs p_k \ge 0$ ist, so folgt
\[
	\bs a_i^T (\bs x_k + \alpha_k \bs p_k) =\bs a_i^T \bs x_k + \underbrace{\alpha_k \bs a_i^T\bs p_k}_{\ge 0} \ge \bs a_i^T \bs x_k \ge b_i \, ,
\]
da $\alpha_k \ge 0$, d.h. für diese Nebenbedingungen müssen wir für die Wahl von $\alpha_k$ nichts beachten.

\underline{Fall 2:} Existiert ein $i \not \in \mcal W_k$, für das $\bs a_i^T\bs p_k < 0$ ist, so gilt
\begin{align}\label{eq:alpha}
\notag	& \bs a_i^T (\bs x_k + \alpha_k \bs p_k) \ge b_i \\
\notag	\Llra \quad &\bs a_i^T \bs x_k + \alpha_k \bs a_i^T\bs p_k \ge b_i \\
\notag	\Llra \quad & \alpha_k \underbrace{\bs a_i^T\bs p_k}_{< 0} \ge b_i - \bs a_i^T \bs x_k \\
	\Llra \quad &  \alpha_k \le \frac{b_i - \bs a_i^T\bs x_k }{\bs a_i^T \bs p_k} \, .
\end{align}

Damit folgt mit \eqref{eq:alpha} und den vorherigen Überlegungen, dass zusammengefasst
\begin{align}\label{eq:alpha_k}
	\alpha_k = \min\left\{ 1,\min_{i \not\in \mcal W_k, \bs a_i^T\bs p_k < 0}  \frac{b_i -\bs a_i^T\bs x_k }{\bs a_i^T \bs p_k}  \right\}
\end{align}
gilt. Eine Restriktion $i \not \in \mcal W_k$, für die das Minimum für $\alpha_k$ angenommen wird, nennen wir \textit{blocking constraint}; diese muss nicht eindeutig sein, da wir beispielsweise anschaulich auch von einer Ecke geblockt werden können. Ist $\alpha_k = 1$, so werden alle Restriktion außerhalb vom {working set} mit dem Schritt $\bs x_{k+1} =\bs x_k + \bs p_k$ erfüllt, d.h. es gibt keine {blocking constraint}. Gibt es eine Nebenbedingung $j \not\in \mcal W_k$, die aktiv ist, obwohl sie nicht zum working set gehört, so gilt
\begin{align*}
	\alpha_k & = \min\left\{ 1,\min_{i \not\in \mcal W_k, \bs a_i^T\bs p_k < 0}  \frac{b_i -\bs a_i^T\bs x_k }{\bs a_i^T\bs p_k}  \right\}  \\
	& = \min\left\{ 1, \frac{b_j - {\bs a_j^T \bs x_k}}{\bs a_j^T\bs p_k}  \right\}  \\
	&  = \min \left\{1,\frac{b_j-b_j}{\bs a_j^T\bs p_k}\right\} = 0 \, .
\end{align*}
Es sei $j \not \in \mcal W_k$ nun ein Index einer {blocking constraint}. Dann ist
\[
	\bs x_{k+1} =\bs x_k + \alpha_k \bs p_k = \bs x_k + \frac{b_j - {\bs a_j^T \bs x_k}}{\bs a_j^T\bs p_k}\bs p_k \, .
\]
Setzen wir $\bs x_{k+1}$ in die $j$-te Restriktion ein, so erhalten wir
\begin{align*}
	\bs a_j^T\bs x_{k+1} & = \bs a_j^T \( \bs x_k +  \frac{b_j - \bs a_j^T\bs x_k}{\bs a_j^T\bs p_k}\bs p_k  \) = \bs a_j^T\bs x_k + \frac{b_j - {\bs a_j^T\bs x_k}}{\cancel{\bs a_j^T\bs p_k}} \cdot \cancel{\bs a_j^T\bs p_k} \\
	& =\bs a_j^T\bs x_k + b_j -\bs a_j^T\bs x_k = b_j \, ,
\end{align*}
d.h. die {blocking constraint} ist für die neue Iterierte $\bs x_{k+1}$ nach Konstruktion aktiv. Daher setzen wir als neues {working set} $\mcal W_{k+1} = \mcal W_k \cup \{ j\}$.

Das oben beschriebene Vorgehen wiederholen wir so lange, bis wir das {working set} $\hat {\mcal W}$ mit dem Minimum des quadratischen Problems $\hat{\bs x}$ gefunden haben. Dies ist leicht zu erkennen, da wir \eqref{eq:QP} auf $\mcal W_k$ nicht weiter minimieren können, sobald es keinen Schritt $p$ gibt, in dessen Richtung wir $q$ verringern können, d.h. wenn $\bs p = \bs 0$ die Lösung für das Teilproblem \eqref{eq:subprob} ist.  Dann ist der optimale Punkt $\hat{\bs x}$ bzgl. des {working sets} $\hat{\mcal W}\subset \mcal A(\hat{\bs x})$ gefunden.

Wir müssen jetzt überprüfen, ob $\hat{\bs x}$ die KKT-Bedingungen erfüllt. Wir wissen, dass für $\bs p= \bs 0$ die KKT-Bedingungen für \eqref{eq:subprob}
\begin{align*}
\begin{pmatrix}
	G & A^T \\
	A & 0
\end{pmatrix} \cdot
\begin{pmatrix} 
-\bs p \\
\hat{\bs\lambda}
 \end{pmatrix} =
 \begin{pmatrix}
 	\hat{\bs g} \\
	\hat{\bs h}
 \end{pmatrix}
\end{align*}
mit $\hat{\bs g} = \bs c + G\hat{\bs x}, \bs h = A\hat{\bs x}+\bs b$ und $\bs p=\bs 0$ erfüllt. Daraus folgt
\begin{align*}
	A^T \hat{\bs \lambda} = \hat{\bs g}  \quad &\Llra \quad \sum_{i \in \hat{\mcal W}}\bs a_i \hat\lambda_i = G\hat{\bs x} +\bs c \, , \\
	\bs 0 = \hat{\bs h} \quad  & \Llra \quad A\hat{\bs x} =\bs b\, , 
\end{align*}
wobei $A$ die Gradienten $\bs a_i^T$ der aktiven Restriktionen $\hat{\mcal W}$ zeilenweise  enthält. Damit werden die ersten beiden KKT-Bedingungen aus \eqref{eq:KKT} erfüllt. Da die Schrittlänge $\alpha_k$ mit \eqref{eq:alpha} so gewählt ist, dass die übrigen Restriktionen erfüllt bleiben, gilt auch die dritte Bedingung aus \eqref{eq:KKT}. Es bleibt zu überprüfen, ob die Lagrange-Multiplikatoren $\hat\lambda_i \ge 0$ sind.

Gilt $\hat\lambda_i \ge 0$ für alle $i \in \hat{\mcal W} \cap \mcal I$, so sind alle KKT-Bedingungen erfüllt und damit $\bs x^* = \hat{\bs x}$. Existiert allerdings ein $j \in \hat{\mcal W} \cap \mcal I$, so dass $\hat \lambda_j < 0$ ist, so können wir den Wert von $q$ noch weiter verringern, indem wir die $j$-te Restriktion wegfallen lassen (vlg. \cite{NocWri}, Kapitel 12.3). Dies zeigt das folgende Theorem.

\begin{theorem}
Der Punkt $\hat{\bs x}$ erfülle die notwendigen Bedingungen 1. Ordnung für das Teilproblem \textnormal{\eqref{eq:subprob}} auf $\hat{\mcal W}$. Weiter seien die Gradienten $\bs a_i, i \in \hat{\mcal W}$, linear unabhängig $($LICQ$)$ und es gebe einen Index $j \in \mcal W$ mit $\hat \lambda_j<0$. Es sei $\bs p$ die Lösung vom Teilproblem \textnormal{\eqref{eq:subprob}} ohne die Restriktion $j$, d.h.
\begin{align*}
	\min_{\bs p} & \quad \frac 1 2\bs p^T G\bs p + (G\hat{ \bs x} +\bs c)^T\bs p \, , \\
	\textnormal{s.t.} & \quad\bs a_i^T\bs p = 0 \, , \quad  \forall \, i \in \hat{\mcal W} \setminus \{j\} \, .
\end{align*}
Dann ist $p$ eine zulässige Richtung für die Nebenbedingung $j$, d.h. $\bs a_j^T \bs p \ge 0$. Weiterhin gilt sogar $\bs a_j^T\bs p > 0$ und $p$ ist eine Abstiegsrichtung für $q$, wenn $\bs p$ die hinreichenden Bedingungen 2. Ordnung erfüllt.
\end{theorem}

Da wir zeigen können, dass der erzielte Abstieg für $q$ durch das Weglassen einer Nebenbedingung mit negativem Lagrange-Multiplikator $\lambda_i$ proportional zu $\abs{\lambda_i}$ ist, eliminieren wir gerade die Restriktion mit kleinstem Langrange-Multiplikator. Es kann allerdings sein, dass der folgende zu berechnende Schritt $\bs p$ aufgrund einer {blocking constraint} kurz ist, wodurch nicht garantiert ist, dass $q$ den größtmöglichen Abstieg erfährt.

%Wie oben beschrieben fordern wir, dass die Gradienten im {working set} $\mcal W_k$ linear unabhängig sind (LICQ-Bedingung). Betrachten wir nun unsere Strategie in der Active-Set-Methode, in das {working set} Restriktionen hinzuzufügen oder welche aus diesem zu eliminieren, sieht man leicht, dass die Elimination von Nebenbedingungen die lineare Abhängigkeit der Gradienten im working set nicht hervorrufen kann. Man kann zudem zeigen, dass eine {blocking constraint} immer linear unabhängig ist zu den schon bestehenden aktiven Nebenbedingungen. Damit kann auch das Hinzufügen die lineare Unabhängigkeit nicht beeinflussen.
%
%Das Eliminieren und Hinzufügen von Nebenbedingungen führt dazu, dass der in Kapitel 3 aufgerührte Algorithmus eine natürliche untere Schranke erhält. Wenn wir beispielsweise eine Lösung $x^*$ eines Problems haben, in der $m$ der Ungleichungs-Restriktionen aktiv sind, in unserer Startnäherung $x_0$ allerdings keine Ungleichungs-Restriktion, so benötigen wir mindestens $m$ Schritte, um von $x_0$ zu $x^*$ zu gelangen, da wir in jedem Schritt genau eine Nebenbedingung hinzufügen oder eliminieren.


\section{Algorithmus}
\label{anhang:B.3}

\begin{algorithm}[H]
\caption{Active-Set-Methode für konvexe quadratische Probleme}
Gegeben sei ein zulässiger Startpunkt $\bs x_0$ für \eqref{eq:QP} und definiere $\mcal W_0$ z.B. mit allen aktiven Restriktionen bzgl. $\bs x_0$.
\begin{algorithmic}
\For{k = 0,1,2,\ldots}
\State Löse \eqref{eq:subprob} zur Berechnung von $\bs p_k$;
\If {$\bs p_k = \bs 0$}
\State Berechne die Lagrange-Multiplikatoren mittels (2.5a)
\State \quad und setze $\hat{\mcal W} = \mcal W_k$;
\If{$\hat\lambda_i \ge 0 \, \forall \, i \in \hat{\mcal W} \cap \mcal I$}
\State \textbf{stop} mit der Lösung $\bs x^* = \hat{\bs x}$;
\Else
\State $j \la \arg\min_{j\in \mcal W_k \cap \mcal I} \hat\lambda_j;$
\State $\bs x_{k+1} \la \bs x_k, \mcal W_{k+1} \la W_k \setminus \{j\}$;
\EndIf
\Else \quad ($\bs p_k \not= \bs 0$)
\State Berechne $\alpha_k$ mit \eqref{eq:alpha_k};
\State $\bs x_{k+1} \la  \bs x_k + \alpha_k\bs p_k$;
\If {$\alpha_k < 1$ ({blocking constraint} existiert)}
\State Bestimme  blocking constraint $j$ und setze $\mcal W_{k+1} \la \mcal W_k \cup \{j\}$
\Else
\State $\mcal W_{k+1} \la \mcal W_k$
\EndIf
\EndIf
\EndFor
\end{algorithmic}
\end{algorithm}







\newchapter{Tensorrechnung}
\label{anhang:C}

hier auch ein paar Integralsätzer???






\newchapter{Quellcode}
\label{anhang:D}

\section{Implementierung des Fehlerschätzers für das Hindernisproblem}

%\newpage
%\newchapter{noch mehr Quellcode}



%%% Local Variables: 
%%% mode: latex
%%% TeX-master: "Skript"
%%% End: 


\addcontentsline{toc}{chapter}{Index}
\printindex

\backmatter
\end{document}

%%% Local Variables: 
%%% mode: latex
%%% TeX-master: t
%%% End: 
