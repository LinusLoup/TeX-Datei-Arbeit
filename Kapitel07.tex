\newchapter{Zusammenfassung und Ausblick}

\begin{itemize}
\item kurz einleiten, worum es ging (Einleitung in einem Absatz zusammenfassen) 
\item Was ist rausgekommen?!
\item Ausblick: Was ist noch offen geblieben, was kann man noch machen... \\
%Hierbei darauf hinweisen, dass wir in dieser Arbeit davon ausgegangen sind, dass wir einen linearisierten Verzerrungstensor betrachten. Dies ist für hyperelastische oder auch plastische Verformungen natürlich nicht mehr erfüllt und daher könnte man die Ideen noch weiter Verallgemeinern, indem man die nichtlinearen Verzerrungstensoren (bzgl. der verschiedenen Konfigurationen) betrachtet.
In dieser Arbeit linearisierte Verzerrung verwendet; kann verallgemeinert werden durch allgemeine Verzerrungstensoren (bzgl. der jeweiligen Konfiguration).\\
Vielleicht statt refinemesh ein Bisectionsverfahren implementieren $\Ra$ weniger Verfeinerung pro Schritt. Nachteil: der Regulatitätsparameter bleibt in keinem Dreieck gleich
\end{itemize}

\newpage

%%% Local Variables: 
%%% mode: latex
%%% TeX-master: "Skript"
%%% End: 
