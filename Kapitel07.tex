\newchapter{Zusammenfassung und Ausblick}
\label{kap:7}

Adaptive Verfeinerungsstrategien ermöglichen für partielle Differentialgleichungen mit Nebenbedingung eine gute Steuerung der Netzverfeinerung, um hinreichend genaue Ergebnisse bei möglichst geringem Aufwand zu erhalten.

Eine solche Verfeinerungsstrategie haben wir in der vorliegendem Arbeit für einen hierarchischen a posteriori Fehlerschätzer für Hindernisprobleme untersucht. In der Theorie konnten wir in Kapitel \ref{kap:4.1} zeigen, dass der von uns betrachtete Fehlerschätzer eine obere Schranke bzgl. des exakten Fehlers bis auf Addition von Termen höherer Ordnung, den Oszillationstermen, darstellt. Durch die Implementierung (s. Anhang \ref{anhang:D.1}) und Auswertung in Matlab im Kapitel \ref{kap:6.1} konnte dies bestätigt werden. Weiter haben wir in Kapitel \ref{kap:6.2} das Hertz'sche Kontaktproblem untersucht, auf das wir die Konzepte des hergeleiteten hierarchischen Fehlerschätzers übertragen haben (vgl. Kapitel \ref{kap:4.4}). Hierbei konnte man sehen, dass die, bzgl. des Schätzers, erzeugten Verfeinerungen grundlegend zu einer hinreichenden Verbesserung der Lösung geführt haben.

Abschließend lässt sich somit feststellen, dass der hergeleitete a posteriori Fehlerschätzer auf Hindernisprobleme  angewendet werden kann und es sinnvoll ist die theoretischen Grundlagen aufgrund der Ergebnisse aus Kapitel \ref{kap:6.2} bzgl. Kontaktproblemen zu erweitern.

\vspace{11pt}

Um eine Erweiterung der theoretischen Grundlagen auf Kontaktprobleme für den hierarchischen a posteriori Fehlerschätzer aus Kapitel \ref{kap:4.1} herzuleiten, kann man grundlegend analog vorgehen. Die lokale Aufteilung des Fehlerindikators und die Oszillationsterme sind in Kapitel \ref{kap:4.4} schon übertragen worden. Die Resultate für die lokalen Projektionen aus Lemma \ref{lem:4.19} und \ref{lem:4.20} sind komponentenweise auf das Verschiebungsfeld $\bs u$ übertrag-bar und können damit auch insgesamt auf den zweidimensionalen Fall bezogen werden. Um jedoch jetzt ein ähnliches Vorgehen wie in Lemma \ref{lem:4.21} anwenden zu können, müssen auch die KKT-Bedingungen \eqref{eq:4.21} auf den zweidimensionalen Fall übertragen werden, damit diese dann in den einzelnen Fällen der disjunkten Zerlegung der Knoten $\mcal N$  aus Lemma \ref{lem:4.21} angewendet werden können. Hierbei zeigt sich jedoch die Schwierigkeit, dass die Nebenbedingung im Kontaktproblem für das zweidimensionale Verschiebungsfeld nur in eine Dimension (nämlich der Normalenrichtung bzgl. des Randes $\partial \Omega$) gilt. Dies bedeutet, dass wir die Bedingungen \eqref{eq:4.21} nicht komponentenweise und damit auch nicht, wie in Kapitel \ref{kap:4.1} vorgestellt, durch Addition einer Hutfunktion $\phi_p$ herleiten können, da im Kontaktfall die Folgerung davon abhängig ist, auf welchem Kontaktrand $\Gamma_c^i, i = 1,2,$ sich der Punkt $p$ befindet. Weiter erzeugt sich, da nur eine Teilmenge von $\Omega$ die Nebenbedingung für den Kontakt erfüllen muss, eine andere disjunkte Vereinigung als in Lemma \ref{lem:4.21} verwendet. Es lässt sich also vermuten, dass die grundlegende Schwierigkeit in der Übertragung der Theorie auf den Kontaktfall in der Kontaktbedingung liegt.

In der vorliegenden Arbeit sind wir von linear elastischen Materialien ausgegangen. Eine interessante Verallgemeinerung der Anwendung des hierarchischen Fehlerschätzers stellt daher beispielsweise die Betrachtung von hyperelastischen oder auch plastischen Verformungen dar (vgl. auch \cite{WriggersContact}). Hierfür sind allgemeinere Prinzipien aus der Kontinuumsmechanik (vgl. \cite{AltKonti}) und der nichtlinearen Finite-Elemente-Methode (vgl. \cite{WriggersFEM}) notwendig.

Da der in dieser Arbeit betrachtete Fehlerschätzer in lokale Anteile bzgl. der Knoten $\mcal N$ aufgeteilt wird, wäre es interessant zu untersuchen, ob eine eigene Verfeinerungsmethode, die auf einem Bisectionsalgorithmus beruht, bessere Verfeinerungen erzeugt. Nach der Auswahl der zu verfeinernden Knoten, werden in der vorliegenden Implementierung diejenigen Elemente gesucht, in denen die ausgewählten Knoten enthalten sind. Da dann in der Funktion {\ttfamily refinemesh} zuzüglich weitere Elemente verfeinert werden, um eine konforme Zerlegung zu garantieren, kann somit eine stärkere Verfeinerung stattfinden, als gewünscht ist. Der Bisectionsalgorithmus beruht darauf, dass die ausgewählten Dreiecke halbiert werden und kann insofern modifiziert werden, dass wir direkt nach der Auswahl der zu verfeinernden Knoten von diesen eine Winkelhalbierende zur gegenüberliegenden Seite ziehen. Mit einer Fallunterscheidung, in welchem Element, wie viele Knoten zur Verfeinerung gewählt wurden, kann man diesen Algorithmus noch weiter verbessern, indem man die Verfeinerungsstrategie aus Abbildung \ref{abb:2.6} implementiert.

\newpage

%%% Local Variables: 
%%% mode: latex
%%% TeX-master: "Skript"
%%% End: 
